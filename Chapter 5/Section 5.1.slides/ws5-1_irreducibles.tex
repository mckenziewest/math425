\documentclass[t]{beamer}

\subtitle{Section 5.1 Irreducibles and Unique Factorization}

\usepackage{amsthm,amsmath,amsfonts,hyperref,graphicx,color,multicol,soul}
\usepackage{enumitem,tikz,tikz-cd,setspace,mathtools}

%%%%%%%%%%
%Beamer Template Customization
%%%%%%%%%%
\setbeamertemplate{navigation symbols}{}
\setbeamertemplate{theorems}[ams style]
\setbeamertemplate{blocks}[rounded]

\definecolor{Blu}{RGB}{43,62,133} % UWEC Blue
\setbeamercolor{structure}{fg=Blu} % Titles

%Unnumbered footnotes:
\newcommand{\blfootnote}[1]{%
	\begingroup
	\renewcommand\thefootnote{}\footnote{#1}%
	\addtocounter{footnote}{-1}%
	\endgroup
}

%%%%%%%%%%
%TikZ Stuff
%%%%%%%%%%
\usetikzlibrary{arrows}
\usetikzlibrary{shapes.geometric}
\tikzset{
	smaller/.style={
		draw,
		regular polygon,
		regular polygon sides=3,
		fill=white,
		node distance=2cm,
		minimum height=1in,
		line width = 2pt
	}
}
\tikzset{
	smsquare/.style={
		draw,
		regular polygon,
		regular polygon sides=4,
		fill=white,
		node distance=2cm,
		minimum height=1in,
		line width = 2pt
	}
}


%%%%%%%%%%
%Custom Commands
%%%%%%%%%%

\newcommand{\C}{\mathbb{C}}
\newcommand{\quats}{\mathbb{H}}
\newcommand{\N}{\mathbb{N}}
\newcommand{\Q}{\mathbb{Q}}
\newcommand{\R}{\mathbb{R}}
\newcommand{\Z}{\mathbb{Z}}

\newcommand{\ds}{\displaystyle}

\newcommand{\fn}{\insertframenumber}

\newcommand{\id}{\operatorname{id}}
\newcommand{\im}{\operatorname{im}}
\newcommand{\Aut}{\operatorname{Aut}}
\newcommand{\Inn}{\operatorname{Inn}}

\newcommand{\blank}[1]{\underline{\hspace*{#1}}}

\newcommand{\abar}{\overline{a}}
\newcommand{\bbar}{\overline{b}}
\newcommand{\cbar}{\overline{c}}

\newcommand{\nml}{\unlhd}

%%%%%%%%%%
%Custom Theorem Environments
%%%%%%%%%%
\theoremstyle{definition}
\newtheorem{exercise}{Exercise}
\newtheorem{question}[exercise]{Question}
\newtheorem{warmup}{Warm-Up}
\newtheorem*{defn}{Definition}
\newtheorem*{exa}{Example}
\newtheorem*{disc}{Group Discussion}
\newtheorem*{nb}{Note}
\newtheorem*{recall}{Recall}
\renewcommand{\emph}[1]{{\color{blue}\texttt{#1}}}

\definecolor{Gold}{RGB}{237, 172, 26}
%Statement block
\newenvironment{statementblock}[1]{%
	\setbeamercolor{block body}{bg=Gold!20}
	\setbeamercolor{block title}{bg=Gold}
	\begin{block}{\textbf{#1.}}}{\end{block}}
\newenvironment{thm}[1]{%
	\setbeamercolor{block body}{bg=Gold!20}
	\setbeamercolor{block title}{bg=Gold}
	\begin{block}{\textbf{Theorem #1.}}}{\end{block}}


%%%%%%%%%%
%Custom Environment Wrappers
%%%%%%%%%%
\newcommand{\enumarabic}[1]{
	\begin{enumerate}[label=\textbf{\arabic*.}]
		#1
	\end{enumerate}
}
\newcommand{\enumalph}[1]{
	\begin{enumerate}[label=(\alph*)]
		#1
	\end{enumerate}
}
\newcommand{\bulletize}[1]{
	\begin{itemize}[label=$\bullet$]
		#1
	\end{itemize}
}
\newcommand{\circtize}[1]{
	\begin{itemize}[label=$\circ$]
		#1
	\end{itemize}
}
\newcommand{\slide}[1]{
	\begin{frame}{\fn}
		#1
	\end{frame}
}
\newcommand{\slidec}[1]{
\begin{frame}[c]{\fn}
	#1
\end{frame}
}
\newcommand{\slidet}[2]{
	\begin{frame}{\fn\ - #1}
		#2
	\end{frame}
}


\newcommand{\startdoc}{
		\title{Math 425: Abstract Algebra 1}
		\author{Mckenzie West}
		\date{Last Updated: \today}
		\begin{frame}
			\maketitle
		\end{frame}
}

\newcommand{\topics}[2]{
	\begin{frame}{\insertframenumber}
		\begin{block}{\textbf{Last Section.}}
			\begin{itemize}[label=--]
				#1
			\end{itemize}
		\end{block}
		\begin{block}{\textbf{This Section.}}
			\begin{itemize}[label=--]
				#2
			\end{itemize}
		\end{block}
	\end{frame}
}

\begin{document} 
	\startdoc
	
	\topics{
		\item Algebraic structure of quotients of polynomial rings
	}
	{
		\item Factorization
		\item Divisibility in Rings
		\item Irreducible Ring Elements
	}

\slide{
	\begin{block}{\textbf{Motivation.}}
		Consider $R=\Z$, factor $a=70$ into prime elements.\vskip 2in\mbox{}
	\end{block}
	\begin{block}{\textbf{Goal.}}
		What do primes mean in other rings what does it mean to factor elements?
	\end{block}
}
\slide{
	\begin{defn}
		Let $R$ be an integral domain (with unity) and $a,b\in R$.  We say $a$ \emph{divides} $b$ and write $a|b$ if $ac=b$ for some $c\in R$.
	\end{defn}
	\begin{block}{\textbf{Some Properties.}}
		Let $R$ be an integral domain and $a,b,c\in R$.
		\enumarabic{\setlength{\itemsep}{1em}
			\item $a|a$
			\item if $a|b$ and $b|c$, then $a|c$
			\item if $a|b$ and $a|c$, then $a|(br+cs)$ for all $r,s\in R$
		}
	\end{block}
}
\slide{
	\begin{nb}
		In $R=\Z$, if $a|b$ and $b|a$, then \blank{1in}.
	\end{nb}
	\begin{thm}{5.1.1}
		If $R$ is an integral domain the following are equivalent for $a,b\in R$:\enumarabic{
			\item $a|b$ and $b|a$
			\item $a=ub$ for some unit $u\in R^*$
			\item $(a)=(b)$
		}
	\end{thm}
}
\slide{\vskip -.25in
	\begin{proof}
		Let $R$ be an integral domain and $a,b\in R$.
		
		$(1)\Rightarrow(2)$ Assume $a|b$ and $b|a$.\vskip 1in
		$(2)\Rightarrow(3)$ Assume $a=ub$ for some unit $u\in R^*$\vskip 1in
		$(3)\Rightarrow(1)$ Assume $(a)=(b)$.\vskip 1in
	\end{proof}
}
\slide{
	\begin{defn}
		We call $a,b\in R$ an integral domain \emph{associates} if $a|b$ and $b|a$.  We may write $a\sim b$.
	\end{defn}
	\begin{exa}
		Show $\sqrt{3}\sim(3+2\sqrt{3})$ in $R=\Z[\sqrt{3}]$.
		
		Book has one method, let's attempt another. Consider
		
		$$\frac{\sqrt{3}}{3+2\sqrt{3}}=\hspace*{2in}$$
		\vskip 2in
	\end{exa}
}
\slide{
	\begin{defn}
		We call an element $p$ in the integral domain $R$ \emph{irreducible} if
		\enumarabic{\item $p\neq0$ and $p$ is not a unit\item If $p=ab$ in $R$, then $a$ or $b$ is a unit in $R$.}
		Anything that is not irreducible is called \emph{reducible}.
	\end{defn}
}


\slide{
\begin{exa}
	Let's talk $\Z[\sqrt{d}]$ where $d\in\Z$ is \emph{squarefree}, that is $p^2\nmid d$ for all primes $p$.
	
	\bulletize{\setlength{\itemsep}{3em}
		\item Norm:
		\item Norm is Multiplicative:
		\item Units:
	}
\end{exa}
}
\slide{
\begin{exercise}
	Show that $4+\sqrt{5}$ is irreducible in $\Z[\sqrt{5}]$.
	\vskip 1in
	Do the same for $1+\sqrt{5}$.
	\vskip 2in\mbox{}
\end{exercise}
}

\slide{
	\begin{thm}{5.1.2}
		If $R$ is an integral domain, the following conditions are equivalent for a nonzero nonunit $p$ in $R$.
		\enumarabic{
			\item $p$ is irreducible.
			\item If $d|p$, then $d\sim 1$ or $d\sim p$.
			\item If $p\sim ab$ in $R$, then $\sim a$ or $p\sim b$.
			\item If $p=ab$ in $R$, then $p\sim a$ or $p\sim b$.
		}
	\end{thm}
	\begin{proof}
		\vskip 2in
	\end{proof}
}
\slide{
	\begin{statementblock}{Corollary}
		If $p\sim q$ in an integral domain $R$, then
		\begin{center} $p$ is irreducible if and only if $q$ is irreducible\end{center}
	\end{statementblock}
}
\slide{
	\begin{defn}
		We call an element $p$ of an integral domain $R$ \emph{prime} if
		\enumarabic{
			\item $p$ is nonzero and not a unit
			\item If $p|ab$ in $R$, then $p|a$ or $p|b$.
		}
	\end{defn}
	\begin{thm}{5.1.5}
		If $R$ is an integral domain and $p$ is prime, then $p$ is irreducible.
	\end{thm}
}


\slide{
	\begin{thm}{5.1.5}
		If $R$ is an integral domain and $p$ is prime, then $p$ is irreducible.
	\end{thm}
	\begin{block}{\textbf{Warning.}}
		The converse does not hold!  For example $p=1+\sqrt{-5}$ in $R=\Z[\sqrt{-5}]$.
		\bulletize{
			\item $p$ is irreducible:\vskip .75in
			\item $p$ is not prime:\vskip .75in\mbox{}
		}
	\end{block}
}
\slide{
	\begin{defn}
		An integral domain $R$ is called a \emph{unique factorization domain} if is satisfies both
		\enumarabic{
			\item Every nonzero nonunit in $R$ is a product of irreducibles.
			\item If $p_1p_2\cdots p_r\sim q_1q_2\cdots q_s$ where the $p_i$ and $q_j$ are irreducibles, then $r=s$ and (after possible relabeling) $p_i\sim q_i$ for each $i$.
		}
	\end{defn}
	\begin{statementblock}{Lemma 5.1.3}
		If $R$ is a UFD, then every irreducible in $R$ is prime.
	\end{statementblock}
}
\slide{
	\begin{exa}
		\bulletize{
			\item $\Z[i]$ is a UFD
			\item $\Z\left[\frac{-1+i\sqrt{3}}{2}\right]=\Z[e^{2\pi i/3}]$ is a UFD, we call this set the \emph{Eistenstein integers}
			\item $\Z[\sqrt{-5}]$ is not a UFD
			\item $d$ positive squarefree integer, $\Z[\sqrt{-d}]$ is a UFD  if and only if $d=1,2,3,7,11,19,43,67,163$, the so-called ``Heegner Numbers''}
	\end{exa}
}

\slide{
	\begin{thm}{5.1.10}
		If $R$ is a UFD, then so is $R[x]$.
	\end{thm}
	\begin{statementblock}{Corollary}
		If $R$ is a UFD, then so is $R[x_1,x_2,\dots,x_n]$.
	\end{statementblock}
}

\slide{
	\begin{exercise}
		If $R[x]$ is a UFD does that mean $R$ is too?\vskip 3in\mbox{}
	\end{exercise}
}
\slide{
	\begin{defn}
		Let $R$ be an integral domain and $s_1,s_2,\dots,s_n\in R$. A \emph{greatest common divisor} of $s_1,s_2,\dots,s_n$, denoted
		\[\gcd(s_1,s_2,\dots,s_n),\]
		is an element $d\in R$ such that
		\enumarabic{
			\item $d\mid s_i$ for all $i$
			\item If $r\in R$ satisfies $r|s_i$ for all $i$, then $r|d$ too.
		}
	\end{defn}
	\begin{nb}
		This element $d$ is unique only up to associates.  
	\end{nb}
}
\slide{
	\begin{defn}
		Let $R$ be an integral domain and $s_1,s_2,\dots,s_n\in R$. A \emph{least common multiple} of $s_1,s_2,\dots,s_n$, denoted
		\[\operatorname{lcm}(s_1,s_2,\dots,s_n),\]
		is an element $\ell\in R$ such that
		\enumarabic{
			\item $s_i|\ell$ for all $i$
			\item If $r\in R$ satisfies $s_i|r$ for all $i$, then $\ell|r$ too.
		}
	\end{defn}
	\begin{nb}
		This element $\ell$ is unique only up to associates.  
	\end{nb}
}

\end{document}

		