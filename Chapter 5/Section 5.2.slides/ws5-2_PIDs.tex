\documentclass[t]{beamer}

\subtitle{Section 5.2 Principal Ideal Domains}

\input{../../_tools/setup}

\begin{document} 
	\startdoc
	
	\topics{
		\item Factorization
		\item Divisibility in Rings
		\item Irreducible Ring Elements
	}
	{
		\item Principal Ideal Domains
	}

\slide{
	\begin{recall}
		Fields $\subseteq$ UFDs $\subseteq$ Integral Domains $\subseteq$ Commutative Rings $\subseteq$ Rings
		\vskip 1in
		In integral domains:
		\bulletize{
			\item \textbf{Irreducible} $p\in R$ such that $p=ab$ $\Rightarrow$ $a\sim 1$ or $b\sim 1$
			\item \textbf{Prime} $p\in R$ such that $p|ab$ $\Rightarrow$ $p|a$ or $p|b$
			\item Always True: Prime $\Rightarrow$ Irreducible
			\item In UFDs: Irreducible $\Rightarrow$ Prime
		}
	\end{recall}
}

\slide{
	\begin{defn}
		An integral domain $R$ is called a \emph{principal ideal domain (PID)} if every ideal of $R$ is principal. (Principal ideals are ideal of the form $(a)=aR$ for some $a\in R$.
	\end{defn}
	\begin{exa}
		\enumarabic{\item $\Z$\vskip .5in\item if $F$ is a field $F[x]$ is a PID\vskip .5in\item $\Z[i]$}
	\end{exa}
}

\slide{
	\begin{exercise}
		Show that $\Z[x]$ is not a PID.
	\end{exercise}
}

\slide{
	\begin{exercise}
		Show that $\Q[x,y]$ is not a PID.
	\end{exercise}
}

\slide{
	\begin{thm}{5.2.1}
		Let $R$ be a PID and let $a_1,a_2,\dots,a_n$ be nonzero elements of $R$.  Then $d\sim \gcd(a_1,a_2,\dots,a_n)$ exists and there exist $r_1,r_2,\dots,r_n\in R$ such that
			\[\gcd(a_1,a_2,\dots,a_n)=r_1a_1+r_2a_2+\cdots+r_na_n.\]
	\end{thm}
	\begin{proof}
		Let $R$ be a PID and let $a_1,a_2,\dots,a_n$ nonzero elements of $R$.
	\end{proof}
}

\slide{
	\begin{thm}{5.2.2}
		Ever PID is a UFD.
	\end{thm}
}

\slide{
	\begin{exercise}
		Let $A_1,A_2,\dots,A_n$ be ideals of a ring $R$.  We can define the \emph{sum} of the ideals as
			\[A_1+A_2+\cdots+A_n=\{x_1+x_2+\cdots+x_n\ |\ x_i\in R\ \forall\ i\}.\]
		How does this work in a PID?
		\vskip 3in\mbox{}
	\end{exercise}
}

\slide{
	If $a_1,a_2,\dots,a_n\in R$ for $R$ a PID.  What is
		\[\langle a_1\rangle\cap\langle a_2\rangle\cap\cdots\cap\langle a_n\rangle?\]
		\vskip 3in\mbox{}
}

\slide{
	\begin{thm}{5.2.3}
		The following are equivalent for a nonzero nonunit $p$ in a PID $R$:
		\enumarabic{
			\item $p$ is a prime
			\item $R/\langle p\rangle$ is a field
			\item $R/\langle p\rangle$ is an integral domain
		}
	\end{thm}
}

\slide{
	\begin{exercise}
		Show that Theorem 5.2.3 does not hold for $\Z[x]$.  In particular, show that $\langle x\rangle$ is a prime. And compute $\Z[x]/\langle x\rangle$.\vskip 3in\mbox{}
	\end{exercise}
	}

\begin{frame}[fragile]
	\frametitle{\fn}
	\begin{block}{\textbf{Some Number Theory}}
		Given $d\in Z$ square free, it's not easy to determine in general if $\Z[\sqrt{d}]$ is a PID.
		
		There's something called the class number that tells us "how far" it is from being a PID.  There isn't a formula for this though....
		
		\begin{block}{Some Sage code}
		\begin{verbatim}
		K.<a> = QuadraticField(-5)
		
		print(K.ideal(23).factor())
		
		J = K.ideal(23).factor()[0][0]
		
		print(K)
		print(J)
		print(J.is_principal())
		\end{verbatim}
		\end{block}
	\end{block}
\end{frame}

%\slide{
%	Let's end here.  Thank you for an amazing semester!
%	\vskip .25in
%	We've learned so much.  
%	\vskip .25in
%	It's been fun, thanks for interacting! 
%	\vskip .25in
%	You're all awesome, and I really look forward to your projects. 
%	\vskip .25in
%	Anything you all want to talk about?
%	\vskip .25in
%	Shall we do some kind of celebration next week?
%}
\end{document}

		