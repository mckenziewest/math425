\documentclass[12pt]{article}

\newcommand{\secname}{Section 0.3: Mappings}

\usepackage{amsthm,amsmath,amsfonts,hyperref,graphicx,color,multicol,soul}
\usepackage{enumitem,tikz,tikz-cd,setspace,mathtools}
\usepackage{colortbl}
\usepackage[margin=1in]{geometry}

%%%%%%%%%%
%Color Customization
%%%%%%%%%%

\definecolor{Blu}{RGB}{43,62,133} % UWEC Blue

%Unnumbered footnotes:
\newcommand{\blfootnote}[1]{%
	\begingroup
	\renewcommand\thefootnote{}\footnote{#1}%
	\addtocounter{footnote}{-1}%
	\endgroup
}

%%%%%%%%%%
%TikZ Stuff
%%%%%%%%%%
\usetikzlibrary{arrows}
\usetikzlibrary{shapes.geometric}
\tikzset{
	smaller/.style={
		draw,
		regular polygon,
		regular polygon sides=3,
		fill=white,
		node distance=2cm,
		minimum height=1in,
		line width = 2pt
	}
}
\tikzset{
	smsquare/.style={
		draw,
		regular polygon,
		regular polygon sides=4,
		fill=white,
		node distance=2cm,
		minimum height=1in,
		line width = 2pt
	}
}

%%%%%%%%%%
%Listing Setup
%%%%%%%%%%
\usepackage{listings}
\usepackage{caption, floatrow, makecell}%
\captionsetup{labelfont = sc}
\setcellgapes{3pt}

\definecolor{backcolour}{RGB}{237,236,230}
\definecolor{myblue}{RGB}{42,157,189}

\lstdefinestyle{mystyle}{
	language=Python,
	keywords=[2]{sage:},
	alsodigit={:,.,<,>},
	backgroundcolor=\color{backcolour},   
	commentstyle=\color{myblue},
	keywordstyle=\bfseries\color{Green},
	keywordstyle=[2]\color{purple},
	numberstyle=\tiny\color{Gray},
	stringstyle=\color{Orange},
	basicstyle=\ttfamily\footnotesize,
	breakatwhitespace=false,         
	breaklines=true,                 
	captionpos=b,                    
	keepspaces=true,                   
	showspaces=false,                
	showstringspaces=false,
	showtabs=false,                  
	tabsize=2
}

\lstset{style=mystyle}


%%%%%%%%%%
%Custom Commands
%%%%%%%%%%

\newcommand{\C}{\mathbb{C}}
\newcommand{\quats}{\mathbb{H}}
\newcommand{\N}{\mathbb{N}}
\newcommand{\Q}{\mathbb{Q}}
\newcommand{\R}{\mathbb{R}}
\newcommand{\Z}{\mathbb{Z}}

\newcommand{\ds}{\displaystyle}

\newcommand{\fn}{\insertframenumber}

\newcommand{\id}{\operatorname{id}}
\newcommand{\im}{\operatorname{im}}
\newcommand{\lcm}{\operatorname{lcm}}
\newcommand{\ord}{\operatorname{ord}}
\newcommand{\Aut}{\operatorname{Aut}}
\newcommand{\Inn}{\operatorname{Inn}}

\newcommand{\blank}[1]{\underline{\hspace*{#1}}}

\newcommand{\abar}{\overline{a}}
\newcommand{\bbar}{\overline{b}}
\newcommand{\cbar}{\overline{c}}

\newcommand{\nml}{\unlhd}

%%%%%%%%%%
%Custom Theorem Environments
%%%%%%%%%%
\theoremstyle{definition}
\newtheorem{exercise}{Exercise}
\newtheorem{question}[exercise]{Question}
\newtheorem{warmup}{Warm-Up}
\newtheorem*{exa}{Example}
\newtheorem*{defn}{Definition}
\newtheorem*{disc}{Group Discussion}
\newtheorem*{recall}{Recall}
\renewcommand{\emph}[1]{{\color{blue}\texttt{#1}}}

\definecolor{Gold}{RGB}{237, 172, 26}
%Statement block
%\newenvironment{statementblock}[1]{%
%	\setbeamercolor{block body}{bg=Gold!20}
%	\setbeamercolor{block title}{bg=Gold}
%	\begin{block}{\textbf{#1.}}}{\end{block}}
%\newenvironment{goldblock}{%
%	\setbeamercolor{block body}{bg=Gold!20}
%	\setbeamercolor{block title}{bg=Gold}
%	\setbeamertemplate{blocks}[shadow=true]
%	\begin{block}{}}{\end{block}}
%\newenvironment{defn}{%
%	\setbeamercolor{block body}{bg=gray!20}
%	\setbeamercolor{block title}{bg=violet, fg=white}
%	\setbeamertemplate{blocks}[shadow=true]
%	\begin{block}{\textbf{Definition.}}}{\end{block}}
%\newenvironment{nb}{%
%	\setbeamercolor{block body}{bg=gray!20}
%	\setbeamercolor{block title}{bg=teal, fg=white}
%	\setbeamertemplate{blocks}[shadow=true]
%	\begin{block}{\textbf{Note.}}}{\end{block}}
%\newenvironment{blockexample}{%
%	\setbeamercolor{block body}{bg=gray!20}
%	\setbeamercolor{block title}{bg=Blu, fg=white}
%	\setbeamertemplate{blocks}[shadow=true]
%	\begin{block}{\textbf{Example.}}}{\end{block}}
%\newenvironment{blocknonexample}{%
%	\setbeamercolor{block body}{bg=gray!20}
%	\setbeamercolor{block title}{bg=purple, fg=white}
%	\setbeamertemplate{blocks}[shadow=true]
%	\begin{block}{\textbf{Non-Example.}}}{\end{block}}
%\newenvironment{thm}[1]{%
%	\setbeamercolor{block body}{bg=Gold!20}
%	\setbeamercolor{block title}{bg=Gold}
%	\begin{block}{\textbf{Theorem #1.}}}{\end{block}}


%%%%%%%%%%
%Custom Environment Wrappers
%%%%%%%%%%
\newcommand{\exer}[1]{
	\begin{exercise}
	#1
	\end{exercise}
}
\newcommand{\exam}[1]{
\textbf{Example: }
	#1
}
\newcommand{\nexam}[1]{
	\textbf{Non-Example: }
	#1
}
\newcommand{\enumarabic}[1]{
	\begin{enumerate}[label=\textbf{\arabic*.}]
		#1
	\end{enumerate}
}
\newcommand{\enumalph}[1]{
	\begin{enumerate}[label=(\alph*)]
		#1
	\end{enumerate}
}
\newcommand{\bulletize}[1]{
	\begin{itemize}[label=$\bullet$]
		#1
	\end{itemize}
}
\newcommand{\circtize}[1]{
	\begin{itemize}[label=$\circ$]
		#1
	\end{itemize}
}
%\newcommand{\slide}[1]{
%	\begin{frame}{\fn}
%		#1
%	\end{frame}
%}
%\newcommand{\slidec}[1]{
%\begin{frame}[c]{\fn}
%	#1
%\end{frame}
%}
%\newcommand{\slidet}[2]{
%	\begin{frame}{\fn\ - #1}
%		#2
%	\end{frame}
%}


\setlength{\parindent}{0pt}



\usepackage{afterpage}
\usepackage{fancyhdr}

\fancyhead[L]{\textbf{Math 425: Abstract Algebra I\\\secname}}
\fancyhead[R]{\textbf{Mckenzie West\\Last Updated: \today}}
\pagestyle{fancy}

\newcommand{\startdoc}{}

\newcommand{\topics}[2]{
		{\textbf{Previously.}}
			\begin{itemize}[label=--]
				#1
			\end{itemize}
		{\textbf{This Section.}}
			\begin{itemize}[label=--]
				#2
			\end{itemize}
}
\begin{document} 
	\startdoc
	\begin{defn}
		A \emph{mapping} or \emph{function} $\alpha$ from $A$ to $B$ is a rule that assigns to every input $a\in A$ exactly one output $\alpha(a)\in B$.  
		
		\textbf{Notation:}
		\[\alpha\colon A\to B\textup{ or }A\xrightarrow{\alpha}B.\]
		
		Once we have verified that each input maps to exactly one output then we say the mapping is \emph{well-defined}.
	\end{defn}


\exam{
	\enumarabic{
		\item (Calculus) The map $\alpha:\R\to[-1,1]$ defined by $\alpha(x)=\sin(x)$.
		\item (Linear Algebra) The map $\beta:\R^2\to\R^2$ defined by $\beta(\vec{v})=\begin{bmatrix}
			1&3\\0&1
		\end{bmatrix}\vec{v}$
	}
}
\exer{
	You define a mapping $\gamma: \Z\to \{0,1\}$.\vskip 1in
}	%	\begin{nb}
		%		We will get into the concept of well-defined later in the course when we start discussing maps on equivalence classes.
		%	\end{nb}
{
	\begin{defn} Assume $\alpha:A \to B$ is a mapping.
		\begin{itemize}[label=--]
			\item We call $A$ the \emph{domain} of $\alpha$ and $B$ the \emph{codomain} of $\alpha$.
			\item If $C\subseteq A$, then the \emph{image} of $C$ is \[\alpha(C)=\{b\in B : b=\alpha(c)\text{ for some } c\in C\}.\] 
			\item The \emph{range} of $\alpha$ is the image of the domain, \[\im(\alpha)=\alpha(A)=\{\alpha(a)\in B : a\in A\}.\]
		\end{itemize}
	\end{defn}
}
{
	\begin{exercise}
		Define $\alpha:\Z\to\Z$ by $\alpha(n)=3n+1$. \enumarabic{\item Compute the image of $C=\{2,4,6\}$. \vskip 1in \item What is the range of $\alpha$?\vskip 1in}
	\end{exercise}
}

\newpage
{
	\begin{defn}
		Let $\alpha\colon A\to B$ be a mapping.
		\enumalph{
			\item We call $\alpha$ \emph{one-to-one} or \emph{injective} if:
				\begin{center}
					for all $a_1,a_2\in A$ if $\alpha(a_1)=\alpha(a_2)$, then $a_1=a_2$.
				\end{center} 
			\item We call $\alpha$ \emph{onto} or \emph{surjective} if 
			\begin{center}
				for all $b\in B$ there is an $a\in A$ such that $\alpha(a)=b$.
			\end{center} 
			\item We call $\alpha$ a \emph{bijection} or \emph{bijective} if $\alpha$ is both one-to-one and onto.
		}
		
	\end{defn}
	\begin{exercise}
		Define $\alpha:\Z\to\Z$ by $\alpha(n)=3n+1$. 
		\enumarabic{
			\item Is $\alpha$ one-to one?  \vfill
			\item Is $\alpha$ onto? \vfill
			\item Is $\alpha$ a bijection?\vfill
		}
	\end{exercise}
}
{
	{\textbf{Generic Proof of One-to-One.}}
	
		Statement: The map $\alpha:A\to B$ defined by $\alpha(a)=....$ is one-to-one.
		\begin{proof}
			Let $a_1,a_2\in A$ and assume $\alpha(a_1)=\alpha(a_2)$.
			
			\begin{center}
			$\vdots$\\
			use the definition of $\alpha$ and whatever theorems\\
			$\vdots$
			\end{center}
			
			Therefore $a_1=a_2$.  And so we can conclude that $\alpha$ is one-to-one.
		\end{proof}
		
	
}
{
	{\textbf{Generic Proof of Onto.}}
	
	Statement: The map $\alpha:A\to B$ defined by $\alpha(a)=....$ is onto.
	\begin{proof}
		Let $b\in B$.
		
		\begin{center}
			$\vdots$\\
			do some reverse engineering to pick just the right $a$\\
			$\vdots$
		\end{center}
		
		Therefore with this $a$ as we have defined it, we have $a\in A$ and $\alpha(a)=b$.  We can conclude that $\alpha$ is onto.
	\end{proof}
		
	
}
\newpage
\exer{
	Prove that the map $f:\R\to\R$ defined by $f(x)=4x-2$ is one-to-one.
	\vfill
}
\exer{
	Prove that the map $g:\R\to\R$ defined by $g(x)=x^3-1$ is onto.
	\vfill
}
\exer{
	Prove that the map $h:\R\to\R$ defined by $h(x)=x^2$ is neither one-to-one nor onto.  
	
	You really should use counterexamples here, not generic ``Let $a\in \R$'' statements.  That is, you should write something like ``Consider $x_1=1$ and $x_2=-1$.''
	\vfill
}
\newpage
{
	\begin{exercise}
		Write down a mapping $\alpha\colon \Z\to \Z$ that is
		\enumalph{
				\item neither one-to-one nor onto,\vskip 1in
				\item one-to-one and not onto,\vskip 1in
				\item onto and not one-to-one,\vskip 1in
				\item a bijection.\vskip 1in
				}
	\end{exercise}
	}
{
%	{\textbf{Brain Break.}}
%		What is your favorite class you've taken and why?
%	
%}
{
		\begin{defn}
				The \emph{identity map} for the set $A$ is the map $1_A:A\to A$ defined by $1_A(a)=a$ for all $a\in A$.
				
				If $\alpha:A\to B$ and $\beta:B\to C$ are mappings, we can write
					\[A\xrightarrow{\alpha}B\xrightarrow{\beta}C,\]
				and the \emph{composition} of the maps is the mapping $\beta\alpha:A\to C$ defined by
					\[\beta\circ\alpha(a)=\beta\alpha(a)=\beta[\alpha(a)]\text{ for all }a\in A.\]
			\end{defn}
	}
\exer{
	Let $\alpha:\N\to\R$ be defined by $\alpha(n)=\sqrt{n}$ and let $\beta:\R\to\Z$ be defined by $\beta(x)=\lfloor x\rfloor$ (the largest integer less than or equal to $x$.)
	\enumalph{
		\item Which of the following are allowable compositions and which are not?
			\begin{multicols}{3}
				\begin{enumerate}[label=(\roman*)]
					\item $1_\N \circ \alpha$\\
					\item $\alpha \circ 1_\N$
					\item $1_\R \circ \alpha$\\
					\item $\alpha \circ 1_\R$
					\item $\alpha \beta$\\
					\item $\beta\alpha$
				\end{enumerate}
			\end{multicols}
		\item Describe $\beta\alpha$.  (Write $\beta\alpha:\underline{\hspace*{.5in}}\to\underline{\hspace*{.5in}}$ and find a formula for $\beta\alpha(n)$.)
			\vskip 1in
	}
}
{
		\textbf{Theorem 0.3.3}
				Let $A\xrightarrow{\alpha}B\xrightarrow{\beta}C\xrightarrow{\gamma}D$ be mappings on sets.  Then
					\enumarabic{
							\item (identity) $\alpha 1_A=\alpha$ and $1_B\alpha=\alpha$
							\item (associativity) $\gamma(\beta\alpha)=(\gamma\beta)\alpha$
							\item If $\alpha$ and $\beta$ are both one-to-one (resp.~onto), then $\beta\alpha$ is one-to-one (resp.~onto) too.
						}
	%	\begin{exa}
		%		We will prove part 3 next time.
		%	\end{exa}
	}

\begin{defn}
	If $\alpha:A\to B$ is a mapping of sets, then we call $\beta:B\to A$ an \emph{inverse} of $\alpha$ if
		\[\beta\alpha = 1_A\text{ and }\alpha\beta=1_B.\]
\end{defn}
\exer{Define
	$\alpha:\R\to\R$ by $\alpha(x)=3x+1$ and 
	$\beta:\R\to \R$ by $\beta(x)=\frac{1}{3}x-\frac{1}{3}$.
	
	Show that $\alpha=\beta^{-1}$ by verifying that both $\alpha\beta(x)$ and $\beta\alpha(x)$ equal $x$ for all $x\in\R$.\vfill
}

{
		\textbf{Theorem 0.3.4}
				If $\alpha:A\to B$ has an inverse, then the inverse mapping is unique.
		\begin{proof}
				Let $\alpha:A\to B$ be a mapping with an inverse.  Let $\beta$ and $\beta'$ be two inverses of $\alpha$.
				We compute
					$$\begin{array}{rcll}
							\beta &=& \beta 1_B&\text{Theorem 0.3.3(a)}\\
								 &=&\beta(\alpha\beta')&\beta'\text{ inverse of }\alpha\\
								 &=&(\beta\alpha)\beta'&\text{Theorem 0.3.3(b)}\\
								 &=&1_A\beta'&\beta\text{ inverse of }\alpha\\
								 &=&\beta'&\text{Theorem 0.3.3(a)}.
						\end{array}$$
				Therefore $\beta=\beta'$, and the inverse of $\alpha$ is unique.
			\end{proof}
		{\textbf{Result.}}
				The notation $\alpha^{-1}$ is valid.

	}
{
		\textbf{Theorem 0.3.5}
				Let $\alpha:A\to B$ and $\beta: B \to C$ denote mappings.
				\enumarabic{
						\item The identity map, $1_A:A\to A$ is invertible and $1_A^{-1}=1_A$.
						\item If $\alpha$ is invertible, then $\alpha^{-1}$ is invertible and $(\alpha^{-1})^{-1}=\alpha$.
						\item If $\alpha$ and $\beta$ are both invertible, then $\beta\alpha$ is invertible with $(\beta\alpha)^{-1}=\alpha^{-1}\beta^{-1}$.
					}

	}
\newpage

\begin{block}{Invertibility Theorem (Theorem 0.3.6)}
		A mapping $\alpha:A\to B$ is invertible if and only if $\alpha$ is a bijection.
	\end{block}
\exer{Here, I have completed the first half of the proof of the Invertibility theorem.  You complete the second half.
	\begin{proof}
		($\Rightarrow$) Assume that $\alpha:A\to B$ is invertible.  Denote its inverse by $\beta:B\to A$.  We now show $\alpha$ is one-to-one and onto.
		
		Let $a_1,a_2\in A$ such that $\alpha(a_1)=\alpha(a_2)$. By definition of inverse, we have
			$$ a_1=\beta\alpha(a_1)\quad\text{and}\quad a_2=\beta\alpha(a_2).$$
		So now we can use substitution to find that
			$$a_1=\beta\alpha(a_1)=\beta\alpha(a_2)=a_2.$$
		Thus $a_1=a_2$, so $\alpha$ is one-to-one.
		
		Now let $b\in B$. Then $a=\beta(b)\in A$.  Furthermore, $$\alpha(a)=\alpha\beta(b)=b,$$
		by definition of inverse.  Therefore $b$ is in the image of $\alpha$, so $\alpha$ is onto.
		
		($\Leftarrow$) Assume that $\alpha$ is a bijection.  \vfill
	\end{proof}
}
	

{
	\begin{defn}
		We will call two maps $\alpha:A\to B$ and $\beta:A\to B$ equal if $\alpha(a)=\beta(a)$ for all $a\in A$.
	\end{defn}
	\begin{exa}
		Consider $\alpha:\Z\to \{0,1\}$ and $\beta:\Z\to \{0,1\}$ defined by 
		\[\alpha(n)=\begin{cases}
			0&\text{ if }n\text{ is even}\\
			1&\text{ if }n\text{ is odd}
		\end{cases}\text{ and }
		\beta(n)=\left\lceil \frac{n}{2}\right\rceil-\left\lfloor \frac{n}{2}\right\rfloor.\]
		These are equal but have very different feeling descriptions.
	\end{exa}
}
\end{document}

