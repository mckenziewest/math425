\documentclass[t]{beamer}

\subtitle{Introduction\\and\\Section 0.1: Proofs}

\usepackage{amsthm,amsmath,amsfonts,hyperref,graphicx,color,multicol,soul}
\usepackage{enumitem,tikz,tikz-cd,setspace,mathtools}

%%%%%%%%%%
%Beamer Template Customization
%%%%%%%%%%
\setbeamertemplate{navigation symbols}{}
\setbeamertemplate{theorems}[ams style]
\setbeamertemplate{blocks}[rounded]

\definecolor{Blu}{RGB}{43,62,133} % UWEC Blue
\setbeamercolor{structure}{fg=Blu} % Titles

%Unnumbered footnotes:
\newcommand{\blfootnote}[1]{%
	\begingroup
	\renewcommand\thefootnote{}\footnote{#1}%
	\addtocounter{footnote}{-1}%
	\endgroup
}

%%%%%%%%%%
%TikZ Stuff
%%%%%%%%%%
\usetikzlibrary{arrows}
\usetikzlibrary{shapes.geometric}
\tikzset{
	smaller/.style={
		draw,
		regular polygon,
		regular polygon sides=3,
		fill=white,
		node distance=2cm,
		minimum height=1in,
		line width = 2pt
	}
}
\tikzset{
	smsquare/.style={
		draw,
		regular polygon,
		regular polygon sides=4,
		fill=white,
		node distance=2cm,
		minimum height=1in,
		line width = 2pt
	}
}


%%%%%%%%%%
%Custom Commands
%%%%%%%%%%

\newcommand{\C}{\mathbb{C}}
\newcommand{\quats}{\mathbb{H}}
\newcommand{\N}{\mathbb{N}}
\newcommand{\Q}{\mathbb{Q}}
\newcommand{\R}{\mathbb{R}}
\newcommand{\Z}{\mathbb{Z}}

\newcommand{\ds}{\displaystyle}

\newcommand{\fn}{\insertframenumber}

\newcommand{\id}{\operatorname{id}}
\newcommand{\im}{\operatorname{im}}
\newcommand{\Aut}{\operatorname{Aut}}
\newcommand{\Inn}{\operatorname{Inn}}

\newcommand{\blank}[1]{\underline{\hspace*{#1}}}

\newcommand{\abar}{\overline{a}}
\newcommand{\bbar}{\overline{b}}
\newcommand{\cbar}{\overline{c}}

\newcommand{\nml}{\unlhd}

%%%%%%%%%%
%Custom Theorem Environments
%%%%%%%%%%
\theoremstyle{definition}
\newtheorem{exercise}{Exercise}
\newtheorem{question}[exercise]{Question}
\newtheorem{warmup}{Warm-Up}
\newtheorem*{defn}{Definition}
\newtheorem*{exa}{Example}
\newtheorem*{disc}{Group Discussion}
\newtheorem*{nb}{Note}
\newtheorem*{recall}{Recall}
\renewcommand{\emph}[1]{{\color{blue}\texttt{#1}}}

\definecolor{Gold}{RGB}{237, 172, 26}
%Statement block
\newenvironment{statementblock}[1]{%
	\setbeamercolor{block body}{bg=Gold!20}
	\setbeamercolor{block title}{bg=Gold}
	\begin{block}{\textbf{#1.}}}{\end{block}}
\newenvironment{thm}[1]{%
	\setbeamercolor{block body}{bg=Gold!20}
	\setbeamercolor{block title}{bg=Gold}
	\begin{block}{\textbf{Theorem #1.}}}{\end{block}}


%%%%%%%%%%
%Custom Environment Wrappers
%%%%%%%%%%
\newcommand{\enumarabic}[1]{
	\begin{enumerate}[label=\textbf{\arabic*.}]
		#1
	\end{enumerate}
}
\newcommand{\enumalph}[1]{
	\begin{enumerate}[label=(\alph*)]
		#1
	\end{enumerate}
}
\newcommand{\bulletize}[1]{
	\begin{itemize}[label=$\bullet$]
		#1
	\end{itemize}
}
\newcommand{\circtize}[1]{
	\begin{itemize}[label=$\circ$]
		#1
	\end{itemize}
}
\newcommand{\slide}[1]{
	\begin{frame}{\fn}
		#1
	\end{frame}
}
\newcommand{\slidec}[1]{
\begin{frame}[c]{\fn}
	#1
\end{frame}
}
\newcommand{\slidet}[2]{
	\begin{frame}{\fn\ - #1}
		#2
	\end{frame}
}


\newcommand{\startdoc}{
		\title{Math 425: Abstract Algebra 1}
		\author{Mckenzie West}
		\date{Last Updated: \today}
		\begin{frame}
			\maketitle
		\end{frame}
}

\newcommand{\topics}[2]{
	\begin{frame}{\insertframenumber}
		\begin{block}{\textbf{Last Section.}}
			\begin{itemize}[label=--]
				#1
			\end{itemize}
		\end{block}
		\begin{block}{\textbf{This Section.}}
			\begin{itemize}[label=--]
				#2
			\end{itemize}
		\end{block}
	\end{frame}
}

\begin{document} 
\startdoc

\topics{
	% Last Time
	\item Hmmm....
}
{
	% This time
	\item Course Introduction
	\item Direct Proofs
	\item Cases
	\item Contrapositive
	\item Contradiction
	\item Counterexamples
}


\slide{
	\begin{exercise}[3 Minutes]
		At your pods do the following: 
			\enumalph{
				\item Introduce yourself, share your name and any relevant information, such as your preferred toothpaste brand.\vskip 1cm
				\item Decide on one ``class time rule'', something that will help strengthen the group's learning experience.\vskip 1cm
				\item Decide who is going to report back to the group.
		}
	\end{exercise}
}
\slide{
	\begin{center}
		Syllabus stuff!  Please take time to read it.
	\end{center}
	\begin{statementblock}{Key Notes}
		\bulletize{
			\setlength{\itemsep}{.5em}
			\item Group work hours: Wed 10-11 and Fri 11-12 (HHH 311)
			\item Office Hours: Tu 2-3 (HHH 524/507)
			\item Welcoming environment
			\item Grading - Specs!
				\begin{itemize}[label=$\circ$]
					\item Objective Exercises
					\item Supplemental Exercises
					\item Project
				\end{itemize}
			\item Collaboration
			\item How to use the internet
		}
	\end{statementblock}
}
\slide{
	\begin{statementblock}{Objective Exercises}
		\bulletize{
			\item 27 Objectives, categorized by topic: Intro, Group, Ring
			\item About 2 assigned per week, due on Fridays
			\item Grading scale: Unsatisfactory/Satisfactory/Exceptional
			\item Credit earned for both S and E marks 
			\item May retry objectives: Must complete a \emph{new} exercise listed for the objective, rewrites due Wednesdays
			\item Your grade: number of objectives complete/22 (max 100\%)
		}
	\end{statementblock}
}
\slide{ 
	\begin{statementblock}{Supplemental Exercises}
	\bulletize{
		\item About 2 assigned per week.
		\item Grading scale: Unsatisfactory/Satisfactory/Exceptional
		\item 1\% earned for E marks
		\item 0.5\% earned for S marks
		\item Maximum of 15\% can be earned in this way
		\item May retry Supplemental Exercises: Must rewrite the \emph{same} exercise, rewrites due Wednesdays
	}
	\end{statementblock}
}
\slide{
	\begin{statementblock}{Benefits}
		\bulletize{
			\item Flexibility
			\item Room for growth and learning
			\item Focus on the main content
			\item More time for me to answer your questions
			\item Faster grading turnaround
		}
	\end{statementblock}
}
\slidec{
	\begin{center}
		\large Section 0.1: Proofs
	\end{center}
}
\slide{
	\begin{block}{Method of Direct Proof.}
		To prove a statement ``if $P$, then $Q$'', we assume $P$ is true and show that $Q$ must then also be true.
		\enumarabic{
			\item Express your statement in the form
				\[\forall x\in D,\ \text{if}\ P(x),\text{ then }Q(x).\]\vskip .5cm
			\item Start the proof by letting $x$ be an arbitrary element of $D$.  Assume $x$ satisfies the condition $P(x)$.\vskip 1cm
			\item Show that $Q(x)$ is true.
		}
	\end{block}
}
\slide{
	\begin{exa}
		Prove that the sum of any two even integers is even.
			\enumarabic{
				\item For all \pause integers $x$ and $y$, if $x$ and $y$ are even, then $x+y$ is even.
			}
			\begin{block}{Proof.}
				Let $x$ and $y$ be integers.  Assume $x$ and $y$ are even.
%			\enumarabic{
%				\item[\textbf{2.}] \pause
%				\item[\textbf{3.}] By the definition of even, there exist integers $r$ and $s$ such that $x=2r$ and $y=2s$.\pause
%				
%				Using basic arithmetic,
%					\[x+y=2r+2s=2(r+s).\]\pause
%				Let $t=r+s$.  Notice that since $r$ and $s$ are integers, so is $t$.  Thus $x+y=2t$ is even by definition.
%			}
		\end{block}
	\end{exa}
}
\slide{
	\begin{block}{Method of Reduction to Cases.}
		\begin{exa}
			Prove that if $n$ is an integer, then $5n^2+n-72$ is even.
		\begin{block}{Proof.}
				Let $n\in \Z$.  We consider two cases: $n$ is even or $n$ is odd.\pause
			\begin{description}
				\item[Case 1.] Assume $n$ is even.  Then there is a $s\in \Z$ such that $n=2s$.  Compute
					\begin{eqnarray*}
					5n^2+n-72 &=& 5(2s)^2+(2s)-72\\
						&=&5(4s^2) + 2s-72\\
						&=&2(10s^2+s-36)
					\end{eqnarray*}
				Since $s$ is an integer, so is $10s^2+s-36$. Thus $5n^2+n-72=2(10s^2+s-36)$ is even by definition whenever $n=2s$ is even.
			\end{description}
		\end{block}
		\end{exa}
	\end{block}
}
\slide{
	\begin{description}
		\item[Case 2.] Assume now that $n$ is odd.  Then there is a $s\in \Z$ such that $n=2s+1$.  Compute
		\begin{eqnarray*}
			5n^2+n-72 &=& 5(2s+1)^2+(2s+1)-72\\
			&=&5(4s^2+4s+1) + 2s-71\\
			&=&20s^2+20s+5+2s-71\\
			&=&2(10s^2+11s-33)
		\end{eqnarray*}
		Since $s$ is an integer, so is $10s^2+11s-33$. Thus $5n^2+n-72=2(10s^2+11s-33)$ is by definition even when $n=2s+1$ is odd.\hfill$\square$
	\end{description}
}

%\begin{frame}{\fn}
%	\begin{block}{\textbf{Brain Break.}}
%		What is your favorite word?
%		
%		\begin{center}
%			\includegraphics[width=2in]{images/word}
%		\end{center}
%	\end{block}
%\end{frame}
\end{document}

