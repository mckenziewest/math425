\documentclass[t]{beamer}

\subtitle{Section 0.3: Mappings}

\usepackage{amsthm,amsmath,amsfonts,hyperref,graphicx,color,multicol,soul}
\usepackage{enumitem,tikz,tikz-cd,setspace,mathtools}

%%%%%%%%%%
%Beamer Template Customization
%%%%%%%%%%
\setbeamertemplate{navigation symbols}{}
\setbeamertemplate{theorems}[ams style]
\setbeamertemplate{blocks}[rounded]

\definecolor{Blu}{RGB}{43,62,133} % UWEC Blue
\setbeamercolor{structure}{fg=Blu} % Titles

%Unnumbered footnotes:
\newcommand{\blfootnote}[1]{%
	\begingroup
	\renewcommand\thefootnote{}\footnote{#1}%
	\addtocounter{footnote}{-1}%
	\endgroup
}

%%%%%%%%%%
%TikZ Stuff
%%%%%%%%%%
\usetikzlibrary{arrows}
\usetikzlibrary{shapes.geometric}
\tikzset{
	smaller/.style={
		draw,
		regular polygon,
		regular polygon sides=3,
		fill=white,
		node distance=2cm,
		minimum height=1in,
		line width = 2pt
	}
}
\tikzset{
	smsquare/.style={
		draw,
		regular polygon,
		regular polygon sides=4,
		fill=white,
		node distance=2cm,
		minimum height=1in,
		line width = 2pt
	}
}


%%%%%%%%%%
%Custom Commands
%%%%%%%%%%

\newcommand{\C}{\mathbb{C}}
\newcommand{\quats}{\mathbb{H}}
\newcommand{\N}{\mathbb{N}}
\newcommand{\Q}{\mathbb{Q}}
\newcommand{\R}{\mathbb{R}}
\newcommand{\Z}{\mathbb{Z}}

\newcommand{\ds}{\displaystyle}

\newcommand{\fn}{\insertframenumber}

\newcommand{\id}{\operatorname{id}}
\newcommand{\im}{\operatorname{im}}
\newcommand{\Aut}{\operatorname{Aut}}
\newcommand{\Inn}{\operatorname{Inn}}

\newcommand{\blank}[1]{\underline{\hspace*{#1}}}

\newcommand{\abar}{\overline{a}}
\newcommand{\bbar}{\overline{b}}
\newcommand{\cbar}{\overline{c}}

\newcommand{\nml}{\unlhd}

%%%%%%%%%%
%Custom Theorem Environments
%%%%%%%%%%
\theoremstyle{definition}
\newtheorem{exercise}{Exercise}
\newtheorem{question}[exercise]{Question}
\newtheorem{warmup}{Warm-Up}
\newtheorem*{defn}{Definition}
\newtheorem*{exa}{Example}
\newtheorem*{disc}{Group Discussion}
\newtheorem*{nb}{Note}
\newtheorem*{recall}{Recall}
\renewcommand{\emph}[1]{{\color{blue}\texttt{#1}}}

\definecolor{Gold}{RGB}{237, 172, 26}
%Statement block
\newenvironment{statementblock}[1]{%
	\setbeamercolor{block body}{bg=Gold!20}
	\setbeamercolor{block title}{bg=Gold}
	\begin{block}{\textbf{#1.}}}{\end{block}}
\newenvironment{thm}[1]{%
	\setbeamercolor{block body}{bg=Gold!20}
	\setbeamercolor{block title}{bg=Gold}
	\begin{block}{\textbf{Theorem #1.}}}{\end{block}}


%%%%%%%%%%
%Custom Environment Wrappers
%%%%%%%%%%
\newcommand{\enumarabic}[1]{
	\begin{enumerate}[label=\textbf{\arabic*.}]
		#1
	\end{enumerate}
}
\newcommand{\enumalph}[1]{
	\begin{enumerate}[label=(\alph*)]
		#1
	\end{enumerate}
}
\newcommand{\bulletize}[1]{
	\begin{itemize}[label=$\bullet$]
		#1
	\end{itemize}
}
\newcommand{\circtize}[1]{
	\begin{itemize}[label=$\circ$]
		#1
	\end{itemize}
}
\newcommand{\slide}[1]{
	\begin{frame}{\fn}
		#1
	\end{frame}
}
\newcommand{\slidec}[1]{
\begin{frame}[c]{\fn}
	#1
\end{frame}
}
\newcommand{\slidet}[2]{
	\begin{frame}{\fn\ - #1}
		#2
	\end{frame}
}


\newcommand{\startdoc}{
		\title{Math 425: Abstract Algebra 1}
		\author{Mckenzie West}
		\date{Last Updated: \today}
		\begin{frame}
			\maketitle
		\end{frame}
}

\newcommand{\topics}[2]{
	\begin{frame}{\insertframenumber}
		\begin{block}{\textbf{Last Section.}}
			\begin{itemize}[label=--]
				#1
			\end{itemize}
		\end{block}
		\begin{block}{\textbf{This Section.}}
			\begin{itemize}[label=--]
				#2
			\end{itemize}
		\end{block}
	\end{frame}
}

\begin{document} 
\startdoc

\topics{
	% Last Time
	\item Counterexamples
	\item Contrapositive
	\item Converse
	\item Sets and Operations
	\item Cardinality
	\item Power Sets
	\item Cartesian Products
}
{
	% This time
	\item Mappings
	\item Codomain vs Range
	\item Image and Inverse Image
	\item One-to-One, Onto, Bijection
	\item Identity Map
	\item Inverse Mappings
	\item Invertibility Theorem
}
\slide{
	\begin{defn}
		A \emph{mapping} or \emph{function} $\alpha$ from $A$ to $B$ is a rule that assigns to every input $a\in A$ exactly one output $\alpha(a)\in B$.  
		
		\textbf{Notation:}
		\[\alpha\colon A\to B\textup{ or }A\xrightarrow{\alpha}B.\]
		
		Once we have verified that each input maps to exactly one output then we say the mapping is \emph{well-defined}.
	\end{defn}
	%	\begin{nb}
		%		We will get into the concept of well-defined later in the course when we start discussing maps on equivalence classes.
		%	\end{nb}
}
\slide{
	\begin{defn} Assume $\alpha:A \to B$ is a mapping.
		\begin{itemize}[label=--]
			\item We call $A$ the \emph{domain} of $\alpha$ and $B$ the \emph{codomain} of $\alpha$.
			\item If $C\subseteq A$, then the \emph{image} of $C$ is \[\alpha(C)=\{b\in B : b=\alpha(c)\text{ for some } c\in C\}.\] 
			\item The \emph{range} of $\alpha$ is the image of the domain, \[\im(\alpha)=\alpha(A)=\{\alpha(a)\in B : a\in A\}.\]
		\end{itemize}
	\end{defn}
}
\slide{
	\begin{exercise}
		Define $\alpha:\Z\to\Z$ by $\alpha(n)=3n+1$. \enumarabic{\item Compute the image of $C=\{2,4,6\}$. \vskip 1in \item What is the range of $\alpha$?}
	\end{exercise}
}
\slide{
	\begin{defn}
		We will call two maps $\alpha:A\to B$ and $\beta:A\to B$ equal if $\alpha(a)=\beta(a)$ for all $a\in A$.
	\end{defn}
	\begin{exa}
		Consider $\alpha:\Z\to \{0,1\}$ and $\beta:\Z\to \{0,1\}$ defined by 
		\[\alpha(n)=\begin{cases}
			0&\text{ if }n\text{ is even}\\
			1&\text{ if }n\text{ is odd}
		\end{cases}\text{ and }
		\beta(n)=\left\lceil \frac{n}{2}\right\rceil-\left\lfloor \frac{n}{2}\right\rfloor.\]
		These are equal but have very different feeling descriptions.
	\end{exa}
}
\slide{
	\begin{defn}
		Let $\alpha\colon A\to B$ be a mapping.
		\enumalph{
			\item We call $\alpha$ \emph{one-to-one} or \emph{injective} if for all $a_1,a_2\in A$ if $\alpha(a_1)=\alpha(a_2)$, then $a_1=a_2$.
			\item We call $\alpha$ \emph{onto} or \emph{surjective} if for all $b\in B$ there is an $a\in A$ such that $\alpha(a)=b$.
			\item We call $\alpha$ a \emph{bijection} or \emph{bijective} if $\alpha$ is both one-to-one and onto.
		}
		
	\end{defn}
	\begin{exercise}
		Define $\alpha:\Z\to\Z$ by $\alpha(n)=3n+1$. 
		\enumarabic{
			\item Is $\alpha$ one-to one?  
			\item Is $\alpha$ onto? 
			\item Is $\alpha$ a bijection?
		}
	\end{exercise}
}
\slide{
	\begin{block}{\textbf{Generic Proof of One-to-One.}}
		
	\end{block}
}
\slide{
	\begin{block}{\textbf{Generic Proof of Onto.}}
		
	\end{block}
}
\slide{
	\begin{exercise}
		Write down a mapping $\alpha\colon \Z\to \Z$ that is
		\enumalph{
				\item neither one-to-one nor onto,
				\item one-to-one and not onto,
				\item onto and not one-to-one,
				\item a bijection.
				}
	\end{exercise}
	}
\slidec{
	\begin{block}{\textbf{Brain Break.}}
		What is your favorite class you've taken and why?
	\end{block}
}
\slide{
		\begin{defn}
				The \emph{identity map} for the set $A$ is the map $1_A:A\to A$ defined by $1_A(a)=a$ for all $a\in A$.
				
				If $\alpha:A\to B$ and $\beta:B\to C$ are mappings, we can write
					\[A\xrightarrow{\alpha}B\xrightarrow{\beta}C,\]
				and the \emph{composition} of the maps is the mapping $\beta\alpha:A\to C$ defined by
					\[\beta\alpha(a)=\beta[\alpha(a)]\text{ for all }a\in A.\]
			\end{defn}
	}
\slide{
		\begin{statementblock}{Theorem 0.3.3}
				Let $A\xrightarrow{\alpha}B\xrightarrow{\beta}C\xrightarrow{\gamma}D$ be mappings on sets.  Then
					\enumarabic{
							\item (identity) $\alpha 1_A=\alpha$ and $1_B\alpha=\alpha$
							\item (associativity) $\gamma(\beta\alpha)=(\gamma\beta)\alpha$
							\item If $\alpha$ and $\beta$ are both one-to-one (resp.~onto), then $\beta\alpha$ is one-to-one (resp.~onto) too.
						}
			\end{statementblock}
	%	\begin{exa}
		%		We will prove part 3 next time.
		%	\end{exa}
	}
\slide{
		\begin{defn}
				If $\alpha:A\to B$ is a mapping of sets, then we call $\beta:B\to A$ an \emph{inverse} of $\alpha$ if
					\[\beta\alpha = 1_A\text{ and }\alpha\beta=1_B.\]
			\end{defn}
		\begin{exa}
				$\alpha:\R\to\R$, $\alpha(x)=3x+1$
				
				$\beta:\R\to \R$, $\beta(x)=\frac{1}{3}x-\frac{1}{3}$
			\end{exa}
	}
\slide{
		\begin{statementblock}{Theorem 0.3.4}
				If $\alpha:A\to B$ has an inverse, then the inverse mapping is unique.
			\end{statementblock}
		\begin{proof}
				Let $\alpha:A\to B$ be a mapping with an inverse.  Let $\beta$ and $\beta'$ be two inverses of $\alpha$.
				We compute
					$$\begin{array}{rcll}
							\beta &=& \beta 1_B&\text{Theorem 0.3.3(a)}\\
								 &=&\beta(\alpha\beta')&\beta'\text{ inverse of }\alpha\\
								 &=&(\beta\alpha)\beta'&\text{Theorem 0.3.3(b)}\\
								 &=&1_A\beta'&\beta\text{ inverse of }\alpha\\
								 &=&\beta'&\text{Theorem 0.3.3(a)}.
						\end{array}$$
				Therefore $\beta=\beta'$, and the inverse of $\alpha$ is unique.
			\end{proof}
		\begin{block}{\textbf{Result.}}
				The notation $\alpha^{-1}$ is valid.
			\end{block}
	}
\slide{
		\begin{statementblock}{Theorem 0.3.5}
				Let $\alpha:A\to B$ and $\beta: B \to C$ denote mappings.
				\enumarabic{
						\item The identity map, $1_A:A\to A$ is invertible and $1_A^{-1}=1_A$.
						\item If $\alpha$ is invertible, then $\alpha^{-1}$ is invertible and $(\alpha^{-1})^{-1}=\alpha$.
						\item If $\alpha$ and $\beta$ are both invertible, then $\beta\alpha$ is invertible with $(\beta\alpha)^{-1}=\alpha^{-1}\beta^{-1}$.
					}
			\end{statementblock}
	}
\slide{
		\begin{statementblock}{Invertibility Theorem (Theorem 0.3.6)}
				A mapping $\alpha:A\to B$ is invertible if and only if $\alpha$ is a bijection.
			\end{statementblock}	
	}
\end{document}

