\documentclass[t]{beamer}

\subtitle{Section 0.2: Sets}

\usepackage{amsthm,amsmath,amsfonts,hyperref,graphicx,color,multicol,soul}
\usepackage{enumitem,tikz,tikz-cd,setspace,mathtools}

%%%%%%%%%%
%Beamer Template Customization
%%%%%%%%%%
\setbeamertemplate{navigation symbols}{}
\setbeamertemplate{theorems}[ams style]
\setbeamertemplate{blocks}[rounded]

\definecolor{Blu}{RGB}{43,62,133} % UWEC Blue
\setbeamercolor{structure}{fg=Blu} % Titles

%Unnumbered footnotes:
\newcommand{\blfootnote}[1]{%
	\begingroup
	\renewcommand\thefootnote{}\footnote{#1}%
	\addtocounter{footnote}{-1}%
	\endgroup
}

%%%%%%%%%%
%TikZ Stuff
%%%%%%%%%%
\usetikzlibrary{arrows}
\usetikzlibrary{shapes.geometric}
\tikzset{
	smaller/.style={
		draw,
		regular polygon,
		regular polygon sides=3,
		fill=white,
		node distance=2cm,
		minimum height=1in,
		line width = 2pt
	}
}
\tikzset{
	smsquare/.style={
		draw,
		regular polygon,
		regular polygon sides=4,
		fill=white,
		node distance=2cm,
		minimum height=1in,
		line width = 2pt
	}
}


%%%%%%%%%%
%Custom Commands
%%%%%%%%%%

\newcommand{\C}{\mathbb{C}}
\newcommand{\quats}{\mathbb{H}}
\newcommand{\N}{\mathbb{N}}
\newcommand{\Q}{\mathbb{Q}}
\newcommand{\R}{\mathbb{R}}
\newcommand{\Z}{\mathbb{Z}}

\newcommand{\ds}{\displaystyle}

\newcommand{\fn}{\insertframenumber}

\newcommand{\id}{\operatorname{id}}
\newcommand{\im}{\operatorname{im}}
\newcommand{\Aut}{\operatorname{Aut}}
\newcommand{\Inn}{\operatorname{Inn}}

\newcommand{\blank}[1]{\underline{\hspace*{#1}}}

\newcommand{\abar}{\overline{a}}
\newcommand{\bbar}{\overline{b}}
\newcommand{\cbar}{\overline{c}}

\newcommand{\nml}{\unlhd}

%%%%%%%%%%
%Custom Theorem Environments
%%%%%%%%%%
\theoremstyle{definition}
\newtheorem{exercise}{Exercise}
\newtheorem{question}[exercise]{Question}
\newtheorem{warmup}{Warm-Up}
\newtheorem*{defn}{Definition}
\newtheorem*{exa}{Example}
\newtheorem*{disc}{Group Discussion}
\newtheorem*{nb}{Note}
\newtheorem*{recall}{Recall}
\renewcommand{\emph}[1]{{\color{blue}\texttt{#1}}}

\definecolor{Gold}{RGB}{237, 172, 26}
%Statement block
\newenvironment{statementblock}[1]{%
	\setbeamercolor{block body}{bg=Gold!20}
	\setbeamercolor{block title}{bg=Gold}
	\begin{block}{\textbf{#1.}}}{\end{block}}
\newenvironment{thm}[1]{%
	\setbeamercolor{block body}{bg=Gold!20}
	\setbeamercolor{block title}{bg=Gold}
	\begin{block}{\textbf{Theorem #1.}}}{\end{block}}


%%%%%%%%%%
%Custom Environment Wrappers
%%%%%%%%%%
\newcommand{\enumarabic}[1]{
	\begin{enumerate}[label=\textbf{\arabic*.}]
		#1
	\end{enumerate}
}
\newcommand{\enumalph}[1]{
	\begin{enumerate}[label=(\alph*)]
		#1
	\end{enumerate}
}
\newcommand{\bulletize}[1]{
	\begin{itemize}[label=$\bullet$]
		#1
	\end{itemize}
}
\newcommand{\circtize}[1]{
	\begin{itemize}[label=$\circ$]
		#1
	\end{itemize}
}
\newcommand{\slide}[1]{
	\begin{frame}{\fn}
		#1
	\end{frame}
}
\newcommand{\slidec}[1]{
\begin{frame}[c]{\fn}
	#1
\end{frame}
}
\newcommand{\slidet}[2]{
	\begin{frame}{\fn\ - #1}
		#2
	\end{frame}
}


\newcommand{\startdoc}{
		\title{Math 425: Abstract Algebra 1}
		\author{Mckenzie West}
		\date{Last Updated: \today}
		\begin{frame}
			\maketitle
		\end{frame}
}

\newcommand{\topics}[2]{
	\begin{frame}{\insertframenumber}
		\begin{block}{\textbf{Last Section.}}
			\begin{itemize}[label=--]
				#1
			\end{itemize}
		\end{block}
		\begin{block}{\textbf{This Section.}}
			\begin{itemize}[label=--]
				#2
			\end{itemize}
		\end{block}
	\end{frame}
}

\begin{document} 
\startdoc

\topics{
	% Last Time
	\item Course Introduction
	\item Direct Proofs
	\item Cases
}
{
	% This time
	\item Contradiction
	\item Counterexamples
	\item Contrapositive/Converse
	\item Sets and Operations
	\item Cardinality
	\item Power Sets
	\item Cartesian Products
}
\slide{
\begin{block}{Method of Proof by Contradiction.}
	To prove ``if $P$ then $Q$'' using contradiction, assume $P$ is true and $Q$ is false then show that something is terribly wrong.
	\begin{exa}
		Prove that there are no integers $x$ and $y$ satisfying $24x+12y=1$.
		%\enumarabic{\item If $x$ and $y$ are integers then $24x+12y=1$.}
	\end{exa}
\end{block}
}
\slide{
\begin{block}{Proof.}
	Let $x$ and $y$ be integers.  Assume, toward contradiction, that $24x+12y=1$.
	%\pause
	%Divide both sides of the equation $24x+12y=1$ by $12$ to get
	%	\[2x+y=\frac{1}{12}.\]\pause
	%Notice that since $x$ and $y$ are integers, so is $2x+y$.  However $\frac{1}{12}$ is not an integer, contradicting that fact that $2x+y=\frac{1}{12}$.
	
	%Therefore, the assumption that $24x+12y=1$ cannot be true.
\end{block}	
}
\slide{
	\begin{exercise}
		Using the last proof as inspiration, what do you think must be true about $m$ and $n$ if 	\[mx+ny=1\]
		is satisfied for integers $x$ and $y$?
		
		Thank about what property(ies) $m=24$ and $n=12$ had that caused a problem.  Why does $m=4$ and $n=9$ have an integer solution with $x=-2$ and $y=1$?
	\end{exercise}
}
\slide{
	\begin{block}{Counterexamples.}
		Sometimes we can prove a universal statement is false by finding a simple counterexample.
		\vskip 1cm
		\begin{exa}
			Prove the result or give a counterexample.
			\begin{quote}
				If $n^2=1$ then $n=1$.
			\end{quote}
			\pause
			\begin{block}{Solution.}
				This statement is false because if $n=-1$ then $n^2=1$ while $n\neq 1$.
			\end{block}
		\end{exa}
	\end{block}
}
\slide{
	\begin{defn}
		The \emph{contrapositive} of the statement $P\Rightarrow Q$ is $\sim Q\Rightarrow \sim P$.
		
		The \emph{converse} of the statement $P\Rightarrow Q$ is $Q\Rightarrow P$.
	\end{defn}
	\vskip 1cm
	\begin{nb}
		The contrapositive of a statement is logically equivalent to it - and is sometimes easier to prove!
		
		The converse of a statement is NOT logically equivalent to it.  
	\end{nb}
}
\slide{
	\begin{exercise}
		\enumarabic{
			\item If $n$ is an even integer then $n^2$ is a multiple of 4.
			\enumalph{
				\item State the converse and the contrapositive. 
				\item Which of the converse and contrapositive are true?
				\item If either is false, give a counterexample.
			}
			\item If $m+n=25$, where $n$ and $m$ are integers, then\\exactly one of $n$ and $m$ is greater than $12$.
			\enumalph{
				\item State the converse and the contrapositive. 
				\item Which of the converse and contrapositive are true?
				\item If either is false, give a counterexample.
			}
			\item For all integers $n$, if $n\geq 2$, then $n^3\geq 2^n$.
			\enumalph{
				\item State the converse and the contrapositive. 
				\item Which of the converse and contrapositive are true?
				\item If either is false, give a counterexample.
			}
		}
	\end{exercise}
}
\slide{
	\begin{defn}
		A \emph{set} is a collection of objects called \emph{elements}.
	\end{defn}
	\begin{block}{\textbf{Notation.}}
		If $S$ is a set with an element $a$, then we write $a{\color{blue}\in} S$.
	\end{block}
	\begin{exa}
		\bulletize{
			\item $\Z$
			\item $\R$
			\item $\C$
			\item $\Q$
			\item $\N$
			\item $\Z^+$
		}
	\end{exa}
}
\slide{
	\begin{exa}
		Sets can be represented in many ways:
		\bulletize{
			\item Listing all elements: $\{1,4,7,322\}$
			\item Giving a formula: $\{3a-2\in\Z\ |\ a\in\Z\}$
			\item Giving a criteria: $\{a\in\R\ :\ 2^a>5\}$
		}
	\end{exa}
	\begin{exercise}
		Determine the elements of the following sets:
		\enumalph{\setlength{\itemsep}{1em}
			\item $\{n\in \Z\ :\ |n-2|\leq 1\}$
			\item $\{10k\ :\ k\in \Z\}$
			\item $\{m+1\ |\ m\in \N \text{ and }m\text{ is even}\}$
		}
	\end{exercise}
}
\slide{
	\begin{defn}
		If $A$ has $n$-elements then we say the \emph{cardinality} of $A$ is $n$ and we write $|A|=n$. Such sets are called \emph{finite} sets.  
		Sets with an infinite number of elements are \emph{infinite} sets.
	\end{defn}
	\begin{exa}
		$|\emptyset|=0$
	\end{exa}
	\begin{exercise}
		Compute $|A|$ and $|B|$ given
		\[A=\{1,\{2\}\}\text{ and }B=\{\emptyset,\{1\},\{2\},\{1,2\}\}.\]
	\end{exercise}
}
\slide{
	\begin{defn}
		For sets $A$ and $B$ we say that \emph{$A$ is a subset of $B$} if every element of $A$ is also an element of $B$. In mathematical notation we would write
		\[\forall a\in A,\ a\in B.\]
	\end{defn}
	\begin{block}{\textbf{Notation.}}
		If $A$ is a subset of $B$ we write $A\ {\color{blue}\subseteq}\ B$.
	\end{block}
}
\slide{
	\begin{exercise}
		Let $A=\{x\in\R\ |\ x^2-1<1\}$ and $B=\{x\in\R\ |\ x<3\}$. Show that $A\subseteq B$.
	\end{exercise}
}
\slide{
	\begin{defn}
		The \emph{power set} of a set $A$ is the set $P(A)$ consisting of all subsets of $A$.
	\end{defn}
	\begin{exa}
		$B=\{\emptyset,\{1\},\{2\},\{1,2\}\}=P(\{1,2\})$
	\end{exa}
	\begin{exercise}
		If $A=\{4,5,6\}$, what is $P(A)$ and $|P(A)|$?
	\end{exercise}
}
\slidec{
	\begin{statementblock}{Principle of Set Equality}
		If $A$ and $B$ are sets then
		\begin{center}
			$A=B$\quad if and only if \quad $A\subseteq B$ and $B\subseteq A$
		\end{center}
	\end{statementblock}
}
\slide{
	\begin{defn}
		The \emph{intersection} of the sets $A$ and $B$, denotes $A\cap B$ is the set of all elements that appear in both $A$ and in $B$.
	\end{defn}
	\begin{defn}
		The \emph{union} of the sets $A$ and $B$, denotes $A\cup B$ is the set of all elements that appear in at least one of $A$ or $B$.
	\end{defn}
	\begin{exercise}
		Use the Principle of Set Equality to show that for sets $A$, $B$, and $C$, 
			\[A\cap (B\cup C) = (A\cap B)\cup (A\cap C).\]
	\end{exercise}
}
\slide{
	\begin{defn}
		The \emph{Cartesian Product} of the sets $A$ and $B$ is the set
		\[A\times B:=\{(a,b):a\in A\text{ and }b\in B\}.\]
		Note that the elements, $(a,b)$, are ordered pairs.
	\end{defn}
	\begin{exercise}
		Compute $A\times B$ if $A=\{3,4\}$ and $B=\{a,b,c\}$.
	\end{exercise}
}
\slide{
	\begin{exercise}
		\enumalph{
			\item Compute $A\times A$ if $A=\{2,4,6\}$.\vskip 1in
			
			\item Consider $A=[-2,6)$ and $B=(5,9)$ as intervals in $\R$.  What is $A\times B$?}
	\end{exercise}
}
\end{document}

