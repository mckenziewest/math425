\documentclass[12pt]{article}
\usepackage{amsmath,amsthm,amssymb}
\usepackage[margin=.75in]{geometry}
\pagestyle{empty}


\newcommand{\C}{\mathbb{C}}
\newcommand{\Q}{\mathbb{Q}}
\newcommand{\R}{\mathbb{R}}
\newcommand{\Z}{\mathbb{Z}}

\setlength{\parindent}{0pt}
\begin{document}
	\begin{center}
		{\Large \bf Math 425 Objective Intro-4 Instructions}
	\end{center}
	\section*{Purpose}
	\begin{itemize}
		\item This document is intended to provide information about completing the objective
		
			\textbf{Intro-4:} Execute the division algorithm and Aryabhata’s (Bézout’s) algorithm.

	\end{itemize}
	\section*{Task}
	\begin{itemize}
		\item If you have not yet earned a \textbf{Satisfactory} mark on an exercise labeled with the objective here, you may submit another attempt to WeBWorK at any time.
		\item Allowed resources: our textbook, classmates, your notes, videos linked in Canvas.
		\item Unacceptable resources: anything you find on an internet search. Do NOT use a homework help website (e.g., Chegg). Their solutions are often wrong or use incorrect context.  I want you to practice making arguments that are yours. Take some ownership.
	\end{itemize}
	\section*{Criteria}
	All items will earn a score using the following scale:
	\begin{itemize}
		\item \textbf{Satisfactory} - WeBWorK assignment was completed for full credit in at most 5 attempts.
		\item \textbf{Unsatisfactory} - WeBWorK assignment either either not completed fully or more than 5 attempts were used.
	\end{itemize}
	Recall from the syllabus
	\begin{itemize}
		\item If you earn a \textbf{Satisfactory} mark on an objective exercise assigned via WeBWorK  then you may consider that item complete. 
		\item If you earn an \textbf{Unsatisfactory} mark on an objective exercise then you have not yet completed this objective.
		\item You may submit a new attempt at completing that objective on a future Wednesday. You must select a new WeBWorK assignment labeled with the given objective, you cannot resubmit a version you have attempted previously.  The only limit you have on number of attempts is the number of exercises available for the objective.
	\end{itemize}
	
\end{document}