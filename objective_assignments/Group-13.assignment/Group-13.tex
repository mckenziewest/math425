\documentclass[12pt]{article}
\usepackage{amsmath,amsthm,amssymb,enumitem}
\usepackage[margin=.75in]{geometry}
\pagestyle{empty}

\newcommand{\C}{\mathbb{C}}
\newcommand{\R}{\mathbb{R}}
\newcommand{\Z}{\mathbb{Z}}
\newcommand{\lcm}{\operatorname{lcm}}

\setlength{\parindent}{0pt}
\begin{document}
	\begin{center}
		{\Large \bf Math 425 Objective Group-13 Exercises}
	\end{center}
	\section*{Purpose}
	\begin{itemize}
		\item This document is intended to provide additional opportunities to complete the objective
		
		\textbf{Group-13:}  Use the first isomorphism theorem to prove that a quotient of a group is isomorphic to another group.
	\end{itemize}
	\section*{Task}
	\begin{itemize}
		\item If you have not yet earned a \textbf{Satisfactory} or \textbf{Exceptional} mark on an exercise labeled with the objective here, you may submit a single one of the following exercises, that you have not yet attempted via Canvas by 4pm on any following Wednesday.
		\item I strongly recommend you use LaTeX to typeset your proofs.
		\item You may work in groups but everyone should submit their own assignment written in their own words.  Do NOT copy your classmates.
		\item Allowed resources: our textbook, classmates, your notes, videos linked in Canvas.
		\item Unacceptable resources: anything you find on an internet search. Do NOT use a homework help website (e.g., Chegg). Their solutions are often wrong or use incorrect context.  I want you to practice making arguments that are yours. Take some ownership.
	\end{itemize}
	\section*{Criteria}
	All items will earn a score using the following scale:
	\begin{itemize}
		\item \textbf{Exceptional} - Solution is succinct, references the correct theorems and definitions, and is entirely correct.
		\item \textbf{Satisfactory} - Solution is nearly correct. It still references the correct theorems and definitions. 
		It may be longer than necessary, have minor errors, or have some grammatical mistakes.
		\item \textbf{Unsatisfactory} - Solution has major errors, references content not covered in class or in the textbook, or is incomplete in some major way.
	\end{itemize}
	Recall from the syllabus
	\begin{itemize}
		\item If you earn either an \textbf{Exceptional} or \textbf{Satisfactory} mark on an objective exercise (labeled Intro-, Group-, or Ring-) then you may consider that item complete. 
		\item If you earn an \textbf{Unsatisfactory} mark on an objective exercise (labeled Intro-, Group-, or Ring-) then you have not yet completed this objective.
		\item You may submit a new attempt at completing that objective on a future Wednesday. You must select a new exercise listed under the given objective, you cannot resubmit a version you have attempted previously.  The only limit you have on number of attempts is the number of exercises available for the objective.
	\end{itemize}
	
	
	\newpage
	Name: \underline{\hspace*{3in}}
	\vskip .25in
	
	\textbf{(Group-13.2)} Let $G=\left.\left\{\begin{bmatrix}a&b\\0&c\end{bmatrix}\ \right|\ a,b,c\in \R \text{ and }ac\neq 0\right\}$. Let $K=\left.\left\{\begin{bmatrix}1&b\\0&1\end{bmatrix}\ \right|\ b\in \R\right\}$. Use the first isomorphism theorem to show that $K\lhd G$ and $G\cong \R^*\times \R^*$.
	
	
	\newpage
	Name: \underline{\hspace*{3in}}
	\vskip .25in
	
	\textbf{(Group-13.3)} If $a,b\in \R$ define $\tau_{a,b}:\R\to \R$ by $\tau_{a,b}(x)=ax+b$ for all $x\in \R$.  Let $G=\{\tau_{a,b}\ |\ a,b\in\R; a\neq 0\}$.  This is a group under the operation of function composition.
	
	Show that $K=\{\tau_{1,b}\ |\ b\in \R\}$ is a normal subgroup of $G$ and $G/K\cong\R^*$.
	
	
	\newpage
	Name: \underline{\hspace*{3in}}
	\vskip .25in
	
	\textbf{(Group-13.4)} Let $s=\lcm(m,n)$.  Show that $\Z/s\Z$ is isomorphic to a subgroup of $\Z_m\times \Z_n$.
	
	
	
	\newpage
	Name: \underline{\hspace*{3in}}
	\vskip .25in
	
	\textbf{(Group-13.5)} Consider $M_2(\Z)$ as a group under addition. For $n\geq 2$, show that $M_2(n\Z)\lhd M_2(\Z)$ and $M_2(\Z)/M_2(n\Z)\cong M_2(\Z_n)$.
		
\end{document}