\documentclass[12pt]{article}
\usepackage{amsmath,amsthm,amssymb}
\usepackage[margin=.75in]{geometry}
\pagestyle{empty}


% Custom Commands

\newcommand{\objective}{Intro-2}
\newcommand{\C}{\mathbb{C}}
\newcommand{\R}{\mathbb{R}}
\newcommand{\Z}{\mathbb{Z}}




\setlength{\parindent}{0pt}
\begin{document}
	\begin{center}
		{\Large \bf Math 425 Objective OBJECTIVE Exercises}
	\end{center}
	\section*{Purpose}
	\begin{itemize}
		\item This document is intended to provide additional opportunities to complete the objective
		
		\textbf{OBJECTIVE:} DESCRIPTION
	\end{itemize}
	\section*{Task}
	\begin{itemize}
		\item If you have not yet earned a \textbf{Satisfactory} or \textbf{Exceptional} mark on an exercise labeled with the objective here, you may submit a single one of the following exercises, that you have not yet attempted via Canvas by 4pm on any following Wednesday.
		\item I strongly recommend you use LaTeX to typeset your proofs.
		\item You may work in groups but everyone should submit their own assignment written in their own words.  Do NOT copy your classmates.
		\item Allowed resources: our textbook, classmates, your notes, videos linked in Canvas.
		\item Unacceptable resources: anything you find on an internet search. Do NOT use a homework help website (e.g., Chegg). Their solutions are often wrong or use incorrect context.  I want you to practice making arguments that are yours. Take some ownership.
	\end{itemize}
	\section*{Criteria}
	All items will earn a score using the following scale:
	\begin{itemize}
		\item \textbf{Exceptional} - Solution is succinct, references the correct theorems and definitions, and is entirely correct.
		\item \textbf{Satisfactory} - Solution is essentially correct. It still references the correct theorems and definitions. 
		It may be longer than necessary, have inconsequential errors, or have some grammatical mistakes.
		\item \textbf{Nearly Complete} - Solution has some errors and should be re-written.
		\item \textbf{Unsatisfactory} - Solution has major errors, references content not covered in class or in the textbook, or is incomplete in some major way.
	\end{itemize}
	Recall from the syllabus
	\begin{itemize}
		\item If you earn either an \textbf{Exceptional} or \textbf{Satisfactory} mark on an objective exercise (labeled Intro-, Group-, or Ring-) then you may consider that item complete. 
		\item If you earn a \textbf{Nearly Complete} or \textbf{Unsatisfactory} mark on a an objective exercise (labeled Intro-, Group-, or Ring-) then you have not yet completed this objective.
		\item You may submit a new attempt at completing that objective on a future Wednesday. You must select a new exercise listed under the given objective, you cannot resubmit a version you have attempted previously.  The only limit you have on number of attempts is the number of exercises available for the objective.
		\item If you earn an \textbf{Exceptional} mark on an additional exercise (labeled A-) then you will earn one half point toward the ten total points in that section of your overall grade.
		\item If you earn an \textbf{Satisfactory} mark on an additional exercise (labeled Supp-) then you will earn 0.25 points toward the ten total points in that section of your overall grade. 
		\item If you earn a \textbf{Nearly Complete} or \textbf{Unsatisfactory} mark on an additional exercise (labeled Supp-) then you will earn 0 points toward the ten total points in that section of your overall grade. 
	\end{itemize}
	
	
	
	
	
	
	%%%%
	
	\newpage
	Name: \underline{\hspace*{3in}}
	\vskip .25in
	
	\textbf{Intro-2.2} 
	
	Let $A$ and $B$ be nonempty sets with a mapping $\alpha\colon A\to B$. Assume that there is some map $\beta:B\to A$ such that $\alpha\beta=1_B$. Prove that $\alpha$ is onto.
	
	
	
	%%%%
	
	\newpage
	Name: \underline{\hspace*{3in}}
	\vskip .25in
	
	\textbf{Intro-2.3} 
	
	Let $A$ and $B$ be nonempty sets with a mapping $\alpha\colon A\to B$. Assume that there is some map $\beta:B\to A$ such that $\beta\alpha=1_A$. Prove that $\alpha$ is one-to-one.
	
	
	
	
	%%%%
	
	\newpage
	Name: \underline{\hspace*{3in}}
	\vskip .25in
	
	\textbf{Intro-2.4} 
	
	Let $A=\left\{\begin{bmatrix}a&b\\c&d\end{bmatrix}\ |\ a,b,c,d\in\R\right\}$. Define $\phi\colon \C\to A$ by $$\phi(a+bi)=\begin{bmatrix}a+bi&0\\0&a-bi\end{bmatrix}.$$ Prove that $\phi$ is one-to-one but not onto.
	
	
	
	
	%%%%
	
	\newpage
	Name: \underline{\hspace*{3in}}
	\vskip .25in
	
	\textbf{Intro-2.5} 
	
	Let $A$ be a non-empty set. Define $\sigma:A\to A\times A$ by $\sigma(a)=(a,a)$. Is $\sigma$ one-to-one? onto? a bijection? You must prove each of your answers.
	
	
	
	
	%%%%
	
	\newpage
	Name: \underline{\hspace*{3in}}
	\vskip .25in
	
	\textbf{Intro-2.6} 
	
	Write down a solution with proof for all three parts. \\
	Define a map $\tau\colon 2\Z\to 5\Z$ such that.
	\begin{enumerate}
		\item $\tau$ is one-to-one but not onto.\vfill
		\item $\tau$ is onto but not one-to-one.\vfill
		\item $\tau$ is a bijection.\vfill
	\end{enumerate}
	
	
	
	
\end{document}