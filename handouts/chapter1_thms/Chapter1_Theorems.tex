\documentclass[12pt]{article}
\usepackage{amsmath,amssymb,graphicx,multicol,enumitem,amsthm}
\usepackage[margin=1in]{geometry}
\pagestyle{empty}
\newtheorem{innercustomthm}{Theorem}
\newtheorem*{cor}{Corollary}
\newenvironment{thm}[1]
{\renewcommand\theinnercustomthm{#1}\innercustomthm}
{\endinnercustomthm}
\newcommand{\enumarabic}[1]{
	\begin{enumerate}[label=\textbf{\arabic*.}]
		#1
	\end{enumerate}
}
\newcommand{\enumalph}[1]{
\begin{enumerate}[label=(\alph*)]
	#1
\end{enumerate}
}
\theoremstyle{definition}
\newtheorem*{defn}{Definition}




%%%%%%%%%%
%Custom Commands
%%%%%%%%%%
\newcommand{\C}{\mathbb{C}}
\newcommand{\N}{\mathbb{N}}
\newcommand{\Q}{\mathbb{Q}}
\newcommand{\R}{\mathbb{R}}
\newcommand{\Z}{\mathbb{Z}}

\newcommand{\ds}{\displaystyle}

\newcommand{\fn}{\insertframenumber}

\newcommand{\im}{\operatorname{im}}

\newcommand{\blank}[1]{\underline{\hspace*{#1}}}

\newcommand{\abar}{\overline{a}}
\newcommand{\bbar}{\overline{b}}
\newcommand{\cbar}{\overline{c}}

\begin{document}
			\noindent Math 425: Abstract Algebra I
		
		\noindent	Theorems from the Textbook - Chapter 1
\vskip .25in


\section*{Theorems}
\begin{thm}{}[Principle of Mathematical Induction]
	Let $P(n)$ be a statement for each integer $n\geq  m$. Suppose the following conditions are satisfied,
	\enumarabic{\item $P(m)$ is true, and \item $P(k)\Rightarrow P(k+1)$ for every $k\geq m$.}
	Then $P(n)$ is true for every $n\geq m$.
\end{thm}
\begin{thm}{}[Another Induction Principle]
Let $P(n)$ be a statement for each integer $n\geq  m$. Suppose the following conditions are satisfied,
\enumarabic{\item $P(m)$ amd $P(m+1)$ are true, and \item If $k\geq m$ and both $P(k)$ and $P(k+1)$ are true then $P(k+2)$ is true.}
Then $P(n)$ is true for every $n\geq m$.
\end{thm}
\begin{thm}{1.2.1}[The Division Algorithm]
Let $n\in\Z$ and $d\geq 1$ be an integer. Then there exists uniquely determined $q,r\in\Z$ such that 
\[n=qd+r\text{ and }0\leq r<d.\]
\end{thm}
\begin{thm}{1.2.2}
Let $m,n$ and $d$ denote integers.
	\enumarabic{\item $n\mid n$ for all $n$.\item If $d\mid m$ and $m\mid n$, then $d\mid n$.\item If $d\mid n$ and $n\mid d$, then $d=\pm n$.\item If $d\mid n$ and $d\mid m$, then $d\mid(xn+ym)$ for all $x,y\in\Z$.}
\end{thm}
\begin{thm}{1.2.3}[B\'ezout's Identity]
Let $a$ and $b$ be integers, not both zero.  Then there exist $r,s\in\Z$ such that $\gcd(a,b)=ra+sb$.
\end{thm}
\begin{thm}{1.2.4}
Let $m,n\in \Z$ not both zero.  Then
\begin{center}
	$m,n$ relatively prime $\Leftrightarrow$ $\exists r,s\in\Z$ such that $1=rm+sn$
\end{center}
\end{thm}
\begin{thm}{1.2.5}
Let $m,n\in \Z$ be relatively prime integers.
\enumarabic{\item If $m\mid k$ and $n\mid k$ for some integer $k$, then $mn\mid k$.
\item If $m\mid kn$ for some integer $k$, then $m\mid k$.}
\end{thm}
\begin{thm}{1.2.6}[Euclid's Lemma]
Let $p$ be a prime number.
\enumarabic{
	\item If $p\mid mn$ where $m,n\in\Z$, then $p\mid m$ or $p\mid n$.
	\item If $p\mid m_1m_2\cdots m_r$ where $m_i\in\Z$ for all $i$, then $p\mid m_i\ \exists i$.
}
\end{thm}
\begin{thm}{1.2.7}[Prime Factorization Theorem]
\enumarabic{
	\item Every integer $n\geq 2$ is a product of (one or more) primes.
	\item This factorization is unique (up to order of the factors).
	
	That is, if
	\[n=p_1p_2\cdots p_r\text{ and }n=q_1q_2\cdots q_2,\]
	then $r=s$ and the $q_j$ can be relabeled so that $p_i=q_i$ for $i=1,2,\dots,r$.
}
\end{thm}
\begin{cor}
Two integers are relatively prime if there exists no prime that divides them both.
\end{cor}
\begin{cor}
Every $n\in\Z_{\geq 2}$ can be written uniquely as
\[n=p_1^{n_1}p_2^{n_2}\cdots p_r^{n_r}\]
where the $p_i$ are distinct primes and $n_i\geq 1$ for all $i$.
\end{cor}

\begin{thm}{1.2.8}
	Let $n\geq 2$ be an integer with prime factorization 
	\[n=p_1^{n_1}p_2^{n_2}\cdots p_r^{n_r},\]
	where the $p_i$ are all distinct primes and $n_i\geq 1$ for all $i$.  Then
	\[d\mid n\Rightarrow d=p_1^{d_1}p_2^{d_2}\cdots p_r^{d_r}\text{ where } 0\leq d_i\leq n_i\ \forall i.\]
\end{thm}
\begin{thm}{1.2.9}
Let $\{a,b,c,\dots\}$ be a finite set of positive integers and write
\begin{eqnarray*}
	a&=&p_1^{a_1}p_2^{a_2}\cdots p_r^{a_r}\\
	b&=&p_1^{b_1}p_2^{b_2}\cdots p_r^{b_r}\\
	c&=&p_1^{c_1}p_2^{c_2}\cdots p_r^{c_r}
\end{eqnarray*}
where there is an exponent of zero if the prime is not a factor.

Then 
\[\gcd(a,b,c,\dots)=p_1^{k_1}p_2^{k_2}\cdots p_r^{k_r},\]
where $k_i=\min(a_i,b_i,c_i,\dots)$ for each $i$, and
\[\operatorname{lcm}(a,b,c,\dots)=p_1^{m_1}p_2^{m_2}\cdots p_r^{m_r},\]
where $m_i=\max(a_i,b_i,c_i,\dots)$ for each $i$.
\end{thm}
\begin{thm}{1.2.10}[Euclid's Theorem]
	There are infinitely many primes.
\end{thm}
\begin{thm}{1.3.1}
Congruence modulo $n$ is an equivalence relation on $\Z$.
\end{thm}
\begin{thm}{1.3.2}
Given $n\geq 2$, $\overline{a}=\overline{b}\Leftrightarrow a\equiv b\pmod n$.
\end{thm}
\begin{thm}{1.3.3}
	Let $n\geq 2$ be an integer.
	\enumarabic{\item If $a\in\Z$, then $\overline{a}=\overline{r}$ for some $r$ where $0\leq r\leq n-1$.\item The residue classes $\overline{0},\overline{1},\dots,\overline{n-1}$ modulo $n$ are distinct.}
\end{thm}
\begin{thm}{1.3.4}
Let $n\geq 2$ be a fixed modulus and let $a,b$ and $c$ denote arbitrary integers. Then the following hold in $\Z_n$.
\enumarabic{\item $\abar+\bbar=\bbar+\abar$ and $\abar\bbar=\bbar\abar$.
	\item $\abar+(\bbar+\cbar)=(\abar+\bbar)+\cbar$ and $\abar(\bbar\cbar)=(\abar\bbar)\cbar$.
	\item $\abar+\overline{0}=\abar$ and $\abar\overline{1}=\abar$.
	\item $\abar+\overline{-a}=\overline{0}$.
	\item $\abar(\bbar+\cbar)=\abar\bbar+\abar\cbar$.
}
\end{thm}
\begin{thm}{1.3.5}
Let $a,n\in\Z$ with $n\geq 2$. Then $\abar$ has a multiplicative inverse in $\Z_n$ if and only if $a$ and $n$ are relatively prime.
\end{thm}
\begin{thm}{1.3.6}[The Chinese Remainder Theorem]
Let $m$ and $n$ be relatively prime integers. If $s$ and $t$ are arbitrary integers, then there is an integer $b$ for which
\[b\equiv s\pmod m\text{ and } b\equiv t\pmod n.\]
\end{thm}
\begin{thm}{1.3.7}
The following are equivalent for any integer $n\geq 2$.
\enumarabic{\item Every element $\abar\neq\overline{0}$ in $\Z_n$ has a multiplicative inverse.
	\item If $\abar\bbar=\overline{0}$ in $\Z_n$, then either $\abar=\overline{0}$ or $\bbar=\overline{0}$.
	\item The integer $n$ is prime.
}
\end{thm}
\begin{thm}{}[Wilson's Theorem]
If $p$ is prime then $(p-1)!\equiv -1\pmod p$.
\end{thm}
\begin{thm}{1.3.8}[Fermat's Theorem]
If $p$ is prime then $a^p\equiv a\pmod p$ for all $a\in\Z$.  Moreover, if $\gcd(a,p)=1$, then $a^{p-1}\equiv 1\pmod p$.
\end{thm}
\begin{thm}{1.4.1}
	The set $S_n$ of permutations on $T_n=\{1,2,\dots,n\}$ has $|S_n|=n!$ elements.
\end{thm}
\begin{thm}{1.4.2}
	Let $\sigma,\tau$ and $\mu$ denote permutations in $S_n$.
		\enumarabic{
			\item the composition $\sigma\tau$ is in $S_n$
			\item $\sigma\varepsilon=\sigma=\varepsilon\sigma$
			\item $\sigma(\tau\mu)=(\sigma\tau)\mu$
			\item $\sigma\sigma^{-1}=\varepsilon=\sigma^{-1}\sigma$
		}
\end{thm}
\begin{thm}{1.4.3}[Disjoint cycles commute]  That is if $\sigma$ and $\tau$ are disjoint cycles then $\sigma\tau=\tau\sigma$.
\end{thm}
\begin{thm}{1.4.4}
	If $\sigma$ is an $r$-cycle, then $\sigma^{-1}$ is also an $r$-cycle. More precisely, if 
		$$\sigma=(k_1\ k_2\ \cdots\ k_{r-1}\ k_r),$$
	then 
	$$\sigma^{-1}=(k_r\ k_{r-1}\ \cdots\ k_{2}\ k_1),$$
\end{thm}
\begin{thm}{1.4.5}[Cycle Decomposition Theorem]
Every $\sigma \in S_n$ with $\sigma\neq\varepsilon$ can be written as a product of disjoint cycles.
\end{thm}
\begin{thm}{1.4.6}
If $n\geq 2$, then every cycle in $S_n$ can be written as a product of transpositions.
\end{thm}
\begin{thm}{1.4.7}[The Parity Theorem]
If a permutation has two factorizations	
\[\sigma = \gamma_n\cdots \gamma_2\gamma_1=\mu_m\cdots\mu_s\mu_1,\]
where each of $\gamma_i$ and $\mu_j$ are transpositions, then $m\equiv n\pmod 2$ ($m$ and $n$ have the same parity).
\end{thm}
\begin{thm}{1.4.8}
	If $n\geq 2$, the set $A_n$ has the following properties:
	\enumarabic{
		\item $\varepsilon$ is in $A_n$ and if $\sigma,\tau\in A_n$, then both $\sigma^-1\in A_n$ and $\sigma\tau\in A_n$.
		\item $|A_n|=\frac{1}{2}n!$.
	}
\end{thm}
\section*{Definitions}
\begin{defn}
	For $a,b,d\in\Z$:
	\begin{itemize}[label=$\bullet$]
		\item 
		We write $a\mid b$ to mean \emph{$a$ divides $b$}, which is defined formally as
		\[a\mid b\Leftrightarrow b=ak\text{ for some }k\in\Z.\]
		\item 
		We say $d$ is \emph{a common divisor of $a$ and $b$} if $d\mid a$ and $d\mid b$.
		\item The \emph{greatest common divisor of $a$ and $b$} is the largest integer that is a common divisor of $a$ and $b$.  Denote this value by $\gcd(a,b)$.
	\end{itemize}
\end{defn}
\begin{defn}
Let $a,b,n\in\Z$ with $n\geq 2$. We say that $a$ and $b$ are \emph{congruent modulo $n$} if 	\[n\mid (a-b).\]
In that case, we write $a\equiv b\pmod n$.
\end{defn}
\begin{defn}
If $a\in\Z$, then its equivalence class, $[a]$, with respect to congruence modulo $n$ is called its \emph{residue class modulo $n$} and we write $\overline{a}$ for convenience.

\[\overline{a}=\{x\in\Z|x\equiv a\pmod n\}.\]
\end{defn}
\begin{defn}
The \emph{set of integers modulo $n$} is denoted $\Z_n$ and is given by
\[\Z_n=\{\overline{0},\overline{1},\overline{2},\dots,\overline{n-1}\}.\]
\end{defn}
\begin{defn}
We call an element $\overline{a}\in\Z_n$ \emph{invertible} if there is some $\overline{b}\in\Z_n$ for which $\overline{ab}=\overline{1}$. We call such a $\overline{b}$ an \emph{inverse} of $\overline{a}$.

We call the set of all units, $\Z_n^\times$ the \emph{group of units} in $\Z_n$.
\end{defn}
\begin{defn}
A \emph{permutation} of $T_n=\{1,2,\dots,n\}$ is a mapping $\sigma:T_n\to T_n$ that is both one-to-one and onto (a bijection).
\vskip 1em
We call the collection of all permutations of $T_n$ the \emph{symmetric group of order $n$}, and we write
\[S_n:=\{\sigma:T_n\to T_n\ |\ \sigma\text{ is a permutation}\}.\]
\end{defn}
\begin{defn}
A \emph{permutation} matrix $A$ is an $n\times n$ matrix that has exactly one 1 in each row and column and every other entry is 0.
\end{defn}
\begin{defn}
The $r$-cycle $(x_1\ x_2\ \dots\ x_r)$ in $S_n$ is the permutation that sends
$$\begin{matrix}
x_1&\mapsto&x_2\\
x_2&\mapsto&x_3\\
x_3&\mapsto&x_4\\
&\vdots\\
x_{r-1}&\mapsto&x_r\\
x_r&\mapsto&x_1.
\end{matrix}$$
\end{defn}
\begin{defn}
Two cycles $(x_1\ x_2\ \dots\ x_r)$ and $(y_1\ y_2\ \dots\ y_s)$ are \emph{disjoint} if 	
\[\{x_1,x_2,\dots,x_r\}\cap\{y_1,y_2,\dots,y_s\}=\emptyset.\]
\end{defn}
\begin{defn}
A \emph{transposition} is a cycle of length 2.
\end{defn}
\begin{defn}
A permutation $\sigma\in S_n$ is called \emph{even} if it can be written as a product of an even number of transpositions.

Similarly, permutations can be called \emph{odd}.
\end{defn}
\begin{defn}
The \emph{alternation group of degree $n$} is the set of even permutations in $S_n$.  We call it $A_n$.
\end{defn}
\begin{defn}
The \emph{order} of a permutation, $\sigma\in S_n$ is the smallest positive integer $k$ such that $\sigma^k=\varepsilon$.
\end{defn}

\end{document}