\documentclass[12pt]{article}
\usepackage{amsmath,amssymb,graphicx,multicol,enumitem,amsthm}
\usepackage[margin=1in]{geometry}
\pagestyle{empty}
\newtheorem{innercustomthm}{Theorem}
\newenvironment{thm}[1]
{\renewcommand\theinnercustomthm{#1}\innercustomthm}
{\endinnercustomthm}
\newcommand{\enumarabic}[1]{
	\begin{enumerate}[label=\textbf{\arabic*.}]
		#1
	\end{enumerate}
}
\newcommand{\enumalph}[1]{
\begin{enumerate}[label=(\alph*)]
	#1
\end{enumerate}
}
\theoremstyle{definition}
\newtheorem*{defn}{Definition}




%%%%%%%%%%
%Custom Commands
%%%%%%%%%%
\newcommand{\C}{\mathbb{C}}
\newcommand{\N}{\mathbb{N}}
\newcommand{\Q}{\mathbb{Q}}
\newcommand{\R}{\mathbb{R}}
\newcommand{\Z}{\mathbb{Z}}

\newcommand{\ds}{\displaystyle}

\newcommand{\fn}{\insertframenumber}

\newcommand{\im}{\operatorname{im}}

\newcommand{\blank}[1]{\underline{\hspace*{#1}}}

\newcommand{\abar}{\overline{a}}
\newcommand{\bbar}{\overline{b}}
\newcommand{\cbar}{\overline{c}}


\usepackage{fancyhdr}
\fancyhead[L]{\textbf{Math 425: Abstract Algebra I}}
\fancyhead[R]{\textbf{Theorems from the Textbook - Chapter 0}}
\pagestyle{fancy}

\begin{document}

\section*{Notation}
\begin{tabular}{cll}
	\textbf{Symbol}&\textbf{Description}&\textbf{Example}\\\hline
	$\N$&Natural Numbers&\{0,1,2,3,\dots\}\\
	$\Z$&Integers&\{\dots,-2,-1,0,1,2,3,\dots\}\\
	$\Q$&Rational Numbers&Ratios of integers\\
	$\R$&Real Numbers&The standard number line\\
	$\C$&Complex Numbers&$\{a+bi\ |\ a,b\in\R,\ i^2=-1, \text{ and } si=is\ \forall\ s\in\R\}$\\
	$\in$&element of&  $2\in\{1,2,3\}$\\
	$\subseteq$&subset of & $\{2\}\subseteq\{1,2,3\}$ and $\{1,2,3\}\subseteq\{1,2,3\}$\\
	$\subset$ or $\subsetneq$&proper subset of &$\{2\}\subset\{1,2,3\}$ but $\{1,2,3\}\not\subset\{1,2,3\}$\\
	$\cap$& intersection&$\{1,2,3\}\cap\{2,3,4\}=\{2,3\}$\\
	$\cup$& union&$\{1,2,3\}\cup\{2,3,4\}=\{1,2,3,4\}$\\
	$\times$&Cartesian product&$\{1,2\}\times\{3,4\}=\{(1,3),(1,4),(2,3),(2,4)\}$\\
	$A\xrightarrow{\alpha}B$&mapping $\alpha$ from $A$ to $B$ & $\alpha:\Z\to\R$ defined by $\alpha(n)=e^n$ for all $n\in \Z$\\
	$1_A=\textrm{id}_A$&identity map on $A$& $1_A:A\to A$ is defined by $1_A(a)=a$ for all $a\in A$\\
	$\im(\alpha)$&the image of the map $\alpha$&Given $\alpha:\Z\to\R$ defined by $\alpha(n)=e^n$ for all $n\in \Z$,\\
	&&$\textrm{im}(\alpha)=\{e^n : n\in\Z\}$\\
	$\beta\alpha$&composition of maps& Given $\alpha:\Z\to \R$ defined by $\alpha(n)=e^n$ and\\&&  $\beta: \R\to \C$ defined by $\beta(x)= \sqrt{x}$,\\&&
	$\beta\alpha:\Z\to\C$ is defined by $\beta\alpha(n)=\beta(\alpha(n))=\sqrt{e^n}$.\\
	$\equiv$&relation&for $a,b\in\Z$ we say $a\equiv b$ if $5$ divides $a-b$\\
	$[\cdot]$&equivalence class& for the relation just above, $[1]=\{\cdots,-4,1,6,11,\dots\}$\\
	$A_\equiv$&quotient of $A$ by $\equiv$&the collection of unique equivalence classes
\end{tabular}

\section*{Theorems}
\begin{thm}{0.3.1}
	If $\alpha\colon A\to B$ and $\beta\colon A\to B$ are mappings then 
		\[\alpha=\beta \quad\text{if and only if}\quad \alpha(a)=\alpha(b)\text{ for all }a\in A.\]
\end{thm}
\begin{thm}{0.3.2}
	Let $\alpha\colon A\to B$ be a mapping where $A$ and $B$ are nonempty finite sets with $|A|=|B|$.  Then $\alpha$ is one-to-one if and only if $\alpha$ is onto.
\end{thm}
\begin{thm}{0.3.3}
	Let $A\xrightarrow{\alpha}B\xrightarrow{\beta}C\xrightarrow{\gamma}D$ be mappings on sets.  Then
	\enumarabic{
		\item (identity) $\alpha 1_A=\alpha$ and $1_B\alpha=\alpha$
		\item (associativity) $\gamma(\beta\alpha)=(\gamma\beta)\alpha$
		\item If $\alpha$ and $\beta$ are both one-to-one (resp.~onto), then $\beta\alpha$ is one-to-one (resp.~onto) too.
	}
\end{thm}
\begin{thm}{0.3.4}
	If $\alpha:A\to B$ has an inverse, then the inverse mapping is unique.
\end{thm}
\begin{thm}{0.3.5}
Let $\alpha:A\to B$ and $\beta: B \to C$ denote mappings.
\enumarabic{
	\item The identity map, $1_A:A\to A$ is invertible and $1_A^{-1}=1_A$.
	\item If $\alpha$ is invertible, then $\alpha^{-1}$ is invertible and $(\alpha^{-1})^{-1}=\alpha$.
	\item If $\alpha$ and $\beta$ are both invertible, then $\beta\alpha$ is invertible with $(\beta\alpha)^{-1}=\alpha^{-1}\beta^{-1}$.
}
\end{thm}
\begin{thm}{0.3.6}[Invertibility Theorem]
A mapping $\alpha:A\to B$ is invertible if and only if $\alpha$ is a bijection.
\end{thm}	
\begin{thm}{0.4.1}
	Let $\equiv$ be an equivalence on a set $A$ and let $a$ and $b$ denote elements of $A$. Then
	\enumarabic{
		\item $a\in[a]$ for every $a\in A$.
		\item $[a]=[b]$ if and only if $a\equiv b$.
		\item If $a\in[b]$, then $[a]=[b]$.
		\item If $[a]\neq [b]$ then $[a]\cap [b]=\emptyset$.
	}
\end{thm}
\begin{thm}{0.4.2}[Partition Theorem]
	If $\equiv$ is any equivalence on a nonempty set $A$, then the collection of all equivalence classes of $A$ under $\equiv$ partitions $A$.
\end{thm}
\section*{Definitions}
\begin{defn}[Method of Direct Proof]
	To prove $p\Rightarrow p$, demonstrate directly that $q$ is true whenever $p$ is true.
\end{defn}
\begin{defn}[Method of Reduction to Cases]
	To prove $p\Rightarrow q$, show that $p$ implies at at least one of a list $p_1,p_2,\dots,p_n$ of statements (the cases) and that $p_i\Rightarrow q$ for each $i$.
\end{defn}
\begin{defn}[Method of Proof by Contradiction]
	To prove $p\Rightarrow p$, show that the assumption that both $p$ is true and $q$ is false leads to a logical contradiction.
\end{defn}
\begin{defn}
	A \emph{counterexample} to the statement $p\Rightarrow q$ is an example set of values and inputs that has $q$ true and $p$ false.
\end{defn}
\begin{defn}
	Two statements $p$ and $q$ are logically equivalent if both $p\Rightarrow q$ and $q\Rightarrow p$ are true.  In which case we write $p\Leftrightarrow q$ and say ``$p$ if and only if $q$''. To prove such a statement, we must prove both that $p\Rightarrow q$ and $q\Rightarrow p$.
\end{defn}
\begin{defn}
	A \emph{set} is a collection of objects called \emph{elements}.  If $a$ is an element of $A$, we write $a\in A$ or $A\ni a$.
\end{defn}
\begin{defn}
	If $A$ and $B$ are sets such that for all $a\in A$, we also have $a\in B$, then we call $A$ a \emph{subset} of $B$.  This is denoted by $A\subseteq B$. If we know further that $A\neq B$, we can write $A\subset B$ or $A\subsetneq B$, in which case $A$ is a \emph{proper subset} of $B$.
\end{defn}
\begin{defn}[Principle of Set Equality]
	If $A$ and $B$ are sets, then
	\begin{center}
		$A=B$\quad if and only if $A\subseteq B$ and $B\subseteq A$.
	\end{center}
\end{defn}
\begin{defn}
	If $A$ has $n$-elements then we say the \emph{cardinality} of $A$ is $n$ and we write $|A|=n$. Such sets are called \emph{finite} sets.  
	Sets with an infinite number of elements are \emph{infinite} sets.
	
	The set with cardinality 0 is called the \emph{emptyset} and is denoted $\emptyset$.
\end{defn}
\begin{defn}
	The \emph{power set} of a set $A$ is the set $P(A)$ consisting of all subsets of $A$.
\end{defn}
\begin{defn}
The \emph{Cartesian Product} of the sets $A$ and $B$ is the set
\[A\times B:=\{(a,b):a\in A\text{ and }b\in B\}.\]
Note that the elements, $(a,b)$, are ordered pairs.
\end{defn}
\begin{defn}
	Let $A$ and $B$ be sets.
	The \emph{union} of $A$ and $B$ is the set of all elements that appear in at least one of $A$ or $B$.  Denote the union of $A$ and $B$ by 
		\[A\cup B:= \{x\ :\ x\in A\text{ or }x\in B\}.\]
	The \emph{intersection} of $A$ and $B$ is the set of all elements that appear in both $A$ and $B$. Denote the intersection of $A$ and $B$ by 
		\[A\cap B:= \{x\ :\ x\in A\text{ and }x\in B\}.\]
	The difference of $B$ in $A$ is the set of all elements in $a$ that do not appear in $B$. Denote the difference of $B$ in $A$ by
		\[A\setminus B:=\{x\in A\ :\ x\not\in B\}.\]
\end{defn}
\begin{defn}
A \emph{mapping} or \emph{function} $\alpha$ from $A$ to $B$ is a rule that assigns to every input $a\in A$ exactly one output $\alpha(a)\in B$.  The notation here is
\[\alpha:A\to B\textup{ or }A\xrightarrow{\alpha}B.\]

Once we have verified that each input maps to exactly one output then we say the mapping is \emph{well-defined}.
\end{defn}
\begin{defn} Assume $\alpha:A \to B$ is a mapping.
\begin{itemize}[label=--]
	\item We call $A$ the \emph{domain} of $\alpha$ and $B$ the \emph{codomain} of $\alpha$.
	\item If $C\subseteq A$, then the \emph{image} of $C$ is \[f(C)=\{b\in B : b=f(c)\text{ for some } c\in C\}.\] 
	\item The \emph{range} of $\alpha$ is the image of the domain, \[\im(\alpha)=f(A)=\{f(a)\in B : a\in A\}.\]
\end{itemize}
\end{defn}
\begin{defn}
Let $\alpha:A\to B$ be a mapping.
\enumalph{
	\item We call $\alpha$ \emph{one-to-one} or \emph{injective} if for all $a_1,a_2\in A$ if $\alpha(a_1)=\alpha(a_2)$, then $a_1=a_2$.
	\item We call $\alpha$ \emph{onto} or \emph{surjective} if for all $b\in B$ there is an $a\in A$ such that $\alpha(a)=b$.
	\item We call $\alpha$ a \emph{bijection} or \emph{bijective} if $\alpha$ is both one-to-one and onto.
}
\end{defn}
\begin{defn}
The \emph{identity map} for the set $A$ is the map $1_A:A\to A$ defined by $1_A(a)=a$ for all $a\in A$.

If $\alpha:A\to B$ and $\beta:B\to C$ are mappings, we can write
\[A\xrightarrow{\alpha}B\xrightarrow{\beta}C,\]
and the \emph{composition} of the maps is the mapping $\beta\alpha:A\to C$ defined by
\[\beta\alpha(a)=\beta[\alpha(a)]\text{ for all }a\in A.\]
\end{defn}
\begin{defn}
If $\alpha:A\to B$ is a mapping of sets, then we call $\beta:B\to A$ an \emph{inverse} of $\alpha$ if
\[\beta\alpha = 1_A\text{ and }\alpha\beta=1_B.\]
Any map that has an inverse is called \emph{invertible}[
\end{defn}
\begin{defn}
If $A$ is a set, any subset of $A\times A$ is called a \emph{relation} on $A$.
\end{defn}
\begin{defn}
A relation $\equiv$ on a set $A$ is called an \emph{equivalence relation} if it satisfies all of the following conditions for all $a,b,c\in A$,
\enumarabic{
	\item $a\equiv a$ (\emph{reflexivity}),
	\item If $a\equiv b$ then $b\equiv a$ (\emph{symmetric}),
	\item If $a\equiv b$ and $b\equiv c$, then $a\equiv c$ (\emph{transitive}).
}
\end{defn}
\begin{defn}
An equivalence relation $\mathcal{R}$ on a set $S$ partitions $S$ into disjoint pieces $S_i$ such that 	
\[S=S_1\cup S_2\cup\cdots.\]
Each $S_i$ is called an \emph{equivalence class} - see next definition.

We can pick any member of each class to be a \emph{representative} of the class $S_i$. We denote this class by square brackets or overbar.
\end{defn}
\begin{defn}
Given an equivalence relation $\equiv$ on a set $A$, we define the \emph{equivalence class} of $a$ to be the set
\[[a]=\{x\in A\ |\ x\equiv a\}.\]
\end{defn}
\begin{defn}
	Two sets are \emph{disjoint} if their intersection is empty.
	
	A collection of sets $\cal P$ is \emph{pairwise disjoint} if $X\cap Y=\emptyset$ for all $X\neq Y$ in $\cal P$.
\end{defn}
\begin{defn}
	A \emph{partition} of the set $A$ is a collection $\cal P$ of subsets of $A$ such that
	\enumarabic{
		\item $\emptyset\not\in\cal P$.
		\item $\cal P$ is pairwise disjoint.
		\item Every element of $A$ is in some element of $\cal P$.
	}
\end{defn}
\begin{defn}
	If $\equiv$ is an equivalence on $A$, the set of equivalence classes is called the \emph{quotient set} and is denoted $A_\equiv$.
\end{defn}
\begin{defn}
	The mapping $\phi:A\to A_{\equiv}$ given by $\phi(a)=[a]$ for all $a\in A$ is called the \emph{natural mapping}.
\end{defn}
\end{document}