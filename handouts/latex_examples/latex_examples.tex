\documentclass[12pt]{article}
\usepackage{amsmath,amssymb,graphicx,multicol,enumitem,amsthm}
\usepackage[margin=1in]{geometry}


%%%%%%%%%%
%Custom Commands
%%%%%%%%%%
\newcommand{\C}{\mathbb{C}}
\newcommand{\N}{\mathbb{N}}
\newcommand{\Q}{\mathbb{Q}}
\newcommand{\R}{\mathbb{R}}
\newcommand{\Z}{\mathbb{Z}}

\newcommand{\ds}{\displaystyle}

\newcommand{\im}{\operatorname{im}}

\newcommand{\abar}{\overline{a}}
\newcommand{\bbar}{\overline{b}}
\newcommand{\cbar}{\overline{c}}

\begin{document}
			 \begin{center}
			 	Math 425: Abstract Algebra I
		
				\LaTeX Examples
			 \end{center}
\vskip .25in
\begin{itemize}
	\item First note that I made some vertical space above this using the command
	\verb|\vksip .25in|
	Use this command liberally to help make your document readable.
	\item Math mode is generated with dollar signs.  For example $a^2+b^2=c^2$. There is a difference between a and $a$.
	\item Centered equations can be made in many ways. Here are a few:
		\begin{enumerate}
			\item Double dollar signs
				$$a_1+a_2+\cdots+a_n = 1$$
			\item Square brackets
				\[\begin{bmatrix}
					a & b\\ c&d
				\end{bmatrix}\]
			\item Equation environment - numbered (only use this if you're going to refer to this equation again)
				\begin{equation}
					x^2-2 = (x-\sqrt{2})(x+\sqrt{2})
				\end{equation}
			\item Equation environment - unnumbered
				\begin{equation*}
					2435\equiv 11\pmod{24}
				\end{equation*}
			\item Multiple lines with lined up equations
				\begin{eqnarray*}
					1+2+\cdots + n + n+1 &=& \frac{n(n+1)}{2} + (n+1)\\
						&=& \frac{n(n+1)}{2} + \frac{2(n+1)}{2}\\
						&=& \frac{(n+1)(n+2)}{2}
				\end{eqnarray*}
			\item Multiple lines with explanations
				$$\begin{array}{rcll}
					(n+1)!&=& n!(n+1)&\text{definition of factorial}\\
					&\geq& 2^n (n+1) & \text{inductive hypothesis}\\
					&\geq& 2^n\cdot 2& \text{assumption of }n\geq 3\\
					&=& 2^{n+1}&\text{exponent laws}
				\end{array}$$
		\end{enumerate}
	\item To write a proof use \verb|\begin{proof}| and \verb|\end{proof}|, and you will get something like this:
		\begin{proof}
			Let $S$ be a set with binary operation $*$. Assume there is an identity element of $*$ in $S$. Let $e_1,e_2\in S$ be identity elements of $S$ with respect to $*$. We want to show $e_1=e_2$. Since $e_1$ is an identity element, we have
				\[e_1*e_2 = e_2.\]
			Similarly since $e_2$ is an identity element,
				\[e_1*e_2 = e_1.\]
			Combining these two equations, we see
				\[e_2=e_1*e_2=e_1.\]
			We conclude that if an identity exits with respect to a binary operation, then that identity is unique.
		\end{proof}
	\item There are also examples in homework assignments. Please let me know if you have further questions.
\end{itemize}
\end{document}