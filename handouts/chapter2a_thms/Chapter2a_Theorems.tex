\documentclass[12pt]{article}
\usepackage{amsmath,amssymb,graphicx,multicol,enumitem,amsthm}
\usepackage[margin=1in]{geometry}
\pagestyle{empty}
\newtheorem{innercustomthm}{Theorem}
\newtheorem*{cor}{Corollary}
\newenvironment{thm}[1]
{\renewcommand\theinnercustomthm{#1}\innercustomthm}
{\endinnercustomthm}
\newcommand{\enumarabic}[1]{
	\begin{enumerate}[label=\textbf{\arabic*.}]
		#1
	\end{enumerate}
}
\newcommand{\enumalph}[1]{
\begin{enumerate}[label=(\alph*)]
	#1
\end{enumerate}
}
\theoremstyle{definition}
\newtheorem*{defn}{Definition}




%%%%%%%%%%
%Custom Commands
%%%%%%%%%%
\newcommand{\C}{\mathbb{C}}
\newcommand{\N}{\mathbb{N}}
\newcommand{\Q}{\mathbb{Q}}
\newcommand{\R}{\mathbb{R}}
\newcommand{\Z}{\mathbb{Z}}

\newcommand{\ds}{\displaystyle}

\newcommand{\fn}{\insertframenumber}

\newcommand{\im}{\operatorname{im}}

\newcommand{\blank}[1]{\underline{\hspace*{#1}}}

\newcommand{\abar}{\overline{a}}
\newcommand{\bbar}{\overline{b}}
\newcommand{\cbar}{\overline{c}}

\begin{document}
			\noindent Math 425: Abstract Algebra I
		
		\noindent	Theorems from the Textbook - Sections 2.1-2.4
\vskip .25in


\section*{Theorems}
\begin{thm}{2.1.1}
	If a binary operation $*$ on a set $S$ has an identity, then it is unique.
\end{thm}
\begin{thm}{2.1.4}
If $(G,*)$ is a group and $g\in G$, then the inverse of $G$ is unique.
\end{thm}
\begin{thm}{2.2.1}
If $(M,*)$ is a monoid, then the set of all unit $M^\times$ is a group using the operation $*$, called the \emph{unit group}.
\end{thm}
\begin{thm}{2.2.2}
If $G_1,G_2,\dots,G_n$ are groups with respective operations $*_1,*_2,\dots,*_n$, then 
\[G_1\times G_2\times\cdots\times G_n\]
is a group under component-wise operation
\[(g_1,g_2,\dots,g_n)*(h_1,h_2,\dots,h_n)=(g_1*_1h_1,g_2*_2h_2,\dots,g_n*_nh_n).\]
\end{thm}
\begin{thm}{2.2.3}
	Let $g,h,g_1,g_2,\dots,g_{n-1},g_n$ be elements of a group $G$ ($n\in \Z_{\geq 1}$).
	\enumarabic{
		\item $e^{-1}=e$.
		\item $(g^{-1})^{-1}=g$.
		\item $(gh)^{-1}=h^{-1}g^{-1}$.
		\item $(g_1g_2\cdots g_n)^{-1}=g_n^{-1}g_{n-1}^{-1}\cdots g_2^{-1} g_1^{-1}$.
		\item $(g^m)^{-1}=(g^{-1})^m$ for all $m\geq 0$.
	}
\end{thm}
\begin{thm}{2.2.4}[Exponent Laws]
	Let $G$ be a group and $g,h\in G$.
	\enumarabic{
		\item $g^ng^m=g^{n+m}$ for all $m,n\in\Z$
		\item $(g^n)^m=g^{n\cdot m}$ for all $m,n\in\Z$
		\item If $gh=hg$, then $(gh)^n=g^nh^n$ for all $n\in\Z$
	}
\end{thm}
\begin{thm}{2.2.5}[Cancellation Laws]
Let $G$ be a group and $g,h,f\in G$.
\enumarabic{
	\item If $gh=gf$ then $h=f$ (\emph{left cancellation})
	\item If $hg=fg$ then $h=f$ (\emph{right cancellation})
}
\end{thm}
\begin{thm}{2.2.6}
Let $G$ be a group and $g,h\in G$.
\enumarabic{
	\item The equation $gx=h$ has a unique solution $x=g^{-1}h$ in $G$.
	\item The equation $xg=h$ has a unique solution $x=hg^{-1}$ in $G$.
}
\end{thm}
\begin{thm}{2.3.1}[Subgoup Test]
A subset $H$ of a group $G$ is a subgroup of $G$ if and only if the following conditions are satisfied.
\enumarabic{
	\item $1_G\in H$, where $1_G$ is the identity element of $G$.
	\item If $h\in H$ and $h_1\in H$, then $hh_1\in H$.
	\item If $h\in H$, then $h^{-1}\in H$, where $h^{-1}\in G$ denotes the inverse of $h$ in $G$.
}

Note that implicit in these statements, if $H\leq G$ then $H$ and $G$ have the same unity and inverses persist.
\end{thm}
\begin{thm}{2.3.3}
If $G$ is any group, then $Z(G)$ is a subgroup of $G$. Moreover, $Z(G)$ is always abelian.
\end{thm}
\begin{thm}{2.4.1}
Let $g$ be an element of a group $G$, and write
\[\langle g\rangle =\{g^k:k\in\Z\}.\]
Then $\langle g\rangle$ is a subgroup of $G$, and $\langle g\rangle\subseteq H$ for every subgroup $H$ of $G$ with $g\in H$.
\end{thm}
\begin{thm}{2.4.2}
Let $g\in G$ with $o(g)=n$.  Then
\enumarabic{
	\item $g^k=1$ if and only if $n|k$.
	\item $g^k=g^m$ if and only if $k\equiv m\pmod n$
	\item $\langle g\rangle=\{1,g,g^2,\dots, g^{n-1}\}$ where $1,g,g^2,\dots,g^{n-1}$ are all distinct.
}
\end{thm}
\begin{thm}{2.4.3}
Let $G$ be a group and let $g\in G$ satisfy $o(g)=\infty$.  Then 
\enumarabic{
	\item $g^k=1$ if and only if $k=0$.
	\item $g^k=g^m$ if and only if $k=m$.
	\item $\langle g\rangle =\{\dots,g^{-2},g^{-1},1,g,g^2,\dots\}$ where the $g^i$ are distinct.
}
\end{thm}
\begin{cor}
For all $g$ in a group $G$, $|g|=|\langle g\rangle|$.
\end{cor}
\begin{thm}{}[{Order in $\Z_n$}]
Given $\overline{a}\in(\Z_n,+)$, with $1\leq a\leq n-1$, 
\[|\overline{a}|=\frac{n}{\gcd(a,n)}.\]
\end{thm}
\begin{thm}{}
If $\gamma=(k_1\ k_2\ \dots\ k_r)$ is an $r$-cycle in $S_n$, then $|\gamma|=r$.
\end{thm}
\begin{thm}{2.4.4}
If $\gamma=\sigma_1\sigma_2\dots\sigma_r$ where $\sigma_i$ are disjoint cycles, then
\[|\gamma|=\operatorname{lcm}(|\sigma_1|,|\sigma_2|,\dots,|\sigma_r|).\]
\end{thm}
\begin{thm}{2.4.6}
Every cyclic group is abelian, but the converse does not hold.
\end{thm}
\begin{thm}{2.4.7}
Every subgroup of a cyclic group is cyclic.
\end{thm}
\begin{thm}{2.4.8}
Let $G=\langle g\rangle$ be a cyclic group, where $o(g)=n$. Then $G=\langle g^k\rangle$ if and only if $\gcd(k,n)=1$.
\end{thm}
\begin{thm}{2.4.9}[{The Fundamental Theorem of Finite Cyclic Groups}]
Let $G=\langle g\rangle$ be a cyclic group of order $n$.
\enumarabic{
	\item If $H$ is a subgroup of $G$, then $H=\langle g^d\rangle$ for some $d|n$. Hence $|H|$ divides $n$.
	\item Conversely if $k|n$, then $\langle g^{n/k}\rangle$ is the unique subgroup of $G$ of order $k$.
}
\end{thm}
\section*{Definitions}
\begin{defn}
	A \emph{binary operation}, $*$ on a set $S$ is a function that associates to each ordered pair $(a,b)\in S\times S$ an element of $S$ which we call $a*b$.
	
	Since we know that $a*b\in S$ for all $a,b\in S$, we say that the binary operation is \emph{closed under $*$}.
\end{defn}
\begin{defn}
A binary operation $*$ on $S$ is \emph{associative} if 
\[a*(b*c)=(a*b)*c,\]
for all $a,b,c\in S$.
\end{defn}
\begin{defn}	
A binary operation $*$ on $S$ is \emph{commutative} if 
\[a*b=b*a,\]
for all $a,b\in S$.
\end{defn}
\begin{defn}
An element $e\in S$ is called an \emph{identity} (or \emph{unity}) for the binary operation $*$ if 	
\[a*e=e*a=a,\]
for all $a\in S$.
\end{defn}
\begin{defn}
A set $S$ along with a binary operation $*$ is called an \emph{monoid} if $*$ is associate and has an identity.

\vskip 1em

If $(S,*)$ is also commutative, then we say $S$ is a \emph{commutative monoid}.
\end{defn}
\begin{defn}
Let $(M,*)$ be a monoid.  

If $x\in M$, we call $y\in M$ an \emph{inverse of $x$} if 	
\[xy=e=yx.\]
An element that has an inverse is called a \emph{unit}.
\end{defn}
\begin{defn}
	Suppose that 
	\enumarabic{
		\item $G$ is a set and $*$ is a binary operation on $G$,
		\item $*$ is associative,
		\item there is some $e\in G$ such that 
		\[g*e=e*g=g,\]
		for all $g\in G$, and
		\item for all $g\in G$, the is an $h\in G$ such that $g*h=e=h*g$.
	}
	Then $(G,*)$ is a \emph{GROUP}.
\end{defn}
\begin{defn}
The \emph{$n$th roots of unity} are the complex numbers that are the roots of 
\[x^n-1.\]
Denote the set of roots as $\mathcal{U}_n$
\end{defn}
\begin{defn}
	If the operation of a group $G$ is commutative, we call $G$ an \emph{abelian group}.
\end{defn}
\begin{defn}
A \emph{Cayley table} is essentially a multiplication table for a given binary operation.
\end{defn}
\begin{defn}
We call $C_n=\{1,a,a^2,\dots,a^{n-1}\}$ the cyclic group of order $n$. Multiplication is defined by $a^xa^y=a^{x+y}$ and $a^n=a^0=1$.
\end{defn}
\begin{defn}
A subsets $H$ of a group $G$ is call a \emph{subgroup} of $G$ if $H$ is also a group using the same operation as $G$.
We denote subgroups using the notation $H\leq G$.

If $H\leq G$ and $H\neq G$, we call $H$ a \emph{proper subgroup of G}.
\end{defn}
\begin{defn}
The \emph{subset of $G$ generated by $g\in G$} in multiplicative notation is
\[\langle g\rangle=\{g^k | k\in Z\}=\{\dots,g^{-3},g^{-2},g^{-1},1,g,g^2,\dots\}.\]
The \emph{subset of $G$ generated by $g\in G$} in additive notation is
\begin{align*}
\langle g\rangle&=\{kg | k\in Z\}\\&=\{\dots,-g-g-g,-g-g,-g,0,g,g+g,g+g+g,\dots\}.
\end{align*}
\end{defn}
\begin{defn}
The \emph{subgroup lattice} of a group $G$ is a schematic picture of the subgroups of $G$.  A line going up from one group to another indicates that the bottom group is a subgroup of the top one.
\end{defn}
\begin{defn}
The \emph{center} of the group $G$ is the set
\[Z(G) = \{z\in G|zg=gz\ \forall g\in G\}.\]
\end{defn}
\begin{defn}
A group $G$ is \emph{cyclic} if there is some $g\in G$ for which $G=\langle g\rangle$.
\end{defn}
\begin{defn}
If $G$ is a finite group, the \emph{order of a group} $G$ is denoted $|G|$ and is the cardinality of the set $G$.

The \emph{order of an element} $g\in G$ is denoted $|g|$ or $o(g)$ and equals the smallest positive integer $n$ such that $g^n=e$.
\end{defn}
\begin{defn}
In general, if $X$ is a nonempty subset of a group $G$, then the \emph{subgroup of $G$ generated by $X$} is defined as	
\[\begin{array}{rcl}
\langle X\rangle &=& \{\text{products of powers (not nec. distinct) of elements of X}\}\\
&=&\{x_1^{k_1}x_2^{k_2}\cdots x_m^{k_m}\ |\ x_i\in X,\ k_i\in \Z,\ m\geq 1\}
\end{array}\]
We will always have $\langle X\rangle\leq G$.
\end{defn}
\end{document}