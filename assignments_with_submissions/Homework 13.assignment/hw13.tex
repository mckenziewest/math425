\documentclass[12pt]{article}
\usepackage{amsmath,amsthm,amssymb,hyperref,enumitem}
\usepackage[margin=.75in]{geometry}
\pagestyle{empty}
\usepackage{tikz}

\newcommand{\C}{\mathbb{C}}
\newcommand{\R}{\mathbb{R}}
\newcommand{\Q}{\mathbb{Q}}
\newcommand{\Z}{\mathbb{Z}}

\setlength{\parindent}{0pt}
\begin{document}
	\begin{center}
		{\Large \bf Math 425 Assignment 13}
	\end{center}
	\section*{Purpose}
	\begin{itemize}
		\item Assess objectives and additional exercises from sections 4.1-4.2, namely
		\begin{itemize} 
			\item Ring-5: Compute roots of a polynomial in positive characteristic rings.
			\item Ring-6:	Prove whether a polynomial is irreducible over a given base field or not.
			\item Supplemental exercises Supp-26 and Supp-27.
		\end{itemize}
		\item Build a strong foundation of creative critical thinking and proof-writing techniques.
	\end{itemize}
	\section*{Task}
	\begin{itemize}
		\item Complete the exercises listed on the following page(s) of this document and submit your solutions as a pdf to Canvas.
		\item I strongly recommend you use LaTeX to typeset your proofs.
		\item You may work in groups but everyone should submit their own assignment written in their own words.  Do NOT copy your classmates.
		\item Allowed resources: our textbook, classmates, your notes, videos linked in Canvas.
		\item Unacceptable resources: anything you find on an internet search. Do NOT use a homework help website (e.g., Chegg). Their solutions are often wrong or use incorrect context.  I want you to practice making arguments that are yours. Take some ownership.
	\end{itemize}
	\section*{Criteria}
		All items will earn a score using the following scale:
		\begin{itemize}
			\item \textbf{Exceptional} - Solution is succinct, references the correct theorems and definitions, and is entirely correct.
			\item \textbf{Satisfactory} - Solution is essentially correct. It still references the correct theorems and definitions. 
					It may be longer than necessary, have inconsequential errors, or have some grammatical mistakes.
			\item \textbf{Nearly Complete} - Solution has some errors and should be re-written.
			\item \textbf{Unsatisfactory} - Solution has major errors, references content not covered in class or in the textbook, or is incomplete in some major way.
		\end{itemize}
		Recall from the syllabus
		\begin{itemize}
			\item If you earn either an \textbf{Exceptional} or \textbf{Satisfactory} mark on an objective exercise (labeled Intro-, Group-, or Ring-) then you may consider that item complete. 
			\item If you earn a \textbf{Nearly Complete} or \textbf{Unsatisfactory} mark on a an objective exercise (labeled Intro-, Group-, or Ring-) then you have not yet completed this objective.
			\item You may submit a new attempt at completing that objective on a future Wednesday. You must select a new exercise listed under the given objective, you cannot resubmit a version you have attempted previously.  The only limit you have on number of attempts is the number of exercises available for the objective.
			\item If you earn an \textbf{Exceptional} mark on an additional exercise (labeled A-) then you will earn one half point toward the ten total points in that section of your overall grade.
			\item If you earn an \textbf{Satisfactory} mark on an additional exercise (labeled Supp-) then you will earn 0.25 points toward the ten total points in that section of your overall grade. 
			\item If you earn a \textbf{Nearly Complete} or \textbf{Unsatisfactory} mark on an additional exercise (labeled Supp-) then you will earn 0 points toward the ten total points in that section of your overall grade. 
		\end{itemize}
	
			
	\newpage

	\textbf{(Ring-5)} Access this exercise on WeBWorK using the link on Canvas.  That is, find the assignment titled \texttt{Ring-5}. Once you open this assignment, there will be a link to WeBWorK. This link will bring you directly to the exercise. You will have 5 attempts to answer all questions for the exercise correctly. After those 5 attempts, a new problem will be generated and you can start a new attempt to complete the objective.
	
	\textbf{(Ring-6)}  Access this exercise on WeBWorK using the link on Canvas.  That is, find the assignment titled \texttt{Ring-6}. Once you open this assignment, there will be a link to WeBWorK. This link will bring you directly to the exercise. You will have 5 attempts to answer all questions for the exercise correctly. After those 5 attempts, a new problem will be generated and you can start a new attempt to complete the objective.
	
	
	\newpage
	Name: \underline{\hspace*{3in}}
	\vskip .25in
	\textbf{(Supp-26)} Show that if $p$ is prime, then the polynomial $x^p-x\in\Z_p[x]$ factors as
		\[x^p-x=x(x-1)(x-2)\cdots(x-(p-1)).\]
		
		(Hint: Thm 4.1.8 and 1.3.8)
	
	
	\newpage
	Name: \underline{\hspace*{3in}}
	\vskip .25in
	\textbf{(Supp-27)} Let $R$ be any ring and let $A\subset R[x]$ be the set of
	polynomials whose coefficients sum to 0. Show that $A$ is an ideal of $R[x]$ and 
	$R[x]/A\cong R$.
	
\end{document}