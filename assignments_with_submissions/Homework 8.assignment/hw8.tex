\documentclass[12pt]{article}
\usepackage{amsmath,amsthm,amssymb,enumitem}
\usepackage[margin=.75in]{geometry}
\pagestyle{empty}
\usepackage{tikz}



\newcommand{\N}{\mathbb{N}}
\newcommand{\R}{\mathbb{R}}
\newcommand{\Z}{\mathbb{Z}}
\newcommand{\abar}{\overline{a}}

\newcommand{\aut}{\operatorname{aut}}

\setlength{\parindent}{0pt}
\begin{document}
	\begin{center}
		{\Large \bf Math 425 Assignment 8}
	\end{center}
	\section*{Purpose}
		\begin{itemize}
			\item Assess objectives and additional exercises from section 2.6 and 2.7, namely
				\begin{itemize}
					\item Group-6:	Compute the right and left cosets of a subgroup of a group.
					\item Group-7:	Use Lagrange’s theorem or its corollaries to prove a property of a group.
					\item Group-8:	Compute the group and Cayley table for the symmetry group of a given shape, and determine which of our standard groups it is isomorphic to.
					\item Supplemental exercise Supp-17.
				\end{itemize}
			\item Build a strong foundation of creative critical thinking and proof-writing techniques.
		\end{itemize}
	\section*{Task}
		\begin{itemize}
			\item Complete the exercises listed on the following page(s) of this document and submit your solutions as a pdf to Canvas.
			\item I strongly recommend you use LaTeX to typeset your proofs.
			\item You may work in groups but everyone should submit their own assignment written in their own words.  Do NOT copy your classmates.
			\item Allowed resources: our textbook, classmates, your notes, videos linked in Canvas.
			\item Unacceptable resources: anything you find on an internet search. Do NOT use a homework help website (e.g., Chegg). Their solutions are often wrong or use incorrect context.  I want you to practice making arguments that are yours. Take some ownership.
		\end{itemize}
	\section*{Criteria}
		All items will earn a score using the following scale:
		\begin{itemize}
			\item \textbf{Exceptional} - Solution is succinct, references the correct theorems and definitions, and is entirely correct.
			\item \textbf{Satisfactory} - Solution is essentially correct. It still references the correct theorems and definitions. 
					It may be longer than necessary, have inconsequential errors, or have some grammatical mistakes.
			\item \textbf{Nearly Complete} - Solution has some errors and should be re-written.
			\item \textbf{Unsatisfactory} - Solution has major errors, references content not covered in class or in the textbook, or is incomplete in some major way.
		\end{itemize}
		Recall from the syllabus
		\begin{itemize}
			\item If you earn either an \textbf{Exceptional} or \textbf{Satisfactory} mark on an objective exercise (labeled Intro-, Group-, or Ring-) then you may consider that item complete. 
			\item If you earn a \textbf{Nearly Complete} or \textbf{Unsatisfactory} mark on a an objective exercise (labeled Intro-, Group-, or Ring-) then you have not yet completed this objective.
			\item You may submit a new attempt at completing that objective on a future Wednesday. You must select a new exercise listed under the given objective, you cannot resubmit a version you have attempted previously.  The only limit you have on number of attempts is the number of exercises available for the objective.
			\item If you earn an \textbf{Exceptional} mark on an additional exercise (labeled A-) then you will earn one half point toward the ten total points in that section of your overall grade.
			\item If you earn an \textbf{Satisfactory} mark on an additional exercise (labeled Supp-) then you will earn 0.25 points toward the ten total points in that section of your overall grade. 
			\item If you earn a \textbf{Nearly Complete} or \textbf{Unsatisfactory} mark on an additional exercise (labeled Supp-) then you will earn 0 points toward the ten total points in that section of your overall grade. 
		\end{itemize}
	
	
\newpage
\textbf{(Group-6)}  Access this exercise on WeBWorK using the link on Canvas.  That is, find the assignment titled \texttt{Group-6}. Once you open this assignment, there will be a link to WeBWorK. This link will bring you directly to the exercise. You will have 5 attempts to answer all questions for the exercise correctly. After those 5 attempts, a new problem will be generated and you can start a new attempt to complete the objective.

	
	\newpage
\mbox{}\hfill {\bf Math 425 Assignment 8}\\
Name: \underline{\hspace*{3in}}\hfill Due 4/5/24
\vskip .25in

\textbf{(Group-7.1)} Let $p$ be a prime number and $G$ be a group with $|G|=p^2$. Show that every proper subgroup of $G$ is cyclic.

\newpage
Name: \underline{\hspace*{3in}}
\vskip .25in

\textbf{(Group-8.1)} Consider the hexagon as follows. (For clarity: The two horizontal edges are the same length. The 4 shorter edges are all the same length, which is smaller than the length of the horizontal edges. The angles at nodes 1,2,4,and 5 are all equal and the angles at nodes 3 and 6 are equal.)

\begin{center}
	\begin{tikzpicture}
		\setlength{\unitlength}{1in}
		\draw (0,0) -- (1,1) -- (5,1) -- (6,0) -- (5,-1) -- (1,-1) -- (0,0) node[anchor=east] {6} (1,1) node[anchor=south] {1} (5,1) node[anchor=south]{2} (6,0) node[anchor=west]{3} (5,-1) node[anchor=north] {4} (1,-1) node[anchor=north]{5};
	\end{tikzpicture}
\end{center}

\begin{enumerate}[label=(\alph*)]
	\item Using $r$ as rotation by $60^\circ$ and $f$ as a flip along the vertical axis at the center of this figure, as in $D_6$, write all of the symmetries of this hexagon in terms of $r$ and $f$.  That is, which combinations of $r$ and $f$ return this shape to the same location, possibly with the numbers swapping?
	\vskip 2in
	\item Write and complete a Cayley table for the group you found in part (a).
	\vfill 
	\item What group ($\Z_n$, $\Z_m\times\Z_n$, $S_n$, $D_n$, $\dots$) is this group of symmetries isomorphic to? Specifically you should be considering the following questions: What is the order of the group? Is this group cyclic? If it is not cyclic is it at least abelian?
	\vskip 1in
\end{enumerate}

\newpage
Name: \underline{\hspace*{3in}}
\vskip .25in

\textbf{(Supp-17)} Let $G$ be a group and fix elements $a,b\in G$. Let $H$ be a subgroup of $G$ such that $Ha=bH$.  Show that $aH=Hb$.

\end{document}