\documentclass[12pt]{article}
\usepackage{amsmath,amsthm,amssymb,hyperref,enumitem}
\usepackage[margin=.75in]{geometry}
\pagestyle{empty}



\newcommand{\N}{\mathbb{N}}
\newcommand{\R}{\mathbb{R}}
\newcommand{\Z}{\mathbb{Z}}
\newcommand{\abar}{\overline{a}}
\newcommand{\bbar}{\overline{b}}

\setlength{\parindent}{0pt}
\begin{document}
	\begin{center}
		{\Large \bf Math 425 Assignment 5}
	\end{center}
	\section*{Purpose}
	\begin{itemize}
		\item Assess objectives and additional exercises from sections 2.2 and 2.3, namely
		\begin{itemize}
			\item Group-2: Use the subgroup test to prove that a subset of a group is a subgroup.
			\item Supplemental exercises Supp-11 and Supp-12.
		\end{itemize}
		\item Build a strong foundation of creative critical thinking and proof-writing techniques.
	\end{itemize}
	\section*{Task}
	\begin{itemize}
		\item Complete the exercises listed on the following page(s) of this document and submit your solutions as a pdf to Canvas.
		\item I strongly recommend you use LaTeX to typeset your proofs.
		\item You may work in groups but everyone should submit their own assignment written in their own words.  Do NOT copy your classmates.
		\item Allowed resources: our textbook, classmates, your notes, videos linked in Canvas.
		\item Unacceptable resources: anything you find on an internet search. Do NOT use a homework help website (e.g., Chegg). Their solutions are often wrong or use incorrect context.  I want you to practice making arguments that are yours. Take some ownership.
	\end{itemize}
	\section*{Criteria}
	All items will earn a score using the following scale:
	\begin{itemize}
		\item \textbf{Exceptional} - Solution is succinct, references the correct theorems and definitions, and is entirely correct.
		\item \textbf{Satisfactory} - Solution is nearly correct. It still references the correct theorems and definitions. 
		It may be longer than necessary, have minor errors, or have some grammatical mistakes.
		\item \textbf{Unsatisfactory} - Solution has major errors, references content not covered in class or in the textbook, or is incomplete in some major way.
	\end{itemize}
	Recall from the syllabus
	\begin{itemize}
		\item If you earn either an \textbf{Exceptional} or \textbf{Satisfactory} mark on an objective exercise (labeled Intro-, Group-, or Ring-) then you may consider that item complete. 
		\item If you earn an \textbf{Unsatisfactory} mark on a an objective exercise (labeled Intro-, Group-, or Ring-) then you have not yet completed this objective.
		\item You may submit a new attempt at completing that objective on a future Wednesday. You must select a new exercise listed under the given objective, you cannot resubmit a version you have attempted previously.  The only limit you have on number of attempts is the number of exercises available for the objective.
		\item Some objectives will be completed through WeBWorK.  You must access these through the course on Canvas. Once you earn full credit on the assignment with at most 5 attempts, then you will have completed that objective.
		\item If you earn an \textbf{Exceptional} mark on an additional exercise (labeled A-) then you will earn one point toward the fifteen total points in that section of your overall grade. You can consider the exercise complete.
		\item If you earn an \textbf{Satisfactory} mark on an additional exercise (labeled Supp-) then you will earn 0.5 points toward the fifteen total points in that section of your overall grade. You may submit a new attempt at this exercise on a future Wednesday. Unlike the objective exercises, you must submit an attempt at the exact same exercise rather than another in a similar theme.
		\item If you earn an \textbf{Unsatisfactory} mark on an additional exercise (labeled Supp-) then you will earn 0 points toward the fifteen total points in that section of your overall grade. You may submit a new attempt at this exercise on a future Wednesday. Unlike the objective exercises, you must submit an attempt at the exact same exercise rather than another in a similar theme.
	\end{itemize}
	
	
	\newpage
	Name: \underline{\hspace*{3in}}
	\vskip .25in
	
	\textbf{(Group-2)} 
	\begin{enumerate}[label=(\alph*)]
		\item If $G$ is an abelian group with operation $*$ and identity $e$, show that $$H=\{a\in G:a*a=e\}$$ is a subgroup of $G$.
		\vfill
		
		\item Give an example of a non-abelian group $G$ such that the subset $H=\{a\in G:a*a=e\}$ is not a subgroup of $G$. 
		
		(And at this point, go back to your proof for part (a) and find the spot where you used the fact that the group was abelian if you haven't already done so.)
		\vskip 2in
	\end{enumerate}
	
	\newpage
	Name: \underline{\hspace*{3in}}
	\vskip .25in
	
	\textbf{(Supp-11)} If $G$ is a group and $g\in G$ is a fixed element, define $C(g)=\{z\in G:z*g=g*z\}$, in other words, $C(g)$ is all of the elements of the group that commute with the specific element $g$.  Show that $C(g)$ is a subgroup of $G$.  (We call $C(g)$ the \emph{centralizer} of $g$ in $G$.)
	
	It might help to compute a few cases first.  Here are some examples:
	\begin{itemize}
		\item Let $G=\Z_4$, then $C(\bar {2})=\{\bar 0,\bar 1,\bar 2,\bar 3\}=\Z_4$.
		\item Let $G= S_3$ and $g=(1\ 2\ 3)$. Then 
		\[C(g)= \{\varepsilon,(1\ 2\ 3),(1\ 3\ 2)\}\]
		\item Let $G= S_4$ and $g = (1\ 2)$. Then 
		\[C(g) = \{\varepsilon,(1\ 2),(3\ 4),(1\ 2)(3\ 4)\}.\]
	\end{itemize}
	
	
	\newpage
	Name: \underline{\hspace*{3in}}
	\vskip .25in
	
	\textbf{(Supp-12)} Let $a$ and $b$ be elements in a group $G$. If $a^n=b^n$ and $a^m=b^m$ where $\gcd(m,n)=1$, show that $a=b$. (Hint: Theorem 1.2.4)
	
	
\end{document}