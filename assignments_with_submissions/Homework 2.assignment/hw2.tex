\documentclass[12pt]{article}
\usepackage{amsmath,amsthm,amssymb,enumitem}
\usepackage[margin=.75in]{geometry}
\pagestyle{empty}



\newcommand{\R}{\mathbb{R}}
\newcommand{\Z}{\mathbb{Z}}

\setlength{\parindent}{0pt}
\begin{document}
	\begin{center}
		{\Large \bf Math 425 Assignment 2}
	\end{center}
	\section*{Purpose}
		\begin{itemize}
			\item Assess objectives and additional exercises from section 0.3-0.4, namely
				\begin{itemize}
					\item Intro-2: Prove a map is or is not one-to-one/onto/bijective.
					\item Intro-3: Prove that a relation is an equivalence relation then compute the set of unique equivalence classes.
					\item Supplemental exercises Supp-3, Supp-4, and Supp-5. 
				\end{itemize}
			\item Build a strong foundation of creative critical thinking and proof-writing techniques.
		\end{itemize}
	\section*{Task}
		\begin{itemize}
			\item Complete the exercises listed on the following page(s) of this document and submit your solutions as a pdf to Canvas.
			\item I strongly recommend you use LaTeX to typeset your proofs.
			\item You may work in groups but everyone should submit their own assignment written in their own words.  Do NOT copy your classmates.
			\item Allowed resources: our textbook, classmates, your notes, videos linked in Canvas.
			\item Unacceptable resources: anything you find on an internet search. Do NOT use a homework help website (e.g., Chegg). Their solutions are often wrong or use incorrect context.  I want you to practice making arguments that are yours. Take some ownership.
		\end{itemize}
	\section*{Criteria}
		All items will earn a score using the following scale:
		\begin{itemize}
			\item \textbf{Exceptional} - Solution is succinct, references the correct theorems and definitions, and is entirely correct.
			\item \textbf{Satisfactory} - Solution is essentially correct. It still references the correct theorems and definitions. 
					It may be longer than necessary, have inconsequential errors, or have some grammatical mistakes.
			\item \textbf{Nearly Complete} - Solution has some errors and should be re-written.
			\item \textbf{Unsatisfactory} - Solution has major errors, references content not covered in class or in the textbook, or is incomplete in some major way.
		\end{itemize}
		Recall from the syllabus
		\begin{itemize}
			\item If you earn either an \textbf{Exceptional} or \textbf{Satisfactory} mark on an objective exercise (labeled Intro-, Group-, or Ring-) then you may consider that item complete. 
			\item If you earn a \textbf{Nearly Complete} or \textbf{Unsatisfactory} mark on a an objective exercise (labeled Intro-, Group-, or Ring-) then you have not yet completed this objective.
			\item You may submit a new attempt at completing that objective on a future Wednesday. You must select a new exercise listed under the given objective, you cannot resubmit a version you have attempted previously.  The only limit you have on number of attempts is the number of exercises available for the objective.
			\item If you earn an \textbf{Exceptional} mark on an additional exercise (labeled A-) then you will earn one half point toward the ten total points in that section of your overall grade.
			\item If you earn an \textbf{Satisfactory} mark on an additional exercise (labeled Supp-) then you will earn 0.25 points toward the ten total points in that section of your overall grade. 
			\item If you earn a \textbf{Nearly Complete} or \textbf{Unsatisfactory} mark on an additional exercise (labeled Supp-) then you will earn 0 points toward the ten total points in that section of your overall grade. 
		\end{itemize}
	
	
\newpage
Name: \underline{\hspace*{3in}}
\vskip .25in

\textbf{(Intro-2.1)} Let $A\xrightarrow{\alpha}B\xrightarrow{\beta}C$ be mappings.  If $\beta\alpha$ is one-to-one and $\alpha$ is onto, show that $\beta$ is one-to-one.


\newpage
\textbf{(Intro-3.1)} Let $U=\{1,3,5\}$ and $A=U\times U$. Define $\equiv$ on $A$ by $(a,b)\equiv (c,d)$ if and only if $a+b=c+d$.
\begin{enumerate}
	\item How many elements are in $A$ and what are the?\vskip 1.5in 
	\item Prove that $\equiv$ is an equivalence relation on $A$.\vfill
	\item Describe the set of unique equivalence classes of $A$ given by $\equiv$.\vfill
\end{enumerate}


\newpage
\textbf{(Supp-3)} Show that $\alpha:\R^+\to\{x\in \R\ :\ x>1\}$ defined by $\alpha(x)=x^2+1$ is invertible and fully describe its inverse.  

To ``fully describe its inverse'', write a sentence like that boxed here, with the blanks filled in.
\begin{center}	
	\fbox{\begin{minipage}{4.5in}
		The inverse of $\alpha$ is the map $\alpha^{-1}:\underline{\hspace*{.5in}}\to\underline{\hspace*{.5in}}$ defined by $$\alpha^{-1}(x)=\underline{\hspace*{.5in}}.$$
	\end{minipage}}
\end{center}


\newpage
\textbf{(Supp-4)} For this problem we will denote the composition of the map $\alpha$ with itself by $\alpha^2$.  
\begin{enumerate}[label=(\alph*)]
	\item Verify that the map $\alpha:\Z\to \Z$ defined by $\alpha(n)=1$ for all $n\in\Z$ satisfies the property that $\alpha^2=\alpha$ by proving that $\alpha^2(n)=\alpha(n)$ for all $n\in \Z$.\vskip 1in
	\item Let $A$ be any set and $\alpha:A\to A$ a map. Show that
		\begin{center}
			$\alpha^2=\alpha$ \quad if and only if \quad $\alpha(x)=x$ for all $x\in \alpha(A)$.
		\end{center}  
		(Note that just as in part(a), $\alpha^2=\alpha$ means that $\alpha^2(x)=\alpha(x)$ for all $x\in A$.)
		\vfill
	\item Let $A$ be any set and $\alpha:A\to A$ a map that satisfies $\alpha^2=\alpha$. Show that
		\begin{center}
			 $\alpha$ is onto if and only if $\alpha$ is one-to-one.
		\end{center}  
		(Note that you should definitely be using part (b) here.) 
		\vfill
\end{enumerate}


\newpage
\textbf{(Supp-5)} Let $\alpha\colon A\to B$ be a mapping of sets. Define a relation on $A$ by $a_1\equiv a_2$ if and only if $\alpha(a_1)=\alpha(a_2)$. This is in fact an equivalence relation. Thus there is a set of equivalence classes we will denote by $A_\equiv$. 

Define $\sigma\colon A_\equiv \to B$ by $\sigma([a]) = \alpha(a)$.
\begin{enumerate}[label=(\alph*)]
	\item Show that $\sigma$ is well-defined. That is, show that if $[a_1]=[a_2]$, then $\sigma([a_1])=\sigma([a_2])$.\vfill
	\item Show that $\sigma$ is one-to-one.\vfill 
	\item Show that if $\alpha$ is onto then $\sigma$ is also onto.\vfill
\end{enumerate}



	

\end{document}