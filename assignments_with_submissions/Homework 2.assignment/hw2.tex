\documentclass[12pt]{article}
\usepackage{amsmath,amsthm,amssymb}
\usepackage[margin=.75in]{geometry}
\pagestyle{empty}



\newcommand{\R}{\mathbb{R}}
\newcommand{\Z}{\mathbb{Z}}

\setlength{\parindent}{0pt}
\begin{document}
	\begin{center}
		{\Large \bf Math 425 Assignment 2}
	\end{center}
	\section*{Purpose}
		\begin{itemize}
			\item Assess objectives and additional exercises from sections 0.4 and 1.1, namely
				\begin{itemize}
					\item Intro-2: Prove that a relation is an equivalence relation then compute the set of unique equivalence classes.
					\item Intro-3: Prove a property of non-negative integers using induction.
					\item Supplemental exercises Supp-4 and Supp-5. 
				\end{itemize}
			\item Build a strong foundation of creative critical thinking and proof-writing techniques.
		\end{itemize}
	\section*{Task}
		\begin{itemize}
			\item Complete the exercises listed on the following page(s) of this document and submit your solutions as a pdf to Canvas.
			\item I strongly recommend you use LaTeX to typeset your proofs.
			\item You may work in groups but everyone should submit their own assignment written in their own words.  Do NOT copy your classmates.
			\item Allowed resources: our textbook, classmates, your notes, videos linked in Canvas.
			\item Unacceptable resources: anything you find on an internet search. Do NOT use a homework help website (e.g., Chegg). Their solutions are often wrong or use incorrect context.  I want you to practice making arguments that are yours. Take some ownership.
		\end{itemize}
	\section*{Criteria}
		All items will earn a score using the following scale:
		\begin{itemize}
			\item \textbf{Exceptional} - Solution is succinct, references the correct theorems and definitions, and is entirely correct.
			\item \textbf{Satisfactory} - Solution is nearly correct. It still references the correct theorems and definitions. 
					It may be longer than necessary, have minor errors, or have some grammatical mistakes.
			\item \textbf{Unsatisfactory} - Solution has major errors, references content not covered in class or in the textbook, or is incomplete in some major way.
		\end{itemize}
		Recall from the syllabus
		\begin{itemize}
			\item If you earn either an \textbf{Exceptional} or \textbf{Satisfactory} mark on an objective exercise (labeled Intro-, Group-, or Ring-) then you may consider that item complete. 
			\item If you earn an \textbf{Unsatisfactory} mark on a an objective exercise (labeled Intro-, Group-, or Ring-) then you have not yet completed this objective.
			\item You may submit a new attempt at completing that objective on a future Wednesday. You must select a new exercise listed under the given objective, you cannot resubmit a version you have attempted previously.  The only limit you have on number of attempts is the number of exercises available for the objective.
			\item If you earn an \textbf{Exceptional} mark on an additional exercise (labeled A-) then you will earn one point toward the fifteen total points in that section of your overall grade. You can consider the exercise complete.
			\item If you earn an \textbf{Satisfactory} mark on an additional exercise (labeled Supp-) then you will earn 0.5 points toward the fifteen total points in that section of your overall grade. You may submit a new attempt at this exercise on a future Wednesday. Unlike the objective exercises, you must submit an attempt at the exact same exercise rather than another in a similar theme.
			\item If you earn an \textbf{Unsatisfactory} mark on an additional exercise (labeled Supp-) then you will earn 0 points toward the fifteen total points in that section of your overall grade. You may submit a new attempt at this exercise on a future Wednesday. Unlike the objective exercises, you must submit an attempt at the exact same exercise rather than another in a similar theme.
		\end{itemize}
	
	
\newpage
Name: \underline{\hspace*{3in}}
\vskip .25in

\textbf{(Intro-2.1)} Let $U=\{1,3,5\}$ and $A=U\times U$. Define $\equiv$ on $A$ by $(a,b)\equiv (c,d)$ if and only if $a+b=c+d$.
	\begin{enumerate}
		\item How many elements are in $A$ and what are the?\vskip 1.5in 
		\item Prove that $\equiv$ is an equivalence relation on $A$.\vfill
		\item Describe the set of unique equivalence classes of $A$ given by $\equiv$.\vfill
	\end{enumerate}

\newpage

Name: \underline{\hspace*{3in}}
\vskip .25in

\textbf{(Intro-3.1)} 
	\begin{enumerate}
	 \item Find the smallest positive integer $N$ such that $3^N<N!$ holds.
	 		\vskip 1in 
		\item Use induction to prove that for all integers $n\geq N$, the inequality $3^n<n!$ holds.
	\end{enumerate}

\newpage
Name: \underline{\hspace*{3in}}
\vskip .25in

\textbf{(Supp-4)} Let $\alpha\colon A\to B$ be a mapping of sets. Define a relation on $A$ by $a_1\equiv a_2$ if and only if $\alpha(a_1)=\alpha(a_2)$. This is in fact an equivalence relation. Thus there is a set of equivalence classes we will denote by $A_\equiv$. 

	Define $\sigma\colon A_\equiv \to B$ by $\sigma([a]) = \alpha(a)$.
	\begin{enumerate}
		\item Show that $\sigma$ is well-defined. That is, show that if $[a_1]=[a_2]$, then $\sigma([a_1])=\sigma([a_2])$.\vfill
		\item Show that $\sigma$ is one-to-one.\vfill 
		\item Show that if $\alpha$ is onto then $\sigma$ is also onto.\vfill
	\end{enumerate}

	
\newpage
Name: \underline{\hspace*{3in}}
\vskip .25in

\textbf{(Supp-5)} For all integers $n$, let $p_n$ denote the statement ``$4n-1$ is divisible by $4$''. 
	\begin{enumerate}
		\item Show that $p_k\Rightarrow p_{k+1}$ for all $k\geq 1$.\vfill
		\item Show that $p_n$ is in fact false for all $n\in\Z$.\vfill 
		\item Why do part (1) and part (2) not contradict one another?\vskip 1in
	\end{enumerate}


	

\end{document}