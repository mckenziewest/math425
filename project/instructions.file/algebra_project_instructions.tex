\documentclass[12pt]{article}
\usepackage{amsmath,amsthm,amssymb,graphicx,hyperref}
\usepackage[margin=.75in]{geometry}


\newcommand{\N}{\mathbb{N}}
\newcommand{\Q}{\mathbb{Q}}
\newcommand{\R}{\mathbb{R}}
\newcommand{\Z}{\mathbb{Z}}

\newcommand{\Aut}{\operatorname{Aut}}
\newcommand{\Inn}{\operatorname{Inn}}

\newcommand{\abar}{\overline{a}}
\newcommand{\bbar}{\overline{b}}


\begin{document}
	\begin{center}
		{\Large \bf Math 425 Final Project}
	\end{center}
	\section*{Purpose}
	\begin{itemize}
		\item This course satisfies the Student Learning Outcome ``Liberal Arts 5'':
			\begin{quote}
				Students will be able to read and synthesize mathematical literature.
			\end{quote}
		\item In this summative assessment you will be displaying your ability to complete this learning outcome as well as showcasing your ability to communicate mathematics effectively.
	\end{itemize}
	\section*{Task}
	\begin{itemize}
		\item Write or present one major artifact that describe your topic to the target audience in the described manner. The artifact must include necessary definitions, theorems, and examples. There must be at least one proof in the artifact. The possible artifacts include
			\begin{itemize}
				\item a 3-5 page paper in the style of a textbook section;
				\item a 10-15 minute video describing the topic to your classmates in this class.
			\end{itemize}
			If you would like to generate a result not on this list, please consult with me. I'm very open to other alternatives such as a poster presentation.
		\item Write three main questions you intend to answer during your research. After completing the project you will be asked to reflect on those questions. You will submit a short summary describing the questions, what part of your project gives the best answers for them, their ultimate relevance, and anything else you want to bring up. 
		\item Work in groups of approximately 3 students to peer review each others major artifacts.
	\end{itemize}
	\subsection*{Detailed Task Deadlines}
	\begin{itemize}
		\item Idea choice: \textbf{4/5} \\
		Submit your preferred topic choice(s), peer review group members, and artifact ideas to Canvas.
		\item Peer Review Agreement: \textbf{4/8}\\
		As a group, submit your contract of completion deadlines and peer review schedule to Canvas.
		\item Source and Question Check: \textbf{4/15} \\
		Submit a list of your preliminary sources and three questions you want to answer through your research.
		\item Artifact Draft: \textbf{4/29}\\
		Submit your first artifact to Canvas to be reviewed by your group.
		\item Peer Review: \textbf{5/3}\\
		Provide useful feedback for your assigned peers. Feedback must be submitted to Canvas.
		\item Artifact Final Submission: \textbf{5/10}\\
		Submit at least one of your artifacts to Canvas.
		\item Reflection on Original Questions: \textbf{5/10}
		\item Reflection on Your Use of Comments You Received: \textbf{5/10}
	\end{itemize}
	\section*{Criteria}
	The rubrics and specifics for each of the criteria below with point values can be found on Canvas.  There will be a separate assignment for each component of the project.
	\begin{itemize}
		\item Your success on the student learning outcome will be based on the criteria of Comprehension, Application, and Adaptation.
		
		\item Major artifacts: 80 pts 
		
			Each of the possible artifacts, as listed in the Task section, will have their own rubric. These rubrics will be posted to Canvas.
			
		\item Timeline Completion: 4$\times$4=16 pts
		
			Did you submit intermediate material on time and to the specifications? These items are:
				\begin{itemize}
					\item Idea Choice
					\item Peer Review Agreement
					\item Source and Question Check
					\item Artifact Draft
				\end{itemize}
			The exact descriptions of what each one should contain are also on Canvas.
			
		\item Peer Reviewing: 10 pts
		
			Your comments to your peers will be graded on their quality in terms of their usefulness to your group member in revising their material.
			
		\item Questions: 20 pts
			
			These points will be awarded based both on your answering of your three questions through your presentation/paper, and on your reflection about the questions. (Note: You will not be docked points if you find your questions unanswerable.  Be sure to mention such instances in your reflection.)
		
		\item Response to Comments: 20 pts
		
			Did you respond to the comments I or your group-mates made during the revision period, why or why not? A reflection on this task will be submitted to Canvas.
	\end{itemize}


\newpage

	\section*{Topic Options}
	For most topics, the textbook will be an excellent source that will serve as a starting place and major resource.  Some of the topics, denoted with a $\diamond$, do not have specicific reference in the textbook. 
	
	\begin{center}
		\fbox{\textbf{You are welcome to propose a topic not on this list.}}
	\end{center}
	\subsection*{Group Theory}
	\begin{itemize}
		\item Cauchy's Theorem (Section 8.2)\\
			
			This is one of the most wonderful theorems of finite group theory.  It states that if $p$ divides $|G|$, then $G$ has an element of order $p$. See how this helps with classification of finite groups!
		\item Group Actions and the Orbit Decomposition Theorem (Section 8.3)\\
		
			Group actions and the Orbit Decomposition Theorem allow us to break groups down to their fundamental components.
		\item Sylow's Theorems (Section 8.4)\\
		
			This theorem states that if $p^k$ is the largest power of $p$ that divides $|G|$, then $G$ has a subgroup of order $p^k$.  This is a powerful tool when describing the possible shapes of groups of order $n$.
		\item Semidirect Products (Section 8.5)\\
		
			Many groups can be written a the product $G\cong H\times K$ where $H,K\leq G$, but not all.  Semi-direct products allow a non-commutative relationship between $H$ and $K$.
		\item The Jordan--H\"older Theorem and composition series (section 9.1)\\
		
			This Theorem is the basis for classification of finite simple groups.
%		\item Solvable Groups (section 9.2)\\
%		
%			Can we use commutators of a group to get a normal series?
%		\item Burnside-Wielandt Theorem (section 9.3)\\
%			
%			Relating nilpotent groups and normal subgroups.
		\item An Application to Binary Linear Codes (section 2.11)
			
				There are several subtopics in this section. The idea being that we can use group multiplication to store data in a way that prevents errors, keeps backups, and doesn't take too much space.  
		\item[$\diamond$] Reed-Solomon Error Correcting Codes
		
			This topic is very related to the previous one but will require some work outside the textbook.  \href{https://en.wikipedia.org/wiki/Reed%E2%80%93Solomon_error_corrections}{Wikipedia}
		\item[$\diamond$] Cayley Graphs (\href{https://en.wikipedia.org/wiki/Cayley_graph}{Wikipedia})
			
			With the same namesake as Cayley tables, Cayley graphs use the generators of a group to encode its structure in a picture. This topic will include a lot of cool pictures and some learning about the possibilities of finite groups.
			
		\item[$\diamond$] Free Groups (\href{https://en.wikipedia.org/wiki/Free_group}{Wikipedia})

			Free groups are groups with little structure. Lovely pictures can be made out of them, forming a basis for geometric group theory.	
				
		\item[$\diamond$] Group Representations (\href{https://en.wikipedia.org/wiki/Group_representation}{Wikipedia})

			We can encode elements of groups as matrices!!
			
		\item[$\diamond$] Symmetry groups of 3d Objects
		
			In class, we did not go into the details of this, however we can describe the symmetries of a tetrahedron using the group $A_4$. There are a lot of of 3-d shapes that have really interesting symmetry groups, such as the dodecahedron and icosohedron. Professor Amethyst's research studies symmetries of an object called the Barth Sextic. Take a look at \url{https://www.ams.org/news?news_id=6961} to see a cool animation she generated using the symmetry group.
		
		\item[$\diamond$] The Rubik's Group
		
			The standard rotations of a Rubik's cube form a group and may help one solve a Rubik's cube.
		
		\item[$\diamond$] Public Key Cryptography
		
			All of our online communications use Public Key cryptography to keep the content secret.  Abstract algebra is required for this task!
			
		\item[$\diamond$] Regular Crystal Structures	
		
			Tools of abstract algebra can be used to model regular crystal structures in chemistry as well as the structures of viruses in biological physics.  The particular use is for the symmetric group of three dimensional objects.  I expect discussion of symmetry groups, and encourage you to take a look at the paper \url{https://www.ncbi.nlm.nih.gov/pmc/articles/PMC8983985/}
			
		\item[$\diamond$] Hexaflexagons
		
			Watch \url{https://www.youtube.com/watch?v=VIVIegSt81k} and \url{https://www.youtube.com/watch?v=paQ10POrZh8}. I bet there's some algebra here. (For sure.)
			
		\item[$\diamond$] Futurama Theorem
		
			In the popular TV show Futurama, the professor invents a mind swapping device.  The only problem is that if you've swapped minds with someone, you can't swap back with that same person.  How can all the minds be swapped back to their rightful owner?  This is a fun question of permutations and their representations as transpositions.
			
		
		\item[$\diamond$] Using Classes to Code Groups
		
			A group is a set and an operation.  In object-oriented languages like Python, Java, or C++, one can define classes with specific operations.  Consider coding a group with an a-typical operation or the group $S_n$ as a class as one of your major artifacts.
	\end{itemize}
	\subsection*{Ring Theory}
	\begin{itemize}
		\item Vector spaces (section 6.1)

			Here we can go further than we did in linear algebra dn talk about more general vector spaces over fields.
		\item Algebraic Extensions (section 6.2)

			If $K\subseteq L$ are fields we call $L$ algebraic over $K$ if every element of $L$ is a root of a polynomial with coefficients in $K$. Fields with these relationship have nice interesting behavior.
		\item Splitting Fields (section 6.3)

			Splitting fields of polynomials are minimal fields that contain roots of a polynomial equation.  Usually we see this with respect to $\Q$.  For example $\Q(i)$ is a splitting field of $x^2+1$.
		\item Finite Fields (section 6.4)

			It turns out that given some $q=p^n$ where $p$ is prime and $n$ a positive integer, there is exaclty one field (up to isomorphism) of that size.  If $m$ is not a prime power then there are no fields of that size.
		\item Geometric Constructions (section 6.5)

			Using only a compass and a straight-edge, we can find the midpoint of a line. (Rulers are not allowed, rulers are not precise.  Can we use a compass and straight edge to divide a line segment into three pieces?
		\item The Fundamental Theorem of Algebra (section 6.6)

			It's really true, every polynomial of degree $n$ has exactly $n$ roots (counting multiplicity).
		\item Modules (section 7.1)

			This is the generalization of a vector space over a field.
		\item[$\diamond$] Algebras (see\href{https://en.wikipedia.org/wiki/Algebra_over_a_field}{Wikipedia})

			This is the generalization/specification of a vector space to include a product.

		\item[$\diamond$] Group Rings (\href{https://en.wikipedia.org/wiki/Group_ring}{Wikipedia})

			If $G$ is a group and $R$ is a ring, an element of the group ring $R[G]$ is a \textit{finite formal sum} of the form 
$\sum_{i=1}^n a_i g_i$ where $a_i\in R$ and $g_i\in G$.  Note that $a_i g_i$ is a ``formal product'' meaning they don't actually interact with on another, they just go next to each other.  In some sense elements of $R[G]$ are polynomials where the variables are elements of $G$ and the coefficients are elements of $R$.  

		\item[$\diamond$] Zariski Topology (\href{https://mathworld.wolfram.com/ZariskiTopology.html}{MathWorld})

			We can use polynomials to define a topology on $n$-dimensional space.  It's pretty neat.
		\item [$\diamond$] Central Simple Algebras (\href{https://en.wikipedia.org/wiki/Central_simple_algebra}{Wikipedia})
		
			An algebra is essential a vector space that also has a multiplication operation.  It is central if the only things that commute multiplicatively with every element are the field, and simple if there are no proper two-sided ideals.  For example $M_n(\R)$ is a central simple $\R$-algebra with center $\left\{\begin{bmatrix}r&0\\0&r\end{bmatrix}\ :\ r\in \R\right\}\cong \R$.
	\end{itemize}
			
\end{document}