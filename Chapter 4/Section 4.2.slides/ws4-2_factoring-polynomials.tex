\documentclass[t]{beamer}

\subtitle{Section 4.2: Factorization of Polynomials over a Field}

\usepackage{amsthm,amsmath,amsfonts,hyperref,graphicx,color,multicol,soul}
\usepackage{enumitem,tikz,tikz-cd,setspace,mathtools}

%%%%%%%%%%
%Beamer Template Customization
%%%%%%%%%%
\setbeamertemplate{navigation symbols}{}
\setbeamertemplate{theorems}[ams style]
\setbeamertemplate{blocks}[rounded]

\definecolor{Blu}{RGB}{43,62,133} % UWEC Blue
\setbeamercolor{structure}{fg=Blu} % Titles

%Unnumbered footnotes:
\newcommand{\blfootnote}[1]{%
	\begingroup
	\renewcommand\thefootnote{}\footnote{#1}%
	\addtocounter{footnote}{-1}%
	\endgroup
}

%%%%%%%%%%
%TikZ Stuff
%%%%%%%%%%
\usetikzlibrary{arrows}
\usetikzlibrary{shapes.geometric}
\tikzset{
	smaller/.style={
		draw,
		regular polygon,
		regular polygon sides=3,
		fill=white,
		node distance=2cm,
		minimum height=1in,
		line width = 2pt
	}
}
\tikzset{
	smsquare/.style={
		draw,
		regular polygon,
		regular polygon sides=4,
		fill=white,
		node distance=2cm,
		minimum height=1in,
		line width = 2pt
	}
}


%%%%%%%%%%
%Custom Commands
%%%%%%%%%%

\newcommand{\C}{\mathbb{C}}
\newcommand{\quats}{\mathbb{H}}
\newcommand{\N}{\mathbb{N}}
\newcommand{\Q}{\mathbb{Q}}
\newcommand{\R}{\mathbb{R}}
\newcommand{\Z}{\mathbb{Z}}

\newcommand{\ds}{\displaystyle}

\newcommand{\fn}{\insertframenumber}

\newcommand{\id}{\operatorname{id}}
\newcommand{\im}{\operatorname{im}}
\newcommand{\Aut}{\operatorname{Aut}}
\newcommand{\Inn}{\operatorname{Inn}}

\newcommand{\blank}[1]{\underline{\hspace*{#1}}}

\newcommand{\abar}{\overline{a}}
\newcommand{\bbar}{\overline{b}}
\newcommand{\cbar}{\overline{c}}

\newcommand{\nml}{\unlhd}

%%%%%%%%%%
%Custom Theorem Environments
%%%%%%%%%%
\theoremstyle{definition}
\newtheorem{exercise}{Exercise}
\newtheorem{question}[exercise]{Question}
\newtheorem{warmup}{Warm-Up}
\newtheorem*{defn}{Definition}
\newtheorem*{exa}{Example}
\newtheorem*{disc}{Group Discussion}
\newtheorem*{nb}{Note}
\newtheorem*{recall}{Recall}
\renewcommand{\emph}[1]{{\color{blue}\texttt{#1}}}

\definecolor{Gold}{RGB}{237, 172, 26}
%Statement block
\newenvironment{statementblock}[1]{%
	\setbeamercolor{block body}{bg=Gold!20}
	\setbeamercolor{block title}{bg=Gold}
	\begin{block}{\textbf{#1.}}}{\end{block}}
\newenvironment{thm}[1]{%
	\setbeamercolor{block body}{bg=Gold!20}
	\setbeamercolor{block title}{bg=Gold}
	\begin{block}{\textbf{Theorem #1.}}}{\end{block}}


%%%%%%%%%%
%Custom Environment Wrappers
%%%%%%%%%%
\newcommand{\enumarabic}[1]{
	\begin{enumerate}[label=\textbf{\arabic*.}]
		#1
	\end{enumerate}
}
\newcommand{\enumalph}[1]{
	\begin{enumerate}[label=(\alph*)]
		#1
	\end{enumerate}
}
\newcommand{\bulletize}[1]{
	\begin{itemize}[label=$\bullet$]
		#1
	\end{itemize}
}
\newcommand{\circtize}[1]{
	\begin{itemize}[label=$\circ$]
		#1
	\end{itemize}
}
\newcommand{\slide}[1]{
	\begin{frame}{\fn}
		#1
	\end{frame}
}
\newcommand{\slidec}[1]{
\begin{frame}[c]{\fn}
	#1
\end{frame}
}
\newcommand{\slidet}[2]{
	\begin{frame}{\fn\ - #1}
		#2
	\end{frame}
}


\newcommand{\startdoc}{
		\title{Math 425: Abstract Algebra 1}
		\author{Mckenzie West}
		\date{Last Updated: \today}
		\begin{frame}
			\maketitle
		\end{frame}
}

\newcommand{\topics}[2]{
	\begin{frame}{\insertframenumber}
		\begin{block}{\textbf{Last Section.}}
			\begin{itemize}[label=--]
				#1
			\end{itemize}
		\end{block}
		\begin{block}{\textbf{This Section.}}
			\begin{itemize}[label=--]
				#2
			\end{itemize}
		\end{block}
	\end{frame}
}

\begin{document} 
	\startdoc
	
	\topics{
		\item Polynomial Rings
		\item The Division Algorithm
		\item The Factor Theorem
		\item The Remainder Theorem
	}
	{
		\item Factoring degree 2 and 3 polynomials
		\item Unique Factorization
		\item Factoring in $\C[x]$, $\R[x]$, $\Q[x]$, $\Z[x]$
	}

\slide{	
	\begin{defn}
		Let $F$ be a field and $p\neq 0$ in $F[x]$ a polynomial.  We call $p$ \emph{irreducible over $F$} if $\deg(p)\geq 1$ and
		\[\text{If }p=fg\text{ for }f,g\in F[x]\text{, then either}\deg f=0\text{ or }\deg g=0.\]
		
		Otherwise we call $p$ \emph{reducible}.
	\end{defn}
%	\begin{exa}
%		\enumalph{	\item $ax+b$ is irreducible in $F[x]$ as long as $a\neq 0$
%			\item $x^2+1$ is irreducible over $\R$ but reducible over $\C$}
%	\end{exa}
}
\slide{
	\begin{thm}{4.2.1}
		Let $F$ be a field and consider $p$ in $F[x]$ where $\deg p\geq 2$.
		\enumarabic{\item If $p$ is irreducible, then $p$ has no root in $F$.
		\item If $\deg p$ is 2 or 3, then $p$ is irreducible if and only if it has no root in $F$.}
	\end{thm}
%	\begin{proof}
%		Let $F$ be a field and $p\in F[x]$ with $\deg p\geq 2.$
%		
%		(1) We prove using the contrapositive.  Let $a\in F$ be a root of $p$.\vskip .5in
%		
%		(2) ($\Rightarrow$) True by (1).  To prove ($\Leftarrow$) we again use contrapositive.  Assume $\deg p$ is 2 or 3 and that $p$ is reducible.\vskip 1in\mbox{}
%	\end{proof}
}

\slide{
	\begin{exa}
		\enumalph{\setlength{\itemsep}{.5in}
			\item $x^2+1$ is irreducible in $\R[x]$ because it has no root in $\R$
			\item $x^2-2$ is irreducible over $\Q$
			\item $p=x^3+3x^2+x+2$ is irreducible over $\Z_5$ because $p(a)\neq 0$ for all $a\in\Z_5$.	
		}
\end{exa}}

\slide{
	\begin{statementblock}{Unique Factorization Theorem (4.2.12)}
		Let $F$ be a field, and $f$ be a nonconstant polynomial in $F[x]$.  Then
		\enumarabic{\item $f=ap_1p_2\cdots p_m$, where $a\in F$ and $p_1,p_2,\dots,p_m$ are monic irreducible polynomials in $F[x]$.\item The factorization is unique up to the order of the factors.}
	\end{statementblock}
	\begin{nb}
		The proof for (1) is a pretty straight-forward induction proof.
		
		The proof for (2) uses the fact that if
			\[p| q_1q_2\cdots q_n,\]
		where $p, q_1,q_2,\dots,q_n$ are irreducible, then $p|q_i$ for some $i$.
	\end{nb}
	\begin{block}{\textbf{Remark.}}
		If $F$ is a field, we call $F[x]$ a \emph{unique factorization domain} because it is a domain and the elements factor uniquely.
	\end{block}
}

\slidet{Factorization over $\C$}{
	\begin{statementblock}{Fundamental Theorem of Algebra (Theorem 4.2.2)}
		If $f\in \C[x]$ with $\deg f>0$, then $f$ has at least one root in $\C$.
	\end{statementblock}
	\begin{thm}{4.2.3}
		\enumarabic{
			\item If $\deg f=n\geq 1$, $f\in \C[x]$, then $f$ factors completely as
				\[f=u(x-a_1)(x-a_2)\cdots(x-a_n),\]
				for $u\neq 0$, $a_1,a_2,\dots,a_n\in \C$.
			\item The only irreducible polynomials in $\C[x]$ are linear.
		}
	\end{thm}
}
\slidet{Factorization over $\C$}{
	\begin{nb}
		Complex roots of real polynomials come in pairs, $z=a+bi$ and $\overline{z}=a-bi$.
		This is because complex conjugation is a ring homomorphism, so \[f(z)=0=\overline{0}=\overline{f(z)}=f(\overline{z}).\]
	\end{nb}
}
\slidet{Factorization over $\R$}{
	\begin{thm}{4.2.4}
		Every nonconstant polynomial $f\in \R[x]$ factors as
			\[f=u(x-r_1)(x-r_2)\cdots(x-r_m)q_1q_2\cdots q_k,\]
		where $r_1,r_2,\dots,r_m$ are the real roots of $f$ and $q_1,q_2,\dots,q_k$ are monic irreducible quadratics in $\R[x]$.
	\end{thm}
	\begin{statementblock}{Corollary}
		The irreducible polynomials in $\R[x]$ are either linear or quadratic.
	\end{statementblock}
}

\slidet{Factoring over $\Q$}{
	\begin{block}{\textbf{Reduction mod $p$}}
		Using the mod $p$ map, $\Z\to\Z_p$, we induce a map
		from $\Z[x]$ to $\Z_p[x]$ given by
		\[f=a_0+a_1x+a_2x^2+\cdots+ a_nx^n\mapsto \bar f =\bar a_0+\bar a_1x+\bar a_2x^2+\cdots +\bar a_nx^n.\]
		We call $\bar f$ the \emph{reduction} of $f$ modulo $p$.  This map is in fact an onto ring homomorphism.
	\end{block}
}
\slidet{Factoring over $\Q$}{
	\begin{statementblock}{Gauss' Lemma (Theorem 4.2.5)}
		Let $f=gh$ in $\Z[x]$.  If a prime $p\in\Z$ divides every coefficient of $f$, then $p$ divides every coefficient of $g$ or $p$ divides every coefficient of $h$.
	\end{statementblock}
%	\begin{proof}[Proof Outline.]
%		\bulletize{\item $f = gh\ \Rightarrow\ \bar f=\bar g\bar h$
%		\item $\bar f=0\Rightarrow \bar g\bar h=0$
%		\item $\Z_p$ a field $\Rightarrow$ $\Z_p[x]$ an integral domain (Theorem 4.1.2)
%		\item $\bar g\bar h=0\ \Rightarrow\ \bar g=0$ or $\bar h=0$
%		\item\begin{tabular}{rcl}
%				$\bar g = 0$&$\Rightarrow$& every coefficient of $g$ is $0\pmod p$
%			\end{tabular}
%		\item[]\begin{tabular}{rcl}
%			{\color{white}$\bar g = 0$}&$\Rightarrow$& every coefficient of $g$ is divisible by $p$
%		\end{tabular}
%		}
%	\end{proof}
}
\slidet{Factoring over $\Q$}{
	\begin{thm}{4.2.6}
		Let $f\in \Z[x]$ be a non-constant polynomial.
	\enumarabic{
		\item If $f=gh$ with $g,h\in\Q[x]$, then $f=g_0h_0$ where $g_0,h_0\in\Z[x]$, $\deg g=\deg g_0$, and $\deg h=\deg h_0$.
		\item $f$ is irreducible in $\Q[x]$ if and only if $f=ag$ where $a\in\Z$ are the only factorizations of $f$ in $\Z[x]$.
	}
	\end{thm}
%	\begin{proof}[Proof Outline.]
%		\bulletize{
%			\item Let $a,b\in\Z$ common denominators of coeffs of $g,h$ respectively.
%			\item Thus $ag,bh\in\Z[x]$ and $abf = (ag)(bh)$.
%			\item $p\mid ab\Rightarrow \overline{abf}=0\Rightarrow \overline{ag}=0$ or $\overline{bh}=0$
%			\item Thus $\frac{ag}{p}\in\Z[x]$ or $\frac{bh}{p}\in\Z[x]$.
%			\item Repeat for all primes dividing $ab$.
%		}
%	\end{proof}
}
\slidet{Factoring Over $\Q$}{
	\begin{exercise}
		Consider \begin{multline*}
		4x^8 + 2x^7 - 4x^6 - 5x^5 - 6x^4 - 7x^3 - 3x^2 - x - 1=\\
		\left(\frac{20}{3} x^{3} + \frac{10}{3} x^{2} + \frac{5}{3}\right)\left(\frac{3}{5} x^{5} - \frac{3}{5} x^{3} - \frac{3}{5} x^{2} - \frac{3}{5} x - \frac{3}{5}\right).
		\end{multline*}  Write this polynomial as a product of polynomials in $\Z[x]$.\vskip 3in\mbox{}
	\end{exercise}
}
\slidet{Factoring Over $\Q$}{
	\begin{statementblock}{Modular Irreducibility (Theorem 4.2.7)}
		Let $0\neq f\in\Z[x]$ and suppose that a prime $p$ exists such that
		\enumarabic{
			\item $p$ does not divide the leading coefficient of $f$.
			\item The reduction, $\bar f$ of $f$ modulo $p$ is irreducible in $\Z_p[x]$.
		}
		Then $f$ is irreducible over $\Q$.
	\end{statementblock}
}
\slidet{Factoring Over $\Q$}{
	\begin{exercise}
		Show that $f=32x^3-51x^2-2x+25$ is irreducible over $\Q$.\\  (Hint: Check mod 3.)\vskip 3in\mbox{}
	\end{exercise}
}
\slidet{Factoring Over $\Q$}{
	\begin{statementblock}{Eisenstein's Criterion (Theorem 4.2.8)}
		Consider $f=a_0+a_1x+a_2x^2+\cdots+a_nx^n$ in $\Z[x]$, where $n\geq 1$ and $a_0\neq 0$.  Let $p\in\Z$ be a prime number satisfying
		\enumarabic{
			\item $p$ divides each of $a_0,a_1,a_2,\dots,a_{n-1}$.
			\item $p$ does not divide $a_n$.
			\item $p^2$ does not divide $a_0$.
		}
		Then $f$ is irreducible in $\Q[x]$.
	\end{statementblock}
}
\slidet{Factoring Over $\Q$}{
	\begin{exercise}
		Show that $x^5-3x^2+6x-12$ is irreducible in $\Q[x]$.\vskip 1in\mbox{}
	\end{exercise}
	\begin{exercise}
		Show that $f=x^n-2$ is irreducible in $\Q[x]$ for all $n$.\vskip 3in\mbox{}
	\end{exercise}
}
\slidet{Factoring Over $\Q$}{
	\begin{block}{\textbf{So What's the Point?}}
		If $f\in \Q[x]$ and we want to find the roots, we can think of $f_1\in\Z[x]$.\vskip 1in\mbox{}
		Polynomials in $\Z[x]$ are ``easier'' than those in $\Q[x]$.\vskip 1in\mbox{}
		Polynomials in $\Z_p[x]$ are way easier than those in $\Q[x]$!!
	\end{block}
}
\end{document}

		