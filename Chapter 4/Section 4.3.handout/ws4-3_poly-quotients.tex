\documentclass[12pt]{article}

\newcommand{\secname}{Section 4.3: Quotient Rings of Polynomials}

\usepackage{amsthm,amsmath,amsfonts,hyperref,graphicx,color,multicol,soul}
\usepackage{enumitem,tikz,tikz-cd,setspace,mathtools}
\usepackage{colortbl}
\usepackage[margin=1in]{geometry}

%%%%%%%%%%
%Color Customization
%%%%%%%%%%

\definecolor{Blu}{RGB}{43,62,133} % UWEC Blue

%Unnumbered footnotes:
\newcommand{\blfootnote}[1]{%
	\begingroup
	\renewcommand\thefootnote{}\footnote{#1}%
	\addtocounter{footnote}{-1}%
	\endgroup
}

%%%%%%%%%%
%TikZ Stuff
%%%%%%%%%%
\usetikzlibrary{arrows}
\usetikzlibrary{shapes.geometric}
\tikzset{
	smaller/.style={
		draw,
		regular polygon,
		regular polygon sides=3,
		fill=white,
		node distance=2cm,
		minimum height=1in,
		line width = 2pt
	}
}
\tikzset{
	smsquare/.style={
		draw,
		regular polygon,
		regular polygon sides=4,
		fill=white,
		node distance=2cm,
		minimum height=1in,
		line width = 2pt
	}
}

%%%%%%%%%%
%Listing Setup
%%%%%%%%%%
\usepackage{listings}
\usepackage{caption, floatrow, makecell}%
\captionsetup{labelfont = sc}
\setcellgapes{3pt}

\definecolor{backcolour}{RGB}{237,236,230}
\definecolor{myblue}{RGB}{42,157,189}

\lstdefinestyle{mystyle}{
	language=Python,
	keywords=[2]{sage:},
	alsodigit={:,.,<,>},
	backgroundcolor=\color{backcolour},   
	commentstyle=\color{myblue},
	keywordstyle=\bfseries\color{Green},
	keywordstyle=[2]\color{purple},
	numberstyle=\tiny\color{Gray},
	stringstyle=\color{Orange},
	basicstyle=\ttfamily\footnotesize,
	breakatwhitespace=false,         
	breaklines=true,                 
	captionpos=b,                    
	keepspaces=true,                   
	showspaces=false,                
	showstringspaces=false,
	showtabs=false,                  
	tabsize=2
}

\lstset{style=mystyle}


%%%%%%%%%%
%Custom Commands
%%%%%%%%%%

\newcommand{\C}{\mathbb{C}}
\newcommand{\quats}{\mathbb{H}}
\newcommand{\N}{\mathbb{N}}
\newcommand{\Q}{\mathbb{Q}}
\newcommand{\R}{\mathbb{R}}
\newcommand{\Z}{\mathbb{Z}}

\newcommand{\ds}{\displaystyle}

\newcommand{\fn}{\insertframenumber}

\newcommand{\id}{\operatorname{id}}
\newcommand{\im}{\operatorname{im}}
\newcommand{\lcm}{\operatorname{lcm}}
\newcommand{\ord}{\operatorname{ord}}
\newcommand{\Aut}{\operatorname{Aut}}
\newcommand{\Inn}{\operatorname{Inn}}

\newcommand{\blank}[1]{\underline{\hspace*{#1}}}

\newcommand{\abar}{\overline{a}}
\newcommand{\bbar}{\overline{b}}
\newcommand{\cbar}{\overline{c}}

\newcommand{\nml}{\unlhd}

%%%%%%%%%%
%Custom Theorem Environments
%%%%%%%%%%
\theoremstyle{definition}
\newtheorem{exercise}{Exercise}
\newtheorem{question}[exercise]{Question}
\newtheorem{warmup}{Warm-Up}
\newtheorem*{exa}{Example}
\newtheorem*{defn}{Definition}
\newtheorem*{disc}{Group Discussion}
\newtheorem*{recall}{Recall}
\renewcommand{\emph}[1]{{\color{blue}\texttt{#1}}}

\definecolor{Gold}{RGB}{237, 172, 26}
%Statement block
%\newenvironment{statementblock}[1]{%
%	\setbeamercolor{block body}{bg=Gold!20}
%	\setbeamercolor{block title}{bg=Gold}
%	\begin{block}{\textbf{#1.}}}{\end{block}}
%\newenvironment{goldblock}{%
%	\setbeamercolor{block body}{bg=Gold!20}
%	\setbeamercolor{block title}{bg=Gold}
%	\setbeamertemplate{blocks}[shadow=true]
%	\begin{block}{}}{\end{block}}
%\newenvironment{defn}{%
%	\setbeamercolor{block body}{bg=gray!20}
%	\setbeamercolor{block title}{bg=violet, fg=white}
%	\setbeamertemplate{blocks}[shadow=true]
%	\begin{block}{\textbf{Definition.}}}{\end{block}}
%\newenvironment{nb}{%
%	\setbeamercolor{block body}{bg=gray!20}
%	\setbeamercolor{block title}{bg=teal, fg=white}
%	\setbeamertemplate{blocks}[shadow=true]
%	\begin{block}{\textbf{Note.}}}{\end{block}}
%\newenvironment{blockexample}{%
%	\setbeamercolor{block body}{bg=gray!20}
%	\setbeamercolor{block title}{bg=Blu, fg=white}
%	\setbeamertemplate{blocks}[shadow=true]
%	\begin{block}{\textbf{Example.}}}{\end{block}}
%\newenvironment{blocknonexample}{%
%	\setbeamercolor{block body}{bg=gray!20}
%	\setbeamercolor{block title}{bg=purple, fg=white}
%	\setbeamertemplate{blocks}[shadow=true]
%	\begin{block}{\textbf{Non-Example.}}}{\end{block}}
%\newenvironment{thm}[1]{%
%	\setbeamercolor{block body}{bg=Gold!20}
%	\setbeamercolor{block title}{bg=Gold}
%	\begin{block}{\textbf{Theorem #1.}}}{\end{block}}


%%%%%%%%%%
%Custom Environment Wrappers
%%%%%%%%%%
\newcommand{\exer}[1]{
	\begin{exercise}
	#1
	\end{exercise}
}
\newcommand{\exam}[1]{
\textbf{Example: }
	#1
}
\newcommand{\nexam}[1]{
	\textbf{Non-Example: }
	#1
}
\newcommand{\enumarabic}[1]{
	\begin{enumerate}[label=\textbf{\arabic*.}]
		#1
	\end{enumerate}
}
\newcommand{\enumalph}[1]{
	\begin{enumerate}[label=(\alph*)]
		#1
	\end{enumerate}
}
\newcommand{\bulletize}[1]{
	\begin{itemize}[label=$\bullet$]
		#1
	\end{itemize}
}
\newcommand{\circtize}[1]{
	\begin{itemize}[label=$\circ$]
		#1
	\end{itemize}
}
%\newcommand{\slide}[1]{
%	\begin{frame}{\fn}
%		#1
%	\end{frame}
%}
%\newcommand{\slidec}[1]{
%\begin{frame}[c]{\fn}
%	#1
%\end{frame}
%}
%\newcommand{\slidet}[2]{
%	\begin{frame}{\fn\ - #1}
%		#2
%	\end{frame}
%}


\setlength{\parindent}{0pt}



\usepackage{afterpage}
\usepackage{fancyhdr}

\fancyhead[L]{\textbf{Math 425: Abstract Algebra I\\\secname}}
\fancyhead[R]{\textbf{Mckenzie West\\Last Updated: \today}}
\pagestyle{fancy}

\newcommand{\startdoc}{}

\newcommand{\topics}[2]{
		{\textbf{Previously.}}
			\begin{itemize}[label=--]
				#1
			\end{itemize}
		{\textbf{This Section.}}
			\begin{itemize}[label=--]
				#2
			\end{itemize}
}

\begin{document} 
	\startdoc
	
	\topics{
		\item Factoring degree 2 and 3 polynomials
		\item Unique Factorization
		\item Factoring in $\C[x]$, $\R[x]$, $\Q[x]$, $\Z[x]$
	}
	{
		\item Algebraic structure of quotients of polynomial rings
	}

\slide{
	\begin{exercise}
		Describe the quotient ring $\Q[x]/A$ where $A=(x^2-2)$ is the ideal containing all multiples of $h=x^2-2$.
	\end{exercise}
}

\vskip 2in

\slide{
	\begin{thm}{4.3.2}
		Let $h$ be a monic polynomial of degree $m\geq 1$ in $F[x]$, there $F$ is a field.  Then
		$$F[x]/(h)\cong\{a_0+a_1 t+a_2 t^2+\cdots + a_{m-1}t^{m-1}\ | a_i\in F, h(t)=0\}.$$
		Moreover, this representation is unique.  That is,
			\[a_0+a_1t+a_2t^2+\cdots+a_{m-1}t^{m-1}=b_0+b_1t+b_2t^2+\cdots+b_{m-1}t^{m-1}\]
		if and only if $a_i=b_i$ for all $i$.
	\end{thm}
}
\slide{
	\begin{exercise}
		Consider $R=\Z_2[x]/(x^2+x+1)$.  Complete the addition and multiplication tables for $R$: (use $t=\bar{x}\in R$)
		
		$$\begin{array}{c||c|c|c|c}
			+&0&1&t&1+t\\\hline\hline
			0&{\color{white}1+t}&{\color{white}1+t}&{\color{white}1+t}&\\\hline
			1&&&&\\\hline
			t&&&&\\\hline
			1+t&&&&
		\end{array}
		$$
		\vskip .5in
		$$\begin{array}{c||c|c|c|c}
			\cdot&0&1&t&1+t\\\hline\hline
			0&{\color{white}1+t}&{\color{white}1+t}&{\color{white}1+t}&\\\hline
			1&&&&\\\hline
			t&&&&\\\hline
			1+t&&&&
		\end{array}
		$$
	\end{exercise}
}
\newpage
\slide{
\begin{exercise}
	Describe the quotient $\Z_2[x]/(x^2)$. That is write an addition and multiplication table as in exercise 2.
	\vskip 2in
\end{exercise}
}
\slide{
\begin{exercise}
	Describe the quotient
	$\Z_2[x]/(x^2-1)$. That is write an addition and multiplication table as in exercise 2.
	\vskip 2in
\end{exercise}
}
\slide{
\begin{exercise}
	Describe the ring $R=\Z_2[x]/(x^3+1)$. That is write an addition and multiplication table as in exercise 2.
	\vskip 2in
\end{exercise}
}
\slide{
\begin{exercise}
	Describe the ring $R=\R[x]/(x^2+1)$. That is write an addition and multiplication table as in exercise 2.
	\vskip 2in
\end{exercise}
}

\slide{
	\begin{exercise}
		Let $R=\Z_3[x]$ and $A=\{f\in R\ |\ f(1)=0\}$.
		\enumarabic{
			\item Use the definition of ideal to show $A$ is an ideal.
			\vskip .75in
			\item Find the monic polynomial $h\in R$ such that $A=(h)$.
			\vskip .75in
			\item Describe $R/A$.
			\vskip .75in
		}
	\end{exercise}
}

\slide{
\begin{thm}{4.3.3}
	Let $h$ be a monic polynomial of degree $m\geq 1$ in $F[x]$, where $F$ is a field.  Then $F[x]/(h)$ is a field if and only if $h$ is irreducible. 
\end{thm}
}

\slide{
	\begin{exercise}
		Find a field of order $9$.\vfill
	\end{exercise}
}
\slide{
	\begin{exercise}
		Consider $F=\Q$ and $h=x^2+7x+3$.  Describe $F[x]/(h)$.\vfill
	\end{exercise}
}
\slide{
	\begin{thm}{4.3.4 (Kronecker's Theorem)}
		Let $F$ be a field and $h\in F[x]$ an irreducible polynomial.  Then there is some field $K$ containing $F$ that has a root of $h$.
	\end{thm}
}
\slide{
	\begin{exercise}
		Consider $F=\Q$ and $h=x^2+bx+c$ with $b^2-4c$ not a square.  Describe $F[x]/(h)$.\vfill
	\end{exercise}
}
\slide{
	\begin{exercise}
		Consider $F=\Q$ and $h=x^3+2$.  Describe $F[x]/(h)$.\vfill
	\end{exercise}
}
\end{document}

		