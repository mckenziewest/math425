\documentclass[12pt]{article}

\newcommand{\secname}{Section 4.1: Polynomials}

\usepackage{amsthm,amsmath,amsfonts,hyperref,graphicx,color,multicol,soul}
\usepackage{enumitem,tikz,tikz-cd,setspace,mathtools}
\usepackage{colortbl}
\usepackage[margin=1in]{geometry}

%%%%%%%%%%
%Color Customization
%%%%%%%%%%

\definecolor{Blu}{RGB}{43,62,133} % UWEC Blue

%Unnumbered footnotes:
\newcommand{\blfootnote}[1]{%
	\begingroup
	\renewcommand\thefootnote{}\footnote{#1}%
	\addtocounter{footnote}{-1}%
	\endgroup
}

%%%%%%%%%%
%TikZ Stuff
%%%%%%%%%%
\usetikzlibrary{arrows}
\usetikzlibrary{shapes.geometric}
\tikzset{
	smaller/.style={
		draw,
		regular polygon,
		regular polygon sides=3,
		fill=white,
		node distance=2cm,
		minimum height=1in,
		line width = 2pt
	}
}
\tikzset{
	smsquare/.style={
		draw,
		regular polygon,
		regular polygon sides=4,
		fill=white,
		node distance=2cm,
		minimum height=1in,
		line width = 2pt
	}
}

%%%%%%%%%%
%Listing Setup
%%%%%%%%%%
\usepackage{listings}
\usepackage{caption, floatrow, makecell}%
\captionsetup{labelfont = sc}
\setcellgapes{3pt}

\definecolor{backcolour}{RGB}{237,236,230}
\definecolor{myblue}{RGB}{42,157,189}

\lstdefinestyle{mystyle}{
	language=Python,
	keywords=[2]{sage:},
	alsodigit={:,.,<,>},
	backgroundcolor=\color{backcolour},   
	commentstyle=\color{myblue},
	keywordstyle=\bfseries\color{Green},
	keywordstyle=[2]\color{purple},
	numberstyle=\tiny\color{Gray},
	stringstyle=\color{Orange},
	basicstyle=\ttfamily\footnotesize,
	breakatwhitespace=false,         
	breaklines=true,                 
	captionpos=b,                    
	keepspaces=true,                   
	showspaces=false,                
	showstringspaces=false,
	showtabs=false,                  
	tabsize=2
}

\lstset{style=mystyle}


%%%%%%%%%%
%Custom Commands
%%%%%%%%%%

\newcommand{\C}{\mathbb{C}}
\newcommand{\quats}{\mathbb{H}}
\newcommand{\N}{\mathbb{N}}
\newcommand{\Q}{\mathbb{Q}}
\newcommand{\R}{\mathbb{R}}
\newcommand{\Z}{\mathbb{Z}}

\newcommand{\ds}{\displaystyle}

\newcommand{\fn}{\insertframenumber}

\newcommand{\id}{\operatorname{id}}
\newcommand{\im}{\operatorname{im}}
\newcommand{\lcm}{\operatorname{lcm}}
\newcommand{\ord}{\operatorname{ord}}
\newcommand{\Aut}{\operatorname{Aut}}
\newcommand{\Inn}{\operatorname{Inn}}

\newcommand{\blank}[1]{\underline{\hspace*{#1}}}

\newcommand{\abar}{\overline{a}}
\newcommand{\bbar}{\overline{b}}
\newcommand{\cbar}{\overline{c}}

\newcommand{\nml}{\unlhd}

%%%%%%%%%%
%Custom Theorem Environments
%%%%%%%%%%
\theoremstyle{definition}
\newtheorem{exercise}{Exercise}
\newtheorem{question}[exercise]{Question}
\newtheorem{warmup}{Warm-Up}
\newtheorem*{exa}{Example}
\newtheorem*{defn}{Definition}
\newtheorem*{disc}{Group Discussion}
\newtheorem*{recall}{Recall}
\renewcommand{\emph}[1]{{\color{blue}\texttt{#1}}}

\definecolor{Gold}{RGB}{237, 172, 26}
%Statement block
%\newenvironment{statementblock}[1]{%
%	\setbeamercolor{block body}{bg=Gold!20}
%	\setbeamercolor{block title}{bg=Gold}
%	\begin{block}{\textbf{#1.}}}{\end{block}}
%\newenvironment{goldblock}{%
%	\setbeamercolor{block body}{bg=Gold!20}
%	\setbeamercolor{block title}{bg=Gold}
%	\setbeamertemplate{blocks}[shadow=true]
%	\begin{block}{}}{\end{block}}
%\newenvironment{defn}{%
%	\setbeamercolor{block body}{bg=gray!20}
%	\setbeamercolor{block title}{bg=violet, fg=white}
%	\setbeamertemplate{blocks}[shadow=true]
%	\begin{block}{\textbf{Definition.}}}{\end{block}}
%\newenvironment{nb}{%
%	\setbeamercolor{block body}{bg=gray!20}
%	\setbeamercolor{block title}{bg=teal, fg=white}
%	\setbeamertemplate{blocks}[shadow=true]
%	\begin{block}{\textbf{Note.}}}{\end{block}}
%\newenvironment{blockexample}{%
%	\setbeamercolor{block body}{bg=gray!20}
%	\setbeamercolor{block title}{bg=Blu, fg=white}
%	\setbeamertemplate{blocks}[shadow=true]
%	\begin{block}{\textbf{Example.}}}{\end{block}}
%\newenvironment{blocknonexample}{%
%	\setbeamercolor{block body}{bg=gray!20}
%	\setbeamercolor{block title}{bg=purple, fg=white}
%	\setbeamertemplate{blocks}[shadow=true]
%	\begin{block}{\textbf{Non-Example.}}}{\end{block}}
%\newenvironment{thm}[1]{%
%	\setbeamercolor{block body}{bg=Gold!20}
%	\setbeamercolor{block title}{bg=Gold}
%	\begin{block}{\textbf{Theorem #1.}}}{\end{block}}


%%%%%%%%%%
%Custom Environment Wrappers
%%%%%%%%%%
\newcommand{\exer}[1]{
	\begin{exercise}
	#1
	\end{exercise}
}
\newcommand{\exam}[1]{
\textbf{Example: }
	#1
}
\newcommand{\nexam}[1]{
	\textbf{Non-Example: }
	#1
}
\newcommand{\enumarabic}[1]{
	\begin{enumerate}[label=\textbf{\arabic*.}]
		#1
	\end{enumerate}
}
\newcommand{\enumalph}[1]{
	\begin{enumerate}[label=(\alph*)]
		#1
	\end{enumerate}
}
\newcommand{\bulletize}[1]{
	\begin{itemize}[label=$\bullet$]
		#1
	\end{itemize}
}
\newcommand{\circtize}[1]{
	\begin{itemize}[label=$\circ$]
		#1
	\end{itemize}
}
%\newcommand{\slide}[1]{
%	\begin{frame}{\fn}
%		#1
%	\end{frame}
%}
%\newcommand{\slidec}[1]{
%\begin{frame}[c]{\fn}
%	#1
%\end{frame}
%}
%\newcommand{\slidet}[2]{
%	\begin{frame}{\fn\ - #1}
%		#2
%	\end{frame}
%}


\setlength{\parindent}{0pt}



\usepackage{afterpage}
\usepackage{fancyhdr}

\fancyhead[L]{\textbf{Math 425: Abstract Algebra I\\\secname}}
\fancyhead[R]{\textbf{Mckenzie West\\Last Updated: \today}}
\pagestyle{fancy}

\newcommand{\startdoc}{}

\newcommand{\topics}[2]{
		{\textbf{Previously.}}
			\begin{itemize}[label=--]
				#1
			\end{itemize}
		{\textbf{This Section.}}
			\begin{itemize}[label=--]
				#2
			\end{itemize}
}

\begin{document} 
	\startdoc
	
	\topics{
		\item Ring Homomorphisms
		\item First Isomorphism Theorem for Rings
	}{
		\item Polynomial Rings
	}
\slide{
	\begin{exercise}
		We encountered $\R[x]=\{a_0+a_1x+\cdots+a_nx^n\ |\ n\geq 0,\ a_i\in \R\}$ briefly before.
		
		It is a ring under polynomial addition and multiplication.
		
		\enumalph{
			\item Give some example elements of $\R[x]$.\vfill
			\item What's the additive identity of $\R[x]$?\vfill
			\item What's the multiplicative identity of $\R[x]$?\vfill
			\item How would you define $\Z[x]$? $\Q[x]$? \vfill
			\item What do elements of $\Z_3[x]$ look like?\vfill
		}	
	\end{exercise}
}

\slide{
	\begin{defn}
		Let $R$ be a ring.  We denote the \emph{set of polynomials over $R$} by $R[x]$ where
			\[R[x]:=\{r_0+r_1x+r_2x^2+\cdots+r_nx^n\ |\ n\in\Z_{\geq 0}\text{ and }r_i\in R\ \forall\  0\leq i\leq n\}.\]
		Any particular element $f=a_0+a_1x+a_2x^2+\cdots+a_nx^n$ in $R[x]$ is called a \emph{polynomial}, the elements $a_i\in R$ are called the \emph{coefficients} of $f$.
	\end{defn}	
	\begin{nb}
		The variable $x$ is called an \emph{indeterminate} it is a symbol representing a position in the list.
	\end{nb}
}
\newpage
\slide{
	\begin{statementblock}{Theorem}
		Let $R$ be a ring with unity.  For $f,g\in R[x]$ such that
			\[f=a_0+a_1x+a_2x^2+\cdots\text{ and } g=b_0+b_1x+b_2x^2+\cdots\]
		define the operations
		\bulletize{
			\item $f+g = (a_0+b_0)+(a_1+b_1)x+(a_2+b_2)x^2+\cdots$
			\item $f\cdot g=c_0+c_1x+c_2x^2+\cdots$ where
				$$c_i=a_0b_i+a_1b_{i-1}+\cdots+a_{i-1}b_1+a_ib_0=\sum_{k=0}^i a_kb_{i-k}.$$
		}
		Then $(R[x],+,\cdot)$ is a ring with unity.
	\end{statementblock}
}
\slide{
	\begin{exercise}
		Let $f=3+2x+4x^2$ and $g=x-3x^2+2x^3$.  Compute the coefficient of $x^4$ in $f\cdot g$ using the formula for $c_4$.
	\end{exercise}
	\vfill
}

\slide{
	\begin{exercise}
		In $\Z_3[x]$ compute $(x+2)^4$.
	\end{exercise}
	\vfill
}
\slide{
	\begin{defn}
		Let $R$ be a ring with unity.  For $f,g\in R[x]$ such that
		\[f=a_0+a_1x+a_2x^2+\cdots\text{ and } g=b_0+b_1x+b_2x^2+\cdots\]
		define the items:
		\bulletize{
			\item We call two polynomials \emph{equal} if the corresponding coefficients are equal.
			
			That is, $f=g$ means that $a_0=b_0$, $a_1=b_1$, $a_2=b_2$, $\dots$.  
		
			\item We call $a_0$ the \emph{constant term} or \emph{constant coefficient}.
		
			\item A polynomial of the form $f=a_0$ is a \emph{constant polynomial}.
		
			\item The \emph{zero} of $R[x]$ is \blank{1in} and the \emph{unity} is \blank{1in}.
		
			\item The \emph{negative} of $f=a_0+a_1x+a_2x^2+\cdots$ is $-f=-a_0-a_1x-a_2x^2-\cdots$}
		\end{defn}
		\newpage
		\begin{defn}
			Let $R$ be a ring with unity.  For $f\in R[x]$ such that
			\[f=a_0+a_1x+a_2x^2+\cdots+a_n x^n\quad a_n\neq 0,\] define the following:
			\bulletize{
			\item Since $a_n\neq 0$, we say that the \emph{degree} of $f$ is $n$ and write $\deg(f)=n$.
			
			\item We call $a_n$ the \emph{leading coefficient} of $f$.
			\item If the leading coefficient of $f$ is $1$, we call $f$ \emph{monic}.
		}
	\end{defn}
}
\slide{
	\begin{thm}{4.1.1}
		Let $R$ be a ring and let $x$ be an indeterminate over $R$.  Then
		\enumarabic{
			\item $R[x]$ is a ring.
			\item $R$ is the subring of all constant polynomials in $R[x]$.
			\item If $Z=Z(R)$ denotes the center of $R$, then the center of $R[x]$ is $Z[x]$.
			\item In fact, $x$ is in the center of $R[x]$.
			\item If $R$ is commutative, then $R[x]$ is commutative.
		}
	\end{thm}
}
\slide{
\begin{defn}
	\bulletize{\setlength{\itemsep}{2em}
		\item If $\deg(f)=1$, we call $f$ a \underline{linear} polynomial.
		\item If $\deg(f)=2$, we call $f$ a \blank{1in} polynomial.
		\item If $\deg(f)=3$, we call $f$ a \blank{1in} polynomial.
		\item If $\deg(f)=4$, we call $f$ a \blank{1in} polynomial.
		\item If $\deg(f)=5$, we call $f$ a \blank{1in} polynomial.
	}
\end{defn}
}
\newpage
\slide{
\begin{exercise}
	Consider $S=\{f\in\Z[x]\ :\ f(1)=0\}$.  Is $S$ a subring or an ideal of $\Z[x]$?\vfill
\end{exercise}
}

\slide{
	\begin{exercise}
		Show that $(x)=\{xf\ |\ f\in \Z[x]\}$ is an ideal of $\Z[x]$ and that $\Z[x]/(x)\cong \Z$.
	\end{exercise}
}
\vfill
\newpage
\slide{
	\begin{thm}{4.1.2}
		Let $R$ be a domain.  Then
		\enumarabic{
			\item $R[x]$ is a domain.
			\item If $f\neq 0$ and $g\neq 0$ in $R[x]$, then $\deg(fg)=\deg(f)+\deg(g)$.
			\item The units in $R[x]$ are the units in $R$.
		}
	\end{thm}
	\vskip 3in
%	\begin{proof}
%		Let $R$ be a domain.  Let $f=a_0+a_1x+a_2x^2+\cdots+a_mx^m$ and $g=b_0+b_1x+\cdots+b_nx^n$ with $a_m\neq 0$ and $b_n\neq 0$.
%		\vskip 3in
%	\end{proof}
}
\slide{
	\begin{exercise}
		As a non-example to the previous theorem, consider $f=1+2x$ in $\Z_4[x]$.  
		\enumalph{
			\item Verify that $\Z_4$ is not a domain.\vfill
			\item Compute $f^2$ to see $\deg(f^2)\neq 2\deg(f)$.\vfill
			\item Find some zero divisors in $\Z_4[x]$.\vfill
		}
	\end{exercise}
}
\newpage
\slide{
	\begin{statementblock}{Division Algorithm (Theorem 4.1.4)}
		Let $R$ be any ring and let $f$ and $g$ be polynomials in $R[x]$.  Assume $f\neq 0$ and that the leading coefficient of $f$ is a unit in $R$.  Then there exist unique $q,r\in R[x]$ such that
			\enumarabic{\item $g=qf+r$.\item Either $r=0$ or $\deg r<\deg f$.}
	\end{statementblock}
	\begin{exercise}
		Use long division to find $q$ and $r$ given $f=x^2+1$ and $g=x^4+3x^3+x+1$ in $\Z[x]$.\vfill
	\end{exercise}
}
\slide{
	\begin{statementblock}{Factor Theorem (Theorem 4.1.6(1)}
		Let $R$ be a commutative ring, $a\in R$, and $f\in R[x]$.
		Then $f(a)=0$ if and only if $f=(x-a)g$ for some $g\in R[x]$.
		
		\textbf{(Remainder Theorem)} Moreover, in general, when dividing $f$ by $x-a$, we get $f=(x-a)q + f(a)$.  That is, the remainder when dividing $f$ by $x-a$ is $f(a)\in R$.
	\end{statementblock}
}

\slide{
	\begin{exercise}
		Consider $R=\Z_6$ and $f=x^3-x$.
		
		Notice $f(0)=f(1)=f(2)=f(3)=f(4)=f(5)=0$. Thus
		\vskip 1em
		\bulletize{\setlength{\itemsep}{1em}
			\item $f = (x-0)(\blank{1in})$
			\item $f = (x-1)(\blank{1in})$
			\item $f = (x-2)(\blank{1in})$
			\item $f = (x-3)(\blank{1in})$
			\item $f = (x-4)(\blank{1in})$
			\item $f = (x-5)(\blank{1in})$
		}
	\end{exercise}
	\begin{question}
		Does this mean $f=x(x-1)(x-2)(x-3)(x-4)(x-5)$?
	\end{question}
}
\newpage

\slide{
	\begin{statementblock}{Corollary}
		Let $R$ be a commutative ring, $a\in R$, and $\phi_a:R[x]\to R$ the evaulation map at $a$. Then $$\ker(\phi_a)=(x-a)=\{(x-a)g\ |\ g\in R[x]\}$$ and $R[x]/(x-a)\cong R$.
	\end{statementblock}
}
\vskip 1in

\slide{
	\begin{defn}
		Let $f\in R[x]$ and $a\in R$.  We call $a$ a \emph{root} or $f$ if the following conditions (which are all equivalent) are true:
		\enumarabic{\item $f(a)=0$.\item $f=(x-a)g$ for some $g\in R[x]$.\item $f$ is in the principal ideal $(x-a)$.}
		
		If $a\in R$ is a root of $f$, we say it has multiplicity $m\in \Z_{>0}$ if $f=(x-a)^mq$ and $q(a)\neq 0$.
	\end{defn}
	\begin{exa}
		What's the multiplicity of $a=-1$ as a root of $f=t^4+t^3+t+1\in \Z_7[t]$?\vskip 1in\mbox{}
	\end{exa}
}

\slide{
	\begin{thm}{4.1.8}
		Let $R$ be an integral domain and let $f$ be a nonzero polynomial of degree $n$ in $R[x]$.  Then $f$ has at most $n$ roots in $R$.
	\end{thm}
}
\slide{
	\begin{statementblock}{Rational Roots Theorem (Theorem 4.1.9)}
		Let $f=a_0+a_1x+a_2x^2+\cdots+a_nx^n$ be a polynomial in $\Z[x]$ with $a_0,a_n\neq 0$.   Then every root of $f$ in $\Q$ is of the form $\frac{c}{d}$ where $c\mid a_0$ and $d\mid a_n$. 
	\end{statementblock}
	\begin{exa}
		Factor the following as much as possible in $\Q[x]$.
		\enumalph{\item $2 \, x^{3} - 7 \, x^{2} -7 \, x + 12$\item $x^{4} + 5x^{3} - 4x^{2} + 3x$\item $12 \, x^{4} - 44 \, x^{3} + 39 \, x^{2} + 8 \, x - 12$}
	\end{exa}
}


\end{document}

	