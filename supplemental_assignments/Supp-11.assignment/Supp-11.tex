\documentclass[12pt]{article}
\usepackage{amsmath,amsthm,amssymb}
\usepackage[margin=.75in]{geometry}
\pagestyle{empty}
\setlength{\parindent}{0pt}

\newcommand{\Z}{\mathbb{Z}}


\begin{document}
Name: \underline{\hspace*{3in}}
\vskip .25in

\textbf{(Supp-11)} If $G$ is a group and $g\in G$ is a fixed element, define $C(g)=\{z\in G:z*g=g*z\}$, in other words, $C(g)$ is all of the elements of the group that commute with the specific element $g$.  Show that $C(g)$ is a subgroup of $G$.  (We call $C(g)$ the \emph{centralizer} of $g$ in $G$.)

It might help to compute a few cases first.  Here are some examples:
\begin{itemize}
	\item Let $G=\Z_4$, then $C(\bar {2})=\{\bar 0,\bar 1,\bar 2,\bar 3\}=\Z_4$.
	\item Let $G= S_3$ and $g=(1\ 2\ 3)$. Then 
	\[C(g)= \{\varepsilon,(1\ 2\ 3),(1\ 3\ 2)\}\]
	\item Let $G= S_4$ and $g = (1\ 2)$. Then 
	\[C(g) = \{\varepsilon,(1\ 2),(3\ 4),(1\ 2)(3\ 4)\}.\]
\end{itemize}



\end{document}