\documentclass[12pt]{article}
\usepackage{amsmath,amsthm,amssymb}
\usepackage[margin=.75in]{geometry}
\pagestyle{empty}
\setlength{\parindent}{0pt}

\newcommand{\Z}{\mathbb{Z}}
\newcommand{\abar}{\overline{a}}
\newcommand{\bbar}{\overline{b}}



\begin{document}
Name: \underline{\hspace*{3in}}
\vskip .25in

\textbf{(Supp-7)} Use proof by induction to show that  If $\abar_1,\abar_2,\dots,\abar_m$ all have inverses in $\Z_n$, show that the same is true for the product $\abar_1\abar_2\cdots\abar_m$ for all $m\geq 2$.
\vskip .25in
\begin{proof} 
	Let $n\geq 2$ be an integer. We use induction on the value of $m$.
	
	\textbf{Base Case:} Let $m=2$. Assume $\abar_1$ and $\abar_2\in\Z_n$ have inverses. 
	
	\fbox{Show that $\abar_1\abar_2$ has an inverse. Use the definition of inverse here.}
	
	\vskip 1.5in
	
	\textbf{Inductive Step:} Let $k\geq 2$ be an integer. Assume the statement holds for $m=k$. Let $\abar_1,\abar_2,\dots,\abar_{k+1}\in\Z_n$ all have inverses. 
	
	\fbox{\begin{minipage}{.95\textwidth}Use the base case and the inductive hypothesis to show that $\abar_1\abar_2\cdots\abar_{k+1}$ has an inverse. Again, use the definition of inverse.\end{minipage}}
	
	\vfill\mbox{}
\end{proof}



\end{document}