\documentclass[12pt]{article}

\newcommand{\secname}{Section 2.1: Binary Operations}

\usepackage{amsthm,amsmath,amsfonts,hyperref,graphicx,color,multicol,soul}
\usepackage{enumitem,tikz,tikz-cd,setspace,mathtools}
\usepackage{colortbl}
\usepackage[margin=1in]{geometry}

%%%%%%%%%%
%Color Customization
%%%%%%%%%%

\definecolor{Blu}{RGB}{43,62,133} % UWEC Blue

%Unnumbered footnotes:
\newcommand{\blfootnote}[1]{%
	\begingroup
	\renewcommand\thefootnote{}\footnote{#1}%
	\addtocounter{footnote}{-1}%
	\endgroup
}

%%%%%%%%%%
%TikZ Stuff
%%%%%%%%%%
\usetikzlibrary{arrows}
\usetikzlibrary{shapes.geometric}
\tikzset{
	smaller/.style={
		draw,
		regular polygon,
		regular polygon sides=3,
		fill=white,
		node distance=2cm,
		minimum height=1in,
		line width = 2pt
	}
}
\tikzset{
	smsquare/.style={
		draw,
		regular polygon,
		regular polygon sides=4,
		fill=white,
		node distance=2cm,
		minimum height=1in,
		line width = 2pt
	}
}

%%%%%%%%%%
%Listing Setup
%%%%%%%%%%
\usepackage{listings}
\usepackage{caption, floatrow, makecell}%
\captionsetup{labelfont = sc}
\setcellgapes{3pt}

\definecolor{backcolour}{RGB}{237,236,230}
\definecolor{myblue}{RGB}{42,157,189}

\lstdefinestyle{mystyle}{
	language=Python,
	keywords=[2]{sage:},
	alsodigit={:,.,<,>},
	backgroundcolor=\color{backcolour},   
	commentstyle=\color{myblue},
	keywordstyle=\bfseries\color{Green},
	keywordstyle=[2]\color{purple},
	numberstyle=\tiny\color{Gray},
	stringstyle=\color{Orange},
	basicstyle=\ttfamily\footnotesize,
	breakatwhitespace=false,         
	breaklines=true,                 
	captionpos=b,                    
	keepspaces=true,                   
	showspaces=false,                
	showstringspaces=false,
	showtabs=false,                  
	tabsize=2
}

\lstset{style=mystyle}


%%%%%%%%%%
%Custom Commands
%%%%%%%%%%

\newcommand{\C}{\mathbb{C}}
\newcommand{\quats}{\mathbb{H}}
\newcommand{\N}{\mathbb{N}}
\newcommand{\Q}{\mathbb{Q}}
\newcommand{\R}{\mathbb{R}}
\newcommand{\Z}{\mathbb{Z}}

\newcommand{\ds}{\displaystyle}

\newcommand{\fn}{\insertframenumber}

\newcommand{\id}{\operatorname{id}}
\newcommand{\im}{\operatorname{im}}
\newcommand{\lcm}{\operatorname{lcm}}
\newcommand{\ord}{\operatorname{ord}}
\newcommand{\Aut}{\operatorname{Aut}}
\newcommand{\Inn}{\operatorname{Inn}}

\newcommand{\blank}[1]{\underline{\hspace*{#1}}}

\newcommand{\abar}{\overline{a}}
\newcommand{\bbar}{\overline{b}}
\newcommand{\cbar}{\overline{c}}

\newcommand{\nml}{\unlhd}

%%%%%%%%%%
%Custom Theorem Environments
%%%%%%%%%%
\theoremstyle{definition}
\newtheorem{exercise}{Exercise}
\newtheorem{question}[exercise]{Question}
\newtheorem{warmup}{Warm-Up}
\newtheorem*{exa}{Example}
\newtheorem*{defn}{Definition}
\newtheorem*{disc}{Group Discussion}
\newtheorem*{recall}{Recall}
\renewcommand{\emph}[1]{{\color{blue}\texttt{#1}}}

\definecolor{Gold}{RGB}{237, 172, 26}
%Statement block
%\newenvironment{statementblock}[1]{%
%	\setbeamercolor{block body}{bg=Gold!20}
%	\setbeamercolor{block title}{bg=Gold}
%	\begin{block}{\textbf{#1.}}}{\end{block}}
%\newenvironment{goldblock}{%
%	\setbeamercolor{block body}{bg=Gold!20}
%	\setbeamercolor{block title}{bg=Gold}
%	\setbeamertemplate{blocks}[shadow=true]
%	\begin{block}{}}{\end{block}}
%\newenvironment{defn}{%
%	\setbeamercolor{block body}{bg=gray!20}
%	\setbeamercolor{block title}{bg=violet, fg=white}
%	\setbeamertemplate{blocks}[shadow=true]
%	\begin{block}{\textbf{Definition.}}}{\end{block}}
%\newenvironment{nb}{%
%	\setbeamercolor{block body}{bg=gray!20}
%	\setbeamercolor{block title}{bg=teal, fg=white}
%	\setbeamertemplate{blocks}[shadow=true]
%	\begin{block}{\textbf{Note.}}}{\end{block}}
%\newenvironment{blockexample}{%
%	\setbeamercolor{block body}{bg=gray!20}
%	\setbeamercolor{block title}{bg=Blu, fg=white}
%	\setbeamertemplate{blocks}[shadow=true]
%	\begin{block}{\textbf{Example.}}}{\end{block}}
%\newenvironment{blocknonexample}{%
%	\setbeamercolor{block body}{bg=gray!20}
%	\setbeamercolor{block title}{bg=purple, fg=white}
%	\setbeamertemplate{blocks}[shadow=true]
%	\begin{block}{\textbf{Non-Example.}}}{\end{block}}
%\newenvironment{thm}[1]{%
%	\setbeamercolor{block body}{bg=Gold!20}
%	\setbeamercolor{block title}{bg=Gold}
%	\begin{block}{\textbf{Theorem #1.}}}{\end{block}}


%%%%%%%%%%
%Custom Environment Wrappers
%%%%%%%%%%
\newcommand{\exer}[1]{
	\begin{exercise}
	#1
	\end{exercise}
}
\newcommand{\exam}[1]{
\textbf{Example: }
	#1
}
\newcommand{\nexam}[1]{
	\textbf{Non-Example: }
	#1
}
\newcommand{\enumarabic}[1]{
	\begin{enumerate}[label=\textbf{\arabic*.}]
		#1
	\end{enumerate}
}
\newcommand{\enumalph}[1]{
	\begin{enumerate}[label=(\alph*)]
		#1
	\end{enumerate}
}
\newcommand{\bulletize}[1]{
	\begin{itemize}[label=$\bullet$]
		#1
	\end{itemize}
}
\newcommand{\circtize}[1]{
	\begin{itemize}[label=$\circ$]
		#1
	\end{itemize}
}
%\newcommand{\slide}[1]{
%	\begin{frame}{\fn}
%		#1
%	\end{frame}
%}
%\newcommand{\slidec}[1]{
%\begin{frame}[c]{\fn}
%	#1
%\end{frame}
%}
%\newcommand{\slidet}[2]{
%	\begin{frame}{\fn\ - #1}
%		#2
%	\end{frame}
%}


\setlength{\parindent}{0pt}



\usepackage{afterpage}
\usepackage{fancyhdr}

\fancyhead[L]{\textbf{Math 425: Abstract Algebra I\\\secname}}
\fancyhead[R]{\textbf{Mckenzie West\\Last Updated: \today}}
\pagestyle{fancy}

\newcommand{\startdoc}{}

\newcommand{\topics}[2]{
		{\textbf{Previously.}}
			\begin{itemize}[label=--]
				#1
			\end{itemize}
		{\textbf{This Section.}}
			\begin{itemize}[label=--]
				#2
			\end{itemize}
}
\begin{document}
	\startdoc

	\topics{
		\item Disjoint Cycles
		\item Transpositions
		\item Even vs Odd Permutations
	}{
		\item Binary Operations
		\item Monoids
	}

\slide{
	\begin{defn}
		A \emph{binary operation}, $*$ on a set $S$ is a function that associates to each ordered pair $(a,b)\in S\times S$ an element of $S$ which we call $a*b$.
	\end{defn}
	\begin{nb}
		Since we know that $a*b\in S$ for all $a,b\in S$, we say that the binary operation is \emph{closed under $*$}.
	\end{nb}
%	\begin{exa}\begin{itemize}[label=$\bullet$]
%			\item $+,-,\cdot,\div$ are all binary operations on $\R$
%			\item $+,-,\cdot$ are all binary operations on $\Z$ and on $\Z_n$
%			\item $\circ$ is a binary operation on $S_n$
%		\end{itemize}
%	\end{exa}
	\begin{nb}
		 Sometimes we write $(S,*)$ to mean $*$ is a binary operation on $S$.
	\end{nb}
}

\slide{
	\begin{defn}
		A binary operation $*$ on $S$ is \emph{associative} if
		\[a*(b*c)=(a*b)*c,\]
		for all $a,b,c\in S$.
	\end{defn}
}
\slide{
	\begin{defn}
		A binary operation $*$ on $S$ is \emph{commutative} if
		\[a*b=b*a,\]
		for all $a,b\in S$.
	\end{defn}
}
\slide{
	\begin{defn}
		An element $e\in S$ is called an \emph{identity} (or \emph{unity}) for the binary operation $*$ if
			\[a*e=e*a=a,\]
		for all $a\in S$.
	\end{defn}
}
\slide{
	\begin{statementblock}{Theorem 2.1.1}
		If a binary operation $*$ on a set $S$ has an identity, then it is unique.
	\end{statementblock}
}

\slide{
	\begin{defn}
		A set $S$ along with a binary operation $*$ is called an \emph{monoid} if $*$ is associative and has an identity.

		\vskip 1em

		If $(S,*)$ is also commutative, then we say $S$ is a \emph{commutative monoid}.
	\end{defn}
	\begin{exa}
		\bulletize{
			\item $(\Z,\cdot)$
			\item $(S_n,\circ)$
		}
	\end{exa}
}
\begin{exercise}
	Let $*$ be the binary operation on $\mathbb{N}$ given by \[a*b:=a^b.\]
	\enumalph{
		\item Associative?
		\vskip 2em
		\item Identity?
		\vskip 2em
		\item Commutative?
		\vskip 2em
	}
	Therefore, $(\mathbb{N},$\verb|^|$)$ is ...
\end{exercise}

\slide{
\begin{exercise}
	Let $*$ be the binary operation on $\mathbb{Z}$ given by \[a*b:=a-b.\]
	\enumalph{
		\item Associative?
		\vskip 2em
		\item Identity?
		\vskip 2em
		\item Commutative?
		\vskip 2em
	}
	Therefore, $(\mathbb{Z},-)$ is ...
\end{exercise}
}
\slide{
\begin{exercise}
	Let $*$ be the binary operation on $GL_2(\R)$given by \[A*B:=AB.\]
	\enumalph{
		\item Associative?
		\vskip 2em
		\item Identity?
		\vskip 2em
		\item Commutative?
		\vskip 2em
	}
	Therefore, $(GL_2(\R),\cdot)$ is ...
\end{exercise}
}
\slide{
	\begin{defn}
		Let $(M,*)$ be a monoid.

		If $x\in M$, we call $y\in M$ an \emph{inverse of $x$} if
			\[xy=e=yx.\]
		An element that has an inverse is called a \emph{unit}.
	\end{defn}
}
\slide{\vskip -.25in
	\begin{exercise}
		Fill in the table

			\def\arraystretch{1.5}
			$\begin{array}{c|c|c}
			\text{Monoid}&\quad e\quad &\quad \text{Set of Units}\quad\\
			\hline\hline
			(\Z,+)&&\\\hline
			(\R,+)&&\\\hline
			(\Z_n,+)&&\\\hline
			(M_n(\R),+)&&\\\hline
			(M_n(\R),\cdot)&&\\\hline
			(\R,\cdot)&&\\\hline
			(\Z_4,\cdot)&&\\\hline
			(\Z_n,\cdot)&&\\\hline
			(\Z_p,\cdot)&&\\\hline
			(S_3,\circ)&&\\\hline
			\end{array}$

		\begin{picture}(0,0)
		\put(195,132){$\leftarrow$ Matrices with real entries}
		\put(195,30){$\leftarrow$ $p$ is prime}
		\end{picture}
	\end{exercise}
}

\end{document}

