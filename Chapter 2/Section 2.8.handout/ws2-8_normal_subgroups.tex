\documentclass[12pt]{article}

\newcommand{\secname}{Section 2.8: Normal Subgroups}

\usepackage{amsthm,amsmath,amsfonts,hyperref,graphicx,color,multicol,soul}
\usepackage{enumitem,tikz,tikz-cd,setspace,mathtools}
\usepackage{colortbl}
\usepackage[margin=1in]{geometry}

%%%%%%%%%%
%Color Customization
%%%%%%%%%%

\definecolor{Blu}{RGB}{43,62,133} % UWEC Blue

%Unnumbered footnotes:
\newcommand{\blfootnote}[1]{%
	\begingroup
	\renewcommand\thefootnote{}\footnote{#1}%
	\addtocounter{footnote}{-1}%
	\endgroup
}

%%%%%%%%%%
%TikZ Stuff
%%%%%%%%%%
\usetikzlibrary{arrows}
\usetikzlibrary{shapes.geometric}
\tikzset{
	smaller/.style={
		draw,
		regular polygon,
		regular polygon sides=3,
		fill=white,
		node distance=2cm,
		minimum height=1in,
		line width = 2pt
	}
}
\tikzset{
	smsquare/.style={
		draw,
		regular polygon,
		regular polygon sides=4,
		fill=white,
		node distance=2cm,
		minimum height=1in,
		line width = 2pt
	}
}

%%%%%%%%%%
%Listing Setup
%%%%%%%%%%
\usepackage{listings}
\usepackage{caption, floatrow, makecell}%
\captionsetup{labelfont = sc}
\setcellgapes{3pt}

\definecolor{backcolour}{RGB}{237,236,230}
\definecolor{myblue}{RGB}{42,157,189}

\lstdefinestyle{mystyle}{
	language=Python,
	keywords=[2]{sage:},
	alsodigit={:,.,<,>},
	backgroundcolor=\color{backcolour},   
	commentstyle=\color{myblue},
	keywordstyle=\bfseries\color{Green},
	keywordstyle=[2]\color{purple},
	numberstyle=\tiny\color{Gray},
	stringstyle=\color{Orange},
	basicstyle=\ttfamily\footnotesize,
	breakatwhitespace=false,         
	breaklines=true,                 
	captionpos=b,                    
	keepspaces=true,                   
	showspaces=false,                
	showstringspaces=false,
	showtabs=false,                  
	tabsize=2
}

\lstset{style=mystyle}


%%%%%%%%%%
%Custom Commands
%%%%%%%%%%

\newcommand{\C}{\mathbb{C}}
\newcommand{\quats}{\mathbb{H}}
\newcommand{\N}{\mathbb{N}}
\newcommand{\Q}{\mathbb{Q}}
\newcommand{\R}{\mathbb{R}}
\newcommand{\Z}{\mathbb{Z}}

\newcommand{\ds}{\displaystyle}

\newcommand{\fn}{\insertframenumber}

\newcommand{\id}{\operatorname{id}}
\newcommand{\im}{\operatorname{im}}
\newcommand{\lcm}{\operatorname{lcm}}
\newcommand{\ord}{\operatorname{ord}}
\newcommand{\Aut}{\operatorname{Aut}}
\newcommand{\Inn}{\operatorname{Inn}}

\newcommand{\blank}[1]{\underline{\hspace*{#1}}}

\newcommand{\abar}{\overline{a}}
\newcommand{\bbar}{\overline{b}}
\newcommand{\cbar}{\overline{c}}

\newcommand{\nml}{\unlhd}

%%%%%%%%%%
%Custom Theorem Environments
%%%%%%%%%%
\theoremstyle{definition}
\newtheorem{exercise}{Exercise}
\newtheorem{question}[exercise]{Question}
\newtheorem{warmup}{Warm-Up}
\newtheorem*{exa}{Example}
\newtheorem*{defn}{Definition}
\newtheorem*{disc}{Group Discussion}
\newtheorem*{recall}{Recall}
\renewcommand{\emph}[1]{{\color{blue}\texttt{#1}}}

\definecolor{Gold}{RGB}{237, 172, 26}
%Statement block
%\newenvironment{statementblock}[1]{%
%	\setbeamercolor{block body}{bg=Gold!20}
%	\setbeamercolor{block title}{bg=Gold}
%	\begin{block}{\textbf{#1.}}}{\end{block}}
%\newenvironment{goldblock}{%
%	\setbeamercolor{block body}{bg=Gold!20}
%	\setbeamercolor{block title}{bg=Gold}
%	\setbeamertemplate{blocks}[shadow=true]
%	\begin{block}{}}{\end{block}}
%\newenvironment{defn}{%
%	\setbeamercolor{block body}{bg=gray!20}
%	\setbeamercolor{block title}{bg=violet, fg=white}
%	\setbeamertemplate{blocks}[shadow=true]
%	\begin{block}{\textbf{Definition.}}}{\end{block}}
%\newenvironment{nb}{%
%	\setbeamercolor{block body}{bg=gray!20}
%	\setbeamercolor{block title}{bg=teal, fg=white}
%	\setbeamertemplate{blocks}[shadow=true]
%	\begin{block}{\textbf{Note.}}}{\end{block}}
%\newenvironment{blockexample}{%
%	\setbeamercolor{block body}{bg=gray!20}
%	\setbeamercolor{block title}{bg=Blu, fg=white}
%	\setbeamertemplate{blocks}[shadow=true]
%	\begin{block}{\textbf{Example.}}}{\end{block}}
%\newenvironment{blocknonexample}{%
%	\setbeamercolor{block body}{bg=gray!20}
%	\setbeamercolor{block title}{bg=purple, fg=white}
%	\setbeamertemplate{blocks}[shadow=true]
%	\begin{block}{\textbf{Non-Example.}}}{\end{block}}
%\newenvironment{thm}[1]{%
%	\setbeamercolor{block body}{bg=Gold!20}
%	\setbeamercolor{block title}{bg=Gold}
%	\begin{block}{\textbf{Theorem #1.}}}{\end{block}}


%%%%%%%%%%
%Custom Environment Wrappers
%%%%%%%%%%
\newcommand{\exer}[1]{
	\begin{exercise}
	#1
	\end{exercise}
}
\newcommand{\exam}[1]{
\textbf{Example: }
	#1
}
\newcommand{\nexam}[1]{
	\textbf{Non-Example: }
	#1
}
\newcommand{\enumarabic}[1]{
	\begin{enumerate}[label=\textbf{\arabic*.}]
		#1
	\end{enumerate}
}
\newcommand{\enumalph}[1]{
	\begin{enumerate}[label=(\alph*)]
		#1
	\end{enumerate}
}
\newcommand{\bulletize}[1]{
	\begin{itemize}[label=$\bullet$]
		#1
	\end{itemize}
}
\newcommand{\circtize}[1]{
	\begin{itemize}[label=$\circ$]
		#1
	\end{itemize}
}
%\newcommand{\slide}[1]{
%	\begin{frame}{\fn}
%		#1
%	\end{frame}
%}
%\newcommand{\slidec}[1]{
%\begin{frame}[c]{\fn}
%	#1
%\end{frame}
%}
%\newcommand{\slidet}[2]{
%	\begin{frame}{\fn\ - #1}
%		#2
%	\end{frame}
%}


\setlength{\parindent}{0pt}



\usepackage{afterpage}
\usepackage{fancyhdr}

\fancyhead[L]{\textbf{Math 425: Abstract Algebra I\\\secname}}
\fancyhead[R]{\textbf{Mckenzie West\\Last Updated: \today}}
\pagestyle{fancy}

\newcommand{\startdoc}{}

\newcommand{\topics}[2]{
		{\textbf{Previously.}}
			\begin{itemize}[label=--]
				#1
			\end{itemize}
		{\textbf{This Section.}}
			\begin{itemize}[label=--]
				#2
			\end{itemize}
}

\begin{document} 
	\startdoc
	
	\topics{
		\item Dihedral Groups
		\item Groups of order 2p
	}{
		\item Normal Subgroups
		\item Products of Groups
		\item Simple Groups
	}

\slide{
	Recall from Section 2.6:
	\begin{defn}
		If $H\leq G$ and $a\in G$, then we call $Ha=\{ha : h\in H\}$ a \emph{right coset} of $H$.  Similarly we call $aH=\{ah : h\in H\}$ a \emph{left coset} of $H$.
	\end{defn}
	\begin{block}{Recall}
		Let $G=S_3$ and $H=\langle(1\ 2)\rangle=\{\varepsilon,(1\ 2)\}$.
		\begin{multicols}{2}
			The right cosets of $H$
			\bulletize{
				\item $H\varepsilon =\{\varepsilon,(1\ 2)\}$
				\item $ H(1\ 2\ 3)=\{(1\ 2\ 3),(2\ 3)\}$
				\item $ H(1\ 3\ 2)=\{(1\ 3\ 2),(1\ 3)\}$
			}
			The left cosets of $H$ are:
			\bulletize{
				\item $\varepsilon H=\{\varepsilon,(1\ 2)\}$
				\item $(1\ 2\ 3) H=\{(1\ 2\ 3),(1\ 3)\}$
				\item $(1\ 3\ 2) H=\{(1\ 3\ 2),(2\ 3)\}$
			}
		\end{multicols}
	\end{block}
		
	\exer{
			\item What are the left and right cosets of $K=\{\varepsilon,(1\ 2\ 3),(1\ 3\ 2)\}$?\vskip 1in
	}
}



\slide{
	\begin{defn}
		A subgroup $H$ of $G$ is called a \emph{normal subgroup} of $G$ if $gH=Hg$ for all $g\in G$.
		If $H$ is a normal subgroup of $G$, we might say \emph{$H$ is normal in $G$} and write $H\nml G$.
	\end{defn}
}
	\vskip 1in
\slide{
	\begin{thm}{2.8.2}
		If $G$ is abelian and $H\leq G$, then $H\nml G$.
	\end{thm}
}
	\vskip 2in
\slide{
	\begin{thm}{2.8.1}
		If $G$ is a group, every subgroup of the center, $Z(G)$, is normal in $G$.  In particular, $Z(G)\nml G$.
	\end{thm}
}
	\vskip 2in

\slide{
	\begin{statementblock}{Normality Test (Theorem 2.8.3)}
		The following conditions are equivalent for a subgroup $H$ of a group $G$.
		\enumarabic{
			\item $H\nml G$.
			\item $gHg^{-1}\subseteq H$ for all $g\in G$.
			\item $gHg^{-1}=H$ for all $g\in G$.
		}
	\end{statementblock}
	\vskip 3in
	\exer{
		Let $G=GL_2(\R)$ and $H=SL_2(\R)$.
		Show $H\nml G$.
	}
}
\newpage
\slide{
	\begin{statementblock}{Corollary 1}
		If $G=\langle g_1,g_2,\dots,g_n\rangle$, then a subgroup $H$ of $G$ is normal if and only if $g_iHg_i^{-1}\subseteq H$ for all $1\leq i\leq n$.
	\end{statementblock}
	\exer{
		Let $G=D_{12}=\langle r,f\rangle$ and $H=\{e,r^4,r^8\}$.  Show that $H\nml G$.\vskip 2in
	}
	
}
\slide{
	\begin{statementblock}{Theorem}
		If $a\in G$, then $\langle a \rangle\nml G$ if and only if $gag^{-1}\in\langle a \rangle$ for all $g\in G$.
	\end{statementblock}
	\begin{statementblock}{Theorem}
		If $a_1,a_2,\dots,a_n\in G$ and $H= \langle a_1,a_2,\dots,a_n\rangle$, then $H\nml G$ if and only if $ga_ig^{-1}\in H$ for all $g\in G$ and all $1\leq i\leq n$.
	\end{statementblock}
}
\exer{
 	Let $G=D_{15}=\langle{r,f}$ and $H=\langle r^3\rangle=\{1,r^3,r^6,r^9,r^{12}\}$.  Show that $H\nml G$.
}
\vskip 2in
\slide{
	\begin{thm}{2.8.4}
		If $H\leq G$ with $|G:H|=2$, then $H\nml G$.
	\end{thm}
	\newpage
}
\slide{
	\begin{exercise}
		Show $A_n\nml S_n$. Using Theorem 2.8.4.\vskip 2in
	\end{exercise}
}
\slide{
\begin{defn}
	If $H$ is a subgroup of $G$ and $g\in G$, we call $gHg^{-1}$ a \emph{conjugate} of $H$ in $G$.
\end{defn}
\begin{statementblock}{Corollary 2}
	If $H$ is a subgroup of $G$, then $gHg^{-1}$ is also a subgroup of $G$, isomorphic to $H$, for all $g\in G$.  Moreover if $G$ has no other subgroups isomorphic to $H$ then $H\nml G$.
\end{statementblock}
\exer{
	What are the conjugate subgroups of $H=\{\varepsilon,(1\ 2)\}$ in $S_3$?}\vskip 2in
}

\slide{
\begin{defn}
	The \emph{product} of the subgroups $H,K\leq G$ is the set
	\[HK=\{hk\ :\ h\in H,\ k\in K\}.\]
\end{defn}
\begin{exercise}
	$S_3=\{\varepsilon,\sigma,\sigma^2,\tau,\tau\sigma,\tau\sigma^2\}$, $H=\{\varepsilon,\tau\}$, $K=\{\varepsilon,\tau\sigma\}$ 
	
	Compute $HK$.
\end{exercise}
}
\newpage
\slide{
\begin{thm}{2.8.5}
	Let $H$ and $K$ be subgroups of a group $G$.
	\enumarabic{\item If $H$ or $K$ is normal in $G$, then $HK=KH$ is a subgroup of $G$.
		\item If both $H$ and $K$ are normal in $G$, then $HK\nml G$ too.}
\end{thm}
\begin{exa}
	$G=D_6$, $H=\{e,r^2,r^4\}$, $K=\{e,r^3\}$\vskip 2in\mbox{}
\end{exa}
}
\slide{
\begin{thm}{2.8.6}
	If $H\nml G$ and $K\nml G$ satisfy $H\cap K=\{e_G\}$, then $HK\cong H\times K$.
\end{thm}
\begin{block}{Proof Idea.}
	Define $\phi: H\times K \to HK$ by $\phi(h,k)=hk$.
	\bulletize{\item (Onto) True no matter $H$ and $K$.\item (One-to-One) True because $H\cap K=\{e_G\}$\item (Homomorphism) True because $H$ and $K$ are normal and $H\cap K=\{e_G\}$.}
\end{block}
}
\slide{
\begin{exercise}
	$G=\Z_6$, $H=\{\overline{0},\overline{2},\overline{4}\}$, $K=\{\overline{0},\overline{3}\}$.
	
	Compute $H+K$ and make a correspondence with the elements of $HK$.\vskip 2in
\end{exercise}
\begin{exercise}
	$G=D_6$, $H=\{e,r^2,r^4\}$, $K=\{e,r^3\}$
	
	Verify $H\times K\cong HK$.\vskip 2in
\end{exercise}
}
\slide{
	\begin{exercise}
		We saw $SL_2(\R)$, the set of $2\times 2$ matrices with determinant 1, is a normal subgroup of $GL_2(\R)$, the multiplicative group of $2\times 2$ invertible matrices.  
		\enumalph{
			\item Verify that $D_2(\R)=\left\{\begin{bmatrix}a&0\\0&a\end{bmatrix}\ |\ a\in\R^\times\right\}$ is also an normal subgroup of $GL_2(\R)$.\vskip 2in
			\item Compute $D_2(\R)\cap SL_2(\R)$.\vskip 1in
			\item Make conclusions about $D_2(\R)\cdot SL_2(\R)$ and $D_2(\R)\times SL_2(\R)$.\vskip 1in
			\item Show that $D_2(\R)$ is the center of $GL_2(\R)$.
			\vskip 3in
		}
	\end{exercise}
}
\newpage
\slide{
	\begin{statementblock}{Corollary 1}
		If $G$ is a finite group and $H,K\leq G$ with $H\cap K=\{e_G\}$, then $|HK|=|H||K|$.
	\end{statementblock}
	\vskip 1in
	\begin{statementblock}{Corollary 2}
		If $G$ is a finite group and $H,K\nml G$ with $H\cap K=\{e_G\}$ and $|HK|=|G|$, then $G\cong H\times K$.
	\end{statementblock}
	\vskip 1in
}
\slide{
\begin{statementblock}{Theorem}
	If $m$ and $n$ are relatively prime integers and $G$ is a cyclic group of order $mn$, then $G\cong C_m\times C_n$.
\end{statementblock}
\vskip 1in
\begin{statementblock}{Theorem}
	Let $G$ be an abelian group of order $p^2$ for some prime $p$.  Then either $G\cong C_{p^2}$ or $G\cong C_p\times C_p$.
\end{statementblock}
\vskip 1in
}
\slide{
\begin{exa}
	A non-abelian group where every subgroup is normal.
	
	$Q=\{\pm 1,\pm i,\pm j\pm k\}$ with relations
	\[i^2=j^2=k^2=-1=ijk\]
	\[ij=k=-ji\]
	\[jk=i=-kj\]
	\[ki=j=-ik\]
\end{exa}
}
\newpage
\slide{
\begin{defn}
	A group $G$ is \emph{simple} if its only normal subgroups are $\{e_G\}$ and $G$.
\end{defn}
\begin{exa}
	$\Z_p$ is simple for all primes $p$
\end{exa}
}
\slide{
\begin{thm}{2.8.7}
	An abelian group $G\neq\{e_G\}$ is simple if and only if $|G|$ is prime.
\end{thm}
	\vskip 1in
}
\slide{
\begin{thm}{2.8.8}
	If $n\geq 5$, then $A_n$ is simple.
\end{thm}
\begin{block}{Proof Idea.} To summarize the argument:
	\bulletize{
		\item Every non-identity element of $A_n$ can be written as a product of 3-cycles.
		\[(ij)(ij)=\varepsilon\quad (i\ j)(i\ k)=(i\ k\ j)\quad (i\ j)(k\ l)=(i\ l\ k)(i\ j\ k).\]
		\item If $H\lhd A_n$ and $H$ contains a 3-cycle, then $H$ contains all 3 cycles. So by the previous note, $H=A_n$.
		\item Use a contradiction to the minimality of the number of elements of $\{1,2,\dots,n\}$ that are changed by $\tau\in H$.
	}
\end{block}
}
\end{document}

