\documentclass[t]{beamer}

\subtitle{Section 2.2: Groups}

\input{../../_tools/setup}

\begin{document} 
	\startdoc

	\topics{ 
		\item Binary Operations
		\item Monoids
	}{
		\item Groups
	}
\slide{
	\begin{defn}
		Suppose that 
		\enumarabic{
			\item $G$ is a set and $*$ is a binary operation on $G$,
			\item $*$ is associative,
			\item there is some $e\in G$ such that 
				\[g*e=e*g=g,\]
				for all $g\in G$, and
			\item for all $g\in G$, there is an $h\in G$ such that $g*h=e=h*g$.
		}
		Then $(G,*)$ is a \emph{GROUP}.
	\end{defn}
	\begin{nb}
		A group is a monoid where every element has an inverse!
	\end{nb}
}
\slide{
	\begin{exercise}
		Group or not?
		\begin{multicols}{2}
			\enumalph{\setlength\itemsep{.75em}
				\item $(\Z,+)$
				\item $(\R,+)$
				\item $(\mathbb{N},+)$
				\item $(\Z_n,+)$
				\item $(M_n(\R),+)$
				\item $(M_n(\R),\cdot)$
				\item $(\R,\cdot)$
				\item $(\Z,\cdot)$
				\item $(\Z_4,\cdot)$
				\item $(\Z_n,\cdot)$
				\item $(\Z_p,\cdot)$
				\item $(S_3,\circ)$
			}
		\end{multicols}
	\end{exercise}
}
\slide{
	\begin{defn}
		The \emph{$n$th roots of unity} are the complex numbers that are the roots of 
		\[x^n-1.\]
		Denote the set of roots as $\mathcal{U}_n$
	\end{defn}
}

\slide{\begin{defn}
		If the operation of a group $G$ is commutative, we call $G$ an \emph{abelian group}.
\end{defn}}

\slide{
	\begin{thm}{2.1.4}
		If $(G,*)$ is a group and $g\in G$, then the inverse of $g$ is unique.
	\end{thm}
}
\slide{
	\begin{thm}{2.2.1}
		If $(M,*)$ is a monoid, then the set of all units $M^\times$ is a group using the operation $*$, called the \emph{unit group}.
	\end{thm}
	\begin{exa}
		\bulletize{
			\setlength{\itemsep}{1em}
			\item $(\R,\cdot)$ has units $\R^\times=$
			\item $(\Z,\cdot)$ has units $\Z^\times=$
			\item $(\Z_4,\cdot)$ has units $\Z_4^\times=$
		}
	\end{exa}
}
\slide{
	\begin{thm}{2.2.2}
		If $G_1,G_2,\dots,G_n$ are groups with respective operations $*_1,*_2,\dots,*_n$, then 
		\[G_1\times G_2\times\cdots\times G_n\]
		is a group under component-wise operation
		\[(g_1,g_2,\dots,g_n)*(h_1,h_2,\dots,h_n)=(g_1*_1h_1,g_2*_2h_2,\dots,g_n*_nh_n).\]
	\end{thm}
}
\slide{
	\begin{thm}{2.2.3}
		Let $g,h,g_1,g_2,\dots,g_{n-1},g_n$ be elements of a group $G$ ($n\in \Z_{\geq 1}$).
		\enumarabic{
			\setlength{\itemsep}{2em}
			\item $e^{-1}=$
			\item $(g^{-1})^{-1}=$
			\item $(gh)^{-1}=$
			\item $(g_1g_2\cdots g_n)^{-1}=$
			\item $(g^m)^{-1}=\qquad\qquad\qquad\qquad$ for all $m\geq 0$.
		}
	\end{thm}
}
\slide{
	\begin{thm}{2.2.4 (Exponent Laws)}
		Let $G$ be a group and $g,h\in G$. Then for all $m,n\in\Z$, the following hold,
		\enumarabic{
			\setlength{\itemsep}{2em}
			\item $g^ng^m=$
			\item $(g^n)^m=$
			\item If $gh=hg$, then $(gh)^n=$
		}
	\end{thm}
}
\slide{
	\begin{thm}{2.2.5 (Cancellation Laws)}
		Let $G$ be a group and $g,h,f\in G$.
		\enumarabic{
			\item If $gh=gf$ then $h=f$ (\emph{left cancellation})
			\item If $hg=fg$ then $h=f$ (\emph{right cancellation})
		}
	\end{thm}
}
\slide{
	\begin{thm}{2.2.6}
		Let $G$ be a group and $g,h\in G$.
		\enumarabic{
			\item The equation $gx=h$ has a unique solution $x=g^{-1}h$ in $G$.
			\item The equation $xg=h$ has a unique solution $x=hg^{-1}$ in $G$.
		}
	\end{thm}
}
\slide{
	\begin{defn}
		A \emph{Cayley table} is essentially a multiplication table for a given binary operation.
	\end{defn}
	\begin{exa}
		$$\begin{array}{c||c|c|c}
			*&e&a&b\\\hline\hline
			e&e&a&b\\\hline
			a&a&a^2&a*b\\\hline
			b&b&b*a&b^2
		\end{array}$$
	\end{exa}
}
\slide{
	\begin{exercise}
		Complete the Table for $(\Z_3,+)$
		\setlength{\arraycolsep}{20pt}
		\renewcommand{\arraystretch}{1.5}
		$$\begin{array}{c||c|c|c}
			+&\overline{0}&\overline{1}&\overline{2}\\\hline\hline
			\overline{0}&&&\\\hline
			\overline{1}&&&\\\hline
			\overline{2}&&&
		\end{array}$$
	\end{exercise}
}
\slide{
	\begin{exercise}
		Consider operation $*$ with the Cayley table
		$$\begin{array}{c||c|c|c|c}
			*&e&a&b&c\\\hline\hline
			e&e&a&b&c\\\hline
			a&a&a&b&e\\\hline
			b&b&b&c&c\\\hline
			c&c&e&c&e
		\end{array}$$
		\enumalph{
			\setlength{\itemsep}{.1in}
			\item Is $*$ commutative? Why?
			\item Is there an identity?
			\item Determine $a*(b*c)$ and $(a*b)*c$.  Is $*$ associative?
			\item Is it a monoid?!
		}
	\end{exercise}
}
\slide{
	\begin{question}
		How many groups are there with 1 element?
		
		Since $e$ must be in EVERY group then $G=\{e\}$ is the \textit{only} group with one element.
	\end{question}
}
\slide{
	\begin{question}
		How many groups are there with 2 elements?
		
		Let $G=\{e,g\}$ be a group with 2 elements. Complete the Cayley table.
		\setlength{\arraycolsep}{20pt}
		\renewcommand{\arraystretch}{1.5}
		$$\begin{array}{c||c|c}
			*&e&g\\\hline\hline
			e&e&g\\\hline
			a&g&{{\color{white}\large e}\hskip .3in}	
		\end{array}$$
		
		(Recall that in a group, every element must have an inverse!)
	\end{question}
}
\slide{
	\begin{nb}
		Note that if $(G,*)$ is a group, every element appears exactly 1 time in every row and every column.
	\end{nb}
	\begin{exercise}
		If $G=\{e,g,h\}$ is a group, complete the Cayley table,
		\setlength{\arraycolsep}{20pt}
		\renewcommand{\arraystretch}{1.5}
		$$\begin{array}{c||c|c|c}
			*&e&g&h\\\hline\hline
			e&e&g&h\\\hline
			g&g&&\\\hline
			h&h&&
		\end{array}$$
	\end{exercise}
}
\slide{
	\begin{nb}
		\setstretch{1.5}
		We will say `up to isomorphism' there is only one group of order 3.  
		
		That is all of the following have the same structure.
		\enumalph{
			\item $(\Z_3,+)$ - equivalences modulo 3
			\item $(A_3,\circ)$ - the alternating group on 3 elements
			\item $(\mathcal{U}_n,\cdot)$ - the third roots of unity
		}
	\end{nb}
}
\slide{
	\begin{exercise}
		What about a group with 4 elements, is there 1 or are there more?
		
		\setlength{\arraycolsep}{20pt}
		\renewcommand{\arraystretch}{1.5}
		$$\begin{array}{c||c|c|c|c}
			*&e&g&h&k\\\hline\hline
			e&e&g&h&k\\\hline
			g&g&&&\\\hline
			h&h&&&\\\hline
			k&k&&&
		\end{array}$$
		
	\end{exercise}
}
\slide{
	\begin{nb}\setstretch{2}
		There are actually two basic structures for groups with 4 elements.
		\begin{itemize}[label=--]
			\item One where $g^2=e$ for every $g\in G$, as with $\Z_2\times \Z_2$.
			
			\item One with some $g\in G$ for which $g,g^2,g^3\neq e$ and $g^4=e$. For example $\Z_4$ or $\mathcal{U}_4=\{1,-1,i,-i\}$.
		\end{itemize}
	\end{nb}
}

\end{document}

