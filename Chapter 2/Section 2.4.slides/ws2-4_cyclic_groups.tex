\documentclass[t]{beamer}

\subtitle{Section 2.4: Cyclic Groups and the Order of an Element}

\usepackage{amsthm,amsmath,amsfonts,hyperref,graphicx,color,multicol,soul}
\usepackage{enumitem,tikz,tikz-cd,setspace,mathtools}

%%%%%%%%%%
%Beamer Template Customization
%%%%%%%%%%
\setbeamertemplate{navigation symbols}{}
\setbeamertemplate{theorems}[ams style]
\setbeamertemplate{blocks}[rounded]

\definecolor{Blu}{RGB}{43,62,133} % UWEC Blue
\setbeamercolor{structure}{fg=Blu} % Titles

%Unnumbered footnotes:
\newcommand{\blfootnote}[1]{%
	\begingroup
	\renewcommand\thefootnote{}\footnote{#1}%
	\addtocounter{footnote}{-1}%
	\endgroup
}

%%%%%%%%%%
%TikZ Stuff
%%%%%%%%%%
\usetikzlibrary{arrows}
\usetikzlibrary{shapes.geometric}
\tikzset{
	smaller/.style={
		draw,
		regular polygon,
		regular polygon sides=3,
		fill=white,
		node distance=2cm,
		minimum height=1in,
		line width = 2pt
	}
}
\tikzset{
	smsquare/.style={
		draw,
		regular polygon,
		regular polygon sides=4,
		fill=white,
		node distance=2cm,
		minimum height=1in,
		line width = 2pt
	}
}


%%%%%%%%%%
%Custom Commands
%%%%%%%%%%

\newcommand{\C}{\mathbb{C}}
\newcommand{\quats}{\mathbb{H}}
\newcommand{\N}{\mathbb{N}}
\newcommand{\Q}{\mathbb{Q}}
\newcommand{\R}{\mathbb{R}}
\newcommand{\Z}{\mathbb{Z}}

\newcommand{\ds}{\displaystyle}

\newcommand{\fn}{\insertframenumber}

\newcommand{\id}{\operatorname{id}}
\newcommand{\im}{\operatorname{im}}
\newcommand{\Aut}{\operatorname{Aut}}
\newcommand{\Inn}{\operatorname{Inn}}

\newcommand{\blank}[1]{\underline{\hspace*{#1}}}

\newcommand{\abar}{\overline{a}}
\newcommand{\bbar}{\overline{b}}
\newcommand{\cbar}{\overline{c}}

\newcommand{\nml}{\unlhd}

%%%%%%%%%%
%Custom Theorem Environments
%%%%%%%%%%
\theoremstyle{definition}
\newtheorem{exercise}{Exercise}
\newtheorem{question}[exercise]{Question}
\newtheorem{warmup}{Warm-Up}
\newtheorem*{defn}{Definition}
\newtheorem*{exa}{Example}
\newtheorem*{disc}{Group Discussion}
\newtheorem*{nb}{Note}
\newtheorem*{recall}{Recall}
\renewcommand{\emph}[1]{{\color{blue}\texttt{#1}}}

\definecolor{Gold}{RGB}{237, 172, 26}
%Statement block
\newenvironment{statementblock}[1]{%
	\setbeamercolor{block body}{bg=Gold!20}
	\setbeamercolor{block title}{bg=Gold}
	\begin{block}{\textbf{#1.}}}{\end{block}}
\newenvironment{thm}[1]{%
	\setbeamercolor{block body}{bg=Gold!20}
	\setbeamercolor{block title}{bg=Gold}
	\begin{block}{\textbf{Theorem #1.}}}{\end{block}}


%%%%%%%%%%
%Custom Environment Wrappers
%%%%%%%%%%
\newcommand{\enumarabic}[1]{
	\begin{enumerate}[label=\textbf{\arabic*.}]
		#1
	\end{enumerate}
}
\newcommand{\enumalph}[1]{
	\begin{enumerate}[label=(\alph*)]
		#1
	\end{enumerate}
}
\newcommand{\bulletize}[1]{
	\begin{itemize}[label=$\bullet$]
		#1
	\end{itemize}
}
\newcommand{\circtize}[1]{
	\begin{itemize}[label=$\circ$]
		#1
	\end{itemize}
}
\newcommand{\slide}[1]{
	\begin{frame}{\fn}
		#1
	\end{frame}
}
\newcommand{\slidec}[1]{
\begin{frame}[c]{\fn}
	#1
\end{frame}
}
\newcommand{\slidet}[2]{
	\begin{frame}{\fn\ - #1}
		#2
	\end{frame}
}


\newcommand{\startdoc}{
		\title{Math 425: Abstract Algebra 1}
		\author{Mckenzie West}
		\date{Last Updated: \today}
		\begin{frame}
			\maketitle
		\end{frame}
}

\newcommand{\topics}[2]{
	\begin{frame}{\insertframenumber}
		\begin{block}{\textbf{Last Section.}}
			\begin{itemize}[label=--]
				#1
			\end{itemize}
		\end{block}
		\begin{block}{\textbf{This Section.}}
			\begin{itemize}[label=--]
				#2
			\end{itemize}
		\end{block}
	\end{frame}
}

\begin{document} 
	\startdoc

\topics{
		\item Cyclic Subgroups
	}{
		\item More about cyclic groups
		\item The order of an element
	}

\slide{
	\begin{thm}{2.4.1}
		Let $g$ be an element of a group $G$, and write
			\[\langle g\rangle =\{g^k:k\in\Z\}.\]
		Then $\langle g\rangle$ is a subgroup of $G$, and $\langle g\rangle\subseteq H$ for every subgroup $H$ of $G$ with $g\in H$.
	\end{thm}
}
\slide{
	\begin{exa}
		$\langle \overline{2}\rangle\leq (\Z_5,+)$
	\end{exa}
	\vskip .75in
	\begin{exa}
		$\langle i\rangle\leq (\C\setminus\{0\},\cdot)$
	\end{exa}
}
\slide{
	\begin{defn}
		A group $G$ is \emph{cyclic} if there is some $g\in G$ for which $G=\langle g\rangle$.
	\end{defn}
}
\slide{
	\begin{exercise}
		Is $\Z_5^\times=\{\overline{1},\overline{2},\overline{3},\overline{4}\}$ cyclic?\vskip 1in\mbox{}
	\end{exercise}
	\begin{exercise}
		Find some $n$ for which $\Z_n^\times$ is not cyclic.  (Recall $\overline{a}\in\Z_n$ is in $\Z_n^\times$ if and only if $\gcd(a,n)=1$.)
	\end{exercise}
}
\slide{
	\begin{defn}
		If $G$ is a finite group, the \emph{order of a group} $G$ is denoted $|G|$ and is the cardinality of the set $G$.
		
		The \emph{order of an element} $g\in G$ is denoted $|g|$ or $o(g)$ and equals the smallest positive integer $n$ such that $g^n=e$.
	\end{defn}
%}
%\slide{
	\begin{exercise}
		\enumalph{\setlength{\itemsep}{2em}
		\item $|\Z_{10}|=$
		\item Using $\overline{2}\in\Z_{10}$, $|\overline{2}|=$	
		\item $|\Z_8^\times|=$
		\item Using $\overline{3}\in \Z_8^\times$, $|\overline{3}|=$
	}
	\end{exercise}
}
\slide{
	\begin{thm}{2.4.2}
		Let $g\in G$ with $o(g)=n$.  Then
		\enumarabic{
			\setlength{\itemsep}{2em}
			\item $g^k=1$ if and only if $n|k$.
			\item $g^k=g^m$ if and only if $k\equiv m\pmod n$
			\item $\langle g\rangle=\{1,g,g^2,\dots, g^{n-1}\}$ where $1,g,g^2,\dots,g^{n-1}$ are all distinct.
		}
	\end{thm}
	* The proof is in the textbook, and uses laws of exponents.
}
\slide{
	\begin{thm}{2.4.3}
		Let $G$ be a group and let $g\in G$ satisfy $o(g)=\infty$.  Then 
			\enumarabic{
				\setlength{\itemsep}{2em}
				\item $g^k=1$ if and only if $k=0$.
				\item $g^k=g^m$ if and only if $k=m$.
				\item $\langle g\rangle =\{\dots,g^{-2},g^{-1},1,g,g^2,\dots\}$ where the $g^i$ are distinct.
			}
	\end{thm}
}
\slide{
	\begin{statementblock}{Corollary}
		For all $g$ in a group $G$, $o(g)=|\langle g\rangle|$.
	\end{statementblock}
	\begin{nb}
		\begin{itemize}[label=$\bullet$]
			\item The identity is the only element of order 1 in a group.
			\item $o(g)=o(g^{-1})$ for all $g\in G$
		\end{itemize}
	\end{nb}
}
\slide{
	\begin{statementblock}{Order in $\Z_n$}
		Given $\overline{a}\in(\Z_n,+)$, with $1\leq a\leq n-1$, 
			\[|\overline{a}|=\frac{n}{\gcd(a,n)}.\]
	\end{statementblock}
	\begin{statementblock}{Order in $\Z_n^\times$}
		There is no formula....... In section 2.6 we'll see that $|g|$ divides $|G|$ if $|G|$ is finite.
	\end{statementblock}
}
\slide{
	\begin{block}{\textbf{Order of an element in $S_n$.}}
		\begin{statementblock}{Theorem}
			If $\gamma=(k_1\ k_2\ \dots\ k_r)$ is an $r$-cycle in $S_n$, then $|\gamma|=r$.
		\end{statementblock}
		* I proved $(1\ 2\ 3\ \dots\ n)$ has order $n$ on homework 3, how might you extend this proof to work for any cycle?
		
		\begin{thm}{2.4.4}
			If $\gamma=\sigma_1\sigma_2\dots\sigma_r$ where $\sigma_i$ are disjoint cycles, then
				\[|\gamma|=\operatorname{lcm}(|\sigma_1|,|\sigma_2|,\dots,|\sigma_r|).\]
		\end{thm}
	\end{block}
}
\slide{\begin{exercise}
		Let $\sigma=(1\ 2\ 3)$ and $\tau=(1\ 5\ 3\ 4\ 2)$.
		
		Then \enumarabic{
			\setlength{\itemsep}{.5in}
			\item $o(\sigma)=$
			\item $o(\tau)=$
			\item $o(\sigma\tau)=$
		}
\end{exercise}}
\slide{
	\begin{thm}{2.4.6}
		Every cyclic group is abelian, but the converse does not hold.
	\end{thm}
	\vskip .25in
%	\vskip 2in\mbox{}
%}
%\slide{
	\begin{thm}{2.4.7}
		Every subgroup of a cyclic group is cyclic.
	\end{thm}
	\vskip .25in
%\vskip 2in\mbox{}
%}
%\slide{
\begin{thm}{2.4.8}
	Let $G=\langle g\rangle$ be a cyclic group, where $o(g)=n$. Then $G=\langle g^k\rangle$ if and only if $\gcd(k,n)=1$.
\end{thm}
}
\slide{
	\begin{statementblock}{The Fundamental Theorem of Finite Cyclic Groups (Theorem 2.4.9)}
		Let $G=\langle g\rangle$ be a cyclic group of order $n$.
		\enumarabic{
			\item If $H$ is a subgroup of $G$, then $H=\langle g^d\rangle$ for some $d|n$. Hence $|H|$ divides $n$.
			\item Conversely if $k|n$, then $\langle g^{n/k}\rangle$ is the unique subgroup of $G$ of order $k$.
		}
	\end{statementblock}
}
\begin{frame}[fragile]
\frametitle{\fn}
\begin{exa}
	The subgroup lattice of $C_{12}=\{1,a,a^2,\dots,a^{11}\}$:
	\begin{center}
		\begin{tikzcd}
		&&\langle a\rangle\arrow[-]{dl} \arrow[-]{dr}\\
		&\langle a^2\rangle\arrow[-]{dl} \arrow[-]{dr}&&\langle a^3\rangle\arrow[-]{dl}\\
		\langle a^4\rangle\arrow[-]{dr}&&\langle a^6\rangle\arrow[-]{dl}\\
		&\langle 1\rangle
		\end{tikzcd}
	\end{center}
\end{exa}
\end{frame}
\begin{frame}[fragile]
	\frametitle{\fn}
	\begin{exa}
		The subgroup lattice of $\Z_{12}=\langle\overline{1}\rangle$:
		\begin{center}
			\begin{tikzcd}
			&&\langle 1\rangle\arrow[-]{dl} \arrow[-]{dr}\\
			&\langle\overline{2}\rangle\arrow[-]{dl} \arrow[-]{dr}&&\langle\overline{3}\rangle\arrow[-]{dl}\\
			\langle\overline{4}\rangle\arrow[-]{dr}&&\langle\overline{6}\rangle\arrow[-]{dl}\\
			&\langle\overline{0}\rangle
		\end{tikzcd}
		\end{center}
	\end{exa}
\end{frame}

\slide{
	\begin{block}{\textbf{Groups Generated by Several Elements}}
		\begin{exa}
			The symmetric group $S_3$:
			From the homework:
				\[S_3=\{e,\sigma,\sigma^2,\tau,\tau\sigma,\tau\sigma^2\},\]
			where $|\sigma|=3$, $|\tau|=2$ and $\sigma\tau=\tau\sigma^2$.
			
			Then $S_3=\langle \sigma,\tau\rangle$.
		\end{exa}
	\end{block}
}
\slide{
	\begin{exa}
		The Klein-4 Group $K_4$:
		$$\Z_2\times \Z_2=\langle(\overline{0},\overline{1}),(\overline{1},\overline{0})\rangle$$
	\end{exa}
}
\slide{
	\begin{defn}
		In general, if $X$ is a nonempty subset of a group $G$, then the \emph{subgroup of $G$ generated by $X$} is defined as	
		\[\begin{array}{rcl}
		\langle X\rangle &=& \{\text{products of powers (not nec. distinct) of elements of X}\}\\
		&=&\{x_1^{k_1}x_2^{k_2}\cdots x_m^{k_m}\ |\ x_i\in X,\ k_i\in \Z,\ m\geq 1\}
		\end{array}\]
	We will always have $\langle X\rangle\leq G$.
	\end{defn}
}
\end{document}

