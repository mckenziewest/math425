\documentclass[12pt]{article}

\newcommand{\secname}{Section 2.9: Factor Groups}

\usepackage{amsthm,amsmath,amsfonts,hyperref,graphicx,color,multicol,soul}
\usepackage{enumitem,tikz,tikz-cd,setspace,mathtools}
\usepackage{colortbl}
\usepackage[margin=1in]{geometry}

%%%%%%%%%%
%Color Customization
%%%%%%%%%%

\definecolor{Blu}{RGB}{43,62,133} % UWEC Blue

%Unnumbered footnotes:
\newcommand{\blfootnote}[1]{%
	\begingroup
	\renewcommand\thefootnote{}\footnote{#1}%
	\addtocounter{footnote}{-1}%
	\endgroup
}

%%%%%%%%%%
%TikZ Stuff
%%%%%%%%%%
\usetikzlibrary{arrows}
\usetikzlibrary{shapes.geometric}
\tikzset{
	smaller/.style={
		draw,
		regular polygon,
		regular polygon sides=3,
		fill=white,
		node distance=2cm,
		minimum height=1in,
		line width = 2pt
	}
}
\tikzset{
	smsquare/.style={
		draw,
		regular polygon,
		regular polygon sides=4,
		fill=white,
		node distance=2cm,
		minimum height=1in,
		line width = 2pt
	}
}

%%%%%%%%%%
%Listing Setup
%%%%%%%%%%
\usepackage{listings}
\usepackage{caption, floatrow, makecell}%
\captionsetup{labelfont = sc}
\setcellgapes{3pt}

\definecolor{backcolour}{RGB}{237,236,230}
\definecolor{myblue}{RGB}{42,157,189}

\lstdefinestyle{mystyle}{
	language=Python,
	keywords=[2]{sage:},
	alsodigit={:,.,<,>},
	backgroundcolor=\color{backcolour},   
	commentstyle=\color{myblue},
	keywordstyle=\bfseries\color{Green},
	keywordstyle=[2]\color{purple},
	numberstyle=\tiny\color{Gray},
	stringstyle=\color{Orange},
	basicstyle=\ttfamily\footnotesize,
	breakatwhitespace=false,         
	breaklines=true,                 
	captionpos=b,                    
	keepspaces=true,                   
	showspaces=false,                
	showstringspaces=false,
	showtabs=false,                  
	tabsize=2
}

\lstset{style=mystyle}


%%%%%%%%%%
%Custom Commands
%%%%%%%%%%

\newcommand{\C}{\mathbb{C}}
\newcommand{\quats}{\mathbb{H}}
\newcommand{\N}{\mathbb{N}}
\newcommand{\Q}{\mathbb{Q}}
\newcommand{\R}{\mathbb{R}}
\newcommand{\Z}{\mathbb{Z}}

\newcommand{\ds}{\displaystyle}

\newcommand{\fn}{\insertframenumber}

\newcommand{\id}{\operatorname{id}}
\newcommand{\im}{\operatorname{im}}
\newcommand{\lcm}{\operatorname{lcm}}
\newcommand{\ord}{\operatorname{ord}}
\newcommand{\Aut}{\operatorname{Aut}}
\newcommand{\Inn}{\operatorname{Inn}}

\newcommand{\blank}[1]{\underline{\hspace*{#1}}}

\newcommand{\abar}{\overline{a}}
\newcommand{\bbar}{\overline{b}}
\newcommand{\cbar}{\overline{c}}

\newcommand{\nml}{\unlhd}

%%%%%%%%%%
%Custom Theorem Environments
%%%%%%%%%%
\theoremstyle{definition}
\newtheorem{exercise}{Exercise}
\newtheorem{question}[exercise]{Question}
\newtheorem{warmup}{Warm-Up}
\newtheorem*{exa}{Example}
\newtheorem*{defn}{Definition}
\newtheorem*{disc}{Group Discussion}
\newtheorem*{recall}{Recall}
\renewcommand{\emph}[1]{{\color{blue}\texttt{#1}}}

\definecolor{Gold}{RGB}{237, 172, 26}
%Statement block
%\newenvironment{statementblock}[1]{%
%	\setbeamercolor{block body}{bg=Gold!20}
%	\setbeamercolor{block title}{bg=Gold}
%	\begin{block}{\textbf{#1.}}}{\end{block}}
%\newenvironment{goldblock}{%
%	\setbeamercolor{block body}{bg=Gold!20}
%	\setbeamercolor{block title}{bg=Gold}
%	\setbeamertemplate{blocks}[shadow=true]
%	\begin{block}{}}{\end{block}}
%\newenvironment{defn}{%
%	\setbeamercolor{block body}{bg=gray!20}
%	\setbeamercolor{block title}{bg=violet, fg=white}
%	\setbeamertemplate{blocks}[shadow=true]
%	\begin{block}{\textbf{Definition.}}}{\end{block}}
%\newenvironment{nb}{%
%	\setbeamercolor{block body}{bg=gray!20}
%	\setbeamercolor{block title}{bg=teal, fg=white}
%	\setbeamertemplate{blocks}[shadow=true]
%	\begin{block}{\textbf{Note.}}}{\end{block}}
%\newenvironment{blockexample}{%
%	\setbeamercolor{block body}{bg=gray!20}
%	\setbeamercolor{block title}{bg=Blu, fg=white}
%	\setbeamertemplate{blocks}[shadow=true]
%	\begin{block}{\textbf{Example.}}}{\end{block}}
%\newenvironment{blocknonexample}{%
%	\setbeamercolor{block body}{bg=gray!20}
%	\setbeamercolor{block title}{bg=purple, fg=white}
%	\setbeamertemplate{blocks}[shadow=true]
%	\begin{block}{\textbf{Non-Example.}}}{\end{block}}
%\newenvironment{thm}[1]{%
%	\setbeamercolor{block body}{bg=Gold!20}
%	\setbeamercolor{block title}{bg=Gold}
%	\begin{block}{\textbf{Theorem #1.}}}{\end{block}}


%%%%%%%%%%
%Custom Environment Wrappers
%%%%%%%%%%
\newcommand{\exer}[1]{
	\begin{exercise}
	#1
	\end{exercise}
}
\newcommand{\exam}[1]{
\textbf{Example: }
	#1
}
\newcommand{\nexam}[1]{
	\textbf{Non-Example: }
	#1
}
\newcommand{\enumarabic}[1]{
	\begin{enumerate}[label=\textbf{\arabic*.}]
		#1
	\end{enumerate}
}
\newcommand{\enumalph}[1]{
	\begin{enumerate}[label=(\alph*)]
		#1
	\end{enumerate}
}
\newcommand{\bulletize}[1]{
	\begin{itemize}[label=$\bullet$]
		#1
	\end{itemize}
}
\newcommand{\circtize}[1]{
	\begin{itemize}[label=$\circ$]
		#1
	\end{itemize}
}
%\newcommand{\slide}[1]{
%	\begin{frame}{\fn}
%		#1
%	\end{frame}
%}
%\newcommand{\slidec}[1]{
%\begin{frame}[c]{\fn}
%	#1
%\end{frame}
%}
%\newcommand{\slidet}[2]{
%	\begin{frame}{\fn\ - #1}
%		#2
%	\end{frame}
%}


\setlength{\parindent}{0pt}



\usepackage{afterpage}
\usepackage{fancyhdr}

\fancyhead[L]{\textbf{Math 425: Abstract Algebra I\\\secname}}
\fancyhead[R]{\textbf{Mckenzie West\\Last Updated: \today}}
\pagestyle{fancy}

\newcommand{\startdoc}{}

\newcommand{\topics}[2]{
		{\textbf{Previously.}}
			\begin{itemize}[label=--]
				#1
			\end{itemize}
		{\textbf{This Section.}}
			\begin{itemize}[label=--]
				#2
			\end{itemize}
}

\begin{document} 
	\startdoc
	
	\topics{
		\item Normal Subgroups
		\item Products of Groups
		\item Simple Groups
	}{
		\item Factor Groups
		\item Commutator Subgroups
	}

\begin{block}{Goal}
	Extend this idea of ``modular arithmetic'' to other groups.
\end{block}
\slide{
\begin{exercise}
	Consider $G=\Z$ and $H=4\Z$.
	
	\enumarabic{
		\item First, find all right cosets of $H$.\vskip 1in
		\item We define the sum of the cosets as follows:
		\[(a+H)+(b+H)=\{m +n\ |\ m\in a+H\text{ and }n\in b+H\}.\]
		
		Verify that $(a+H) + (b+H)=(a+b)+H$ for all $a,b\in \Z$.
		
	}
	\vskip 1in \mbox{}
\end{exercise}
}


\slide{
	Notice, if $G=\Z$ and $H=4\Z$, then
		\bulletize{
			\item $(a+H) = \overline{a} \in \Z_4$,
			\item $(a+H)+(b+H)=\overline{a}+\overline{b}=\overline{a+b}=(a+b)+H$
		}
}
\slide{
	\begin{statementblock}{Lemma}
		The following conditions are equivalent for a subgroup $K$ of $G$.
		\enumarabic{
			\item $K$ is normal in $G$
			\item $aK*bK=(a*b)K$ is a well-defined operation of left cosets.
		}
	\end{statementblock}
}
\newpage
\slide{
	\begin{thm}{2.9.1}
		Let $K\nml G$ and write $G/K=\{aK\ |\ a\in G\}$, the set of left cosets of $K$. Then
		\enumarabic{
			\item $G/K$ is a group under the operation $(aK)(bK)=(ab)K$.
			\item The mapping $\varphi: G\to G/K$ defined by $\varphi(a)=aK$ is an onto homorphism.
			\item If $G$ is abelian, then $G/K$ is abelian.
			\item If $G=\langle a\rangle$, then $G/K$ is also cyclic with $G/K=\langle Ka\rangle$.
			\item If $|G:K|$ is finite then $|G/K|=|G:K|$.  If $|G|$ is finite, then $|G/K|=\frac{|G|}{|K|}$.
		}
	\end{thm}
}
\slide{
\vskip 3in
}
\slide{
	\begin{defn}
		If $K$ is a normal subgroup of the group $G$, then the group $G/K$ is called the \emph{factor group}, or \emph{quotient group}, of $G$ by $K$.
		
		We call the homomorphism $\varphi:G\to G/K$ with $\varphi(a)=Ka$ the \emph{coset map}.
	\end{defn}
	\begin{exa} The ``trivial'' examples:
		\enumalph{
			\item $G/G = \{G\}$
			\item $G/\{e_G\}=\{\{a\}\ |\ a\in G\}\cong G$
		}
	\end{exa}
}
\slide{
	\begin{exercise}
		Consider the group $G=S_3$ and $K=\{\varepsilon,(1\ 2\ 3),(1\ 3\ 2)\}$.
	\end{exercise}
}
\newpage
\slide{
	\begin{exercise}
		Consider the group $G=S_4$ and $K=\{\varepsilon,(1\ 2)(3\ 4),(1\ 3)(2\ 4),(1\ 4)(2\ 3)\}$.
		\vskip 2in
	\end{exercise}
}
\slide{
	\begin{exercise}
		Consider the group $G=\{...,\frac{1}{4},\frac{1}{2},1,2,4,8,16,\dots\}$ under the operation multiplication and $K=\langle 8\rangle$.
		\vskip 2in
	\end{exercise}
}
\slide{
	\begin{exercise}
		Consider $G=D_4=\{e,r,r^2,r^3,f,fr,fr^2,fr^3\}$. 
		\enumalph{
			\item Show that $Z=Z(G)=\{e,r^2\}$.\vskip 2em
			\item Certify $Z\nml D_4$.\vskip 2em
			\item Determine the cosets of $Z$ in $D_4$.\vskip 2em
			\item Complete the Cayley table for $D_4/Z$.\vskip 2em
			\item What is the group $D_4/Z$?
		}
	\end{exercise}
}
\newpage
\slide{
	\begin{thm}{2.9.2}
		If $G$ is a group and $G/Z(G)$ is cyclic, then $G$ is abelian.
	\end{thm}
	\vfill
}
\slide{
	\begin{exercise}
		In general, how do we know if $G/H$ is abelian?
		That is, explore what it means for $HaHb=HbHa$.
		\vfill		
	\end{exercise}
}
\slide{
	\begin{defn}
		For $a,b\in G$ we define the \emph{commutator} of $a$ and $b$ to be
			\[[a,b]=aba^{-1}b^{-1}.\]
	\end{defn}
	\begin{nb}
		If $G$ is abelian, then for all $a,b\in G$, $[a,b]=e_G$.
	\end{nb}
	\begin{statementblock}{Fact}
		If $H\nml G$, then $G/H$ is abelian if and only if $H$ contains every commutator.
	\end{statementblock}
}
\slide{
	\begin{defn}
		The \emph{commutator subgroup} of $G$ is the group
			\[\begin{array}{rcl}
			G'&=&\{\text{all finite products of commutators from }G\}\\\\
			&=&\langle [a,b]\ |\ a,b\in G\rangle.
			\end{array}\]
	\end{defn}
	\begin{nb}
		$[a,b]^{-1}=[b,a]$
	\end{nb}
}
\slide{
	\begin{thm}{2.9.3}
		Let $G$ be a group and let $H$ be a subgroup of $G$.
		\enumarabic{
			\item $G'$ is a normal subgroup of $G$ and $G/G'$ is abelian.
			\item $G'\subseteq H$ if and only if $H$ is normal in $G$ and $G/H$ is abelian.
		}
	\end{thm}
}
\slide{
	\begin{exercise}
		 Compute $D'_4$.\vfill
	\end{exercise}
}
\end{document}

