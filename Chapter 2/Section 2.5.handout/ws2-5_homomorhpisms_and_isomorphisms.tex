\documentclass[12pt]{article}

\newcommand{\secname}{Section 2.5: Homomorphisms and Isomorphisms}

\usepackage{amsthm,amsmath,amsfonts,hyperref,graphicx,color,multicol,soul}
\usepackage{enumitem,tikz,tikz-cd,setspace,mathtools}
\usepackage{colortbl}
\usepackage[margin=1in]{geometry}

%%%%%%%%%%
%Color Customization
%%%%%%%%%%

\definecolor{Blu}{RGB}{43,62,133} % UWEC Blue

%Unnumbered footnotes:
\newcommand{\blfootnote}[1]{%
	\begingroup
	\renewcommand\thefootnote{}\footnote{#1}%
	\addtocounter{footnote}{-1}%
	\endgroup
}

%%%%%%%%%%
%TikZ Stuff
%%%%%%%%%%
\usetikzlibrary{arrows}
\usetikzlibrary{shapes.geometric}
\tikzset{
	smaller/.style={
		draw,
		regular polygon,
		regular polygon sides=3,
		fill=white,
		node distance=2cm,
		minimum height=1in,
		line width = 2pt
	}
}
\tikzset{
	smsquare/.style={
		draw,
		regular polygon,
		regular polygon sides=4,
		fill=white,
		node distance=2cm,
		minimum height=1in,
		line width = 2pt
	}
}

%%%%%%%%%%
%Listing Setup
%%%%%%%%%%
\usepackage{listings}
\usepackage{caption, floatrow, makecell}%
\captionsetup{labelfont = sc}
\setcellgapes{3pt}

\definecolor{backcolour}{RGB}{237,236,230}
\definecolor{myblue}{RGB}{42,157,189}

\lstdefinestyle{mystyle}{
	language=Python,
	keywords=[2]{sage:},
	alsodigit={:,.,<,>},
	backgroundcolor=\color{backcolour},   
	commentstyle=\color{myblue},
	keywordstyle=\bfseries\color{Green},
	keywordstyle=[2]\color{purple},
	numberstyle=\tiny\color{Gray},
	stringstyle=\color{Orange},
	basicstyle=\ttfamily\footnotesize,
	breakatwhitespace=false,         
	breaklines=true,                 
	captionpos=b,                    
	keepspaces=true,                   
	showspaces=false,                
	showstringspaces=false,
	showtabs=false,                  
	tabsize=2
}

\lstset{style=mystyle}


%%%%%%%%%%
%Custom Commands
%%%%%%%%%%

\newcommand{\C}{\mathbb{C}}
\newcommand{\quats}{\mathbb{H}}
\newcommand{\N}{\mathbb{N}}
\newcommand{\Q}{\mathbb{Q}}
\newcommand{\R}{\mathbb{R}}
\newcommand{\Z}{\mathbb{Z}}

\newcommand{\ds}{\displaystyle}

\newcommand{\fn}{\insertframenumber}

\newcommand{\id}{\operatorname{id}}
\newcommand{\im}{\operatorname{im}}
\newcommand{\lcm}{\operatorname{lcm}}
\newcommand{\ord}{\operatorname{ord}}
\newcommand{\Aut}{\operatorname{Aut}}
\newcommand{\Inn}{\operatorname{Inn}}

\newcommand{\blank}[1]{\underline{\hspace*{#1}}}

\newcommand{\abar}{\overline{a}}
\newcommand{\bbar}{\overline{b}}
\newcommand{\cbar}{\overline{c}}

\newcommand{\nml}{\unlhd}

%%%%%%%%%%
%Custom Theorem Environments
%%%%%%%%%%
\theoremstyle{definition}
\newtheorem{exercise}{Exercise}
\newtheorem{question}[exercise]{Question}
\newtheorem{warmup}{Warm-Up}
\newtheorem*{exa}{Example}
\newtheorem*{defn}{Definition}
\newtheorem*{disc}{Group Discussion}
\newtheorem*{recall}{Recall}
\renewcommand{\emph}[1]{{\color{blue}\texttt{#1}}}

\definecolor{Gold}{RGB}{237, 172, 26}
%Statement block
%\newenvironment{statementblock}[1]{%
%	\setbeamercolor{block body}{bg=Gold!20}
%	\setbeamercolor{block title}{bg=Gold}
%	\begin{block}{\textbf{#1.}}}{\end{block}}
%\newenvironment{goldblock}{%
%	\setbeamercolor{block body}{bg=Gold!20}
%	\setbeamercolor{block title}{bg=Gold}
%	\setbeamertemplate{blocks}[shadow=true]
%	\begin{block}{}}{\end{block}}
%\newenvironment{defn}{%
%	\setbeamercolor{block body}{bg=gray!20}
%	\setbeamercolor{block title}{bg=violet, fg=white}
%	\setbeamertemplate{blocks}[shadow=true]
%	\begin{block}{\textbf{Definition.}}}{\end{block}}
%\newenvironment{nb}{%
%	\setbeamercolor{block body}{bg=gray!20}
%	\setbeamercolor{block title}{bg=teal, fg=white}
%	\setbeamertemplate{blocks}[shadow=true]
%	\begin{block}{\textbf{Note.}}}{\end{block}}
%\newenvironment{blockexample}{%
%	\setbeamercolor{block body}{bg=gray!20}
%	\setbeamercolor{block title}{bg=Blu, fg=white}
%	\setbeamertemplate{blocks}[shadow=true]
%	\begin{block}{\textbf{Example.}}}{\end{block}}
%\newenvironment{blocknonexample}{%
%	\setbeamercolor{block body}{bg=gray!20}
%	\setbeamercolor{block title}{bg=purple, fg=white}
%	\setbeamertemplate{blocks}[shadow=true]
%	\begin{block}{\textbf{Non-Example.}}}{\end{block}}
%\newenvironment{thm}[1]{%
%	\setbeamercolor{block body}{bg=Gold!20}
%	\setbeamercolor{block title}{bg=Gold}
%	\begin{block}{\textbf{Theorem #1.}}}{\end{block}}


%%%%%%%%%%
%Custom Environment Wrappers
%%%%%%%%%%
\newcommand{\exer}[1]{
	\begin{exercise}
	#1
	\end{exercise}
}
\newcommand{\exam}[1]{
\textbf{Example: }
	#1
}
\newcommand{\nexam}[1]{
	\textbf{Non-Example: }
	#1
}
\newcommand{\enumarabic}[1]{
	\begin{enumerate}[label=\textbf{\arabic*.}]
		#1
	\end{enumerate}
}
\newcommand{\enumalph}[1]{
	\begin{enumerate}[label=(\alph*)]
		#1
	\end{enumerate}
}
\newcommand{\bulletize}[1]{
	\begin{itemize}[label=$\bullet$]
		#1
	\end{itemize}
}
\newcommand{\circtize}[1]{
	\begin{itemize}[label=$\circ$]
		#1
	\end{itemize}
}
%\newcommand{\slide}[1]{
%	\begin{frame}{\fn}
%		#1
%	\end{frame}
%}
%\newcommand{\slidec}[1]{
%\begin{frame}[c]{\fn}
%	#1
%\end{frame}
%}
%\newcommand{\slidet}[2]{
%	\begin{frame}{\fn\ - #1}
%		#2
%	\end{frame}
%}


\setlength{\parindent}{0pt}



\usepackage{afterpage}
\usepackage{fancyhdr}

\fancyhead[L]{\textbf{Math 425: Abstract Algebra I\\\secname}}
\fancyhead[R]{\textbf{Mckenzie West\\Last Updated: \today}}
\pagestyle{fancy}

\newcommand{\startdoc}{}

\newcommand{\topics}[2]{
		{\textbf{Previously.}}
			\begin{itemize}[label=--]
				#1
			\end{itemize}
		{\textbf{This Section.}}
			\begin{itemize}[label=--]
				#2
			\end{itemize}
}

\begin{document} 
	\startdoc
	
	\topics{
		\item Subgroups generated by one or more elements of a group
		\item Cyclic groups
		\item The order of an element
		\item Subgroup lattices
	}{
		\item Mappings between groups
		\item Homomorphisms
		\item Isomorphisms
		\item Image
		\item Kernel
		\item Automorphisms
	}


\slide{
	\begin{statementblock}{\textbf{Goal}}
		Study ``sensible'' functions from one group to another.
	\end{statementblock}
%	\begin{question}
%		What makes a function between groups ``sensible''?
%		\vskip 1in
%%		
%%		Considering $\phi\colon(G,*)\to(H,\diamond)$, if $g_1*g_2=g\in G$, then how should $\phi(g_1), \phi(g_2)$ and $\phi(g)\in H$ relate?
%	\end{question}
}
\slide{
	\begin{defn}
		Let $(G,*)$ and $(H,\diamond)$ be groups. Then a mapping $\phi\colon G\to H$ is a \emph{[group] homomorphism} if $\phi(g_1*g_2)=\phi(g_1)\diamond\phi(g_2)$ for all $g_1,g_2\in G$.
	\end{defn}
	\begin{exercise}
		Consider the groups $\Z$ and $2\Z$ and the map
		$\phi:\Z\to2\Z$ defined by $\phi(n)=-2n$ for all $n\in\Z$.
		
		Show that $\phi$ is a homomorphism.
	\end{exercise}
	\begin{exa}
		The \emph{trivial homomorphism},	
			\[\phi\colon G\to H,\quad \phi(g)=e_H\ \forall g\in G\]
		\vskip 1in
	\end{exa}
	\begin{block}{\textbf{Remark}}
		We might leave operations out and write
			\[\phi(ab)=\phi(a)\phi(b).\]
	\end{block}
}
\exer{
	Let $H=\{\varepsilon,(1\ 2)\}$ and $G=S_3$.  Define
	$\pi:H\to G$ by $\pi(\sigma)=\sigma$ for all $\sigma\in H$.  Show that $\pi$ is a homomorhpism.
	
}
\newpage
\slide{
	\begin{exercise}
		$G=(\Z_2,+)$, $H=(\Z_5^*,\cdot)$
			\[\phi\colon \Z_2\to \Z_5^*,\quad \phi(\overline{0}_2)=\overline{0}_5\ \text{ and }\phi(\overline{1}_2)=\overline{4}_5.\]
		 Answer each of the following to verify that $\phi$ is a homomorphism:
		\enumalph{
			\item Is $\phi(\overline{0}_2+\overline{0}_2)$ equal to $\phi(\overline{0}_2)\cdot\phi(\overline{0}_2)$?\vskip .5in
			\item Is $\phi(\overline{0}_2+\overline{1}_2)$ equal to $\phi(\overline{0}_2)\cdot\phi(\overline{1}_2)$?\vskip .5in
			\item Is $\phi(\overline{1}_2+\overline{0}_2)$ equal to $\phi(\overline{1}_2)\cdot\phi(\overline{0}_2)$?\vskip .5in
			\item Is $\phi(\overline{1}_2+\overline{1}_2)$ equal to $\phi(\overline{1}_2)\cdot\phi(\overline{1}_2)$?\vskip .5in
		}
	\end{exercise}
}
\exer{
	Let $\sigma \in S_4$. Define 
	$\phi:S_4\to S_4$ by $\phi(\tau)=\sigma\tau\sigma^{-1}$ for all $\tau\in S_4$.  Show that $\phi$ is a homomorhism by showing that $\phi(\tau_1\tau_2)=\phi(\tau_1)\phi(\tau_2)$ for all $\tau_1,\tau_2\in S_4$.
	\vfill
}
\begin{defn}
 A homomorphism that is both injective and surjective is called an \emph{isomorphism}. If an isomorphism exists from $G$ to $H$, we call $G$ and $H$ \emph{isomorphic} and we write $G\cong H$ (\verb|$G\cong H$|).
\end{defn}
\begin{exa}
	Considering the homomorphism from before, $\phi\colon\Z\to 2\Z$ defined by $\phi(n)=-2n$ for all $n\in\Z$.  Show that $\phi$ is:
		\bulletize{
			\item (One-to-One)  \vfill
			\item (Onto) \vfill
		}
	Therefore $\Z\cong 2\Z$.
\end{exa}
\newpage
\slide{
	\begin{exercise}
		Let $G=(\R,+)$ and $H=(\R_{>0},\cdot)$.  Define $\phi\colon G\to H$ by
			\[\phi(x)=e^x\ \forall x\in\R.\]
		\enumalph{
			\item Is $\phi$ a homomorphism?\vskip .5in
			\item Is $\phi$ one-to-one?\vskip .5in
			\item Is $\phi$ onto?\vskip .5in
			\item Is $\phi$ an isomorphism?\vskip .5in
		}
	\end{exercise}
}
\slide{
	\begin{exercise}
		Show that $\Z_3\cong C_3$ (here $C_3=\{1,a,a^2\}$).
		
		\vskip 2in\mbox{}
	\end{exercise}
}
\section*{Properties of Homomorphisms}
\slide{
	\begin{thm}{2.5.1}
		Let $\phi\colon G\to H$ be a group homomorphism. Then
		\enumalph{\item $\phi(e_G)=e_H$ \hfill($\phi$ preserves identities)
			\item $\phi(g^{-1})=\phi(g)^{-1}$ $\forall g\in G$\hfill($\phi$ preserves inverses)
			\item $\phi(g^k)=\phi(g)^k$ $\forall g\in G,k\in\Z$ \hfill($\phi$ preserves powers)
		}
	\end{thm}
\newpage
}
\slide{
	\begin{statementblock}{Corollary}
		Let $\phi\colon G\to H$ be a homomorphism.  If $g\in G$ has $|g|=n<\infty$, then $|\phi(g)|<\infty$.  Moreover $|\phi(g)|$ divides $|g|$.\vskip 1in
	\end{statementblock}
	\begin{exa}
		$\phi\colon \Z_8\to\Z_4$, $\phi(\overline{a}_8)=\overline{a}_4$
		\vskip 1in
		Warning: A map $\phi\colon \Z_n\to\Z_m$ with $\phi(\overline{a}_n)=\overline{a}_m$ only exists if $m\mid n$.
	\end{exa}
}
%\slide{
%	\begin{statementblock}{The Power of Injectivity [\textsc{AND}] Surjectivity}
%		If $\phi$ is an isomorphism, then
%		\enumarabic{
%			\item $|\phi(g)|=|g|,\ \forall g\in G$
%			\item $|G|=|H|$
%			\item $G$ abelian $\Leftrightarrow$ $H$ abelian
%			\item $G$ finite $\Leftrightarrow$ $H$ finite
%			\item $G$ cyclic $\Leftrightarrow$ $H$ cyclic
%			\item $G$ has no elements of order $n$ $\Leftrightarrow$ $H$ has no elements of order $n$
%			\item $G$ has exactly $m$ elements of order $n$ $\Leftrightarrow$ $H$ has exactly $m$ elements of order $n$
%			\item $\dots$	
%		}
%	\end{statementblock}
%}
\slide{
	\begin{exercise}
		Show $(\Z_4,+)\cong (\{1,-1,i,-i\},\cdot)$ by writing down a group isomorphism.\vfill
	\end{exercise}
}
\slide{
	\begin{exercise}
		Show $\Z_4$ has a subgroup isomorphic to $\Z_2$. (Meaning there is an isomorphism between that subgroup and $\Z_2$.)
		\vfill
	\end{exercise}
}
\slide{
	\begin{exercise}
		Give some reason why...
		\enumalph{
		
			\item $K_4\not\cong(\Z_4,+)$
			\vskip .5in
			\item $S_3\ncong C_6$
			\vskip .5in
			\item $(\Z_{12},+)\ncong(\Q^+,\cdot)$
		}
	\end{exercise}
}
\newpage
\slide{
	\begin{exercise}
		Is $(2\Z,+)\cong(3\Z,+)$?
	\end{exercise}\vskip 1.5in
}

\section*{Image of a Homomorphism}
\slide{
	\begin{defn}
		Let $\phi\colon G\to H$ be a group homomorphism.
		
		The \emph{image of $\phi$} is denoted $\phi(G)$ or $\im(\phi)$ and is defined to be the set
		\[\{\phi(g)\in H\ |\ g\in G\}=\{h\in H\ |\ \exists g\in G\text{ s.t. }\phi(g)=h\}.\]
	\end{defn}
\vskip 1in
%	\begin{block}{Picture}
%%		\begin{picture}(50,30)
%%			\put(75,0){\oval(50,75)}
%%			\put(30,-30){$G$}
%%			\put(200,0){\oval(50,75)}
%%			\put(240,-30){$H$}
%%		\end{picture}
%	\end{block}
}


\slide{
	\begin{statementblock}{Corollary of Thm 2.5.1}
		Let $\phi\colon G\to H$ be a homomorphism. Then $\im(\phi)\leq H$.
	\end{statementblock}
}

\slide{
	\begin{exercise}
		Consider $\phi\colon  \Z\to GL_2(\R)$ defined by
		\[\phi(n)=\begin{bmatrix}
			1&n\\0&1
		\end{bmatrix}.\]
		\enumalph{
			\setlength{\itemsep}{1in}
			\item Show that $\phi$ is a homomorphism.
			\item Verify $\ds\im(\phi)=\left\{\begin{bmatrix}1&n\\0&1\end{bmatrix}:n\in\Z\right\}$. (Call this set $H$.)
			\item Conclude $H$ is a group.
			\vskip 1in
		}
	\end{exercise}
}
\newpage

\section*{Isomorphisms are Equivalence Relations}
\slide{
	\begin{thm}{2.5.3}
		Let $G$, $H$, and $K$ denote groups.
		\enumarabic{\setlength{\itemsep}{1em}
			\item The identity map $id_G\colon G\to G$ is an isomorphism for every group $G$.
			\item If $\sigma\colon G\to H$ is an isomorphism then the inverse mapping $\sigma^{-1}\colon H\to G$ is an isomorphism.
			\item If $\sigma\colon G\to H$ and $\tau\colon H\to K$ are isomorphisms then $\tau\sigma\colon G\to K$ is an isomorphism.
		}
	\end{thm}
	\vfill\vfill
}
\slide{
	\begin{statementblock}{Corollary 1}
		This isomorphism relation, $\cong$ is an equivalence relation on groups.  That is, for all groups $G, H,$ and $K$,
		\enumarabic{
			\item $G\cong G$,
			\item if $G\cong H$, then $H\cong G$, and
			\item if $G\cong H$ and $H\cong K$, then $G\cong K$.
		}
	\end{statementblock}
	\vfill
}
\newpage
\section*{Group of Homomorphisms}
\slide{
	\begin{statementblock}{Corollary}
		If $G$ is a group, then the set of all isomorphisms $G\to G$ forms a group under composition.
	\end{statementblock}
	\begin{proof}
		Notice that the set of all isomorphisms $G\to G$ is a subset of $S_G$.  Therefore we can use the subgroup test.  Theorem 2.5.3 completes the proof.
	\end{proof}
	\begin{question}
		How does Theorem 2.5.3 show that we have (1) Non-empty, (2) Closure, (3) Inverses?
	\end{question}
\vskip 2in
}


\slide{
	\begin{defn}Let $G$ be a group.
		\enumarabic{
			\item An \emph{automorphism of $G$} is an isomorphism from $G$ to itself.
			\item The set $\Aut(G)$ is the set of all automorphisms of $G$.
		}
	\end{defn}
	\begin{nb}
		The previous Corollary says $\Aut(G)\leq S_G$, so $\Aut(G)$ is a group.
	\end{nb}
}
\slide{
	\begin{exercise}
		\enumalph{
			\item If $G$ is abelian, then $\phi\colon G\to G$ defined by $\phi(g)=g^{-1}$ is an automorphism of $G$.
			
			\begin{enumerate}[label=\roman*.]
				\item Verify $\phi$ is a homomorphism.\item Check $\phi$ is injective. \item Check $\phi$ is surjective.
			\end{enumerate}
			\vskip .5in
			\item Let $G=S_3$ (which is not abelian). Show that $\phi$ from (a) is not a homomorphism.\vskip 1in\mbox{}
		}
	\end{exercise}
}
%\slide{
%	\begin{exercise}
%		
%		This is actually a choose your own adventure!! Since we know $\Z_6\cong C_6$, we can compute either $\Aut(\Z_6)$ or $\Aut(C_6)$.
%		\begin{enumerate}[label=(\alph*)]
%			\item $\Aut(\Z_6)$ go to the next exercise
%			\item $\Aut(C_6)$ go to exercise \ref{c6}
%		\end{enumerate}
%	\end{exercise}
%}

	\begin{exercise}
		Compute the automorphism group of the cyclic group of order 6.
		
		$\Z_6=\{\overline{0},\overline{1},\overline{2},\overline{3},\overline{4},\overline{5}\}$
		\enumalph{
			\item Show that $\lambda:\Z_6\to \Z_6$ defined by $\lambda(\overline{n})=-\overline{n}$ is an automorphism of $\Z_6$.
			\item Verify that if $\phi:\Z_6\to \Z_6$ is an automorphism, then $\phi(\overline{1})=\overline{1}$ or $\phi(\overline{1})=\overline{5}$.
			\item Conclude that $\Aut(\Z_6)=\{id_{\Z_6},\lambda\}$.
			\item Discuss why $\Aut(\Z_6)\cong \Z_2$.
		}
	\end{exercise}\vskip 1in

%	\begin{exercise}\label{c6}
%		$C_6=\{1,a,a^2,a^3,a^4,a^5\}$
%		\enumalph{
%			\item Show that $\lambda:C_6\to C_6$ defined by $\lambda(a^i)=(a^{i})^{-1}$ is an automorphism of $C_6$.
%			\item Verify that if $\phi:C_6\to C_6$ is an automorphism, then $\phi(a)=a$ or $\phi(a)=a^5$.
%			\item Conclude that $\Aut(C_6)=\{id_{C_6},\lambda\}$.
%			\item Discuss why $\Aut(C_6)\cong C_2$.
%		}
%	\end{exercise}\vskip 1in
\slide{
	\begin{exercise}
		Let $G$ be a group and $a\in G$.
		
		Define $\sigma_a\colon G\to G$ by $\sigma_a(g)=aga^{-1}$.
		
		We call $\sigma_a$ an \emph{inner automorphism of $G$}.
		\enumalph{\setlength{\itemsep}{3em}
			\item Verify $\sigma_a$ is a homomorphism.
			\item Check $\sigma_a$ is injective.
			\item Check $\sigma_a$ is surjective.\vskip 3em\mbox{}
		}
	\end{exercise}
}
\slide{
	\begin{defn}
		The set $\Inn(G)=\{\sigma_a|a\in G\}$ is the set of all inner autmorphisms of $G$.
	\end{defn}
	\begin{exercise}
		Prove that $\Inn(G)\leq \Aut(G)$.
		\enumalph{
			\item Find a fixed $a\in G$ for which $\sigma_a=id_G$.
			\item If $\sigma_a,\sigma_b\in\Inn(G)$ what $c$ satisfies $\sigma_{a}\sigma_b=\sigma_{c}$?
			\item For each $\sigma_a\in\Inn(G)$ what might be $\sigma_a^{-1}$?
		}
	\end{exercise}
}
\section*{Kernel of a Homomorphism}
\slide{
	\begin{defn}
		Let $\phi\colon G\to H$ be a group homomorphism.
		
		The \emph{kernel of $\phi$} is denoted $\ker(\phi)$ and is defined to be the set
		\[\{g\in G\ |\ \phi(g)=e_H\}.\]
	\end{defn}
	\vskip 1in
	%	\begin{block}{Picture}\vskip 1in
		%%		\begin{picture}(200,30)
			%%			\put(75,0){\oval(50,75)}
			%%			\put(30,-30){$G$}
			%%			\put(200,0){\oval(50,75)}
			%%			\put(240,-30){$H$}
			%%		\end{picture}
		%	\end{block}
}


\slide{
	\begin{exercise}
		Let $\phi:G\to H$ be a homomorphism.  Prove that $\ker(\phi)\leq G$.\vskip 2in\mbox{}
	\end{exercise}
}

\slide{
	\begin{exercise}
		Let $\phi:\Z\to \Z_5$ be defined by $\phi(n)=\overline{n}$.  Computer $\ker(\phi)$.\vskip 2in
	\end{exercise}
}

\end{document}

