
\documentclass[t]{beamer}

\subtitle{Section 2.6: Cosets and Lagrange's Theorem}

\usepackage{amsthm,amsmath,amsfonts,hyperref,graphicx,color,multicol,soul}
\usepackage{enumitem,tikz,tikz-cd,setspace,mathtools}

%%%%%%%%%%
%Beamer Template Customization
%%%%%%%%%%
\setbeamertemplate{navigation symbols}{}
\setbeamertemplate{theorems}[ams style]
\setbeamertemplate{blocks}[rounded]

\definecolor{Blu}{RGB}{43,62,133} % UWEC Blue
\setbeamercolor{structure}{fg=Blu} % Titles

%Unnumbered footnotes:
\newcommand{\blfootnote}[1]{%
	\begingroup
	\renewcommand\thefootnote{}\footnote{#1}%
	\addtocounter{footnote}{-1}%
	\endgroup
}

%%%%%%%%%%
%TikZ Stuff
%%%%%%%%%%
\usetikzlibrary{arrows}
\usetikzlibrary{shapes.geometric}
\tikzset{
	smaller/.style={
		draw,
		regular polygon,
		regular polygon sides=3,
		fill=white,
		node distance=2cm,
		minimum height=1in,
		line width = 2pt
	}
}
\tikzset{
	smsquare/.style={
		draw,
		regular polygon,
		regular polygon sides=4,
		fill=white,
		node distance=2cm,
		minimum height=1in,
		line width = 2pt
	}
}


%%%%%%%%%%
%Custom Commands
%%%%%%%%%%

\newcommand{\C}{\mathbb{C}}
\newcommand{\quats}{\mathbb{H}}
\newcommand{\N}{\mathbb{N}}
\newcommand{\Q}{\mathbb{Q}}
\newcommand{\R}{\mathbb{R}}
\newcommand{\Z}{\mathbb{Z}}

\newcommand{\ds}{\displaystyle}

\newcommand{\fn}{\insertframenumber}

\newcommand{\id}{\operatorname{id}}
\newcommand{\im}{\operatorname{im}}
\newcommand{\Aut}{\operatorname{Aut}}
\newcommand{\Inn}{\operatorname{Inn}}

\newcommand{\blank}[1]{\underline{\hspace*{#1}}}

\newcommand{\abar}{\overline{a}}
\newcommand{\bbar}{\overline{b}}
\newcommand{\cbar}{\overline{c}}

\newcommand{\nml}{\unlhd}

%%%%%%%%%%
%Custom Theorem Environments
%%%%%%%%%%
\theoremstyle{definition}
\newtheorem{exercise}{Exercise}
\newtheorem{question}[exercise]{Question}
\newtheorem{warmup}{Warm-Up}
\newtheorem*{defn}{Definition}
\newtheorem*{exa}{Example}
\newtheorem*{disc}{Group Discussion}
\newtheorem*{nb}{Note}
\newtheorem*{recall}{Recall}
\renewcommand{\emph}[1]{{\color{blue}\texttt{#1}}}

\definecolor{Gold}{RGB}{237, 172, 26}
%Statement block
\newenvironment{statementblock}[1]{%
	\setbeamercolor{block body}{bg=Gold!20}
	\setbeamercolor{block title}{bg=Gold}
	\begin{block}{\textbf{#1.}}}{\end{block}}
\newenvironment{thm}[1]{%
	\setbeamercolor{block body}{bg=Gold!20}
	\setbeamercolor{block title}{bg=Gold}
	\begin{block}{\textbf{Theorem #1.}}}{\end{block}}


%%%%%%%%%%
%Custom Environment Wrappers
%%%%%%%%%%
\newcommand{\enumarabic}[1]{
	\begin{enumerate}[label=\textbf{\arabic*.}]
		#1
	\end{enumerate}
}
\newcommand{\enumalph}[1]{
	\begin{enumerate}[label=(\alph*)]
		#1
	\end{enumerate}
}
\newcommand{\bulletize}[1]{
	\begin{itemize}[label=$\bullet$]
		#1
	\end{itemize}
}
\newcommand{\circtize}[1]{
	\begin{itemize}[label=$\circ$]
		#1
	\end{itemize}
}
\newcommand{\slide}[1]{
	\begin{frame}{\fn}
		#1
	\end{frame}
}
\newcommand{\slidec}[1]{
\begin{frame}[c]{\fn}
	#1
\end{frame}
}
\newcommand{\slidet}[2]{
	\begin{frame}{\fn\ - #1}
		#2
	\end{frame}
}


\newcommand{\startdoc}{
		\title{Math 425: Abstract Algebra 1}
		\author{Mckenzie West}
		\date{Last Updated: \today}
		\begin{frame}
			\maketitle
		\end{frame}
}

\newcommand{\topics}[2]{
	\begin{frame}{\insertframenumber}
		\begin{block}{\textbf{Last Section.}}
			\begin{itemize}[label=--]
				#1
			\end{itemize}
		\end{block}
		\begin{block}{\textbf{This Section.}}
			\begin{itemize}[label=--]
				#2
			\end{itemize}
		\end{block}
	\end{frame}
}

\begin{document} 
	\startdoc
	
	\topics{
		\item Mappings between groups
		\item Homomorphisms
		\item Isomorphisms
		\item Image
		\item Kernel
		\item Automorphisms
	}{
		\item Cosets 
		\item Lagrange's Theorem
		\item Corollaries to Lagrange!
	}


\slide{
	\begin{block}{\textbf{Goal.}}
		Prove that if $H\leq G$ and $|G|<\infty$, then
			\[|H| \text{ divides } |G|.\]
	\end{block}
}
\slide{
	\begin{defn}
		Let $H\leq G$ and let $a\in G$.  Define the two sets
			\enumarabic{
			\item $H*a=\{h*a|h\in H\}$ called the \emph{right coset of $H$ by $a$}.
			\item $a*H=\{a*h|h\in H\}$ called the \emph{left coset of $H$ by $a$}.
			}
	\end{defn}
	\begin{nb}
		We often write $Ha$ and $aH$ if the group operation is unknown/unspecified.
	\end{nb}
}
\slide{
	\begin{exa}
		If $G=\Z$ the right cosets of $H=\langle4\rangle$ are
		
			$$\begin{array}{l}
				H+0=4\Z\\
				H+1=4\Z+1\\
				H+2=4\Z+2\\
				H+3=4\Z+3
			\end{array}$$
		
		What are the left cosets?
	\end{exa}
}
\slide{
	\begin{exa}
		For $G=(\R,+)$, $H=(\Z,+)$ is a subgroup.
		
		The right coset $\Z+\frac{1}{2}$ is the set of all points if we shift all integers one $\frac{1}{2}$ unit to the right.  That is
		
			\[\Z+\frac{1}{2}=\{\dots,-\frac{5}{2},-\frac{3}{2},-\frac{1}{2},\frac{1}{2},\frac{3}{2},\frac{5}{2},\dots\}.\]
	\end{exa}
}
\slide{
	\begin{exercise}
		Let $G=\Z_{12}$ and $H=\langle \bar 3\rangle=\{\bar 0,\bar 3,\bar 6,\bar 9\}$.
		
		We claim that $H$, $H+\bar 1$, and $H+\bar 2$ are the only right cosets of $H$.
		\enumalph{\item $H=$\vskip 2em \item $H+\bar1=$\vskip 2em\item $H+\bar2=$\vskip 2em
		\item What is $H+\bar 7$?\vskip 2em\item What is $H+\bar 9$?\vskip 2em\mbox{}}
	\end{exercise}
}
\slide{
	\begin{exercise}
		Let $G=S_3$ and $H=\langle(1\ 2)\rangle=\{\varepsilon,(1\ 2)\}$.
		
		Compute the right cosets
		\enumalph{\setlength{\itemsep}{1.5em}
			\item $H\varepsilon=$
			\item $H(1\ 2\ 3)=$
			\item $H(1\ 3\ 2)=$
		}
		Notice that these three cosets cover all six elements of $S_3$. 
		\\ Now compute
		\enumalph{\setlength{\itemsep}{1.5em}
			\item $H(1\ 2)=$
			\item $H(1\ 3)=$
			\item $H(2\ 3)=$
		}
	\end{exercise}
}
\slide{
\begin{exercise}
	Let $G=S_3$ and $H=\langle(1\ 2)\rangle=\{\varepsilon,(1\ 2)\}$.
	
	Compute the left cosets
	\enumalph{\setlength{\itemsep}{1.5em}
		\item $\varepsilon H=$
		\item $(1\ 2\ 3)H=$
		\item $(1\ 3\ 2)H=$
	}
\end{exercise}
\begin{exercise}
	Does $\varepsilon H=H\varepsilon$?
	
	\vskip 2em
	
	Does $(1\ 2\ 3)H=H(1\ 2\ 3)$?
\end{exercise}
}

\slide{
	\begin{thm}{2.6.1}
		Let $H$ be a subgroup of a group $G$ and let $a,b\in G$.
		\enumarabic{
			\item $H=H{e_G}$.
			\item $Ha=H$ if and only if $a\in H$.
			\item $Ha=Hb$ if and only if $ab^{-1}\in H$.
			\item If $a\in Hb$, then $Ha=Hb$.
			\item Either $Ha=Hb$ or $Ha\cap Hb=\emptyset$.
			\item The distinct right cosets of $H$ partition $G$.
		}
	\end{thm}
	\begin{nb}
		Here's a recommendation. Go back to the previous 4 examples and verify these facts.
	\end{nb}
}
\slide{
	%	\begin{block}{\textbf{Goal.}}
		%		Prove that if $H\leq G$ and $|G|<\infty$, then
		%			\[|H| \text{ divides } |G|.\]
		%	\end{block}
	\begin{thm}{2.6.1}
		Let $H$ be a subgroup of a group $G$ and let $a,b\in G$.
		\enumarabic{
			\item $H=H{e_G}$.
			\item $Ha=H$ if and only if $a\in H$.
			\item $Ha=Hb$ if and only if $ab^{-1}\in H$.
			\item If $a\in Hb$, then $Ha=Hb$.
			\item Either $Ha=Hb$ or $Ha\cap Hb=\emptyset$.
			\item The distinct right cosets of $H$ partition $G$.
		}
	\end{thm}
	
	\begin{statementblock}{Corollary}
		Corresponding statements hold for left cosets. In particular, part (3) becomes
		\[aH=bH\Leftrightarrow a^{-1}b\in H\Leftrightarrow b^{-1}a\in H.\]
	\end{statementblock}
}

\slide{
	\begin{statementblock}{Lemma}
		There is a 1-1 correspondence (aka bijection) between the elements in any two right cosets of $H$ in $G$.
	\end{statementblock}
	%	\begin{proof}
		%		Let $H\leq G$, and let $a,b\in G$.  Consider the right cosets $Ha$ and $Hb$. Define the map $f:Ha\to Hb$ by $f(ha)=hb$ for all $h\in H$.  
		%		
		%		\underline{Claim:} $f$ is a bijection.\vskip 2in\mbox{}
		%	\end{proof}
}

\slide{
	\begin{statementblock}{Lemma}
		If $H\leq G$ and $a,b\in G$, then $|Ha|=|Hb|$.
	\end{statementblock}
	\begin{exercise}
		Why is this a re-statement of the last Lemma?
	\end{exercise}
}

\slide{
	\begin{nb}
		``The single most important result about finite groups!'' - W.~Keith Nicholson (textbook author)
	\end{nb}
	\begin{statementblock}{Lagrange's Theorem (Theorem 2.6.2)}
		Let $H$ be a subgroup of a finite group $G$.  Then $|H|$ divides $|G|$.
	\end{statementblock}
}

\slide{
	\begin{exa}
		Here's a quick review of examples from before.
		
		$G=\Z_{12}$, $H=\langle\overline{3}\rangle=\{\overline{0},\overline{3},\overline{6},\overline{9}\}$
		\vskip 2in\mbox{}
	\end{exa}
}
\slide{
	\begin{exa}
		Here's a quick review of examples from before.
		
		$G=S_3=\{\varepsilon,(1\ 2),(1\ 3),(2\ 3),(1\ 2\ 3),(1\ 3\ 2)$, $H=\langle(1\ 2)\rangle=\{\varepsilon,(1\ 2)\}$
		\vskip 2in\mbox{}
	\end{exa}
}
\slide{
	\begin{statementblock}{Corollary 1}
		If $G$ is a finite group and $g\in G$, then $|g|$ divides $|G|$.
	\end{statementblock}
	\begin{block}{Warning}
		Corollary 1 does \textbf{not} say that $G$ has a subgroup of order $d$ for every divisor $d$ of $|G|$.
	\end{block}
	\begin{exercise}
		$A_4$, the alternating group on 4 indices, satisfies $|A_4|=12$ but $A_4$ has no subgroup of order 6.
	\end{exercise}
}

\slide{
	\begin{statementblock}{Corollary 2}
		If $G$ is a group and $|G|=n$, then $g^n=e$ for all $g\in G$.
	\end{statementblock}
	\begin{proof}
		Let $G$ be a group with $|G|=n$.  Let $g\in G$. If $|g|=m$ then $m$ divides $n$ \fbox{why?}.  So $n=mk$ for some $k\in \Z$.  Thus
		\[g^n=g^{mk}=(g^m)^k=e^k=e.\]
	\end{proof}
}
\slide{
	\begin{statementblock}{Corollary 3}
		If $p$ is a prime, then every group of order $p$ is cyclic.  In fact, $G=\langle g\rangle$ for every non-identity element $g$ in $G$, so the only subgroups of $G$ are $\{e\}$ and $G$ itself.
	\end{statementblock}
	\begin{exercise}
		Complete the proof.
		\begin{proof}
			Let $G$ be a group of prime order $p$.  Let $g$ be non-identity element of $G$.
			Consider the subgroup $H=\langle g\rangle$ of $G$.\vskip 2in\mbox{}
		\end{proof}
	\end{exercise}
}
\slide{
	\begin{statementblock}{Corollary 4}
		Let $H$ and $K$ be finite subgroups of a group $G$.  If $|H|$ and $|K|$ are relatively prime, then $H\cap K=\{e\}$.
	\end{statementblock}
	\begin{exercise}
		Complete the proof.
		\begin{proof}
			Let $H$ and $K$ be finite subgroup of a group $G$.  Let $|H|=m$, $|K|=n$, and $\gcd(m,n)=1$.  We want to show that $H\cap K=\{e\}$.  It suffices to show $|H\cap K|=1$ because ....
			[Hint: Prove $H\cap K\leq H$ and $H\cap K\leq K$, then use Lagrange.]\vskip 2in\mbox{}
		\end{proof}
	\end{exercise}
}
\slide{
	\begin{defn}
		The \emph{index} of $H$ in $G$, denoted $|G:H|$, is defined to the number of distinct right (or left if you prefer) cosets of $H$ in $G$.
	\end{defn}
	\begin{exa}
		\enumarabic{\item $|\Z_{12}:\langle\overline{4}\rangle|=4$
			\item $|\Z:2\Z|=2$
			\item $|S_3:\{\varepsilon,(1\ 2)\}|=3$
		}
	\end{exa}
}
\slide{
	\begin{statementblock}{Corollary 5}
		If $H$ is a subgroup of a finite group $G$, then 
		\[|G:H|=\frac{|G|}{|H|}.\]
	\end{statementblock}
}
\end{document}

