\documentclass[12pt]{article}

\newcommand{\secname}{Section 2.6: Cosets and Lagrange's Theorems}

\usepackage{amsthm,amsmath,amsfonts,hyperref,graphicx,color,multicol,soul}
\usepackage{enumitem,tikz,tikz-cd,setspace,mathtools}
\usepackage{colortbl}
\usepackage[margin=1in]{geometry}

%%%%%%%%%%
%Color Customization
%%%%%%%%%%

\definecolor{Blu}{RGB}{43,62,133} % UWEC Blue

%Unnumbered footnotes:
\newcommand{\blfootnote}[1]{%
	\begingroup
	\renewcommand\thefootnote{}\footnote{#1}%
	\addtocounter{footnote}{-1}%
	\endgroup
}

%%%%%%%%%%
%TikZ Stuff
%%%%%%%%%%
\usetikzlibrary{arrows}
\usetikzlibrary{shapes.geometric}
\tikzset{
	smaller/.style={
		draw,
		regular polygon,
		regular polygon sides=3,
		fill=white,
		node distance=2cm,
		minimum height=1in,
		line width = 2pt
	}
}
\tikzset{
	smsquare/.style={
		draw,
		regular polygon,
		regular polygon sides=4,
		fill=white,
		node distance=2cm,
		minimum height=1in,
		line width = 2pt
	}
}

%%%%%%%%%%
%Listing Setup
%%%%%%%%%%
\usepackage{listings}
\usepackage{caption, floatrow, makecell}%
\captionsetup{labelfont = sc}
\setcellgapes{3pt}

\definecolor{backcolour}{RGB}{237,236,230}
\definecolor{myblue}{RGB}{42,157,189}

\lstdefinestyle{mystyle}{
	language=Python,
	keywords=[2]{sage:},
	alsodigit={:,.,<,>},
	backgroundcolor=\color{backcolour},   
	commentstyle=\color{myblue},
	keywordstyle=\bfseries\color{Green},
	keywordstyle=[2]\color{purple},
	numberstyle=\tiny\color{Gray},
	stringstyle=\color{Orange},
	basicstyle=\ttfamily\footnotesize,
	breakatwhitespace=false,         
	breaklines=true,                 
	captionpos=b,                    
	keepspaces=true,                   
	showspaces=false,                
	showstringspaces=false,
	showtabs=false,                  
	tabsize=2
}

\lstset{style=mystyle}


%%%%%%%%%%
%Custom Commands
%%%%%%%%%%

\newcommand{\C}{\mathbb{C}}
\newcommand{\quats}{\mathbb{H}}
\newcommand{\N}{\mathbb{N}}
\newcommand{\Q}{\mathbb{Q}}
\newcommand{\R}{\mathbb{R}}
\newcommand{\Z}{\mathbb{Z}}

\newcommand{\ds}{\displaystyle}

\newcommand{\fn}{\insertframenumber}

\newcommand{\id}{\operatorname{id}}
\newcommand{\im}{\operatorname{im}}
\newcommand{\lcm}{\operatorname{lcm}}
\newcommand{\ord}{\operatorname{ord}}
\newcommand{\Aut}{\operatorname{Aut}}
\newcommand{\Inn}{\operatorname{Inn}}

\newcommand{\blank}[1]{\underline{\hspace*{#1}}}

\newcommand{\abar}{\overline{a}}
\newcommand{\bbar}{\overline{b}}
\newcommand{\cbar}{\overline{c}}

\newcommand{\nml}{\unlhd}

%%%%%%%%%%
%Custom Theorem Environments
%%%%%%%%%%
\theoremstyle{definition}
\newtheorem{exercise}{Exercise}
\newtheorem{question}[exercise]{Question}
\newtheorem{warmup}{Warm-Up}
\newtheorem*{exa}{Example}
\newtheorem*{defn}{Definition}
\newtheorem*{disc}{Group Discussion}
\newtheorem*{recall}{Recall}
\renewcommand{\emph}[1]{{\color{blue}\texttt{#1}}}

\definecolor{Gold}{RGB}{237, 172, 26}
%Statement block
%\newenvironment{statementblock}[1]{%
%	\setbeamercolor{block body}{bg=Gold!20}
%	\setbeamercolor{block title}{bg=Gold}
%	\begin{block}{\textbf{#1.}}}{\end{block}}
%\newenvironment{goldblock}{%
%	\setbeamercolor{block body}{bg=Gold!20}
%	\setbeamercolor{block title}{bg=Gold}
%	\setbeamertemplate{blocks}[shadow=true]
%	\begin{block}{}}{\end{block}}
%\newenvironment{defn}{%
%	\setbeamercolor{block body}{bg=gray!20}
%	\setbeamercolor{block title}{bg=violet, fg=white}
%	\setbeamertemplate{blocks}[shadow=true]
%	\begin{block}{\textbf{Definition.}}}{\end{block}}
%\newenvironment{nb}{%
%	\setbeamercolor{block body}{bg=gray!20}
%	\setbeamercolor{block title}{bg=teal, fg=white}
%	\setbeamertemplate{blocks}[shadow=true]
%	\begin{block}{\textbf{Note.}}}{\end{block}}
%\newenvironment{blockexample}{%
%	\setbeamercolor{block body}{bg=gray!20}
%	\setbeamercolor{block title}{bg=Blu, fg=white}
%	\setbeamertemplate{blocks}[shadow=true]
%	\begin{block}{\textbf{Example.}}}{\end{block}}
%\newenvironment{blocknonexample}{%
%	\setbeamercolor{block body}{bg=gray!20}
%	\setbeamercolor{block title}{bg=purple, fg=white}
%	\setbeamertemplate{blocks}[shadow=true]
%	\begin{block}{\textbf{Non-Example.}}}{\end{block}}
%\newenvironment{thm}[1]{%
%	\setbeamercolor{block body}{bg=Gold!20}
%	\setbeamercolor{block title}{bg=Gold}
%	\begin{block}{\textbf{Theorem #1.}}}{\end{block}}


%%%%%%%%%%
%Custom Environment Wrappers
%%%%%%%%%%
\newcommand{\exer}[1]{
	\begin{exercise}
	#1
	\end{exercise}
}
\newcommand{\exam}[1]{
\textbf{Example: }
	#1
}
\newcommand{\nexam}[1]{
	\textbf{Non-Example: }
	#1
}
\newcommand{\enumarabic}[1]{
	\begin{enumerate}[label=\textbf{\arabic*.}]
		#1
	\end{enumerate}
}
\newcommand{\enumalph}[1]{
	\begin{enumerate}[label=(\alph*)]
		#1
	\end{enumerate}
}
\newcommand{\bulletize}[1]{
	\begin{itemize}[label=$\bullet$]
		#1
	\end{itemize}
}
\newcommand{\circtize}[1]{
	\begin{itemize}[label=$\circ$]
		#1
	\end{itemize}
}
%\newcommand{\slide}[1]{
%	\begin{frame}{\fn}
%		#1
%	\end{frame}
%}
%\newcommand{\slidec}[1]{
%\begin{frame}[c]{\fn}
%	#1
%\end{frame}
%}
%\newcommand{\slidet}[2]{
%	\begin{frame}{\fn\ - #1}
%		#2
%	\end{frame}
%}


\setlength{\parindent}{0pt}



\usepackage{afterpage}
\usepackage{fancyhdr}

\fancyhead[L]{\textbf{Math 425: Abstract Algebra I\\\secname}}
\fancyhead[R]{\textbf{Mckenzie West\\Last Updated: \today}}
\pagestyle{fancy}

\newcommand{\startdoc}{}

\newcommand{\topics}[2]{
		{\textbf{Previously.}}
			\begin{itemize}[label=--]
				#1
			\end{itemize}
		{\textbf{This Section.}}
			\begin{itemize}[label=--]
				#2
			\end{itemize}
}

\begin{document} 
	\startdoc
	
	\topics{
		\item Mappings between groups
		\item Homomorphisms
		\item Isomorphisms
		\item Image
		\item Kernel
		\item Automorphisms
	}{
		\item Cosets 
		\item Lagrange's Theorem
		\item Corollaries to Lagrange!
	}


\slide{
	\begin{block}{Goal}
		Prove that if $H\leq G$ and $|G|<\infty$, then
			\[|H| \text{ divides } |G|.\]
	\end{block}
}
\slide{
	\begin{defn}
		Let $H\leq G$ and let $a\in G$.  Define the two sets
			\enumarabic{
			\item $H*a=\{h*a|h\in H\}$ called the \emph{right coset of $H$ by $a$}.
			\item $a*H=\{a*h|h\in H\}$ called the \emph{left coset of $H$ by $a$}.
			}
	\end{defn}
	\begin{nb}
		We often write $Ha$ and $aH$ if the group operation is unknown/unspecified.
	\end{nb}
}
\slide{
	\begin{exa}
		If $G=\Z$ the right cosets of $H=\langle4\rangle$ are
		
			$$\begin{array}{l}
				H+0=4\Z\\
				H+1=4\Z+1\\
				H+2=4\Z+2\\
				H+3=4\Z+3
			\end{array}$$
		
		What are the left cosets?
		\vskip 1in
	\end{exa}
}
\slide{
	\begin{exa}
		For $G=(\R,+)$, $H=(\Z,+)$ is a subgroup.
		
		The right coset $\Z+\frac{1}{2}$ is the set of all points if we shift all integers one $\frac{1}{2}$ unit to the right.  That is
		
			\[\Z+\frac{1}{2}=\{\dots,-\frac{5}{2},-\frac{3}{2},-\frac{1}{2},\frac{1}{2},\frac{3}{2},\frac{5}{2},\dots\}.\]
	\end{exa}
}
\slide{
	\begin{exercise}
		Let $G=\Z_{12}$ and $H=\langle \bar 3\rangle=\{\bar 0,\bar 3,\bar 6,\bar 9\}$.
		
		What are the right cosets of $H$?\vfill
	\end{exercise}
}
\newpage
\slide{
	\begin{exercise}
		Let $G=S_3$ and $H=\langle(1\ 2)\rangle=\{\varepsilon,(1\ 2)\}$.
		
		Compute the right cosets
		\enumalph{\setlength{\itemsep}{1.5em}
			\item $H\varepsilon=$
			\item $H(1\ 2\ 3)=$
			\item $H(1\ 3\ 2)=$
			\item $H(1\ 2)=$
			\item $H(1\ 3)=$
			\item $H(2\ 3)=$
		}
		\vskip 1.5em
		Any additional observations?\vskip .5in
	\end{exercise}
}
\slide{
\begin{exercise}
	Let $G=S_3$ and $H=\langle(1\ 2)\rangle=\{\varepsilon,(1\ 2)\}$.
	
	Compute the left cosets
	\enumalph{\setlength{\itemsep}{1.5em}
		\item $\varepsilon H=$
		\item $(1\ 2\ 3)H=$
		\item $(1\ 3\ 2)H=$
		\item $(1\ 2)H=$
		\item $(1\ 3)H=$
		\item $(2\ 3)H=$
	}
	\vskip 1.5em
	What do you notice about the left vs right cosets in the case with $G=S_3$ and $H=\{\varepsilon,(1\ 2)\}$?\vfill
\end{exercise}
}
\newpage
\slide{
	\begin{thm}{2.6.1}
		Let $H$ be a subgroup of a group $G$ and let $a,b\in G$.
		\enumarabic{
			\item $H=H{e_G}$.
			\item $Ha=H$ if and only if $a\in H$.
			\item $Ha=Hb$ if and only if $ab^{-1}\in H$.
			\item If $a\in Hb$, then $Ha=Hb$.
			\item Either $Ha=Hb$ or $Ha\cap Hb=\emptyset$.
			\item The distinct right cosets of $H$ partition $G$.
		}
	\end{thm}
	\begin{nb}
		Here's a recommendation. Go back to the previous 4 examples and verify these facts.
	\end{nb}
}
\slide{	
	\begin{statementblock}{Corollary}
		Corresponding statements hold for left cosets. In particular, part (3) becomes
		\[aH=bH\Leftrightarrow a^{-1}b\in H\Leftrightarrow b^{-1}a\in H.\]
	\end{statementblock}
}

\slide{
	\begin{statementblock}{Lemma}
		There is a 1-1 correspondence (aka bijection) between the elements in any two right cosets of $H$ in $G$.
	\end{statementblock}
	\vfill
	%	\begin{proof}
		%		Let $H\leq G$, and let $a,b\in G$.  Consider the right cosets $Ha$ and $Hb$. Define the map $f:Ha\to Hb$ by $f(ha)=hb$ for all $h\in H$.  
		%		
		%		\underline{Claim:} $f$ is a bijection.\vskip 2in\mbox{}
		%	\end{proof}
}

\slide{
	\begin{statementblock}{Lemma}
		If $H$ is a finite subgroup of the group $G$ and $a,b\in G$, then $|Ha|=|Hb|$.
	\end{statementblock}
	\vfill
}

\newpage
\slide{
	\begin{nb}
		``The single most important result about finite groups!'' - W.~Keith Nicholson (textbook author)
	\end{nb}
	\begin{statementblock}{Lagrange's Theorem (Theorem 2.6.2)}
		Let $G$ be a finite group.  Then if $H$ is a subgroup of $G$, we have $|H|$ divides $|G|$.
	\end{statementblock}
		\vskip 2in
}

\slide{
	\begin{exa}
		Recall the previous exercises
		\enumalph{
			\item 
			$G=\Z_{12}$,\quad $H=\langle\overline{3}\rangle=\{\overline{0},\overline{3},\overline{6},\overline{9}\}$
			\vfill
			\item 
			$G=S_3=\{\varepsilon,(1\ 2),(1\ 3),(2\ 3),(1\ 2\ 3),(1\ 3\ 2)\}$,\quad $H=\langle(1\ 2)\rangle=\{\varepsilon,(1\ 2)\}$
			\vfill
		}
	\end{exa}
}
\newpage
\slide{
	\begin{statementblock}{Corollary 1}
		If $G$ is a finite group and $g\in G$, then $|g|$ divides $|G|$.
	\end{statementblock}
	\vskip 1in
	\begin{block}{Warning}
		Corollary 1 does \textbf{not} say that $G$ has a subgroup of order $d$ for every divisor $d$ of $|G|$.
	\end{block}
	\begin{exa} Consider the alternating group $A_4$,
		\begin{equation*}
		\begin{split}
			A_4=\{\varepsilon,(1\ 2\ 3),(1\ 3\ 2),(1\ 2\ 4),(1\ 4\ 2),(1\ 3\ 4),(1\ 4\ 3),\\(2\ 3\ 4),(2\ 4\ 3),(1\ 2)(3\ 4),(1\ 3)(2\ 4),(1\ 4)(2\ 3)\}
		\end{split}
		\end{equation*}
		
		Satisfies $|A_4|=12$ but $A_4$ has no subgroup of order 6.
	\end{exa}\vskip 1in
}

\slide{
	\begin{statementblock}{Corollary 2}
		If $G$ is a group and $|G|=n$, then $g^n=e$ for all $g\in G$.
	\end{statementblock}
	\begin{proof}
		Let $G$ be a group with $|G|=n$.  Let $g\in G$. If $|g|=m$ then $m$ divides $n$ \fbox{why?}.  So $n=mk$ for some $k\in \Z$.  Thus
		\[g^n=g^{mk}=(g^m)^k=e^k=e.\]
	\end{proof}
}
\slide{
	\begin{statementblock}{Corollary 3}
		If $p$ is a prime, then every group of order $p$ is cyclic.  In fact, $G=\langle g\rangle$ for every non-identity element $g$ in $G$, so the only subgroups of $G$ are $\{e\}$ and $G$ itself.
	\end{statementblock}
	\begin{exercise}
		Complete the proof.
		\vfill
	\end{exercise}
	\newpage
}
\slide{
	\begin{statementblock}{Corollary 4}
		Let $H$ and $K$ be finite subgroups of a group $G$.  If $|H|$ and $|K|$ are relatively prime, then $H\cap K=\{e\}$.
	\end{statementblock}
	\begin{exercise}Prove Corollary 4.
		\vfill
	\end{exercise}
}
\slide{
	\begin{defn}
		The \emph{index} of $H$ in $G$, denoted $|G:H|$, is defined to the number of distinct right (or left if you prefer) cosets of $H$ in $G$.
	\end{defn}
	\begin{exercise} Compute each of the following indexes.
		\enumarabic{\item $|\Z_{12}:\langle\overline{4}\rangle|$\vskip 1in
			\item $|\Z:2\Z|$\vskip 1in
			\item $|S_3:\{\varepsilon,(1\ 2)\}|$\vskip 1in
		}
	\end{exercise}
}
\slide{
	\begin{statementblock}{Corollary 5}
		If $H$ is a subgroup of a finite group $G$, then 
		\[|G:H|=\frac{|G|}{|H|}.\]
	\end{statementblock}
		\vfill
}
\end{document}

