\documentclass[12pt]{article}

\newcommand{\secname}{Section 1.1: Induction}

\usepackage{amsthm,amsmath,amsfonts,hyperref,graphicx,color,multicol,soul}
\usepackage{enumitem,tikz,tikz-cd,setspace,mathtools}
\usepackage{colortbl}
\usepackage[margin=1in]{geometry}

%%%%%%%%%%
%Color Customization
%%%%%%%%%%

\definecolor{Blu}{RGB}{43,62,133} % UWEC Blue

%Unnumbered footnotes:
\newcommand{\blfootnote}[1]{%
	\begingroup
	\renewcommand\thefootnote{}\footnote{#1}%
	\addtocounter{footnote}{-1}%
	\endgroup
}

%%%%%%%%%%
%TikZ Stuff
%%%%%%%%%%
\usetikzlibrary{arrows}
\usetikzlibrary{shapes.geometric}
\tikzset{
	smaller/.style={
		draw,
		regular polygon,
		regular polygon sides=3,
		fill=white,
		node distance=2cm,
		minimum height=1in,
		line width = 2pt
	}
}
\tikzset{
	smsquare/.style={
		draw,
		regular polygon,
		regular polygon sides=4,
		fill=white,
		node distance=2cm,
		minimum height=1in,
		line width = 2pt
	}
}

%%%%%%%%%%
%Listing Setup
%%%%%%%%%%
\usepackage{listings}
\usepackage{caption, floatrow, makecell}%
\captionsetup{labelfont = sc}
\setcellgapes{3pt}

\definecolor{backcolour}{RGB}{237,236,230}
\definecolor{myblue}{RGB}{42,157,189}

\lstdefinestyle{mystyle}{
	language=Python,
	keywords=[2]{sage:},
	alsodigit={:,.,<,>},
	backgroundcolor=\color{backcolour},   
	commentstyle=\color{myblue},
	keywordstyle=\bfseries\color{Green},
	keywordstyle=[2]\color{purple},
	numberstyle=\tiny\color{Gray},
	stringstyle=\color{Orange},
	basicstyle=\ttfamily\footnotesize,
	breakatwhitespace=false,         
	breaklines=true,                 
	captionpos=b,                    
	keepspaces=true,                   
	showspaces=false,                
	showstringspaces=false,
	showtabs=false,                  
	tabsize=2
}

\lstset{style=mystyle}


%%%%%%%%%%
%Custom Commands
%%%%%%%%%%

\newcommand{\C}{\mathbb{C}}
\newcommand{\quats}{\mathbb{H}}
\newcommand{\N}{\mathbb{N}}
\newcommand{\Q}{\mathbb{Q}}
\newcommand{\R}{\mathbb{R}}
\newcommand{\Z}{\mathbb{Z}}

\newcommand{\ds}{\displaystyle}

\newcommand{\fn}{\insertframenumber}

\newcommand{\id}{\operatorname{id}}
\newcommand{\im}{\operatorname{im}}
\newcommand{\lcm}{\operatorname{lcm}}
\newcommand{\ord}{\operatorname{ord}}
\newcommand{\Aut}{\operatorname{Aut}}
\newcommand{\Inn}{\operatorname{Inn}}

\newcommand{\blank}[1]{\underline{\hspace*{#1}}}

\newcommand{\abar}{\overline{a}}
\newcommand{\bbar}{\overline{b}}
\newcommand{\cbar}{\overline{c}}

\newcommand{\nml}{\unlhd}

%%%%%%%%%%
%Custom Theorem Environments
%%%%%%%%%%
\theoremstyle{definition}
\newtheorem{exercise}{Exercise}
\newtheorem{question}[exercise]{Question}
\newtheorem{warmup}{Warm-Up}
\newtheorem*{exa}{Example}
\newtheorem*{defn}{Definition}
\newtheorem*{disc}{Group Discussion}
\newtheorem*{recall}{Recall}
\renewcommand{\emph}[1]{{\color{blue}\texttt{#1}}}

\definecolor{Gold}{RGB}{237, 172, 26}
%Statement block
%\newenvironment{statementblock}[1]{%
%	\setbeamercolor{block body}{bg=Gold!20}
%	\setbeamercolor{block title}{bg=Gold}
%	\begin{block}{\textbf{#1.}}}{\end{block}}
%\newenvironment{goldblock}{%
%	\setbeamercolor{block body}{bg=Gold!20}
%	\setbeamercolor{block title}{bg=Gold}
%	\setbeamertemplate{blocks}[shadow=true]
%	\begin{block}{}}{\end{block}}
%\newenvironment{defn}{%
%	\setbeamercolor{block body}{bg=gray!20}
%	\setbeamercolor{block title}{bg=violet, fg=white}
%	\setbeamertemplate{blocks}[shadow=true]
%	\begin{block}{\textbf{Definition.}}}{\end{block}}
%\newenvironment{nb}{%
%	\setbeamercolor{block body}{bg=gray!20}
%	\setbeamercolor{block title}{bg=teal, fg=white}
%	\setbeamertemplate{blocks}[shadow=true]
%	\begin{block}{\textbf{Note.}}}{\end{block}}
%\newenvironment{blockexample}{%
%	\setbeamercolor{block body}{bg=gray!20}
%	\setbeamercolor{block title}{bg=Blu, fg=white}
%	\setbeamertemplate{blocks}[shadow=true]
%	\begin{block}{\textbf{Example.}}}{\end{block}}
%\newenvironment{blocknonexample}{%
%	\setbeamercolor{block body}{bg=gray!20}
%	\setbeamercolor{block title}{bg=purple, fg=white}
%	\setbeamertemplate{blocks}[shadow=true]
%	\begin{block}{\textbf{Non-Example.}}}{\end{block}}
%\newenvironment{thm}[1]{%
%	\setbeamercolor{block body}{bg=Gold!20}
%	\setbeamercolor{block title}{bg=Gold}
%	\begin{block}{\textbf{Theorem #1.}}}{\end{block}}


%%%%%%%%%%
%Custom Environment Wrappers
%%%%%%%%%%
\newcommand{\exer}[1]{
	\begin{exercise}
	#1
	\end{exercise}
}
\newcommand{\exam}[1]{
\textbf{Example: }
	#1
}
\newcommand{\nexam}[1]{
	\textbf{Non-Example: }
	#1
}
\newcommand{\enumarabic}[1]{
	\begin{enumerate}[label=\textbf{\arabic*.}]
		#1
	\end{enumerate}
}
\newcommand{\enumalph}[1]{
	\begin{enumerate}[label=(\alph*)]
		#1
	\end{enumerate}
}
\newcommand{\bulletize}[1]{
	\begin{itemize}[label=$\bullet$]
		#1
	\end{itemize}
}
\newcommand{\circtize}[1]{
	\begin{itemize}[label=$\circ$]
		#1
	\end{itemize}
}
%\newcommand{\slide}[1]{
%	\begin{frame}{\fn}
%		#1
%	\end{frame}
%}
%\newcommand{\slidec}[1]{
%\begin{frame}[c]{\fn}
%	#1
%\end{frame}
%}
%\newcommand{\slidet}[2]{
%	\begin{frame}{\fn\ - #1}
%		#2
%	\end{frame}
%}


\setlength{\parindent}{0pt}



\usepackage{afterpage}
\usepackage{fancyhdr}

\fancyhead[L]{\textbf{Math 425: Abstract Algebra I\\\secname}}
\fancyhead[R]{\textbf{Mckenzie West\\Last Updated: \today}}
\pagestyle{fancy}

\newcommand{\startdoc}{}

\newcommand{\topics}[2]{
		{\textbf{Previously.}}
			\begin{itemize}[label=--]
				#1
			\end{itemize}
		{\textbf{This Section.}}
			\begin{itemize}[label=--]
				#2
			\end{itemize}
}

\begin{document} 
	\startdoc
\topics{
	\item Relations
	\item Equivalence Relations
	\item Equivalence Classes
}{
	\item Induction
}

\slide{
	\begin{statementblock}{Principle of Mathematical Induction}
		Let $P(n)$ be a statement for each integer $n\geq  m$. Suppose the following conditions are satisfied,
			\enumarabic{\item $P(m)$ is true, and \item $P(k)\Rightarrow P(k+1)$ for every $k\geq m$.}
		Then $P(n)$ is true for every $n\geq m$.
	\end{statementblock}
%	\begin{exa}
%		\textbf{Distribution:} It is known that if $a,b,c\in\R$, then $a(b+c)=ab+ac$.
%		
%		\textbf{Claim:} For all $x\in \R$ and $n\geq 1$,
%			\[(1-x)(1+x+\cdots+x^{n-1})=1-x^n.\]
%		\begin{proof}
%			\vskip 1in
%		\end{proof}
%	\end{exa}
}

\slide{
	\begin{exa}
		\textbf{Claim:} For all $n\geq 4$,
			\[2^n<n!.\]
		\vskip 3in
%		\begin{proof}
%			\textbf{Base Case:} Notice that when $n=4$, $2^n=16$ and $n!=24$.  Thus $2^n<n!$.
%			
%			\textbf{Inductive Step:} Let $k\geq 4$ be an integer.  Assume $2^k<k!$.
%			\vskip 3in
%		\end{proof}
	\end{exa}
}
\slide{
	\begin{exa}
		Show that $3^{3n}+1$ is a multiple of 7 for all odd $n\geq 1$.
		\vskip 3in
%		\begin{proof}
%			\textbf{Base Case:} Notice that when $n=1$, \vskip .5in
%			
%			\textbf{Inductive Step:} Let $k\geq 1$ be an odd integer.
%			\vskip 3in
%		\end{proof}
	\end{exa}
}
%\slide{
%	\begin{exercise}
%		Consider the statement, for all $n\geq 1$,
%			\[1^3+2^3+\cdots+n^3=\frac{1}{4}n^2(n+1)^2.\]
%		Use Induction to prove this statement.
%			\enumalph{
%				\item What is the base case?\vskip .5in
%				\item What is the inductive hypothesis?\vskip .5in
%				\item Use the inductive hypothesis to complete the proof.
%			}
%	\end{exercise}
%}
\slide{
	\begin{block}{\textbf{Induction in Algebra}.}
		We assume the distributive property $a(b+c)=ab+ac$.
		
		Prove that this property can be extended, that is show that for all integers $n\geq 2$, for all $a,b_1,\dots,b_n\in\Z$, we have
			 $$a(b_1+b_2+\cdots+b_n) = ab_1+ab_2+\cdots+ab_n.$$
	\end{block}
}
\slide{
\begin{exercise}
	Consider the statement, for all $n\geq 1$,
	\[n^3+(n+1)^3+(n+2)^3\]
	is a multiple of 9.
	
	Use Induction to prove this statement.
	\enumalph{
		\item What is the base case?\vskip .5in
		\item What is the inductive hypothesis?\vskip .5in
		\item Use the inductive hypothesis to complete the proof.
		
			Hint: Do NOT do any expanding of the terms $(k+1)^3$ and $(k+2)^3$.
			\vskip 3in
	}
\end{exercise}
}
\slide{
\begin{exercise}
	Let $a_n=2^{3n}-1$ for $n\geq 0$. Guess a common divisor for each $a_n$ and prove your assertion.
	(The common divisor should be greater than 1.)
	\enumalph{
		\item Use a calculator or math software co compute $a_1,a_2,a_3,a_4$.  What common divisor do you notice?\vskip 1in
		\item Use induction to prove your claim.
	}
		\vskip 3in
\end{exercise}
}
\slide{
	\begin{statementblock}{Another Induction Principle}
		Let $P(n)$ be a statement for each integer $n\geq  m$. Suppose the following conditions are satisfied,
		\enumarabic{\item $P(m)$ and $P(m+1)$ are true, and \item If $k\geq m$ and both $P(k)$ and $P(k+1)$ are true then $P(k+2)$ is true.}
		Then $P(n)$ is true for every $n\geq m$.
	\end{statementblock}
\begin{exercise}
	Let $a_n$ denote a number for each integer $n\geq 0$ and assume that $a_{n+2}=a_{n+1}+2a_n$ holds for every $n\geq 0$. 
	
	Show that if $a_0=1$ and $a_2=2$, then $a_n=2^n$ for each $n\geq 0$.
\end{exercise}
}
\end{document}

