\documentclass[12pt]{article}

\newcommand{\secname}{Section 1.4: Permutations}

\usepackage{amsthm,amsmath,amsfonts,hyperref,graphicx,color,multicol,soul}
\usepackage{enumitem,tikz,tikz-cd,setspace,mathtools}
\usepackage{colortbl}
\usepackage[margin=1in]{geometry}

%%%%%%%%%%
%Color Customization
%%%%%%%%%%

\definecolor{Blu}{RGB}{43,62,133} % UWEC Blue

%Unnumbered footnotes:
\newcommand{\blfootnote}[1]{%
	\begingroup
	\renewcommand\thefootnote{}\footnote{#1}%
	\addtocounter{footnote}{-1}%
	\endgroup
}

%%%%%%%%%%
%TikZ Stuff
%%%%%%%%%%
\usetikzlibrary{arrows}
\usetikzlibrary{shapes.geometric}
\tikzset{
	smaller/.style={
		draw,
		regular polygon,
		regular polygon sides=3,
		fill=white,
		node distance=2cm,
		minimum height=1in,
		line width = 2pt
	}
}
\tikzset{
	smsquare/.style={
		draw,
		regular polygon,
		regular polygon sides=4,
		fill=white,
		node distance=2cm,
		minimum height=1in,
		line width = 2pt
	}
}

%%%%%%%%%%
%Listing Setup
%%%%%%%%%%
\usepackage{listings}
\usepackage{caption, floatrow, makecell}%
\captionsetup{labelfont = sc}
\setcellgapes{3pt}

\definecolor{backcolour}{RGB}{237,236,230}
\definecolor{myblue}{RGB}{42,157,189}

\lstdefinestyle{mystyle}{
	language=Python,
	keywords=[2]{sage:},
	alsodigit={:,.,<,>},
	backgroundcolor=\color{backcolour},   
	commentstyle=\color{myblue},
	keywordstyle=\bfseries\color{Green},
	keywordstyle=[2]\color{purple},
	numberstyle=\tiny\color{Gray},
	stringstyle=\color{Orange},
	basicstyle=\ttfamily\footnotesize,
	breakatwhitespace=false,         
	breaklines=true,                 
	captionpos=b,                    
	keepspaces=true,                   
	showspaces=false,                
	showstringspaces=false,
	showtabs=false,                  
	tabsize=2
}

\lstset{style=mystyle}


%%%%%%%%%%
%Custom Commands
%%%%%%%%%%

\newcommand{\C}{\mathbb{C}}
\newcommand{\quats}{\mathbb{H}}
\newcommand{\N}{\mathbb{N}}
\newcommand{\Q}{\mathbb{Q}}
\newcommand{\R}{\mathbb{R}}
\newcommand{\Z}{\mathbb{Z}}

\newcommand{\ds}{\displaystyle}

\newcommand{\fn}{\insertframenumber}

\newcommand{\id}{\operatorname{id}}
\newcommand{\im}{\operatorname{im}}
\newcommand{\lcm}{\operatorname{lcm}}
\newcommand{\ord}{\operatorname{ord}}
\newcommand{\Aut}{\operatorname{Aut}}
\newcommand{\Inn}{\operatorname{Inn}}

\newcommand{\blank}[1]{\underline{\hspace*{#1}}}

\newcommand{\abar}{\overline{a}}
\newcommand{\bbar}{\overline{b}}
\newcommand{\cbar}{\overline{c}}

\newcommand{\nml}{\unlhd}

%%%%%%%%%%
%Custom Theorem Environments
%%%%%%%%%%
\theoremstyle{definition}
\newtheorem{exercise}{Exercise}
\newtheorem{question}[exercise]{Question}
\newtheorem{warmup}{Warm-Up}
\newtheorem*{exa}{Example}
\newtheorem*{defn}{Definition}
\newtheorem*{disc}{Group Discussion}
\newtheorem*{recall}{Recall}
\renewcommand{\emph}[1]{{\color{blue}\texttt{#1}}}

\definecolor{Gold}{RGB}{237, 172, 26}
%Statement block
%\newenvironment{statementblock}[1]{%
%	\setbeamercolor{block body}{bg=Gold!20}
%	\setbeamercolor{block title}{bg=Gold}
%	\begin{block}{\textbf{#1.}}}{\end{block}}
%\newenvironment{goldblock}{%
%	\setbeamercolor{block body}{bg=Gold!20}
%	\setbeamercolor{block title}{bg=Gold}
%	\setbeamertemplate{blocks}[shadow=true]
%	\begin{block}{}}{\end{block}}
%\newenvironment{defn}{%
%	\setbeamercolor{block body}{bg=gray!20}
%	\setbeamercolor{block title}{bg=violet, fg=white}
%	\setbeamertemplate{blocks}[shadow=true]
%	\begin{block}{\textbf{Definition.}}}{\end{block}}
%\newenvironment{nb}{%
%	\setbeamercolor{block body}{bg=gray!20}
%	\setbeamercolor{block title}{bg=teal, fg=white}
%	\setbeamertemplate{blocks}[shadow=true]
%	\begin{block}{\textbf{Note.}}}{\end{block}}
%\newenvironment{blockexample}{%
%	\setbeamercolor{block body}{bg=gray!20}
%	\setbeamercolor{block title}{bg=Blu, fg=white}
%	\setbeamertemplate{blocks}[shadow=true]
%	\begin{block}{\textbf{Example.}}}{\end{block}}
%\newenvironment{blocknonexample}{%
%	\setbeamercolor{block body}{bg=gray!20}
%	\setbeamercolor{block title}{bg=purple, fg=white}
%	\setbeamertemplate{blocks}[shadow=true]
%	\begin{block}{\textbf{Non-Example.}}}{\end{block}}
%\newenvironment{thm}[1]{%
%	\setbeamercolor{block body}{bg=Gold!20}
%	\setbeamercolor{block title}{bg=Gold}
%	\begin{block}{\textbf{Theorem #1.}}}{\end{block}}


%%%%%%%%%%
%Custom Environment Wrappers
%%%%%%%%%%
\newcommand{\exer}[1]{
	\begin{exercise}
	#1
	\end{exercise}
}
\newcommand{\exam}[1]{
\textbf{Example: }
	#1
}
\newcommand{\nexam}[1]{
	\textbf{Non-Example: }
	#1
}
\newcommand{\enumarabic}[1]{
	\begin{enumerate}[label=\textbf{\arabic*.}]
		#1
	\end{enumerate}
}
\newcommand{\enumalph}[1]{
	\begin{enumerate}[label=(\alph*)]
		#1
	\end{enumerate}
}
\newcommand{\bulletize}[1]{
	\begin{itemize}[label=$\bullet$]
		#1
	\end{itemize}
}
\newcommand{\circtize}[1]{
	\begin{itemize}[label=$\circ$]
		#1
	\end{itemize}
}
%\newcommand{\slide}[1]{
%	\begin{frame}{\fn}
%		#1
%	\end{frame}
%}
%\newcommand{\slidec}[1]{
%\begin{frame}[c]{\fn}
%	#1
%\end{frame}
%}
%\newcommand{\slidet}[2]{
%	\begin{frame}{\fn\ - #1}
%		#2
%	\end{frame}
%}


\setlength{\parindent}{0pt}



\usepackage{afterpage}
\usepackage{fancyhdr}

\fancyhead[L]{\textbf{Math 425: Abstract Algebra I\\\secname}}
\fancyhead[R]{\textbf{Mckenzie West\\Last Updated: \today}}
\pagestyle{fancy}

\newcommand{\startdoc}{}

\newcommand{\topics}[2]{
		{\textbf{Previously.}}
			\begin{itemize}[label=--]
				#1
			\end{itemize}
		{\textbf{This Section.}}
			\begin{itemize}[label=--]
				#2
			\end{itemize}
}

\begin{document} 
	\startdoc
	\topics{
			\item Arithmetic Modulo $n$
			\item Some further results
		}
		{
			\item Permutations
			\item Notation for Permutations
			\item Composition of Permutations
			\item Cycles
			\item Disjoint Cycles
			\item Transpositions
			\item Even vs Odd Permutations
		}

	\begin{defn}
		A \emph{permutation} of $T_n=\{1,2,\dots,n\}$ is 
		a mapping $\sigma\colon T_n\to T_n$ that is both one-to-one and onto (a bijection).
		\vskip 1em
		We call the collection of all permutations of $T_n$ the \emph{symmetric group of order $n$}, and we write
			\[S_n:=\{\sigma\colon T_n\to T_n\ |\ \sigma\text{ is a permutation}\}.\]
	\end{defn}
	\begin{exa}
		$n=2$:  \quad $T_2=\{1,2\}$ and $S_n=\{\mathrm{id},\sigma\}$ where $\mathrm{id}$ is the identity map and $\sigma$ is the map that swaps 1 and 2
		\begin{center}
			
			\begin{tikzcd}
				1\arrow[rr]&&1\\2\arrow[rr]&&2\\&\mathrm{id}
			\end{tikzcd}
			\qquad
			\begin{tikzcd}
				1\arrow[rrd]&&1\\2\arrow[rru]&&2\\&\sigma
			\end{tikzcd}
		\end{center}
	\end{exa}
	\exer{
		What are a couple of elements of $S_3$?\vfill
	}
%}
%\slide{
	\begin{nb}
		We can define a \emph{permutation on any set $X$} to be a bijection $\sigma\colon X\to X$.  And the set of all permutation on $X$ is the set of \emph{symmetries of $X$}:
			\[S_X:=\{\sigma\colon X\to X\ |\ \sigma\text{ is a bijection}\}.\]
	\end{nb}
\newpage

\slidet{Notation 1 - Two-Line Notation}{
	\begin{defn}[\emph{Two-Line Notation}]
			For $\sigma:T_n\to T_n$, we can write
				\[\left(\begin{array}{cccc}1&2&\dots&n\\
					\sigma(1)&\sigma(2)&\dots&\sigma(n)\end{array}\right).\]
	\end{defn}
	Think of the top row as the input and the bottom row as the output.
	\begin{exercise}
		In the case of $\sigma:T_4\to T_4$, $\left(\begin{array}{cccc}1&2&3&4\\3&2&4&1\end{array}\right)$ means
			\begin{multicols}{4}
				\enumalph{\item $\sigma(1)=3$\item $\sigma(2)={\color{white}3}$\item $\sigma(3)={\color{white}3}$\item $\sigma(4)={\color{white}3}$}
			\end{multicols}
		\vskip .5in
	\end{exercise}
}
\slidet{Notation 2 - One-Line Notation}{
	\begin{defn}[\emph{One-Line Notation}]
		For $\sigma:T_n\to T_n$, we can write
		\[\sigma = \sigma(1)\ \sigma(2)\ \dots\ \sigma(n).\]
	\end{defn}
	Think of the one line notation as only the bottom row of two-line notation.
	\begin{exa}
		For the previous example, the permutation in two-line notation \[\left(\begin{array}{cccc}1&2&3&4\\3&2&4&1\end{array}\right)\] can instead be written as
			\[\sigma = 3\ 2\ 4\ 1.\]
	\end{exa}
	\vskip .15in
}
\slide{\begin{exercise}
	There are 6 permutations in $S_3$. Write them in both one-line and two line notation.\vfill
\end{exercise}}

\newpage
%	\fbox{\textbf{Notation 3 - Braid or Arrow Notation}}
%		\begin{exa}
%			For $\sigma=3\ 2\ 4\ 1$:
%		\begin{center}
%			\setlength{\unitlength}{0.8cm}
%			\begin{tikzcd}
%				1\arrow[->,bend right=20]{rrdd}&&1\\2\arrow[->]{rr}&&2\\3\arrow[->,bend left=10]{rrd}&&3\\4\arrow[->,bend left=10]{rruuu}&&4
%			\end{tikzcd}
%		\end{center}
%		\end{exa}
\slidet{Notation 3 - Cycle Notation}{
	\begin{defn}
		The $r$-cycle $(x_1\ x_2\ \dots\ x_r)$ in $S_n$ is the permutation that sends
		$$\begin{matrix}
			x_1&\mapsto&x_2\\
			x_2&\mapsto&x_3\\
			x_3&\mapsto&x_4\\
			&\vdots\\
			x_{r-1}&\mapsto&x_r\\
			x_r&\mapsto&x_1.
		\end{matrix}$$
	\end{defn}
}
\slide{
	\begin{exa}
		$(2\ 4\ 1)\in S_5$ does the following
		$$\begin{matrix}
			2&\mapsto& 4\\
			4&\mapsto&1\\
			1&\mapsto&2
		\end{matrix}$$
	\end{exa}
	\begin{question}
		What does  this permutation do to $3$ and $5$?\vskip 1in
	\end{question}
}
\slide{
	\begin{nb}
		There are several equivalent ways to write $(2\ 4\ 1)$:\vskip 1in
		%\[(4\ 1\ 2)=(1\ 2\ 4)=(2\ 4\ 1).\]
		%Our convention: Write the smallest element in the cycle first.
	\end{nb}
}
\slide{
	\begin{exercise}
		Convert the permutation from two-line notation to cycle notation:
		\vskip 2em
		$\sigma=\begin{pmatrix}
			1&2&3&4&5&6&7&8\\8&3&1&4&2&7&6&5
		\end{pmatrix}$
		\vskip 1em
		$\tau=\begin{pmatrix}
			1&2&3&4&5&6&7\\
			5&6&1&7&3&2&4
		\end{pmatrix}$
	\end{exercise}
	\begin{exercise}
		Convert the permutation from cycle notation to two-line notation:  Assume that $\alpha,\beta\in S_8$.
		\vfill
		$\alpha=(1\ 4\ 5\ 7)(2\ 3)(6\ 8)$
		\vfill
		$\beta=(3\ 8\ 7)$
	\end{exercise}
}
\newpage
\begin{block}{Conventions}
	We establish the following conventions for cycle notation.
	\bulletize{
		\item The smallest number in a cycle will be written first.
		\item When there are multiple cycles, sort them according to their smallest element. (Do not do any sorting if any cycles have the same number!)
	}
\end{block}
\slide{
\begin{defn}
	Two cycles $(x_1\ x_2\ \dots\ x_r)$ and $(y_1\ y_2\ \dots\ y_s)$ are \emph{disjoint} if 	
	\[\{x_1,x_2,\dots,x_r\}\cap\{y_1,y_2,\dots,y_s\}=\emptyset.\]
\end{defn}
%	\begin{exa}
	%		\enumalph{
		%			\item $(2\ 4\ 5\ 6)$ and $(3\ 7\ 8)$ are disjoint cycles
		%			\item Non-example: $(1\ 2\ 3)$ and $(3\ 2\ 4)$ are not disjoint cycles
		%		}
	%	\end{exa}
%}
%\slide{
\begin{statementblock}{Theorem}
	Disjoint cycles commute.  That is if $\sigma$ and $\tau$ are disjoint cycles then $\sigma\tau=\tau\sigma$.
\end{statementblock}
}
\slide{
\begin{statementblock}{Theorem 1.4.5 (Cycle Decomposition Theorem)}
	Every $\sigma \in S_n$ with $\sigma\neq\varepsilon$ can be written as a product of disjoint cycles.
\end{statementblock}
%	\begin{exa}
	%		$\begin{pmatrix}
		%		1&2&3&4&5&6\\4&5&3&1&6&2
		%		\end{pmatrix}=(14)(256)$
	%	\end{exa}
%}
%\slide{
%	\begin{exercise}
	%		Write
	%		\[\sigma=\begin{pmatrix}
		%			1&2&3&4&5&6&7&8\\
		%			3&5&7&4&2&8&1&6
		%		\end{pmatrix}\in S_8\]
	%		as a product of disjoint cycles.
	%		%		\enumalph{
		%			%			\item Start with 1 to get the first cycle.
		%			%			\item Continue to the next smallest number not in the cycle above.
		%			%			\item Continue with the next smallest number not in the previous cycles.
		%			%			\item[$\dots$]
		%			%		}
	%	\end{exercise}
}

\exer{
	Let's complete this table of various notation for elements of $S_3$.
	\begin{center}
		\renewcommand*{\arraystretch}{4}
		\begin{tabular}{||c|c|c|c||}
			\hline\hline
			Verbal&Two-Line&One-Line&Cycle\\
			\hline\hline
			Identity&\hspace*{1.5in}&\hspace*{1.5in}&\hspace*{1.5in}\\
			\hline
			Swap $1\leftrightarrow2$&&&\\
			\hline
			Swap $1\leftrightarrow3$&&&\\
			\hline
			Swap $2\leftrightarrow3$&&&\\
			\hline
			$1\rightarrow2\rightarrow3\rightarrow1$&&&\\
			\hline
			$1\rightarrow3\rightarrow2\rightarrow1$&&&\\
			\hline\hline
		\end{tabular}
	\end{center}
}

\newpage

\slidet{Notation 4 - Permutation Matrices}{
	Recall: Standard basis for $\R^n$, $\{\vec{e}_1,\vec{e}_2,\dots,\vec{e}_n\}$
	
	We can view permutations as ``permuting'' the indices of the $\vec{e}_i$.
	
	\begin{exa}
		The permutation $\sigma=3\ 2\ 4\ 1$ corresponds to the linear transformation $T\colon\R^4~\to~\R^4$ defined by
		\[T(\vec{e}_1)=\vec{e}_3,\quad T(\vec{e}_2)=\vec{e}_2,\quad T(\vec{e}_3)=\vec{e}_4,\quad T(\vec{e}_4)=\vec{e}_1\]
		
		Which corresponds to the matrix
		%				\[\small 	
		%				T\begin{pmatrix}1\\0\\0\\0\end{pmatrix}=\begin{pmatrix}0\\0\\1\\0\end{pmatrix},\ 
		%				T\begin{pmatrix}0\\1\\0\\0\end{pmatrix}=\begin{pmatrix}0\\1\\0\\0\end{pmatrix},\ 
		%				T\begin{pmatrix}0\\0\\1\\0\end{pmatrix}=\begin{pmatrix}0\\0\\0\\1\end{pmatrix},\]
		\[\begin{pmatrix}0&0&0&1\\0&1&0&0\\1&0&0&0\\0&0&1&0\end{pmatrix}.\]
		Read from the columns - 1 goes to 3 because the first column has a 1 in the 3rd row.  That being said, as long as you're consistent with whether you read the row or the column, all will work out in the end.
	\end{exa}
}
\begin{defn}
	A \emph{permutation matrix} is an $n\times n$ matrix that has exactly one 1 in each row and column and every other entry is 0.
\end{defn}
\begin{exercise}
	Write down all of the $3\times 3$ permutation matrices.\vfill
\end{exercise}
\begin{nb}
	Every permutation matrix has determinant $\pm 1$, and can be constructed by swapping columns (or rows) of the identity matrix.
\end{nb}



\newpage
\slidet{Composition - The operation of permutations}{
}
\begin{exa} 
	Suppose that, in cycle notation, $\sigma = (1\ 3\ 4)$ and $\tau=(1\ 2\ 4\ 3)$ and we want to compute $\tau\circ\sigma=\tau\sigma$.  We could first translate to arrow diagrams.
	
	Note that the composition notation means that $\tau\sigma(1)=\tau(\sigma(1))$, so we apply $\sigma$ first.  Thus we draw $\sigma$ on the left, then we write $\tau$.  
	
	\begin{center}
		\small
		\setlength{\unitlength}{0.5cm}
		\begin{tikzcd}
			1\arrow[->,bend right=20]{rrdd}&&1\arrow[->,bend right=10]{rrd}&&1\\
			2\arrow[->]{rr}&&2\arrow[->,bend left=10]{rrdd}&&2\\
			3\arrow[->,bend left=10]{rrd}&&3\arrow[->,bend right=10]{rruu}&&3\\
			4\arrow[->,bend left=10]{rruuu}&&4\arrow[->,bend left=10]{rru}&&4\\
			&\sigma&&\tau
		\end{tikzcd}
		$\qquad=\qquad$
		\begin{tikzcd}
			1\arrow[->]{rr}&&1\\2\arrow[->,bend right=10]{rrdd}&&2\\3\arrow[->]{rr}&&3\\4\arrow[->,bend right=10]{rruu}&&4\\&\tau\sigma
		\end{tikzcd}
	\end{center}
\end{exa}
\slide{
	\begin{exercise}
		Let $\sigma=\begin{pmatrix}
		1&2&3&4&5&6\\6&5&3&2&4&1
		\end{pmatrix}$ and $\tau=\begin{pmatrix}
		1&2&3&4&5&6\\2&5&4&1&3&6
		\end{pmatrix}$.
		\vskip 1em
		Write $\tau\sigma=\tau \circ \sigma$ in two-line notation.\vskip .25in
		$\tau\sigma=\begin{pmatrix}
			1&2&3&4&5&6\\&&&&&
		\end{pmatrix}$
		\vskip .25in
		Write $\sigma\tau=\sigma\circ\tau  $ in two-line notation.\vskip .25in
		$\sigma\tau=\begin{pmatrix}
			1&2&3&4&5&6\\&&&&&
		\end{pmatrix}$
		\vskip .25in
	\end{exercise}
}

\slidet{Multiplication in Cycle Notation}{
	\begin{nb}
		When multiplying cycles, work from right to left one cycle at a time.
	\end{nb}
	\begin{exercise}
		Let $\alpha=(1\ 3\ 2)$, and $\beta=(1\ 5\ 3)$.  Write $\alpha\beta$ and $\beta\alpha$ in cycle notation.
	\end{exercise}
}
\newpage
\slide{
	\begin{block}{\textbf{Exponents.}}
		We write exponents to mean the repeated composition of :
		\[\sigma^k\coloneqq\underbrace{\sigma\sigma\cdots\sigma}_k.\]
	\end{block}
	\begin{exercise}
		Let $\sigma=(1\ 4\ 2\ 6)\in S_6$. Compute $\sigma^k$ for $k=2,3,4,5,\dots$.
		\vskip 2in
	\end{exercise}
}
\slidet{Order of a Permutation}{
	\begin{defn}
		The \emph{order} of a permutation, $\sigma\in S_n$ is the smallest positive integer $k$ such that $\sigma^k=\varepsilon$.
	\end{defn}
	\begin{exercise}
		\enumalph{
			\item What is the order of $(1\ 2)(3\ 4)$?\vfill
			\item What is the order of $(1\ 2)(3\ 4\ 5)$?\vfill
		}
	\end{exercise}
	\begin{nb}
		In the homework: Prove that the order of an $r$-cycle is $r$.
	\end{nb}
}
\slidet{Inverse Permutations}{
	\begin{exercise}
		Recall that permutations are bijective maps, so they ALL have inverses! (Woo!)
		
		\enumalph{
			\item Find the inverse of the cycle $(1\ 2)$.\vfill
			\item Find the inverse of the cycle $(1\ 4\ 2\ 5\ 3\ 6)$.\vfill
			%\item Find the inverse of the permutation $(1\ 2\ 3)(4\ 5)$.
		}
	\end{exercise}
}
\newpage
\slide{
	\begin{block}{Theorem}
		The inverse of the cycle $(x_1\ x_2\ \dots\ x_r)$ is the cycle $(x_r\ x_{r-1}\ \dots\ x_2\ x_1)$.
	\end{block}
	\begin{exercise}
		If $\sigma = (1\ 2\ 3)$ and $\tau = (3\ 2\ 1)$, verify that $\sigma\tau=\varepsilon$ and $\tau\sigma=\varepsilon$.
	\end{exercise}
	\vskip 1in
}
\slide{
	\begin{statementblock}{Theorem 0.3.5(3) - Specialized}
		Let $\sigma,\tau\in S_n$, then 
		\[(\sigma\tau)^{-1}=\tau^{-1}\sigma^{-1}.\]
	\end{statementblock}
	\begin{exercise}
		Let $\sigma = (1\ 2\ 3\ 4)$ and $\tau = ( 5\ 6)$.
		
		Use Theorem 0.3.5 to compute $(\sigma\tau)^{-1}$.\vskip 1in
	\end{exercise}
}
\slidet{Transpositions}{
	\begin{defn}
		A \emph{transposition} is a cycle of length 2.
	\end{defn}
	\begin{statementblock}{Theorem 1.4.6}
		If $n\geq 2$, then every cycle in $S_n$ can be written as a product of transpositions.
	\end{statementblock}
	\begin{proof}[``Proof.'']
		\begin{equation}\label{eq:transpositions}
			(x_1\ x_2\ \dots\ x_r)=(x_1\ x_r)(x_1\ x_{r-1})\cdots(x_1\ x_3)(x_1\ x_2)
		\end{equation}
	\end{proof}
	\begin{exercise}
		Verify that $(1\ 2\ 5\ 3) = (1\ 3)(1\ 5)(1\ 2)$.
	\end{exercise}
	\vskip 1in
	\begin{exercise}
		Write $(1\ 5\ 4)(2\ 6\ 7\ 8\ 3) $ as a product of transpositions.
	\end{exercise}
	\vskip 2in
}

\newpage
\slidet{Even Permutations and the Alternating Group}{
	\begin{defn}
		A permutation $\sigma\in S_n$ is called \emph{even} if it can be written as a product of an even number of transpositions.
		
		Similarly, permutations can be called \emph{odd}.
	\end{defn}
	\begin{statementblock}{The Parity Theorem (Theorem 1.4.7)}
		If a permutation has two factorizations	
		\[\sigma = \gamma_n\cdots \gamma_2\gamma_1=\mu_m\cdots\mu_s\mu_1,\]
		where each of $\gamma_i$ and $\mu_j$ are transpositions, then $m\equiv n\pmod 2$ ($m$ and $n$ have the same parity).
	\end{statementblock}
}
\slide{
	\begin{defn}
		The \emph{alternating group of degree $n$} is the set of even permutations in $S_n$.  We call it $A_n$.
	\end{defn}
	\begin{exercise}
		Determine $A_3$. 
	\end{exercise}
	\vskip 1in
	\begin{question}
		How do you think $|A_n|$ compares with $|S_n|$?
	\end{question}
	\vskip 1in
}
\exer{
	Determine whether each of the following permutations is even or odd.
	
	\enumalph{
		\item $(2\ 3\ 6\ 8\ 5\ 7)$\vfill
		\item $(2\ 8\ 5)(3\ 7)$\vfill
		\item $(1\ 4)(2\ 9\ 8)(3\ 7)$\vfill
		\item $(1\ 4\ 6)(2\ 5)(3\ 8\ 7)$\vfill
	}
}
\newpage
\slidet{More Practice}{
	\begin{exercise}
		Let $f=(1\ 3)(2\ 5\ 6\ 8\ 4)$ and $g=(1\ 5\ 2\ 4)(3\ 7)(6\ 8)$.  
		\enumalph{
			\item Compute 
			\begin{enumerate}[label=\roman*.]
				\setlength{\itemsep}{1em}
				\item $fg$\vfill
				\item $g^{-1}$\vfill
				\item $f^{-1}$\vfill
				\item $fgf^{-1}$\vfill
			\end{enumerate}
		}
	\end{exercise}
}
%\slide{
%	\begin{question}
%		Assuming $\sigma\in S_n$ can be written as $k$ disjoint cycles $\rho_1\rho_2\cdots\rho_k$ where cycle $\rho_i$ has order $n_i$.  Conjecture a value for the order of $\sigma\in S_n$.
%	\end{question}
%	\begin{nb}
%		You might want to first try some more examples.
%	\end{nb}
%}
\slide{
\fbox{\begin{minipage}{\textwidth}
		\begin{nb}
			The set $S_n$ has an operation, composition. With this operation on $S_n$, we have
			\begin{enumerate}[label=(\alph*)]
				\item an identity, $\mathrm{id}$, the identity map, usually denoted $\varepsilon$
				\item associativity, $\sigma\circ(\tau\circ\gamma)=(\sigma\circ\tau)\circ\gamma$, and
				\item inverses, if $\sigma\in S_n$, then $\sigma^{-1}\in S_n$.
			\end{enumerate}
			\vskip 1em
			A look ahead to the future: This is why we can call $S_n$ a \textit{group}.
		\end{nb}
\end{minipage}}
}
\end{document}

