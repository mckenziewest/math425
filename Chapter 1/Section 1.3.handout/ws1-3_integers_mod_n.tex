\documentclass[12pt]{article}

\newcommand{\secname}{Section 1.3: Integers mod $n$}

\usepackage{amsthm,amsmath,amsfonts,hyperref,graphicx,color,multicol,soul}
\usepackage{enumitem,tikz,tikz-cd,setspace,mathtools}
\usepackage{colortbl}
\usepackage[margin=1in]{geometry}

%%%%%%%%%%
%Color Customization
%%%%%%%%%%

\definecolor{Blu}{RGB}{43,62,133} % UWEC Blue

%Unnumbered footnotes:
\newcommand{\blfootnote}[1]{%
	\begingroup
	\renewcommand\thefootnote{}\footnote{#1}%
	\addtocounter{footnote}{-1}%
	\endgroup
}

%%%%%%%%%%
%TikZ Stuff
%%%%%%%%%%
\usetikzlibrary{arrows}
\usetikzlibrary{shapes.geometric}
\tikzset{
	smaller/.style={
		draw,
		regular polygon,
		regular polygon sides=3,
		fill=white,
		node distance=2cm,
		minimum height=1in,
		line width = 2pt
	}
}
\tikzset{
	smsquare/.style={
		draw,
		regular polygon,
		regular polygon sides=4,
		fill=white,
		node distance=2cm,
		minimum height=1in,
		line width = 2pt
	}
}

%%%%%%%%%%
%Listing Setup
%%%%%%%%%%
\usepackage{listings}
\usepackage{caption, floatrow, makecell}%
\captionsetup{labelfont = sc}
\setcellgapes{3pt}

\definecolor{backcolour}{RGB}{237,236,230}
\definecolor{myblue}{RGB}{42,157,189}

\lstdefinestyle{mystyle}{
	language=Python,
	keywords=[2]{sage:},
	alsodigit={:,.,<,>},
	backgroundcolor=\color{backcolour},   
	commentstyle=\color{myblue},
	keywordstyle=\bfseries\color{Green},
	keywordstyle=[2]\color{purple},
	numberstyle=\tiny\color{Gray},
	stringstyle=\color{Orange},
	basicstyle=\ttfamily\footnotesize,
	breakatwhitespace=false,         
	breaklines=true,                 
	captionpos=b,                    
	keepspaces=true,                   
	showspaces=false,                
	showstringspaces=false,
	showtabs=false,                  
	tabsize=2
}

\lstset{style=mystyle}


%%%%%%%%%%
%Custom Commands
%%%%%%%%%%

\newcommand{\C}{\mathbb{C}}
\newcommand{\quats}{\mathbb{H}}
\newcommand{\N}{\mathbb{N}}
\newcommand{\Q}{\mathbb{Q}}
\newcommand{\R}{\mathbb{R}}
\newcommand{\Z}{\mathbb{Z}}

\newcommand{\ds}{\displaystyle}

\newcommand{\fn}{\insertframenumber}

\newcommand{\id}{\operatorname{id}}
\newcommand{\im}{\operatorname{im}}
\newcommand{\lcm}{\operatorname{lcm}}
\newcommand{\ord}{\operatorname{ord}}
\newcommand{\Aut}{\operatorname{Aut}}
\newcommand{\Inn}{\operatorname{Inn}}

\newcommand{\blank}[1]{\underline{\hspace*{#1}}}

\newcommand{\abar}{\overline{a}}
\newcommand{\bbar}{\overline{b}}
\newcommand{\cbar}{\overline{c}}

\newcommand{\nml}{\unlhd}

%%%%%%%%%%
%Custom Theorem Environments
%%%%%%%%%%
\theoremstyle{definition}
\newtheorem{exercise}{Exercise}
\newtheorem{question}[exercise]{Question}
\newtheorem{warmup}{Warm-Up}
\newtheorem*{exa}{Example}
\newtheorem*{defn}{Definition}
\newtheorem*{disc}{Group Discussion}
\newtheorem*{recall}{Recall}
\renewcommand{\emph}[1]{{\color{blue}\texttt{#1}}}

\definecolor{Gold}{RGB}{237, 172, 26}
%Statement block
%\newenvironment{statementblock}[1]{%
%	\setbeamercolor{block body}{bg=Gold!20}
%	\setbeamercolor{block title}{bg=Gold}
%	\begin{block}{\textbf{#1.}}}{\end{block}}
%\newenvironment{goldblock}{%
%	\setbeamercolor{block body}{bg=Gold!20}
%	\setbeamercolor{block title}{bg=Gold}
%	\setbeamertemplate{blocks}[shadow=true]
%	\begin{block}{}}{\end{block}}
%\newenvironment{defn}{%
%	\setbeamercolor{block body}{bg=gray!20}
%	\setbeamercolor{block title}{bg=violet, fg=white}
%	\setbeamertemplate{blocks}[shadow=true]
%	\begin{block}{\textbf{Definition.}}}{\end{block}}
%\newenvironment{nb}{%
%	\setbeamercolor{block body}{bg=gray!20}
%	\setbeamercolor{block title}{bg=teal, fg=white}
%	\setbeamertemplate{blocks}[shadow=true]
%	\begin{block}{\textbf{Note.}}}{\end{block}}
%\newenvironment{blockexample}{%
%	\setbeamercolor{block body}{bg=gray!20}
%	\setbeamercolor{block title}{bg=Blu, fg=white}
%	\setbeamertemplate{blocks}[shadow=true]
%	\begin{block}{\textbf{Example.}}}{\end{block}}
%\newenvironment{blocknonexample}{%
%	\setbeamercolor{block body}{bg=gray!20}
%	\setbeamercolor{block title}{bg=purple, fg=white}
%	\setbeamertemplate{blocks}[shadow=true]
%	\begin{block}{\textbf{Non-Example.}}}{\end{block}}
%\newenvironment{thm}[1]{%
%	\setbeamercolor{block body}{bg=Gold!20}
%	\setbeamercolor{block title}{bg=Gold}
%	\begin{block}{\textbf{Theorem #1.}}}{\end{block}}


%%%%%%%%%%
%Custom Environment Wrappers
%%%%%%%%%%
\newcommand{\exer}[1]{
	\begin{exercise}
	#1
	\end{exercise}
}
\newcommand{\exam}[1]{
\textbf{Example: }
	#1
}
\newcommand{\nexam}[1]{
	\textbf{Non-Example: }
	#1
}
\newcommand{\enumarabic}[1]{
	\begin{enumerate}[label=\textbf{\arabic*.}]
		#1
	\end{enumerate}
}
\newcommand{\enumalph}[1]{
	\begin{enumerate}[label=(\alph*)]
		#1
	\end{enumerate}
}
\newcommand{\bulletize}[1]{
	\begin{itemize}[label=$\bullet$]
		#1
	\end{itemize}
}
\newcommand{\circtize}[1]{
	\begin{itemize}[label=$\circ$]
		#1
	\end{itemize}
}
%\newcommand{\slide}[1]{
%	\begin{frame}{\fn}
%		#1
%	\end{frame}
%}
%\newcommand{\slidec}[1]{
%\begin{frame}[c]{\fn}
%	#1
%\end{frame}
%}
%\newcommand{\slidet}[2]{
%	\begin{frame}{\fn\ - #1}
%		#2
%	\end{frame}
%}


\setlength{\parindent}{0pt}



\usepackage{afterpage}
\usepackage{fancyhdr}

\fancyhead[L]{\textbf{Math 425: Abstract Algebra I\\\secname}}
\fancyhead[R]{\textbf{Mckenzie West\\Last Updated: \today}}
\pagestyle{fancy}

\newcommand{\startdoc}{}

\newcommand{\topics}[2]{
		{\textbf{Previously.}}
			\begin{itemize}[label=--]
				#1
			\end{itemize}
		{\textbf{This Section.}}
			\begin{itemize}[label=--]
				#2
			\end{itemize}
}

\begin{document} 
	\startdoc
	\topics{
		\item Division Algorithm
		\item GCD
		\item B\'ezout's Identity
		\item Euclidean Algorithm
		\item Prime Factorization Theorem
	}{
		\item Congruence modulo $n$
		\item Relations and Equivalence Classes
		\item Integers and Arithmetic modulo $n$
		\item Arithmetic Modulo $n$
		\item Inverses Modulo $n$
	}

\slide{
	\begin{defn}
		Let $a,b,n\in\Z$ with $n\geq 2$. We say that $a$ and $b$ are \emph{congruent modulo $n$} if 	\[n\mid (a-b).\]
		In that case, we write $a\equiv b\pmod n$.
	\end{defn}
}
\slide{
	\begin{thm}{1.3.1}
		Congruence modulo $n$ is an equivalence relation on $\Z$.
	\end{thm}
}

\slide{
	\begin{exercise}
		Write the equivalence classes of $(\Z,\equiv\pmod 2)$.
	\end{exercise}
	\vskip 1in
	\begin{exercise}
		Write the equivalence classes of $(\Z,\equiv\pmod 3)$.
	\end{exercise}
	\vskip 1in
}
\slide{
	\begin{defn}
		If $a\in\Z$, then its equivalence class, $[a]$, with respect to congruence modulo $n$ is called its \emph{residue class modulo $n$} and we write $\overline{a}$ for convenience.
		
			\[\overline{a}=\{x\in\Z : x\equiv a\pmod n\}.\]
	\end{defn}
}
%\slide{
%	\begin{statementblock}{Theorem 1.3.2}
%		Given $n\geq 2$, $\overline{a}=\overline{b}\Leftrightarrow a\equiv b\pmod n$.
%	\end{statementblock}
%%	\begin{block}{Proof Notes.}
%%		$(\Rightarrow)$ Follows from the definition of $\overline{a}$.
%%		
%%		$(\Leftarrow)$ Show the sets are equal!  Let $c\in\overline{a}$, use the fact that\\ $a\equiv b~\pmod n$ to show that $c\in\overline{b}$. 
%%		
%%		Do the same for $\overline{b}$ to $\overline{a}$. (Or explain that the argument will be functionally the same.)
%%	\end{block}
%}
%\slide{
%	\begin{block}{\textbf{Brain Break.}}
%		What might you say to a non-math person if you wanted to teach them about integers mod $n$?
%	\end{block}
%}
\slide{
	\begin{defn}
		The \emph{set of integers modulo $n$} is denoted $\Z_n$ and is given by
			\[\Z_n=\{\overline{0},\overline{1},\overline{2},\dots,\overline{n-1}\}.\]
	\end{defn}
	\begin{exa}
		$\Z_7=$
	\end{exa}
	\vskip 2em
	\begin{exercise}
		What is $\overline{47}$ in $\Z_7$? What is $\overline{-16}$?
	\end{exercise}
	\vskip 2in
}
\slide{
	\begin{statementblock}{Claim}
		Addition and multiplication in $\Z_n$, as defined below, are well-defined:
			\begin{itemize}
				\item[(1)] $\overline{a}+\overline{b}=\overline{a+b}$
				\item[(2)] $\overline{a}\overline{b}=\overline{ab}$
			\end{itemize}
	\end{statementblock}
	\begin{nb}
		The important point here is that any well-defined arithmetic operation on $\Z_n$ should NOT depend on the choice of residue class representative.
	\end{nb}
	\begin{exa}
		In $\Z_7$, $\overline{48}=\overline{6}$ and $\overline{3}=\overline{10}$.  Is it true that $\overline{48}+\overline{3}=\overline{6}+\overline{10}$?
	\end{exa}
}
\slide{
	\begin{block}{Proof.}
		It suffices to show that if $\overline{a_1}=\overline{a_2}$ and $\overline{b_1}=\overline{b_2}$ in $\Z_n$, then
			\[\overline{a_1+b_1}=\overline{a_2+b_2}\text{ and }\overline{a_1b_1}=\overline{a_2b_2}.\]
	\end{block}
	\vskip 2in
}
\slide{
	\begin{exercise}
		Fill out the addition and multiplication tables for $\Z_4$.
		\vskip .5in
		\Large
		$$
		\begin{array}{c|c|c|c|c}
		+_4&\overline{0}&\overline{1}&\overline{2}&\overline{3}\\\hline
		\overline{0}&&&&\\\hline
		\overline{1}&&&&\\\hline
		\overline{2}&&&&\\\hline
		\overline{3}&&&&
		\end{array}
		\hskip 3em
		\begin{array}{c|c|c|c|c}
		\times_4&\overline{0}&\overline{1}&\overline{2}&\overline{3}\\\hline
		\overline{0}&&&&\\\hline
		\overline{1}&&&&\\\hline
		\overline{2}&&&&\\\hline
		\overline{3}&&&&
		\end{array}$$
	\end{exercise}
}
\slide{
	\begin{exa}
		We can show that an integer $n\in\Z$ is divisible by 9 if and only if the sum of its digits is divisible by 9, using arithmetic mod 9!\vskip 2in
	\end{exa}
}
\slide{
	\begin{block}{\textbf{Summary.}}
		\begin{itemize}[label=$\bullet$]
			\item The set of integers modulo $n$ is
			\[\Z_n=\{\overline{0},\overline{1},\overline{2},\dots,\overline{n-1}\}.\]
			\item If $r$ is the remainder you get when dividing $a$ by $n$, then
			\[a\equiv r\pmod{n} \text{ or equivalently } \abar=\overline{r}.\]
			\item Addition in $\Z_n$ is defined by:
			\[\overline{a}+\overline{b}=\overline{a+b}.\]
			\item Multiplication in $\Z_n$ is defined by 
			\[\overline{a}\overline{b}=\overline{ab}.\]
		\end{itemize}
	\end{block}
}
\slide{
	\begin{statementblock}{Theorem 1.3.4}
		Let $n\geq 2$ be a fixed modulus and let $a,b$ and $c$ denote arbitrary integers. Then the following hold in $\Z_n$.
		\enumarabic{\item $\abar+\bbar=\bbar+\abar$ and $\abar\bbar=\bbar\abar$.
			\item $\abar+(\bbar+\cbar)=(\abar+\bbar)+\cbar$ and $\abar(\bbar\cbar)=(\abar\bbar)\cbar$.
			\item $\abar+\overline{0}=\abar$ and $\abar\overline{1}=\abar$.
			\item $\abar+\overline{-a}=\overline{0}$.
			\item $\abar(\bbar+\cbar)=\abar\bbar+\abar\cbar$.
		}
	\end{statementblock}
	\begin{nb}
		The proof of (5) is in the book.  And (2) is proved in a video.
	\end{nb}
}
\slide{	

	\fbox{
	\begin{minipage}{\textwidth}
		\textbf{Moral from last Theorem:}
		
		Arithmetic in $\Z_n$ behaves very similarly to arithmetic in $\Z$! 
		
		There's a \emph{zero}, $\overline{0}$, and \emph{unity}, $\overline{1}$, in $\Z_n$.
		
		Every $\abar\in\Z_n$ has an \emph{negative} or \emph{additive inverse}, $\overline{-a}$, in $\Z_n$, which we write as $-\abar$ and satisfies
		\[\abar+\overline{-a}=\overline{0}.\]
		
		\emph{Subtraction} is then naturally defined as
		\[\abar-\bbar=\abar+\overline{-b}=\overline{a-b}.\]
\end{minipage}}
}
\slide{
	\begin{exercise}
		What is the additive inverse of $\overline{6}$ in $\Z_8$?
		\vskip 1in
		%		Be sure to write it as a class in $\{\overline{0},\overline{1},\overline{2},\dots,\overline{7}\}$.
	\end{exercise}
}
\newpage
\slide{
	\begin{defn}
		We call a class $\abar\in\Z_n$ \emph{invertible} if there is some $\bbar\in Z_n$ such that $\abar\bbar=\overline{1}$.
	\end{defn}
	\begin{exa}
		Consider $\Z_4$.
		\vskip 1in
%		 $\overline{3}$ is invertible 
%		%because $\overline{3}\cdot\overline{3}=\overline{9}=\overline{1}$.
%		\vskip .5in
%		However, $\overline{2}$ is not invertible 
		%because we can test everything:
		%			\enumarabic{\item $\overline{2}\cdot\overline{0}=\overline{0}$
			%			\item $\overline{2}\cdot\overline{1}=\overline{1}$ 
			%			\item $\overline{2}\cdot\overline{2}=\overline{4}=\overline{0}$
			%			\item $\overline{2}\cdot\overline{3}=\overline{6}=\overline{2}$}
	\end{exa}
}
\slide{
	\begin{exercise}
		Show $\overline 6\in\Z_8$ has no multiplicative inverse.
	\end{exercise}
	\vskip 2in
	\begin{nb}
		Looking at the question of whether $\overline 6\in\Z_8$ has a multiplicative inverse, we can rephrase it by saying there is no solution to the congruence equation $\overline{6}x=\overline{1}$ in $\Z_8$.
	\end{nb}
}
	\frametitle{\fn}
	\begin{exercise}
		\enumalph{
			\item Solve $\overline{5}x=\overline{1}$ in $\Z_8$, if possible.
			\vfill
			\item Solve $\overline{5}x=\overline{2}$ in $\Z_8$, if possible.
			\vfill
			\item Solve $\overline{6}x=\overline{2}$ in $\Z_8$, if possible.
			\vfill
		}
	\end{exercise}
	\begin{nb}
		Here's some Sage code for some brute force that will print it nicely.
		\begin{quote}
			\begin{lstlisting}
sage: Zmod8=Integers(8)
sage: for a in Zmod8:
sage:     print(f"5*{a}={5*a} mod 8")\end{lstlisting}
		\end{quote}
		Use at \url{https://sagecell.sagemath.org/}.
	\end{nb}
\newpage
\slide{
	\begin{question}
		What do you notice about the relationship between $n$ and the values in $\Z_n$ that have inverses?
		
		Here we have multiplication tables for $\Z_7$, $\Z_8$, $\Z_9$, and $\Z_{10}$.  Identify the rows that have a 1 in them - these are the classes with inverses.
	\end{question}
	\vskip 1em
	\begin{minipage}{.45\textwidth}\small
		Multiplication in $\Z_7$
		
		$\begin{array}{r|rrrrrrr}
			\times & 0 & 1 & 2 & 3 & 4 & 5 & 6 \\\hline
			0 & 0 & 0 & 0 & 0 & 0 & 0 & 0 \\
			1 & 0 & 1 & 2 & 3 & 4 & 5 & 6 \\
			2 & 0 & 2 & 4 & 6 & 1 & 3 & 5 \\
			3 & 0 & 3 & 6 & 2 & 5 & 1 & 4 \\
			4 & 0 & 4 & 1 & 5 & 2 & 6 & 3 \\
			5 & 0 & 5 & 3 & 1 & 6 & 4 & 2 \\
			6 & 0 & 6 & 5 & 4 & 3 & 2 & 1
		\end{array}$
	\end{minipage}
	\hskip 2em
	\begin{minipage}{.45\textwidth}\small
		Multiplication in $\Z_8$
		
		$
		\begin{array}{r|rrrrrrrr}
			\times & 0 & 1 & 2 & 3 & 4 & 5 & 6 & 7 \\\hline
			0 & 0 & 0 & 0 & 0 & 0 & 0 & 0 & 0 \\
			1 & 0 & 1 & 2 & 3 & 4 & 5 & 6 & 7 \\
			2 & 0 & 2 & 4 & 6 & 0 & 2 & 4 & 6 \\
			3 & 0 & 3 & 6 & 1 & 4 & 7 & 2 & 5 \\
			4 & 0 & 4 & 0 & 4 & 0 & 4 & 0 & 4 \\
			5 & 0 & 5 & 2 & 7 & 4 & 1 & 6 & 3 \\
			6 & 0 & 6 & 4 & 2 & 0 & 6 & 4 & 2 \\
			7 & 0 & 7 & 6 & 5 & 4 & 3 & 2 & 1
		\end{array}
		$
	\end{minipage}
	
}
\slide{\hskip -1em
	\begin{minipage}{.45\textwidth}
		\small
		Multiplication in $\Z_9$
		
		$
		\begin{array}{r|rrrrrrrrr}
			\times & 0 & 1 & 2 & 3 & 4 & 5 & 6 & 7 & 8 \\\hline
			0 & 0 & 0 & 0 & 0 & 0 & 0 & 0 & 0 & 0 \\
			1 & 0 & 1 & 2 & 3 & 4 & 5 & 6 & 7 & 8 \\
			2 & 0 & 2 & 4 & 6 & 8 & 1 & 3 & 5 & 7 \\
			3 & 0 & 3 & 6 & 0 & 3 & 6 & 0 & 3 & 6 \\
			4 & 0 & 4 & 8 & 3 & 7 & 2 & 6 & 1 & 5 \\
			5 & 0 & 5 & 1 & 6 & 2 & 7 & 3 & 8 & 4 \\
			6 & 0 & 6 & 3 & 0 & 6 & 3 & 0 & 6 & 3 \\
			7 & 0 & 7 & 5 & 3 & 1 & 8 & 6 & 4 & 2 \\
			8 & 0 & 8 & 7 & 6 & 5 & 4 & 3 & 2 & 1
		\end{array}
		$
	\end{minipage}
	\hskip 2em
	\begin{minipage}{.45\textwidth}\small
		Multiplication in $\Z_{10}$
		
		$\begin{array}{r|rrrrrrrrrr}
			0 & 0 & 1 & 2 & 3 & 4 & 5 & 6 & 7 & 8 & 9 \\\hline
			0 & 0 & 0 & 0 & 0 & 0 & 0 & 0 & 0 & 0 & 0 \\
			1 & 0 & 1 & 2 & 3 & 4 & 5 & 6 & 7 & 8 & 9 \\
			2 & 0 & 2 & 4 & 6 & 8 & 0 & 2 & 4 & 6 & 8 \\
			3 & 0 & 3 & 6 & 9 & 2 & 5 & 8 & 1 & 4 & 7 \\
			4 & 0 & 4 & 8 & 2 & 6 & 0 & 4 & 8 & 2 & 6 \\
			5 & 0 & 5 & 0 & 5 & 0 & 5 & 0 & 5 & 0 & 5 \\
			6 & 0 & 6 & 2 & 8 & 4 & 0 & 6 & 2 & 8 & 4 \\
			7 & 0 & 7 & 4 & 1 & 8 & 5 & 2 & 9 & 6 & 3 \\
			8 & 0 & 8 & 6 & 4 & 2 & 0 & 8 & 6 & 4 & 2 \\
			9 & 0 & 9 & 8 & 7 & 6 & 5 & 4 & 3 & 2 & 1
		\end{array}$
	\end{minipage}
}
\slide{
	\begin{statementblock}{Theorem 1.3.5}
		Let $a,n\in\Z$ with $n\geq 2$. Then $\abar$ has a multiplicative inverse in $\Z_n$ if and only if $a$ and $n$ are relatively prime.
	\end{statementblock}
}
%\slide{
%	\begin{block}{Brain Break}
%		If you were a dog, what kind of toy would you prefer?
%		\vskip .25in
%		\begin{minipage}{.5\textwidth}	
%			\enumarabic{\item Fluffy and squeaky \item Crinkle toy \item BALL \item Tough chewer \item Puzzle toy with treats to find}
%		\end{minipage}
%	\end{block}
%}
\slide{
	Before starting the proof of Theorem 1.3.5, we recall two important Theorems:
	\begin{statementblock}{Theorem 1.2.4}
		Let $m,n\in Z$ not both zero.  Then
		\begin{center}
			$m,n$ relatively prime $\Leftrightarrow$ $\exists r,s\in\Z$ such that $1=rm+sn$
		\end{center}
	\end{statementblock}
	\vskip 1em
	\begin{statementblock}{Theorem 1.3.2}
		Given $n\geq 2$, $\overline{a}=\overline{b}\Leftrightarrow a\equiv b\pmod n$.
	\end{statementblock}
}
\slide{
	\begin{statementblock}{Theorem 1.3.5}
		Let $a,n\in\Z$ with $n\geq 2$. Then $\abar$ has a multiplicative inverse in $\Z_n$ if and only if $a$ and $n$ are relatively prime.
	\end{statementblock}
%	\begin{proof}
%		($\Rightarrow$) Assume $\abar$ has a multiplicative inverse in $\Z_n$.
%		\vskip 4in
%	\end{proof}
}
\slide{
	\begin{nb}
		The proof of the reverse direction of Theorem 1.3.5 helps us to find inverses.  
	\end{nb}
	\begin{exa}
		Find the inverse of $\overline{16}$ in $\Z_{35}$.
		
		\fbox{\begin{minipage}{2in}
				Euclidean Algorithm:
				
				$\begin{array}{rcl}
					35&=&2(16)+3\\
					16&=&5(3)+1\\
					3&=&3(1)+0
				\end{array}$
		\end{minipage}}
		\fbox{\begin{minipage}{2in}
				
				B\'ezout:
				
				$\begin{array}{rcl}
					1&=&16-5(3)\\
					&=&16-5(35-2(16))\\
					&=&11(16)-5(35)
				\end{array}$
		\end{minipage}}
		\vskip 1em
		The equation $1=11(16)-5(35)$ modulo 35 gives:
		\[1\equiv 11\cdot 16\pmod{35}.\]
		Therefore, the multiplicative inverse of $\overline{16}$ in $\Z_{35}$ is $\overline{11}$.
	\end{exa}
}
\slide{
	\begin{exercise}
		Solve the equation $\overline{16}x=\overline{9},$ in $\Z_{35}$.
	\end{exercise}
	\vskip 2in
}
\slide{
	\begin{exercise}
		Solve the system of equations in $\Z_{13}$
		\[\begin{cases}
			\overline{5}x+\overline{2}y=\overline{1}\\
			\overline{2}x+\overline{10}y=\overline{2}.
		\end{cases}\]
	\end{exercise}
	\vskip 2in
}
\newpage
\slide{
	\begin{statementblock}{Theorem 1.3.6 (The Chinese Remainder Theorem)}
		Let $m$ and $n$ be relatively prime integers. If $s$ and $t$ are arbitrary integers, then there is an integer $b$ for which
		\[b\equiv s\pmod m\text{ and } b\equiv t\pmod n.\]
	\end{statementblock}
	\begin{nb}
		How do we find this $b$?
		
		Since $\gcd(m,n)=1$, we can find $p,q\in \Z$ such that $1=mp+nq$. \fbox{why?}
		
		Set $b=(mp)t+(nq)s$. \fbox{why does this work???}
	\end{nb}
}
\slide{
	\begin{statementblock}{Theorem 1.3.7}
		The following are equivalent for any integer $n\geq 2$.
		\enumarabic{\item Every element $\abar\neq\overline{0}$ in $\Z_n$ has a multiplicative inverse.
			\item If $\abar\bbar=\overline{0}$ in $\Z_n$, then either $\abar=\overline{0}$ or $\bbar=\overline{0}$.
			\item The integer $n$ is prime.
		}
	\end{statementblock}
}
\slide{
	\begin{statementblock}{Wilson's Theorem - A Corollary to 1.3.7}
		If $p$ is prime then $(p-1)!\equiv -1\pmod p$.
	\end{statementblock}
	\begin{nb}
		Think about how numbers and their inverses mod $p$ appear in the product
		\[1\cdot 2\cdot3\cdots (p-1).\]
	\end{nb}
	\begin{statementblock}{Theorem 1.3.8 (Fermat's Theorem)}
		If $p$ is prime then $a^p\equiv a\pmod p$ for all $a\in\Z$.  Moreover, if $\gcd(a,p)=1$, then $a^{p-1}\equiv 1\pmod p$.
	\end{statementblock}
}
\end{document}

