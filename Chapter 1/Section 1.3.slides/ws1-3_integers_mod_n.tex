\documentclass[t]{beamer}

\subtitle{Section 1.3: Integers mod $n$}

\input{../../_python_tools/setup}

\begin{document} 
	\startdoc
	\topics{
		\item Division Algorithm
		\item GCD
		\item B\'ezout's Identity
		\item Euclidean Algorithm
		\item Prime Factorization Theorem
	}{
		\item Congruence modulo $n$
		\item Relations and Equivalence Classes
		\item Integers and Arithmetic modulo $n$
		\item Arithmetic Modulo $n$
		\item Inverses Modulo $n$
	}

\slide{
	\begin{defn}
		Let $a,b,n\in\Z$ with $n\geq 2$. We say that $a$ and $b$ are \emph{congruent modulo $n$} if 	\[n\mid (a-b).\]
		In that case, we write $a\equiv b\pmod n$.
	\end{defn}
}
\slide{
	\begin{statementblock}{Theorem 1.3.1}
		Congruence modulo $n$ is an equivalence relation on $\Z$.
	\end{statementblock}
}

\slide{
	\begin{exercise}
		Write the equivalence classes of $(\Z,\equiv\pmod 2)$.
	\end{exercise}
	\vskip 1in
	\begin{exercise}
		Write the equivalence classes of $(\Z,\equiv\pmod 3)$.
	\end{exercise}
}
\slide{
	\begin{defn}
		If $a\in\Z$, then its equivalence class, $[a]$, with respect to congruence modulo $n$ is called its \emph{residue class modulo $n$} and we write $\overline{a}$ for convenience.
		
			\[\overline{a}=\{x\in\Z : x\equiv a\pmod n\}.\]
	\end{defn}
}
%\slide{
%	\begin{statementblock}{Theorem 1.3.2}
%		Given $n\geq 2$, $\overline{a}=\overline{b}\Leftrightarrow a\equiv b\pmod n$.
%	\end{statementblock}
%%	\begin{block}{Proof Notes.}
%%		$(\Rightarrow)$ Follows from the definition of $\overline{a}$.
%%		
%%		$(\Leftarrow)$ Show the sets are equal!  Let $c\in\overline{a}$, use the fact that\\ $a\equiv b~\pmod n$ to show that $c\in\overline{b}$. 
%%		
%%		Do the same for $\overline{b}$ to $\overline{a}$. (Or explain that the argument will be functionally the same.)
%%	\end{block}
%}
%\slide{
%	\begin{block}{\textbf{Brain Break.}}
%		What might you say to a non-math person if you wanted to teach them about integers mod $n$?
%	\end{block}
%}
\slide{
	\begin{defn}
		The \emph{set of integers modulo $n$} is denoted $\Z_n$ and is given by
			\[\Z_n=\{\overline{0},\overline{1},\overline{2},\dots,\overline{n-1}\}.\]
	\end{defn}
	\begin{exa}
		$\Z_7=$
	\end{exa}
	\begin{exercise}
		What is $\overline{47}$ in $\Z_7$? What is $\overline{-16}$?
	\end{exercise}
}
\slide{
	\begin{statementblock}{Claim}
		Addition and multiplication in $\Z_n$, as defined below, are well-defined:
			\begin{itemize}
				\item[(1)] $\overline{a}+\overline{b}=\overline{a+b}$
				\item[(2)] $\overline{a}\overline{b}=\overline{ab}$
			\end{itemize}
	\end{statementblock}
	\begin{nb}
		The important point here is that any well-defined arithmetic operation on $\Z_n$ should NOT depend on the choice of residue class representative.
	\end{nb}
	\begin{exa}
		In $\Z_7$, $\overline{48}=\overline{6}$ and $\overline{3}=\overline{10}$.  Is it true that $\overline{48}+\overline{3}=\overline{6}+\overline{10}$?
	\end{exa}
}
\slide{
	\begin{block}{Proof.}
		It suffices to show that if $\overline{a_1}=\overline{a_2}$ and $\overline{b_1}=\overline{b_2}$ in $\Z_n$, then
			\[\overline{a_1+b_1}=\overline{a_2+b_2}\text{ and }\overline{a_1b_1}=\overline{a_2b_2}.\]
	\end{block}
}
\slide{
	\begin{exercise}
		Fill out the addition and multiplication tables for $\Z_4$.
		\Large
		$$
		\begin{array}{c|c|c|c|c}
		+_4&\overline{0}&\overline{1}&\overline{2}&\overline{3}\\\hline
		\overline{0}&&&&\\\hline
		\overline{1}&&&&\\\hline
		\overline{2}&&&&\\\hline
		\overline{3}&&&&
		\end{array}
		\hskip 3em
		\begin{array}{c|c|c|c|c}
		\times_4&\overline{0}&\overline{1}&\overline{2}&\overline{3}\\\hline
		\overline{0}&&&&\\\hline
		\overline{1}&&&&\\\hline
		\overline{2}&&&&\\\hline
		\overline{3}&&&&
		\end{array}$$
	\end{exercise}
}
\slide{
	\begin{statementblock}{Claim}
		An integer $n\in\Z$ is divisible by 9 if and only if the sum of its digits is divisible by 9.
	\end{statementblock}
}
\slide{
	\begin{block}{\textbf{Summary.}}
		\begin{itemize}[label=$\bullet$]
			\item The set of integers modulo $n$ is
			\[\Z_n=\{\overline{0},\overline{1},\overline{2},\dots,\overline{n-1}\}.\]
			\item If $r$ is the remainder you get when dividing $a$ by $n$, then
			\[a\equiv r\pmod{n} \text{ or equivalently } \abar=\overline{r}.\]
			\item Addition in $\Z_n$ is defined by:
			\[\overline{a}+\overline{b}=\overline{a+b}.\]
			\item Multiplication in $\Z_n$ is defined by 
			\[\overline{a}\overline{b}=\overline{ab}.\]
		\end{itemize}
	\end{block}
}
\slide{
	\begin{statementblock}{Theorem 1.3.4}
		Let $n\geq 2$ be a fixed modulus and let $a,b$ and $c$ denote arbitrary integers. Then the following hold in $\Z_n$.
		\enumarabic{\item $\abar+\bbar=\bbar+\abar$ and $\abar\bbar=\bbar\abar$.
			\item $\abar+(\bbar+\cbar)=(\abar+\bbar)+\cbar$ and $\abar(\bbar\cbar)=(\abar\bbar)\cbar$.
			\item $\abar+\overline{0}=\abar$ and $\abar\overline{1}=\abar$.
			\item $\abar+\overline{-a}=\overline{0}$.
			\item $\abar(\bbar+\cbar)=\abar\bbar+\abar\cbar$.
		}
	\end{statementblock}
	\begin{nb}
		The proof of (5) is in the book.  And (2) is proved in a video.
	\end{nb}
}
\slide{	
	Moral from last Theorem: Arithmetic in $\Z_n$ behaves very similarly to arithmetic in $\Z$! 
	\vskip 2em
	There's a \emph{zero}, $\overline{0}$, and \emph{unity}, $\overline{1}$, in $\Z_n$.
	\vskip 2em
	Every $\abar\in\Z_n$ has an \emph{negative} or \emph{additive inverse}, $\overline{-a}$, in $\Z_n$, which we write as $-\abar$ and satisfies
	\[\abar+\overline{-a}=\overline{0}.\]
	\vskip 1em
	\emph{Subtraction} is then naturally defined as
	\[\abar-\bbar=\abar+\overline{-b}=\overline{a-b}.\]
}
\slide{
	\begin{exercise}
		What is the additive inverse of $\overline{6}$ in $\Z_8$?
		
		%		Be sure to write it as a class in $\{\overline{0},\overline{1},\overline{2},\dots,\overline{7}\}$.
	\end{exercise}
}
\slide{
	\begin{defn}
		We call a class $\abar\in\Z_n$ \emph{invertible} if there is some $\bbar\in Z_n$ such that $\abar\bbar=\overline{1}$.
	\end{defn}
	\begin{exa}
		Consider $\Z_4$.
%		 $\overline{3}$ is invertible 
%		%because $\overline{3}\cdot\overline{3}=\overline{9}=\overline{1}$.
%		\vskip .5in
%		However, $\overline{2}$ is not invertible 
		%because we can test everything:
		%			\enumarabic{\item $\overline{2}\cdot\overline{0}=\overline{0}$
			%			\item $\overline{2}\cdot\overline{1}=\overline{1}$ 
			%			\item $\overline{2}\cdot\overline{2}=\overline{4}=\overline{0}$
			%			\item $\overline{2}\cdot\overline{3}=\overline{6}=\overline{2}$}
	\end{exa}
}
\slide{
	\begin{exercise}
		Show $\overline 6\in\Z_8$ has no multiplicative inverse.
	\end{exercise}
	\vskip 2in
	\begin{nb}
		Looking at this question as a polynomial equation, there is no solution to $\overline{6}x=\overline{1}$ in $\Z_8$.
	\end{nb}
}
\begin{frame}[fragile]
	\frametitle{\fn}
	\begin{exercise}
		\enumalph{
			\item Solve $\overline{5}x=\overline{1}$ in $\Z_8$, if possible.
			
			Brute force is a great plan.
			\item Solve $\overline{5}x=\overline{2}$ in $\Z_8$, if possible.
			\item Solve $\overline{6}x=\overline{2}$ in $\Z_8$, if possible.
		}
	\end{exercise}
	\begin{nb}
		Here's some Sage code for some brute force that will print it nicely.
		\begin{quote}
			\begin{verbatim}
				Zmod8=Integers(8)
				for a in Zmod8:
				print(f"5*{a}={5*a} mod 8")
			\end{verbatim}
		\end{quote}
		Use at \url{https://sagecell.sagemath.org/}.
	\end{nb}
\end{frame}
\slide{
	\begin{question}
		What do you notice about the relationship between $n$ and the values in $\Z_n$ that have inverses?
		
		This slide and the next have multiplication tables for $\Z_7$, $\Z_8$, $\Z_9$, and $\Z_{10}$.  Identify the rows that have a 1 in them - these are the classes with inverses.
	\end{question}
	\vskip 1em
	\begin{minipage}{.45\textwidth}\small
		Multiplication in $\Z_7$
		
		$\begin{array}{r|rrrrrrr}
			\times & 0 & 1 & 2 & 3 & 4 & 5 & 6 \\\hline
			0 & 0 & 0 & 0 & 0 & 0 & 0 & 0 \\
			1 & 0 & 1 & 2 & 3 & 4 & 5 & 6 \\
			2 & 0 & 2 & 4 & 6 & 1 & 3 & 5 \\
			3 & 0 & 3 & 6 & 2 & 5 & 1 & 4 \\
			4 & 0 & 4 & 1 & 5 & 2 & 6 & 3 \\
			5 & 0 & 5 & 3 & 1 & 6 & 4 & 2 \\
			6 & 0 & 6 & 5 & 4 & 3 & 2 & 1
		\end{array}$
	\end{minipage}
	\hskip 2em
	\begin{minipage}{.45\textwidth}\small
		Multiplication in $\Z_8$
		
		$
		\begin{array}{r|rrrrrrrr}
			\times & 0 & 1 & 2 & 3 & 4 & 5 & 6 & 7 \\\hline
			0 & 0 & 0 & 0 & 0 & 0 & 0 & 0 & 0 \\
			1 & 0 & 1 & 2 & 3 & 4 & 5 & 6 & 7 \\
			2 & 0 & 2 & 4 & 6 & 0 & 2 & 4 & 6 \\
			3 & 0 & 3 & 6 & 1 & 4 & 7 & 2 & 5 \\
			4 & 0 & 4 & 0 & 4 & 0 & 4 & 0 & 4 \\
			5 & 0 & 5 & 2 & 7 & 4 & 1 & 6 & 3 \\
			6 & 0 & 6 & 4 & 2 & 0 & 6 & 4 & 2 \\
			7 & 0 & 7 & 6 & 5 & 4 & 3 & 2 & 1
		\end{array}
		$
	\end{minipage}
	
	{\small * Overlines omitted for the sake of visual appearance.}
}
\slide{\hskip -1em
	\begin{minipage}{.45\textwidth}
		\small
		Multiplication in $\Z_9$
		
		$
		\begin{array}{r|rrrrrrrrr}
			\times & 0 & 1 & 2 & 3 & 4 & 5 & 6 & 7 & 8 \\\hline
			0 & 0 & 0 & 0 & 0 & 0 & 0 & 0 & 0 & 0 \\
			1 & 0 & 1 & 2 & 3 & 4 & 5 & 6 & 7 & 8 \\
			2 & 0 & 2 & 4 & 6 & 8 & 1 & 3 & 5 & 7 \\
			3 & 0 & 3 & 6 & 0 & 3 & 6 & 0 & 3 & 6 \\
			4 & 0 & 4 & 8 & 3 & 7 & 2 & 6 & 1 & 5 \\
			5 & 0 & 5 & 1 & 6 & 2 & 7 & 3 & 8 & 4 \\
			6 & 0 & 6 & 3 & 0 & 6 & 3 & 0 & 6 & 3 \\
			7 & 0 & 7 & 5 & 3 & 1 & 8 & 6 & 4 & 2 \\
			8 & 0 & 8 & 7 & 6 & 5 & 4 & 3 & 2 & 1
		\end{array}
		$
	\end{minipage}
	\hskip 2em
	\begin{minipage}{.45\textwidth}\small
		Multiplication in $\Z_{10}$
		
		$\begin{array}{r|rrrrrrrrrr}
			0 & 0 & 1 & 2 & 3 & 4 & 5 & 6 & 7 & 8 & 9 \\\hline
			0 & 0 & 0 & 0 & 0 & 0 & 0 & 0 & 0 & 0 & 0 \\
			1 & 0 & 1 & 2 & 3 & 4 & 5 & 6 & 7 & 8 & 9 \\
			2 & 0 & 2 & 4 & 6 & 8 & 0 & 2 & 4 & 6 & 8 \\
			3 & 0 & 3 & 6 & 9 & 2 & 5 & 8 & 1 & 4 & 7 \\
			4 & 0 & 4 & 8 & 2 & 6 & 0 & 4 & 8 & 2 & 6 \\
			5 & 0 & 5 & 0 & 5 & 0 & 5 & 0 & 5 & 0 & 5 \\
			6 & 0 & 6 & 2 & 8 & 4 & 0 & 6 & 2 & 8 & 4 \\
			7 & 0 & 7 & 4 & 1 & 8 & 5 & 2 & 9 & 6 & 3 \\
			8 & 0 & 8 & 6 & 4 & 2 & 0 & 8 & 6 & 4 & 2 \\
			9 & 0 & 9 & 8 & 7 & 6 & 5 & 4 & 3 & 2 & 1
		\end{array}$
	\end{minipage}
	* Overlines omitted for the sake of visual appearance.
}
\slide{
	\begin{statementblock}{Theorem 1.3.5}
		Let $a,n\in\Z$ with $n\geq 2$. Then $\abar$ has a multiplicative inverse in $\Z_n$ if and only if $a$ and $n$ are relatively prime.
	\end{statementblock}
}
%\slide{
%	\begin{block}{Brain Break}
%		If you were a dog, what kind of toy would you prefer?
%		\vskip .25in
%		\begin{minipage}{.5\textwidth}	
%			\enumarabic{\item Fluffy and squeaky \item Crinkle toy \item BALL \item Tough chewer \item Puzzle toy with treats to find}
%		\end{minipage}
%	\end{block}
%}
\slide{
	Before starting the proof of Theorem 1.3.5, we recall two important Theorems:
	\begin{statementblock}{Theorem 1.2.4}
		Let $m,n\in Z$ not both zero.  Then
		\begin{center}
			$m,n$ relatively prime $\Leftrightarrow$ $\exists r,s\in\Z$ such that $1=rm+sn$
		\end{center}
	\end{statementblock}
	\vskip 1em
	\begin{statementblock}{Theorem 1.3.2}
		Given $n\geq 2$, $\overline{a}=\overline{b}\Leftrightarrow a\equiv b\pmod n$.
	\end{statementblock}
}
\slide{
	\begin{statementblock}{Theorem 1.3.5}
		Let $a,n\in\Z$ with $n\geq 2$. Then $\abar$ has a multiplicative inverse in $\Z_n$ if and only if $a$ and $n$ are relatively prime.
	\end{statementblock}
%	\begin{proof}
%		($\Rightarrow$) Assume $\abar$ has a multiplicative inverse in $\Z_n$.
%		\vskip 4in
%	\end{proof}
}
\slide{
	\begin{nb}
		The proof of the reverse direction of Theorem 1.3.5 helps us to find inverses.  
	\end{nb}
	\begin{exa}
		Find the inverse of $\overline{16}$ in $\Z_{35}$.
		
		\fbox{\begin{minipage}{2in}
				Euclidean Algorithm:
				
				$\begin{array}{rcl}
					35&=&2(16)+3\\
					16&=&5(3)+1\\
					3&=&3(1)+0
				\end{array}$
		\end{minipage}}
		\fbox{\begin{minipage}{2in}
				
				B\'ezout:
				
				$\begin{array}{rcl}
					1&=&16-5(3)\\
					&=&16-5(35-2(16))\\
					&=&11(16)-5(35)
				\end{array}$
		\end{minipage}}
		\vskip 1em
		The equation $1=11(16)-5(35)$ modulo 35 gives:
		\[1\equiv 11\cdot 16\pmod{35}.\]
		Therefore, the multiplicative inverse of $\overline{16}$ in $\Z_{35}$ is $\overline{11}$.
	\end{exa}
}
\slide{
	\begin{exercise}
		Solve the equation $\overline{16}x=\overline{9},$ in $\Z_{35}$.
	\end{exercise}
}
\slide{
	\begin{exercise}
		Solve the system of equations in $\Z_{13}$
		\[\begin{cases}
			\overline{5}x+\overline{2}y=\overline{1}\\
			\overline{2}x+\overline{10}y=\overline{2}.
		\end{cases}\]
	\end{exercise}
}
\slide{
	\begin{statementblock}{Theorem 1.3.6 (The Chinese Remainder Theorem)}
		Let $m$ and $n$ be relatively prime integers. If $s$ and $t$ are arbitrary integers, then there is an integer $b$ for which
		\[b\equiv s\pmod m\text{ and } b\equiv t\pmod n.\]
	\end{statementblock}
	\begin{nb}
		How do we find this $b$?
		
		Since $\gcd(m,n)=1$, we can find $p,q\in \Z$ such that $1=mp+nq$. \fbox{why?}
		
		Set $b=(mp)t+(nq)s$. \fbox{why does this work???}
	\end{nb}
}
\slide{
	\begin{statementblock}{Theorem 1.3.7}
		The following are equivalent for any integer $n\geq 2$.
		\enumarabic{\item Every element $\abar\neq\overline{0}$ in $\Z_n$ has a multiplicative inverse.
			\item If $\abar\bbar=\overline{0}$ in $\Z_n$, then either $\abar=\overline{0}$ or $\bbar=\overline{0}$.
			\item The integer $n$ is prime.
		}
	\end{statementblock}
}
\slide{
	\begin{statementblock}{Wilson's Theorem - A Corollary to 1.3.7}
		If $p$ is prime then $(p-1)!\equiv -1\pmod p$.
	\end{statementblock}
	\begin{nb}
		Think about how numbers and their inverses mod $p$ appear in the product
		\[1\cdot 2\cdot3\cdots (p-1).\]
	\end{nb}
	\begin{statementblock}{Theorem 1.3.8 (Fermat's Theorem)}
		If $p$ is prime then $a^p\equiv a\pmod p$ for all $a\in\Z$.  Moreover, if $\gcd(a,p)=1$, then $a^{p-1}\equiv 1\pmod p$.
	\end{statementblock}
}
\end{document}

