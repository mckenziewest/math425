\documentclass[t]{beamer}

\subtitle{Section 1.1: Induction}

\input{../../_tools/setup}

\begin{document} 
	\startdoc
\topics{
	\item Relations
	\item Equivalence Relations
	\item Equivalence Classes
}{
	\item Induction
}

\slide{
	\begin{statementblock}{Principle of Mathematical Induction}
		Let $P(n)$ be a statement for each integer $n\geq  m$. Suppose the following conditions are satisfied,
			\enumarabic{\item $P(m)$ is true, and \item $P(k)\Rightarrow P(k+1)$ for every $k\geq m$.}
		Then $P(n)$ is true for every $n\geq m$.
	\end{statementblock}
%	\begin{exa}
%		\textbf{Distribution:} It is known that if $a,b,c\in\R$, then $a(b+c)=ab+ac$.
%		
%		\textbf{Claim:} For all $x\in \R$ and $n\geq 1$,
%			\[(1-x)(1+x+\cdots+x^{n-1})=1-x^n.\]
%		\begin{proof}
%			\vskip 1in
%		\end{proof}
%	\end{exa}
}

\slide{
	\begin{exa}
		\textbf{Claim:} For all $n\geq 4$,
			\[2^n<n!.\]
%		\begin{proof}
%			\textbf{Base Case:} Notice that when $n=4$, $2^n=16$ and $n!=24$.  Thus $2^n<n!$.
%			
%			\textbf{Inductive Step:} Let $k\geq 4$ be an integer.  Assume $2^k<k!$.
%			\vskip 3in
%		\end{proof}
	\end{exa}
}
\slide{
	\begin{exa}
		Show that $3^{3n}+1$ is a multiple of 7 for all odd $n\geq 1$.
%		\begin{proof}
%			\textbf{Base Case:} Notice that when $n=1$, \vskip .5in
%			
%			\textbf{Inductive Step:} Let $k\geq 1$ be an odd integer.
%			\vskip 3in
%		\end{proof}
	\end{exa}
}
\slide{
	\begin{block}{\textbf{Brain Break.}}
		What candies (if any) do you think all of the colors taste the same?
	\end{block}
}
%\slide{
%	\begin{exercise}
%		Consider the statement, for all $n\geq 1$,
%			\[1^3+2^3+\cdots+n^3=\frac{1}{4}n^2(n+1)^2.\]
%		Use Induction to prove this statement.
%			\enumalph{
%				\item What is the base case?\vskip .5in
%				\item What is the inductive hypothesis?\vskip .5in
%				\item Use the inductive hypothesis to complete the proof.
%			}
%	\end{exercise}
%}
\slide{
	\begin{block}{\textbf{Induction in Algebra}.}
		We assume the distributive property $a(b+c)=ab+ac$.
		
		Prove that this property can be extended, that is show that for all integers $n\geq 2$, for all $a,b_1,\dots,b_n\in\Z$, we have
			 $$a(b_1+b_2+\cdots+b_n) = ab_1+ab_2+\cdots+ab_n.$$
	\end{block}
}
\slide{
\begin{exercise}
	Consider the statement, for all $n\geq 1$,
	\[n^3+(n+1)^3+(n+2)^3\]
	is a multiple of 9.
	
	Use Induction to prove this statement.
	\enumalph{
		\item What is the base case?\vskip .5in
		\item What is the inductive hypothesis?\vskip .5in
		\item Use the inductive hypothesis to complete the proof.
		
			Hint: Do NOT do any expanding of the terms $(k+1)^3$ and $(k+2)^3$.
	}
\end{exercise}
}
\slide{
\begin{exercise}
	Let $a_n=2^{3n}-1$ for $n\geq 0$. Guess a common divisor for each $a_n$ and prove your assertion.
	(The common divisor should be greater than 1.)
	\enumalph{
		\item Use a calculator or math software co compute $a_1,a_2,a_3,a_4$.  What common divisor do you notice?\vskip 1in
		\item Use induction to prove your claim.
	}
\end{exercise}
}
\slide{
	\begin{statementblock}{Another Induction Principle}
		Let $P(n)$ be a statement for each integer $n\geq  m$. Suppose the following conditions are satisfied,
		\enumarabic{\item $P(m)$ and $P(m+1)$ are true, and \item If $k\geq m$ and both $P(k)$ and $P(k+1)$ are true then $P(k+2)$ is true.}
		Then $P(n)$ is true for every $n\geq m$.
	\end{statementblock}
\begin{exercise}
	Let $a_n$ denote a number for each integer $n\geq 0$ and assume that $a_{n+2}=a_{n+1}+2a_n$ holds for every $n\geq 0$. 
	
	Show that if $a_0=1$ and $a_2=2$, then $a_n=2^n$ for each $n\geq 0$.
\end{exercise}
}
\end{document}

