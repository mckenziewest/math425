\documentclass[12pt]{article}

\newcommand{\secname}{Section 3.2: Integral Domains and Fields}

\usepackage{amsthm,amsmath,amsfonts,hyperref,graphicx,color,multicol,soul}
\usepackage{enumitem,tikz,tikz-cd,setspace,mathtools}
\usepackage{colortbl}
\usepackage[margin=1in]{geometry}

%%%%%%%%%%
%Color Customization
%%%%%%%%%%

\definecolor{Blu}{RGB}{43,62,133} % UWEC Blue

%Unnumbered footnotes:
\newcommand{\blfootnote}[1]{%
	\begingroup
	\renewcommand\thefootnote{}\footnote{#1}%
	\addtocounter{footnote}{-1}%
	\endgroup
}

%%%%%%%%%%
%TikZ Stuff
%%%%%%%%%%
\usetikzlibrary{arrows}
\usetikzlibrary{shapes.geometric}
\tikzset{
	smaller/.style={
		draw,
		regular polygon,
		regular polygon sides=3,
		fill=white,
		node distance=2cm,
		minimum height=1in,
		line width = 2pt
	}
}
\tikzset{
	smsquare/.style={
		draw,
		regular polygon,
		regular polygon sides=4,
		fill=white,
		node distance=2cm,
		minimum height=1in,
		line width = 2pt
	}
}

%%%%%%%%%%
%Listing Setup
%%%%%%%%%%
\usepackage{listings}
\usepackage{caption, floatrow, makecell}%
\captionsetup{labelfont = sc}
\setcellgapes{3pt}

\definecolor{backcolour}{RGB}{237,236,230}
\definecolor{myblue}{RGB}{42,157,189}

\lstdefinestyle{mystyle}{
	language=Python,
	keywords=[2]{sage:},
	alsodigit={:,.,<,>},
	backgroundcolor=\color{backcolour},   
	commentstyle=\color{myblue},
	keywordstyle=\bfseries\color{Green},
	keywordstyle=[2]\color{purple},
	numberstyle=\tiny\color{Gray},
	stringstyle=\color{Orange},
	basicstyle=\ttfamily\footnotesize,
	breakatwhitespace=false,         
	breaklines=true,                 
	captionpos=b,                    
	keepspaces=true,                   
	showspaces=false,                
	showstringspaces=false,
	showtabs=false,                  
	tabsize=2
}

\lstset{style=mystyle}


%%%%%%%%%%
%Custom Commands
%%%%%%%%%%

\newcommand{\C}{\mathbb{C}}
\newcommand{\quats}{\mathbb{H}}
\newcommand{\N}{\mathbb{N}}
\newcommand{\Q}{\mathbb{Q}}
\newcommand{\R}{\mathbb{R}}
\newcommand{\Z}{\mathbb{Z}}

\newcommand{\ds}{\displaystyle}

\newcommand{\fn}{\insertframenumber}

\newcommand{\id}{\operatorname{id}}
\newcommand{\im}{\operatorname{im}}
\newcommand{\lcm}{\operatorname{lcm}}
\newcommand{\ord}{\operatorname{ord}}
\newcommand{\Aut}{\operatorname{Aut}}
\newcommand{\Inn}{\operatorname{Inn}}

\newcommand{\blank}[1]{\underline{\hspace*{#1}}}

\newcommand{\abar}{\overline{a}}
\newcommand{\bbar}{\overline{b}}
\newcommand{\cbar}{\overline{c}}

\newcommand{\nml}{\unlhd}

%%%%%%%%%%
%Custom Theorem Environments
%%%%%%%%%%
\theoremstyle{definition}
\newtheorem{exercise}{Exercise}
\newtheorem{question}[exercise]{Question}
\newtheorem{warmup}{Warm-Up}
\newtheorem*{exa}{Example}
\newtheorem*{defn}{Definition}
\newtheorem*{disc}{Group Discussion}
\newtheorem*{recall}{Recall}
\renewcommand{\emph}[1]{{\color{blue}\texttt{#1}}}

\definecolor{Gold}{RGB}{237, 172, 26}
%Statement block
%\newenvironment{statementblock}[1]{%
%	\setbeamercolor{block body}{bg=Gold!20}
%	\setbeamercolor{block title}{bg=Gold}
%	\begin{block}{\textbf{#1.}}}{\end{block}}
%\newenvironment{goldblock}{%
%	\setbeamercolor{block body}{bg=Gold!20}
%	\setbeamercolor{block title}{bg=Gold}
%	\setbeamertemplate{blocks}[shadow=true]
%	\begin{block}{}}{\end{block}}
%\newenvironment{defn}{%
%	\setbeamercolor{block body}{bg=gray!20}
%	\setbeamercolor{block title}{bg=violet, fg=white}
%	\setbeamertemplate{blocks}[shadow=true]
%	\begin{block}{\textbf{Definition.}}}{\end{block}}
%\newenvironment{nb}{%
%	\setbeamercolor{block body}{bg=gray!20}
%	\setbeamercolor{block title}{bg=teal, fg=white}
%	\setbeamertemplate{blocks}[shadow=true]
%	\begin{block}{\textbf{Note.}}}{\end{block}}
%\newenvironment{blockexample}{%
%	\setbeamercolor{block body}{bg=gray!20}
%	\setbeamercolor{block title}{bg=Blu, fg=white}
%	\setbeamertemplate{blocks}[shadow=true]
%	\begin{block}{\textbf{Example.}}}{\end{block}}
%\newenvironment{blocknonexample}{%
%	\setbeamercolor{block body}{bg=gray!20}
%	\setbeamercolor{block title}{bg=purple, fg=white}
%	\setbeamertemplate{blocks}[shadow=true]
%	\begin{block}{\textbf{Non-Example.}}}{\end{block}}
%\newenvironment{thm}[1]{%
%	\setbeamercolor{block body}{bg=Gold!20}
%	\setbeamercolor{block title}{bg=Gold}
%	\begin{block}{\textbf{Theorem #1.}}}{\end{block}}


%%%%%%%%%%
%Custom Environment Wrappers
%%%%%%%%%%
\newcommand{\exer}[1]{
	\begin{exercise}
	#1
	\end{exercise}
}
\newcommand{\exam}[1]{
\textbf{Example: }
	#1
}
\newcommand{\nexam}[1]{
	\textbf{Non-Example: }
	#1
}
\newcommand{\enumarabic}[1]{
	\begin{enumerate}[label=\textbf{\arabic*.}]
		#1
	\end{enumerate}
}
\newcommand{\enumalph}[1]{
	\begin{enumerate}[label=(\alph*)]
		#1
	\end{enumerate}
}
\newcommand{\bulletize}[1]{
	\begin{itemize}[label=$\bullet$]
		#1
	\end{itemize}
}
\newcommand{\circtize}[1]{
	\begin{itemize}[label=$\circ$]
		#1
	\end{itemize}
}
%\newcommand{\slide}[1]{
%	\begin{frame}{\fn}
%		#1
%	\end{frame}
%}
%\newcommand{\slidec}[1]{
%\begin{frame}[c]{\fn}
%	#1
%\end{frame}
%}
%\newcommand{\slidet}[2]{
%	\begin{frame}{\fn\ - #1}
%		#2
%	\end{frame}
%}


\setlength{\parindent}{0pt}



\usepackage{afterpage}
\usepackage{fancyhdr}

\fancyhead[L]{\textbf{Math 425: Abstract Algebra I\\\secname}}
\fancyhead[R]{\textbf{Mckenzie West\\Last Updated: \today}}
\pagestyle{fancy}

\newcommand{\startdoc}{}

\newcommand{\topics}[2]{
		{\textbf{Previously.}}
			\begin{itemize}[label=--]
				#1
			\end{itemize}
		{\textbf{This Section.}}
			\begin{itemize}[label=--]
				#2
			\end{itemize}
}

\begin{document} 
	\startdoc
	
	\topics{
		\item Rings
		\item Commutative Rings
		\item Fields
		\item Subrings
		\item Ring Isomorphisms
	}{
		\item Domains
		\item Integral Domains
		\item Fields
	}

\slide{
	\begin{recall}
		We call $a\in R$ a \emph{zero-divisor} if $a\neq 0_R$ and there is some $b\neq 0_R$ in $R$ with $ab=0_R$. 
	\end{recall}
	\exam{
		In $\Z_6$, $2$ is a zero divisor.  (Verify this!)\vskip .5in
	}
	\begin{defn}
		A ring $R\neq\{0\}$ is called a \emph{domain} if $ab=0$ implies that either $a=0$ or $b=0$.
	\end{defn}
	\exer{
		Which of the following are domains?
		\enumalph{
			\item $\Z_6$\vskip 2em
			\item $\Z_5$\vskip 2em
			\item $\Q$\vskip 2em
			\item $M_2(\R)$\vskip 2em
		}
	}
	
}

\slide{
	\begin{defn}
		A commutative domain is called an \emph{integral domain}.
	\end{defn}
	\begin{exa}The following are all integral domains.
		\enumalph{
			\item $\Z$\vskip 2em
			\item $\Z[\sqrt{2}]$\vskip 2em
			\item $\Z[i]$\vskip 2em
			\item Rings of polynomials whose coefficients come from an integral domain, such as $\Z[x]$
		}
	\end{exa}
}
\newpage
\slide{
	\begin{statementblock}{Theorem}
		If $u\in R$ is a unit then $u$ is not a zero divisor.
	\end{statementblock}
	\vskip 2in
}

\slide{
	\begin{exercise}
		Use the last theorem to show that $\Z_p$ is an integral domain if $p$ is prime.\vskip 3in
	\end{exercise}
}

\slide{
	\begin{exercise}
		Some rings that are \textit{not} integral domains. Why are they not?
		\enumalph{\item $M_2(\R)$\vfill \item $\Z_m$ where $m$ is a composite number\vfill\item $\Z\times \Z$\vfill}
	\end{exercise}
}

\newpage
\slide{
	\begin{thm}{3.2.1}
		The following are equivalent for a ring $R$.
		\enumarabic{\item If $ab=0$ in $R$, then $a=0$ or $b=0$.\item If $ab=ac$ in $R$ and $a\neq 0$, then $b=c$.\item If $ba=ca$ in $R$ and $a\neq 0$, then $b=c$.}
	\end{thm}
	\begin{nb}
		Note that what this is say is that we can only cancel in multiplication if there are no zero divisors. In which case, we can always cancel whether or not $a^{-1}$ exists!!!
	\end{nb}
	\vskip 2in
}

\slide{
\begin{recall}
	A \emph{field} is a commutative ring such that every non-zero element is a unit.
\end{recall}
\begin{nb}
	From the last Theorem, every field is a division ring.
\end{nb}
}
\slide{
	\begin{exercise}
		 Claim: $\Q(\sqrt{2})=\{a+b\sqrt{2}\ |\ a,b\in\Q\}$ is a field
%		 \begin{quote}
%		 	Define the \emph{conjugate} of $r=a+b\sqrt{2}\in\Q(\sqrt{2})$ as $r^*$ where $r^*=a-b\sqrt{2}$.
%		 	
%		 	Define the \emph{norm} $N(r)$ to be $a^2-2b^2$.
%		 \end{quote}
	 	
	 	Given $r=a+b\sqrt{2}\in \Q(\sqrt{2})$ with $r\neq 0$, what is $r^{-1}$?  
	 	
	 	That is, write $r^{-1}=c+d\sqrt{2}$ where $c,d$ are rational numbers in terms of $a$ and $b$.
	\end{exercise}
	\vfill
}
\slide{
	\begin{defn}
		We say $z\in \C$ is \emph{algebraic over $\Q$} if there is some polynomial $p\in \Q[x]$ such that $p(z)=0$.
	\end{defn}
	\begin{defn}
		The \emph{number field} generated by $z$ is the field $\Q(z)$, which is the set of complex numbers of the form $a_0+a_1z+a_2z^2+\cdots+a_k z^k$ where $k\in\N$ and $a_0,a_1,\dots,a_k\in\Q$.
	\end{defn}
}
\newpage

\slide{
	\begin{exercise}
		Why is $\Z[i]$ not a field?  (It is an integral domain though!)
		\vfill
	\end{exercise}
}

\slide{
	\begin{exercise}
		Show $\Z_3(i)=\{a+bi\ :\ a,b\in\Z_3,i^2=-1\}$ is a field.
		\vfill
	\end{exercise}
}

\slide{
	\begin{thm}{3.2.2}
		The characteristic of any domain is either zero or a prime.
	\end{thm}
	\begin{question}
		Verify $\operatorname{char}(\Z_n)=n$.  How does this theorem imply $\Z_n$ is a domain if and only if $n$ is prime.
	\end{question}
}
\slide{
	\begin{thm}{3.2.3}
		Every finite integral domain is a field.
	\end{thm}
	\begin{question}
		How does this theorem imply $\Z_p$ if a field for $p$ prime?
	\end{question}
	\vskip 1in
}
\slide{
	\begin{statementblock}{Wedderburn's Theorem}
		Every finite division ring is a field.
	\end{statementblock}
	\begin{nb}
		A division ring is a ring in which every nonzero element has an inverse. (i.e., every $r\neq0\in R$ is a unit). But division rings are called fields if they are commutative.
	\end{nb}
	\begin{question}
		Can you think of a non-commutative division ring?
	\end{question}
	\vskip 1in
}

%\slidet{Field of Quotients}{
%	See pages 170-172
%	
%	The motivation here is the relationship between $\Z$ and $\Q$.  Here $\Q$ is the ring we get if we make every element of $\Z$ a unit.
%	
%	In fact, we can do this for every integral domain.
%	
%	\begin{defn}
%		If $R$ is an integral domain, then the field $Q=\{\frac{r}{u}\ |\ r,u\in R \text{ and }u\neq 0\}$ is called the \emph{field of quotients} or \emph{fraction field of $R$}.
%		
%		In $Q$, we say $\frac{r}{u}=\frac{s}{v}$ if $rv-us=0$.
%	\end{defn}
%}

\end{document}

	