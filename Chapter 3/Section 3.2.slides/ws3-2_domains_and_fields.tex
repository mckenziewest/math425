\documentclass[t]{beamer}

\subtitle{Section 3.2: Integral Domains and Fields}

\input{../../_tools/setup}

\begin{document} 
	\startdoc
	
	\topics{
		\item Rings
		\item Commutative Rings
		\item Fields
		\item Subrings
		\item Ring Isomorphisms
	}{
		\item Domains
		\item Integral Domains
		\item Fields
	}

\slide{
	\begin{recall}
		We call $a\in R$ a \emph{zero-divisor} if $a\neq 0_R$ and there is some $b\neq 0_R$ in $R$ with $ab=0_R$. 
	\end{recall}
	\begin{defn}
		A ring $R\neq\{0\}$ is called a \emph{domain} if $ab=0$ implies that either $a=0$ or $b=0$.
	\end{defn}
}

\slide{
	\begin{defn}
		A commutative domain is called an \emph{integral domain}.
	\end{defn}
	\begin{exa}
		\enumalph{
			\item $\Z$
			\item $\Z[\sqrt{2}]$
			\item $\Z[i]$
			\item Rings of polynomials whose coefficients come from an integral domain.
		}
	\end{exa}
}
\slide{
	\begin{statementblock}{Theorem}
		If $u\in R$ is a unit then $u$ is not a zero divisor.
	\end{statementblock}
}

\slide{
	\begin{exercise}
		Use the last theorem to show that $\Z_p$ is an integral domain if $p$ is prime.\vskip 3in\mbox{}
	\end{exercise}
}

\slide{
	\begin{exa}
		Some rings that are \textit{not} integral domains.
		\enumalph{\item $M_2(\R)$\vskip .75in \item $\Z_m$ where $m$ is a composite number\vskip .75in\item $\Z\times \Z$\vskip .75in \mbox{}}
	\end{exa}
}

\slide{\begin{thm}{3.2.1}
		The following are equivalent for a ring $R$.
		\enumarabic{\item If $ab=0$ in $R$, then $a=0$ or $b=0$.\item If $ab=ac$ in $R$ and $a\neq 0$, then $b=c$.\item If $ba=ca$ in $R$ and $a\neq 0$, then $b=c$.}
\end{thm}}

\slide{
\begin{recall}
	A \emph{field} is a commutative ring such that every non-zero element is a unit.
\end{recall}
\begin{nb}
	From the last Theorem, every field is a division ring.
\end{nb}
}
\slide{
	\begin{exercise}
		 Claim: $\Q(\sqrt{2})=\{a+b\sqrt{2}\ |\ a,b\in\Q\}$ is a field
%		 \begin{quote}
%		 	Define the \emph{conjugate} of $r=a+b\sqrt{2}\in\Q(\sqrt{2})$ as $r^*$ where $r^*=a-b\sqrt{2}$.
%		 	
%		 	Define the \emph{norm} $N(r)$ to be $a^2-2b^2$.
%		 \end{quote}
	 	
	 	Given $r=a+b\sqrt{2}\in \Q(\sqrt{2})$ with $r\neq 0$, what is $r^{-1}$?
	\end{exercise}
}
\slide{
	\begin{defn}
		We say $z\in \C$ is \emph{algebraic over $\Q$} if there is some polynomial $p\in \Q[x]$ such that $p(z)=0$.
		
		The \emph{number field} generated by $z$ is the field $\Q(z)$, which is the set of complex numbers of the form $a_0+a_1z+a_2z^2+\cdots+a_k z^k$ where $k\in\N$ and $a_0,a_1,\dots,a_k\in\Q$.
	\end{defn}
}

\slide{
	\begin{exa}
		Why is $\Z[i]$ not a field?  (It is an integral domain though!)
	\end{exa}
}

\slide{
	\begin{exa}
		Show $\Z_3(i)$ is a field.
	\end{exa}
}

\slide{
	\begin{thm}{3.2.2}
		The characteristic of any domain is either zero or a prime.
	\end{thm}
	\begin{question}
		Verify $\operatorname{char}(\Z_n)=n$.  How does this theorem imply $\Z_n$ is a domain if and only if $n$ is prime.
	\end{question}
}
\slide{
	\begin{thm}{3.2.3}
		Every finite integral domain is a field.
	\end{thm}
	\begin{question}
		How does this theorem imply $\Z_p$ if a field for $p$ prime?
	\end{question}
}
\slide{
	\begin{statementblock}{Wedderburn's Theorem}
		Every finite division ring is a field.
	\end{statementblock}
	\begin{nb}
		A division ring is a ring in which every nonzero element has an inverse. (i.e., every $r\neq0\in R$ is a unit). But division rings are called fields if they are commutative.
	\end{nb}
	\begin{question}
		Can you name a non-commutative division ring?
	\end{question}
}

\slidet{Field of Quotients}{
	See pages 170-172
	
	The motivation here is the relationship between $\Z$ and $\Q$.  Here $\Q$ is the ring we get if we make every element of $\Z$ a unit.
	
	In fact, we can do this for every integral domain.
	
	\begin{defn}
		If $R$ is an integral domain, then the field $Q=\{\frac{r}{u}\ |\ r,u\in R \text{ and }u\neq 0\}$ is called the \emph{field of quotients} or \emph{fraction field of $R$}.
		
		In $Q$, we say $\frac{r}{u}=\frac{s}{v}$ if $rv-us=0$.
	\end{defn}
}

\end{document}

	