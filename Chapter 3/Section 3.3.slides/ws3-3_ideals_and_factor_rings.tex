\documentclass[t]{beamer}

\subtitle{Section 3.3: Ideals and Factor Rings}

\input{../../_tools/setup}

\begin{document} 
	\startdoc
	
	\topics{
		\item Domains
		\item Integral Domains
		\item Fields
	}{
		\item Ideals
		\item Principal Ideals
		\item Prime Ideals
		\item Maximal Ideals
	}

\slide{
	\begin{statementblock}{Lemma}
		If $(S,+)\leq (R,+)$, then \vskip 1em
			\[(S+a)(S+b)=\hspace*{2in}\]\vskip 1em
		is well-defined if and only if 
			\[rS\subseteq S\quad\text{and}\quad Sr\subseteq S,\quad\text{for all}r\in R.\]
	\end{statementblock}
	\begin{defn}
		Let $(R,+,\cdot)$ be a ring. An additive subgroup $(I,+)$ of $(R,+)$ is an \emph{ideal of $R$} if $rI\subseteq I$ and $Ir\subseteq I$ for all $r\in R$.
	\end{defn}
}
\slide{
	\begin{defn}
		Equivalent definitions of an \emph{ideal} $I$ of a ring $R$: (given $(I,+)\leq (R,+)$)
		\bulletize{
			\item for all $i\in I$, $iR\subseteq I$ and $Ri\subseteq I$
			\item for all $i\in I$ and $r\in R$, $ir\in I$ and $ri\in I$.
		}
	\end{defn}
	\begin{nb}
		Some may call this a ``two-sided ideal''.  There also exist ``left ideals'' and ``right ideals''.
		
		In a commutative ring, every ideal is two-sided.
	\end{nb}
}

\slide{
	\begin{exa}
		Show that $2\Z$ is an ideal of the ring $\Z$.\vskip 3in\mbox{}
\end{exa}}
\slide{
	\begin{exercise}
		Show that $\Z$ is not an ideal of the ring $\Q$.\vskip 3in \mbox{} 
	\end{exercise}
}
\slide{
	\begin{block}{\textbf{Warning.}}
		The following is up for debate:
		
			If $I$ is an ideal of $R$, it does not guarantee that $I$ is a subring of $R$.
			
		Why?  It depends on the requirements of having a 1.
	\end{block}
}

\slide{
	\begin{thm}{3.3.2}
		If $I$ is an ideal of the ring $R$ (that has unity), then the following are equivalent
		\enumarabic{\item $1\in I$\item $I$ contains a unit\item $I=R$}
	\end{thm}
}

\slide{
	\begin{thm}{3.3.1}
		Let $I$ be an ideal of the ring $R$ (with unity).  Then the additive group $(R/I,+)$ becomes a ring with multiplication $(r+I)(s+I)=rs+I$ called the \emph{factor ring} or \emph{quotient ring}.  The unity of $R/I$ is $1+I$ and if $R$ is commutative, then $R/I$ is commutative.
	\end{thm}
}
\slide{
	\begin{block}{\textbf{Observations.}}
		\enumarabic{
			\item For an ring $R$, $\{0\}$ and $R$ are ideals of $R$.
			\item $R/R\cong \{0\}$ and $R/\{0\}\cong R$
			\item Everything from quotient groups extends to quotient rings
				\enumalph{
						\setlength{\itemsep}{1.5em}
					\item $r+I=s+I$ if and only if $r-s\in I$
					\item $(r+I)+(s+I)=(r+s)+I$
					\item $0+I=I$
					\item $-(r+I)=-r+I$
					\item $k(r+I)=kr+I$ for all $k\in \Z$
				}
		}
	\end{block}
}
\slide{
	Much like $2\Z$, an ideal of $\Z$, we can write
		\begin{quote}
			$Ra=\{ra\ |\ r\in R\}$\\
			$aR=\{ar\ |\ r\in R\}$
		\end{quote}
	\begin{defn}
		If $a\in Z(R)$, then $Ra=aR$ and we call this set the \emph{principal ideal of $R$ generated by $a$}.  Denote this set by $(a)$.
	\end{defn}
	\begin{block}{\textbf{Warning.}}
		The book uses $\langle a \rangle$ for the ideal generated by $a$.  To avoid mixing it up with cyclic groups, we'll use $(a)$.
	\end{block}
}
\slide{
	\begin{exercise}
		Is $6\Z$ a pincipal ideal of $\Z$?\vskip 3in\mbox{}
	\end{exercise}
}
\slide{
	\begin{exercise}
		Consider $R=\Z[i]$ and $I=(2+i)$.  Show that $$R/I=\{0+I,1+I,2+I,3+I,4+I\}.$$
	\end{exercise}
}
\slide{
\begin{nb}
	There are many examples of ideals that are not principal.  One example of this is the ideal \[(2,1+\sqrt{-5})=\{r(2)+s(1+\sqrt{-5})\ |\ r,s\in\Z[\sqrt{-5}]\}\] of $\Z[\sqrt{-5}]$.  See: {\footnotesize\url{https://math.stackexchange.com/questions/543216/proving-that-a-ring-is-not-a-principal-ideal-domain}}
\end{nb}
}

\slide{
\begin{defn}
	We call a proper ideal $P$ of a ring $R$ \emph{prime} if 
	\[rs\in P\quad\Rightarrow\quad r\in P\text{ or }s\in P.\]
\end{defn}
\begin{exa}
	Let $R=\Z$, what ideals are prime?\vskip 2in\mbox{}
\end{exa}
}
\slide{
\begin{thm}{3.3.3}
	If $R$ is a commutative ring, an ideal $P\neq R$ of $R$ is a prime ideal if and only if $R/P$ is an integral domain.
\end{thm}
}

\slide{
\begin{thm}{3.3.4}
	Let $I$ be an ideal of the ring $R$. There is a correspondence
	\[\left\{
	\begin{array}{c}
		\text{ideals of }R\\
		\text{containing }I
	\end{array}
	\right\}\leftrightarrow\left\{
	\text{ideals of }R/I
	\right\}.\]
	Moreover, this correspondence respects containment.
\end{thm}
}
\slide{
\begin{defn}
	A ring $R$ is a \emph{simple ring} if $R\neq\{0\}$ and the only ideals of $R$ are $\{0\}$ and $R$.
\end{defn}
\begin{exa}
	$\Q$, $\R$, $\C$, $\Z_p$
\end{exa}
\begin{exa}
	A less trivial example, $M_2(\R)$, or any matrix ring over a field.  
\end{exa}
}
\slide{
\begin{thm}{3.3.5}
	If $R$ is a commutative ring with identity, then $R$ is simple if and only if it is a field.
\end{thm}
%\begin{proof}
%	Let $R$ be a commutative ring with identity.
%	
%	($\Leftarrow$) Assume $R$ is a field. Let $I$ be a non-trivial ideal of $R$.  It will suffice to show that $I=R$.  Let $a\in I$ with $a\neq 0$.  This exists because $I\neq\{0\}$. Notice that since $R$ is a field, $a$ is a unit.  Therefore by Theorem 3.3.2, $I=R$ because $I$ contains a unit.
%	
%	($\Rightarrow$) Assume $R$ is simple. Let $a\in R$ with $a\neq0$. We want to show $a$ is a unit.  Consider the ideal $Ra$ of $R$.
%\end{proof}
}

\slide{
\begin{defn}
	Let $R$ be a ring (not necessarily commutative), and let $M$ be an ideal of $R$.  We call $M$ a \emph{maximal ideal} of $R$ if
	\enumarabic{\item $M\neq R$, and\item if $I$ is an ideal of $R$ satisfying $M\subseteq I\subseteq R$, then $I=M$ or $I=R$.}
\end{defn}
}
\slide{
\begin{exercise}
	Is $5\Z$ maximal in $\Z$?\vskip 1in
	Is $6\Z$ maximal in $\Z$?\vskip 1in\mbox{}
\end{exercise}
}
\slide{
\begin{thm}{3.3.6}
	Let $M$ be an ideal of a ring $R$.  Then $M$ is maximal if and only if $R/A$ is simple.
\end{thm}
\begin{statementblock}{Corollary 1}
	Let $R$ be a commutative ring, with unity. Let $M$ be an ideal of $R$.  Then $M$ is maximal if and only if $R/M$ is a field.
\end{statementblock}
\textbf{Exercise.} Prove this using Theorems 5 and 6  of section 3.3.
\begin{statementblock}{Corollary 2}
	Let $R$ be a commutative ring, with unity.  If $M$ is a maximal ideal of $R$, then $M$ is a prime ideal.
\end{statementblock}
\textbf{Exercise.} Prove this using the previous Corollary and Theorem 3 of section 3.3.
}
\slide{
\begin{exercise}
	Show that the converse of the second corollary is false:
	
	Let $R=\Z\times\Z$ and $I=\{(a,0)\ |\ a\in\Z\}$.
	\enumarabic{\item Verify $I$ is an ideal of $R$.\vskip .75in\item Verify that $I$ is a prime ideal. \vskip .75in\item Let $J=\{(a,2b)\ |\ a,b\in\Z\}$.  Show that $J$ is also an ideal of $R$ and $I\subsetneq J\subsetneq R$. Thus showing $I$ is not maximal.\vskip .75in\mbox{}}
\end{exercise}
}
\slide{
These will be important in Math 426.

\begin{statementblock}{Lemma 3.3.3}
	Let $R$ be a ring with unity and $n\geq 1$.  Every ideal of $M_n(R)$ has the form $M_n(A)$ for some ideal $A$ of $R$.
\end{statementblock}
\begin{thm}{3.3.7}
	If $R$ is a ring with unity then $M_n(R)$ is simple if and only if $R$ is simple.
\end{thm}
\begin{statementblock}{Corollary}
	If $R$ is a division ring then $M_n(R)$ is simple.
\end{statementblock}
\textbf{Note.} This last one is HUGE in my research!
}

\end{document}

	