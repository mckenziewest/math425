\documentclass[t]{beamer}

\subtitle{Section 3.4: Homomorphisms}

\usepackage{amsthm,amsmath,amsfonts,hyperref,graphicx,color,multicol,soul}
\usepackage{enumitem,tikz,tikz-cd,setspace,mathtools}

%%%%%%%%%%
%Beamer Template Customization
%%%%%%%%%%
\setbeamertemplate{navigation symbols}{}
\setbeamertemplate{theorems}[ams style]
\setbeamertemplate{blocks}[rounded]

\definecolor{Blu}{RGB}{43,62,133} % UWEC Blue
\setbeamercolor{structure}{fg=Blu} % Titles

%Unnumbered footnotes:
\newcommand{\blfootnote}[1]{%
	\begingroup
	\renewcommand\thefootnote{}\footnote{#1}%
	\addtocounter{footnote}{-1}%
	\endgroup
}

%%%%%%%%%%
%TikZ Stuff
%%%%%%%%%%
\usetikzlibrary{arrows}
\usetikzlibrary{shapes.geometric}
\tikzset{
	smaller/.style={
		draw,
		regular polygon,
		regular polygon sides=3,
		fill=white,
		node distance=2cm,
		minimum height=1in,
		line width = 2pt
	}
}
\tikzset{
	smsquare/.style={
		draw,
		regular polygon,
		regular polygon sides=4,
		fill=white,
		node distance=2cm,
		minimum height=1in,
		line width = 2pt
	}
}


%%%%%%%%%%
%Custom Commands
%%%%%%%%%%

\newcommand{\C}{\mathbb{C}}
\newcommand{\quats}{\mathbb{H}}
\newcommand{\N}{\mathbb{N}}
\newcommand{\Q}{\mathbb{Q}}
\newcommand{\R}{\mathbb{R}}
\newcommand{\Z}{\mathbb{Z}}

\newcommand{\ds}{\displaystyle}

\newcommand{\fn}{\insertframenumber}

\newcommand{\id}{\operatorname{id}}
\newcommand{\im}{\operatorname{im}}
\newcommand{\Aut}{\operatorname{Aut}}
\newcommand{\Inn}{\operatorname{Inn}}

\newcommand{\blank}[1]{\underline{\hspace*{#1}}}

\newcommand{\abar}{\overline{a}}
\newcommand{\bbar}{\overline{b}}
\newcommand{\cbar}{\overline{c}}

\newcommand{\nml}{\unlhd}

%%%%%%%%%%
%Custom Theorem Environments
%%%%%%%%%%
\theoremstyle{definition}
\newtheorem{exercise}{Exercise}
\newtheorem{question}[exercise]{Question}
\newtheorem{warmup}{Warm-Up}
\newtheorem*{defn}{Definition}
\newtheorem*{exa}{Example}
\newtheorem*{disc}{Group Discussion}
\newtheorem*{nb}{Note}
\newtheorem*{recall}{Recall}
\renewcommand{\emph}[1]{{\color{blue}\texttt{#1}}}

\definecolor{Gold}{RGB}{237, 172, 26}
%Statement block
\newenvironment{statementblock}[1]{%
	\setbeamercolor{block body}{bg=Gold!20}
	\setbeamercolor{block title}{bg=Gold}
	\begin{block}{\textbf{#1.}}}{\end{block}}
\newenvironment{thm}[1]{%
	\setbeamercolor{block body}{bg=Gold!20}
	\setbeamercolor{block title}{bg=Gold}
	\begin{block}{\textbf{Theorem #1.}}}{\end{block}}


%%%%%%%%%%
%Custom Environment Wrappers
%%%%%%%%%%
\newcommand{\enumarabic}[1]{
	\begin{enumerate}[label=\textbf{\arabic*.}]
		#1
	\end{enumerate}
}
\newcommand{\enumalph}[1]{
	\begin{enumerate}[label=(\alph*)]
		#1
	\end{enumerate}
}
\newcommand{\bulletize}[1]{
	\begin{itemize}[label=$\bullet$]
		#1
	\end{itemize}
}
\newcommand{\circtize}[1]{
	\begin{itemize}[label=$\circ$]
		#1
	\end{itemize}
}
\newcommand{\slide}[1]{
	\begin{frame}{\fn}
		#1
	\end{frame}
}
\newcommand{\slidec}[1]{
\begin{frame}[c]{\fn}
	#1
\end{frame}
}
\newcommand{\slidet}[2]{
	\begin{frame}{\fn\ - #1}
		#2
	\end{frame}
}


\newcommand{\startdoc}{
		\title{Math 425: Abstract Algebra 1}
		\author{Mckenzie West}
		\date{Last Updated: \today}
		\begin{frame}
			\maketitle
		\end{frame}
}

\newcommand{\topics}[2]{
	\begin{frame}{\insertframenumber}
		\begin{block}{\textbf{Last Section.}}
			\begin{itemize}[label=--]
				#1
			\end{itemize}
		\end{block}
		\begin{block}{\textbf{This Section.}}
			\begin{itemize}[label=--]
				#2
			\end{itemize}
		\end{block}
	\end{frame}
}

\begin{document} 
	\startdoc
	
	\topics{
		\item Ideals
		\item Principal Ideals
		\item Prime Ideals
		\item Maximal Ideals
	}{
		\item Ring Homomorphisms
		\item First Isomorphism Theorem for Rings
	}

\slide{
	\begin{exercise}
		Consider $\theta\colon M_2(\Z)\to M_2(\Z)$ defined by
		\[\theta\left(\begin{bmatrix}a&b\\c&d\end{bmatrix}\right)=\begin{bmatrix}a& b\\0&d\end{bmatrix}.\]
		\enumalph{
			\item What is $\theta\left(\begin{bmatrix}7&23\\67&-535\end{bmatrix} \right)$?
				\vskip .25in
			\item Does $\theta$ satisfy $\theta(A+B)=\theta(A)+\theta(B)$?
			\vskip .25in
			\item Does $\theta$ satisfy $\theta(AB)=\theta(A)\theta(B)$?
			\vskip .25in
			\item What is $\theta^{-1}\left(\begin{bmatrix}\bar0&\bar0\\\bar0&\bar0\end{bmatrix} \right)$?
		}	
	\end{exercise}
}

\slide{
	\begin{defn}
		If $R$ and $S$ are rings with unity, we call a map $\theta:R\to S$ a \emph{ring homomorphism} if
		\enumarabic{
			\item $\theta(r_1+r_2)=\blank{3in}$\vskip .25in
			\item $\theta(r_1r_2)=\blank{3in}$\vskip .25in
			\item $\theta(1_R)=1_S$
		}
	\end{defn}
}

\slide{
	\begin{exa}
		\enumalph{\setlength{\itemsep}{2em}
			\item $\theta: \Z\to \Z_n$, $\theta(a)=\overline{a}$
			\item $\theta: R\to R/I$, $\theta(r) = r+I$
			\item $\pi_1: R_1\times R_2\to R_1$, $\pi_1(r_1,r_2)=r_1$
		}
	\end{exa}
}

\slide{
	\begin{thm}{3.4.1}
		Let $\theta: R\to R_1$ be a ring homomorphism and let $r\in R$.
		\enumarabic{
			\item $\theta(0)=0$
			\item $\theta(-r)=-\theta(r)$ for all $r\in R$
			\item $\theta(kr)=k\theta(r)$ for all $r\in R$ and $k\in\Z$
			\item $\theta(r^n)=\theta(r)^n$ for all $r\in R$ and $n\geq0$ in $\Z$
			\item If $u\in R^*$, $\theta(u^k)=\theta(u)^k$ for all $k\in\Z$.
		}
	\end{thm}
}
\slide{
	\begin{exercise}
		Consider $\theta:\Z\to \Z_9$ with $\theta(a)=\overline{a}$.
		
		Use this map and contradiction to show there are no integers $a,b,c$ for which 
			\[a^3+b^3+c^3=31.\]\vskip 2in\mbox{}
	\end{exercise}
}
\slide{
	\begin{thm}{3.4.2}
		Let $R\neq 0$ be a commutative ring with characteristic $p$, and define \[\phi:R\to R\quad\text{by}\quad\phi(r)=r^p\text{ for all }r\in R.\]
		Then $\phi$ is a ring homomorphism.
		
		We call this $\phi$ the \emph{Frobenius Endomorphism}.  If $\phi$ is a finite field, we call $\phi$ the \emph{Frobenius Automorphism}, which is an isomorphism.
	\end{thm}
}

\begin{frame}{\fn}
	\begin{defn}
		If $\theta:R\to S$ is a ring homomorphism, then the \emph{kernel} of $\theta$ is
			\[\ker(\theta)=\{r\in R\ |\ \theta(r)=0_{S}\}.\]
		The \emph{image} of $\theta$ is
			\[\im(\theta)=\theta(R)=\{\theta(r)\ |\ r\in R\}.\]
	\end{defn}
	\begin{thm}{3.4.3}
		Let $\theta:R\to S$ be a ring homomorphism.  Then\vskip 2em
			\enumarabic{\setlength{\itemsep}{2em}
				\item $\theta(R)$ is \blank{2in} of \blank{.5in}
				\item $\ker\theta$ is \blank{2in} of \blank{.5in}
			}
	\end{thm}
\end{frame}

\slide{
	\begin{statementblock}{First Isomorphism Theorem for Rings (Theorem 3.4.4)}
		Let $\theta:R\to S$ be a ring homomorphism and write $A=\ker \theta$.  Then $\theta$ induces a ring isomorphism
			\[\bar\theta:R/A\to \theta(R)\quad\text{given by}\quad \bar\theta(r+A)=\theta(r)\text{ for all }r\in R.\]
	\end{statementblock}
	\begin{proof}
		Let $\theta:R\to S$ be a ring homomorphism and write $A=\ker \theta$. Define $\bar\theta: R/A\to \theta(R)$ by 
		\[\bar\theta(r+A)=\theta(r)\text{ for all }r\in R.\]
		\vskip 3in
	\end{proof}
}

\slide{
	\begin{nb}
		Now's a great time to discuss some notation.
		
		In $R/A$, elements look like $r+A$ where $r\in R$.
		
		If $A$ is clear, we can simplify our notation by writing $\overline{r}$.
		
		\begin{exa}
			In $\Z/n\Z$, elements are of the form $k+n\Z$ for $k\in \Z$.  These correspond to $\overline{k}$ in $\Z_n$.
		\end{exa}
		\begin{exa}
			Though to make our lives easier, if $m|n$, we might want to use the longer version in the restriction map, $$\phi:\Z_n\to \Z_m\quad\text{with}\quad\phi(k+n\Z)=k+m\Z.$$
		\end{exa}
	\end{nb}
}
\slide{
	\begin{statementblock}{Corollary}
		Let $A$ and $B$ be ideals of the rings $R$ and $S$, respectively.  Then $A\times B$ is an ideal of $R\times S$ and 	
			\[\frac{R\times S}{A\times B}\cong \frac{R}{A}\times\frac{S}{B}.\]
	\end{statementblock}
	\begin{proof}
		\vskip 3in
	\end{proof}
}
\slide{
	\begin{statementblock}{Corollary}
		Let $A$ be an ideal of the ring $R$.  Then $M_n(A)$ is an ideal of $M_n(R)$ and 	
		\[\frac{M_n(R)}{M_n(A)}\cong M_n(R/A).\]
	\end{statementblock}
	\begin{proof}
		\vskip 3in
	\end{proof}
}


\end{document}

	