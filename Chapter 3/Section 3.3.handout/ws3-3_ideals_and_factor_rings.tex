\documentclass[12pt]{article}

\newcommand{\secname}{Section 3.3: Ideals and Factor Rings}

\usepackage{amsthm,amsmath,amsfonts,hyperref,graphicx,color,multicol,soul}
\usepackage{enumitem,tikz,tikz-cd,setspace,mathtools}
\usepackage{colortbl}
\usepackage[margin=1in]{geometry}

%%%%%%%%%%
%Color Customization
%%%%%%%%%%

\definecolor{Blu}{RGB}{43,62,133} % UWEC Blue

%Unnumbered footnotes:
\newcommand{\blfootnote}[1]{%
	\begingroup
	\renewcommand\thefootnote{}\footnote{#1}%
	\addtocounter{footnote}{-1}%
	\endgroup
}

%%%%%%%%%%
%TikZ Stuff
%%%%%%%%%%
\usetikzlibrary{arrows}
\usetikzlibrary{shapes.geometric}
\tikzset{
	smaller/.style={
		draw,
		regular polygon,
		regular polygon sides=3,
		fill=white,
		node distance=2cm,
		minimum height=1in,
		line width = 2pt
	}
}
\tikzset{
	smsquare/.style={
		draw,
		regular polygon,
		regular polygon sides=4,
		fill=white,
		node distance=2cm,
		minimum height=1in,
		line width = 2pt
	}
}

%%%%%%%%%%
%Listing Setup
%%%%%%%%%%
\usepackage{listings}
\usepackage{caption, floatrow, makecell}%
\captionsetup{labelfont = sc}
\setcellgapes{3pt}

\definecolor{backcolour}{RGB}{237,236,230}
\definecolor{myblue}{RGB}{42,157,189}

\lstdefinestyle{mystyle}{
	language=Python,
	keywords=[2]{sage:},
	alsodigit={:,.,<,>},
	backgroundcolor=\color{backcolour},   
	commentstyle=\color{myblue},
	keywordstyle=\bfseries\color{Green},
	keywordstyle=[2]\color{purple},
	numberstyle=\tiny\color{Gray},
	stringstyle=\color{Orange},
	basicstyle=\ttfamily\footnotesize,
	breakatwhitespace=false,         
	breaklines=true,                 
	captionpos=b,                    
	keepspaces=true,                   
	showspaces=false,                
	showstringspaces=false,
	showtabs=false,                  
	tabsize=2
}

\lstset{style=mystyle}


%%%%%%%%%%
%Custom Commands
%%%%%%%%%%

\newcommand{\C}{\mathbb{C}}
\newcommand{\quats}{\mathbb{H}}
\newcommand{\N}{\mathbb{N}}
\newcommand{\Q}{\mathbb{Q}}
\newcommand{\R}{\mathbb{R}}
\newcommand{\Z}{\mathbb{Z}}

\newcommand{\ds}{\displaystyle}

\newcommand{\fn}{\insertframenumber}

\newcommand{\id}{\operatorname{id}}
\newcommand{\im}{\operatorname{im}}
\newcommand{\lcm}{\operatorname{lcm}}
\newcommand{\ord}{\operatorname{ord}}
\newcommand{\Aut}{\operatorname{Aut}}
\newcommand{\Inn}{\operatorname{Inn}}

\newcommand{\blank}[1]{\underline{\hspace*{#1}}}

\newcommand{\abar}{\overline{a}}
\newcommand{\bbar}{\overline{b}}
\newcommand{\cbar}{\overline{c}}

\newcommand{\nml}{\unlhd}

%%%%%%%%%%
%Custom Theorem Environments
%%%%%%%%%%
\theoremstyle{definition}
\newtheorem{exercise}{Exercise}
\newtheorem{question}[exercise]{Question}
\newtheorem{warmup}{Warm-Up}
\newtheorem*{exa}{Example}
\newtheorem*{defn}{Definition}
\newtheorem*{disc}{Group Discussion}
\newtheorem*{recall}{Recall}
\renewcommand{\emph}[1]{{\color{blue}\texttt{#1}}}

\definecolor{Gold}{RGB}{237, 172, 26}
%Statement block
%\newenvironment{statementblock}[1]{%
%	\setbeamercolor{block body}{bg=Gold!20}
%	\setbeamercolor{block title}{bg=Gold}
%	\begin{block}{\textbf{#1.}}}{\end{block}}
%\newenvironment{goldblock}{%
%	\setbeamercolor{block body}{bg=Gold!20}
%	\setbeamercolor{block title}{bg=Gold}
%	\setbeamertemplate{blocks}[shadow=true]
%	\begin{block}{}}{\end{block}}
%\newenvironment{defn}{%
%	\setbeamercolor{block body}{bg=gray!20}
%	\setbeamercolor{block title}{bg=violet, fg=white}
%	\setbeamertemplate{blocks}[shadow=true]
%	\begin{block}{\textbf{Definition.}}}{\end{block}}
%\newenvironment{nb}{%
%	\setbeamercolor{block body}{bg=gray!20}
%	\setbeamercolor{block title}{bg=teal, fg=white}
%	\setbeamertemplate{blocks}[shadow=true]
%	\begin{block}{\textbf{Note.}}}{\end{block}}
%\newenvironment{blockexample}{%
%	\setbeamercolor{block body}{bg=gray!20}
%	\setbeamercolor{block title}{bg=Blu, fg=white}
%	\setbeamertemplate{blocks}[shadow=true]
%	\begin{block}{\textbf{Example.}}}{\end{block}}
%\newenvironment{blocknonexample}{%
%	\setbeamercolor{block body}{bg=gray!20}
%	\setbeamercolor{block title}{bg=purple, fg=white}
%	\setbeamertemplate{blocks}[shadow=true]
%	\begin{block}{\textbf{Non-Example.}}}{\end{block}}
%\newenvironment{thm}[1]{%
%	\setbeamercolor{block body}{bg=Gold!20}
%	\setbeamercolor{block title}{bg=Gold}
%	\begin{block}{\textbf{Theorem #1.}}}{\end{block}}


%%%%%%%%%%
%Custom Environment Wrappers
%%%%%%%%%%
\newcommand{\exer}[1]{
	\begin{exercise}
	#1
	\end{exercise}
}
\newcommand{\exam}[1]{
\textbf{Example: }
	#1
}
\newcommand{\nexam}[1]{
	\textbf{Non-Example: }
	#1
}
\newcommand{\enumarabic}[1]{
	\begin{enumerate}[label=\textbf{\arabic*.}]
		#1
	\end{enumerate}
}
\newcommand{\enumalph}[1]{
	\begin{enumerate}[label=(\alph*)]
		#1
	\end{enumerate}
}
\newcommand{\bulletize}[1]{
	\begin{itemize}[label=$\bullet$]
		#1
	\end{itemize}
}
\newcommand{\circtize}[1]{
	\begin{itemize}[label=$\circ$]
		#1
	\end{itemize}
}
%\newcommand{\slide}[1]{
%	\begin{frame}{\fn}
%		#1
%	\end{frame}
%}
%\newcommand{\slidec}[1]{
%\begin{frame}[c]{\fn}
%	#1
%\end{frame}
%}
%\newcommand{\slidet}[2]{
%	\begin{frame}{\fn\ - #1}
%		#2
%	\end{frame}
%}


\setlength{\parindent}{0pt}



\usepackage{afterpage}
\usepackage{fancyhdr}

\fancyhead[L]{\textbf{Math 425: Abstract Algebra I\\\secname}}
\fancyhead[R]{\textbf{Mckenzie West\\Last Updated: \today}}
\pagestyle{fancy}

\newcommand{\startdoc}{}

\newcommand{\topics}[2]{
		{\textbf{Previously.}}
			\begin{itemize}[label=--]
				#1
			\end{itemize}
		{\textbf{This Section.}}
			\begin{itemize}[label=--]
				#2
			\end{itemize}
}

\begin{document} 
	\startdoc
	
	\topics{
		\item Domains
		\item Integral Domains
		\item Fields
	}{
		\item Ideals
		\item Principal Ideals
		\item Prime Ideals
		\item Maximal Ideals
	}

\slide{
	\begin{recall}
		Consider the ring $(R,+,\cdot)$. 
		Recall that $(R,+)$ is an abelian group.  So any subgroup $S\leq R$ is automatically normal.
		In particular, we can construct $R/S$, the set of cosets of $S$ in $R$ as a group.
	\end{recall}
	\begin{exa}
		The ring $R=\Z[i]$ is a group under addition and it has the subgroup $$S=(2+i)\Z[i]=\{(2+i)z\ :\ z\in\Z[i]\}=\{(a+bi)(2+i)\ :\ a,b\in\Z\}.$$
		
		The cosets of $S$ are of the form $r+S=r+(2+i)\Z[i]$ where $r\in R$.
		
		Some cosets are 
		\begin{itemize}
			\item $2+S=\{2+(2+i)z\ :\ z\in\Z[i]\}=\{(2+2a-b)+(a+2b)i\ :\ a,b\in\Z\}$
			\item $(1-i)+S=\{1-i+(2+i)z\ :\ z\in\Z[i]\}=\{(1+2a-b)+(a+2b-1)i\ :\ a,b\in\Z\}$
		\end{itemize}
		
		On question we might ask is ``Is $R/S$ a ring?''
	\end{exa}
	\exer{
		Take the $R=\Z[i]$ and $S=(2+i)\Z[i]$ as in the example above.  Let's try multiplying cosets. We expect $(a+S)(b+S)=(ab)+S$, right?
		\enumalph{
			\item (Elements)
			Consider the cosets $2+S$ and $(1-i)+S$.
			
			Take generic elements $r_1=2+(2+i)z_1\in 2+S$ and $r_2=(1-i)+(2+i)z_2\in (1-i)+S$. 
			
			Can you write the product $r_1r_2$ in the form $2(1-i)+(2+i)z_3$ for some $z_3\in \Z[i]$?  
			 \vfill
			\item (Sets)  What would we expect for the product
			\[(2+S)((1-i)+S)?\]
			\vfill
			\item (Verification) Is the value you computed for the first part of this exercise in this expected set?
			\vfill
		}
	}
	
	\newpage
		
	\begin{block}{Lemma}
		$(S,+)\leq (R,+)$, then 
			\[(a+S)(b+S)=(ab)+S\]
		is well-defined if and only if 
			\[rS\subseteq S\quad\text{and}\quad Sr\subseteq S,\quad\text{for all }r\in R.\]
	\end{block}
	\exer{
		Do some FOILing of $(a+S)(b+S)$ and see how this relates to the containment of $rS\subseteq S$ and $Sr\subseteq S$, and the additive closure property of subgroups.
		\vskip 1in
	}
	\begin{defn}
		Let $(R,+,\cdot)$ be a ring. An additive subgroup $(I,+)$ of $(R,+)$ is an \emph{ideal of $R$} if $rI\subseteq I$ and $Ir\subseteq I$ for all $r\in R$.
	\end{defn}
	\begin{exercise}
		Verify that $(2+i)\Z[i]$ is an ideal of $\Z[i]$.\vskip 1in
	\end{exercise}
}
\slide{
	\begin{defn}
		Equivalent definitions of an \emph{ideal} $I$ of a ring $R$: (given $(I,+)\leq (R,+)$)
		\bulletize{
			\item for all $i\in I$, $iR\subseteq I$ and $Ri\subseteq I$
			\item for all $i\in I$ and $r\in R$, $ir\in I$ and $ri\in I$.
		}
	\end{defn}
	\begin{nb}
		Some may call this a ``two-sided ideal''.  By considering just one of the containments, we could also define ``left ideals'' and ``right ideals''.
		In a commutative ring, every ideal is two-sided.  WHY??
	\end{nb}
}
\begin{block}{Warning}
	Not all subgroups are ideals, as we will see in the following exercise!
\end{block}
\slide{
	\begin{exercise}
		Show that $\Z$ is not an ideal of the ring $\Q$.\vskip 2in 
	\end{exercise}
}
%\slide{
%	\begin{block}{\textbf{Warning.}}
%		The following is up for debate:
%		
%			If $I$ is an ideal of $R$, it does not guarantee that $I$ is a subring of $R$.
%			
%		Why?  It depends on the requirements of having a 1.
%	\end{block}
%}

\newpage

\slide{
	\begin{thm}{3.3.1}
		Let $I$ be an ideal of the ring $R$ (with unity).  Then the additive group $(R/I,+)$ becomes a ring with multiplication $(r+I)(s+I)=rs+I$ called the \emph{factor ring} or \emph{quotient ring}.  The unity of $R/I$ is $1+I$ and if $R$ is commutative, then $R/I$ is commutative.
	\end{thm}
}
\vskip 2in
\slide{
	\begin{block}{Observations} Let $R$ be a ring (with unity)
		\enumarabic{
			\item $\{0\}$ and $R$ are ideals of $R$.
			\item $R/R\cong \{0\}$ and $R/\{0\}\cong R$
			\item Everything from quotient groups extends to quotient rings
				\enumalph{
					\item $r+I=s+I$ if and only if $r-s\in I$
					\item $(r+I)+(s+I)=(r+s)+I$
					\item $0+I=I$
					\item $-(r+I)=-r+I$
					\item $k(r+I)=kr+I$ for all $k\in \Z$
				}
		}
	\end{block}
}
\slide{
\begin{thm}{3.3.2}
	If $I$ is an ideal of the ring $R$ (that has unity), then the following are equivalent
	\enumarabic{\item $1\in I$\item $I$ contains a unit\item $I=R$}
\end{thm}
}
\newpage
\section*{Principal Ideals}
\slide{
	Given a fixed element $a$ in a ring $R$, we can get an ideal easily by taking all of the multiples of that element.
		\begin{quote}
			$Ra=\{ra\ |\ r\in R\}$\\
			$aR=\{ar\ |\ r\in R\}$
		\end{quote}
	\begin{defn}
		If $a\in Z(R)$, then we call $Ra=aR$ the \emph{principal ideal of $R$ generated by $a$}.  Denote such a principal ideal by $(a)$.
	\end{defn}
	\exer{
		Show that if $a\in Z(R)$, then $(a)=Ra=aR$ is an ideal of $R$ by showing that (1) $(a)$ is a subgroup of $R$ under addition and (2) for all $r\in R$, we have the following inclusions of sets $r(a)\subseteq (a)$ and $(a)r\subseteq (a)$.\vskip 2in
	}
	\begin{block}{Warning}
		The book uses $\langle a \rangle$ for the ideal generated by $a$.  To avoid mixing it up with cyclic groups, we'll use $(a)$ in these notes.
	\end{block}
}
\slide{
	\begin{exercise}
		Is the set of multiples of $6$ a principal ideal of $\Z$?\vfill
	\end{exercise}
}
%\exer{
%	Show that if $I=(i)=i\Z$ and $R=\Z[i]$, then $$R/I=\{a+I\ :\ a\in\Z\}\cong \Z.$$
%}
\newpage
\slide{
	\begin{exercise}
		Consider $R=\Z[i]$ and $I=(2+i)$, the ideal from earlier in the packet.  Follow the listed steps to show that $$R/I=\{0+I,1+I,2+I,3+I,4+I\}.$$
		\enumalph{
			\item Show that $5\in I$ by writing $5=r(2+i)$ for some $r\in \Z[i]$.\vfill
			\item Show that if $n\in\Z$, then $n+I$ is the same as one of $0+I,1+I,2+I,3+I,4+I$.\vfill
			\item Show that $i+I=-2+I$. (Hint: Observation 3a on page 3 of the packet.)\vfill
			\item Show that if $a+bi\in\Z[i]$ then $(a+bi)+I=(a-2b)+I$.\vfill
			\item Conclude that every coset of $I$ in $Z[i]$ is equal to one of $0+I,1+I,2+I,3+I,4+I$.\vfill
			\item (Challenge) Show that if $0\leq m<n\leq 4$, then $m+I\neq n+I$.\vfill
		}
	\end{exercise}
	\newpage
}
\slide{
\begin{nb}
	There are many examples of ideals that are not principal.  One example of this is the ideal \[(2,1+\sqrt{-5})=\{r(2)+s(1+\sqrt{-5})\ |\ r,s\in\Z[\sqrt{-5}]\}\] of $\Z[\sqrt{-5}]$.  See: {\tiny\url{https://math.stackexchange.com/questions/543216/proving-that-a-ring-is-not-a-principal-ideal-domain}}
\end{nb}
}

\slide{
\begin{defn}
	We call a proper ideal $P$ of a ring $R$ \emph{prime} if 
	\[rs\in P\quad\Rightarrow\quad r\in P\text{ or }s\in P.\]
\end{defn}
\begin{exa}
	Let $R=\Z$, what ideals are prime? (This is a thought exercise, and the answer is what you expect, but why??)\vskip 2in
\end{exa}
}
\slide{
\begin{thm}{3.3.3}
	If $R$ is a commutative ring, an ideal $P\neq R$ of $R$ is a prime ideal if and only if $R/P$ is an integral domain.
\end{thm}
	\vskip 2in
}

\slide{
\begin{thm}{3.3.4}
	Let $I$ be an ideal of the ring $R$. There is a correspondence
	\[\left\{
	\begin{array}{c}
		\text{ideals of }R\\
		\text{containing }I
	\end{array}
	\right\}\leftrightarrow\left\{
	\text{ideals of }R/I
	\right\}.\]
	Moreover, this correspondence respects containment.
\end{thm}
}
\newpage
\slide{
\begin{defn}
	Let $R$ be a ring (not necessarily commutative), and let $M$ be an ideal of $R$.  We call $M$ a \emph{maximal ideal} of $R$ if
	\enumarabic{\item $M\neq R$, and\item if $I$ is an ideal of $R$ satisfying $M\subseteq I\subseteq R$, then $I=M$ or $I=R$.}
\end{defn}
}
\slide{
\begin{exercise}
	Is $5\Z$ maximal in $\Z$?\vskip 1in
	Is $6\Z$ maximal in $\Z$?\vskip 1in\mbox{}
\end{exercise}
}




\slide{
	\begin{defn}
		A ring $R$ is a \emph{simple ring} if $R\neq\{0\}$ and the only ideals of $R$ are $\{0\}$ and $R$.
	\end{defn}
	\begin{exa}
		$\Q$, $\R$, $\C$, $\Z_p$
	\end{exa}
	\begin{exa}
		A less trivial example, $M_2(\R)$, or any matrix ring over a field.  
	\end{exa}
}
\slide{
	\begin{thm}{3.3.5}
		If $R$ is a commutative ring with identity, then $R$ is simple if and only if it is a field.
	\end{thm}
	%\begin{proof}
	%	Let $R$ be a commutative ring with identity.
	%	
	%	($\Leftarrow$) Assume $R$ is a field. Let $I$ be a non-trivial ideal of $R$.  It will suffice to show that $I=R$.  Let $a\in I$ with $a\neq 0$.  This exists because $I\neq\{0\}$. Notice that since $R$ is a field, $a$ is a unit.  Therefore by Theorem 3.3.2, $I=R$ because $I$ contains a unit.
	%	
	%	($\Rightarrow$) Assume $R$ is simple. Let $a\in R$ with $a\neq0$. We want to show $a$ is a unit.  Consider the ideal $Ra$ of $R$.
	%\end{proof}
}

\slide{
\begin{thm}{3.3.6}
	Let $M$ be an ideal of a ring $R$.  Then $M$ is maximal if and only if $R/A$ is simple.
\end{thm}
\begin{statementblock}{Corollary 1}
	Let $R$ be a commutative ring, with unity. Let $M$ be an ideal of $R$.  Then $M$ is maximal if and only if $R/M$ is a field.
\end{statementblock}
\newpage
\begin{statementblock}{Corollary 2}
	Let $R$ be a commutative ring, with unity.  If $M$ is a maximal ideal of $R$, then $M$ is a prime ideal.
\end{statementblock}
\vskip 2in
}
\slide{
\begin{exercise}
	Show that the converse of the second corollary is false:
	
	Let $R=\Z\times\Z$ and $I=\{(a,0)\ |\ a\in\Z\}$.
	\enumarabic{\item Verify $I$ is an ideal of $R$.\vskip .75in\item Verify that $I$ is a prime ideal. \vskip .75in\item Let $J=\{(a,2b)\ |\ a,b\in\Z\}$.  Show that $J$ is also an ideal of $R$ and $I\subset J\subset R$ with $I\neq J\neq R$. Thus showing $I$ is not maximal.\vskip .75in\mbox{}}
\end{exercise}
}


\slide{
These will be important in Math 426.

\begin{statementblock}{Lemma 3.3.3}
	Let $R$ be a ring with unity and $n\geq 1$.  Every ideal of $M_n(R)$ has the form $M_n(A)$ for some ideal $A$ of $R$.
\end{statementblock}
\begin{thm}{3.3.7}
	If $R$ is a ring with unity then $M_n(R)$ is simple if and only if $R$ is simple.
\end{thm}
\begin{statementblock}{Corollary}
	If $R$ is a division ring then $M_n(R)$ is simple.
\end{statementblock}
\textbf{Note.} This last one is HUGE in my research!
}




\end{document}

	