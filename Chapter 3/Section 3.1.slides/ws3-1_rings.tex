\documentclass[t]{beamer}

\subtitle{Section 3.1: Rings}

\usepackage{amsthm,amsmath,amsfonts,hyperref,graphicx,color,multicol,soul}
\usepackage{enumitem,tikz,tikz-cd,setspace,mathtools}

%%%%%%%%%%
%Beamer Template Customization
%%%%%%%%%%
\setbeamertemplate{navigation symbols}{}
\setbeamertemplate{theorems}[ams style]
\setbeamertemplate{blocks}[rounded]

\definecolor{Blu}{RGB}{43,62,133} % UWEC Blue
\setbeamercolor{structure}{fg=Blu} % Titles

%Unnumbered footnotes:
\newcommand{\blfootnote}[1]{%
	\begingroup
	\renewcommand\thefootnote{}\footnote{#1}%
	\addtocounter{footnote}{-1}%
	\endgroup
}

%%%%%%%%%%
%TikZ Stuff
%%%%%%%%%%
\usetikzlibrary{arrows}
\usetikzlibrary{shapes.geometric}
\tikzset{
	smaller/.style={
		draw,
		regular polygon,
		regular polygon sides=3,
		fill=white,
		node distance=2cm,
		minimum height=1in,
		line width = 2pt
	}
}
\tikzset{
	smsquare/.style={
		draw,
		regular polygon,
		regular polygon sides=4,
		fill=white,
		node distance=2cm,
		minimum height=1in,
		line width = 2pt
	}
}


%%%%%%%%%%
%Custom Commands
%%%%%%%%%%

\newcommand{\C}{\mathbb{C}}
\newcommand{\quats}{\mathbb{H}}
\newcommand{\N}{\mathbb{N}}
\newcommand{\Q}{\mathbb{Q}}
\newcommand{\R}{\mathbb{R}}
\newcommand{\Z}{\mathbb{Z}}

\newcommand{\ds}{\displaystyle}

\newcommand{\fn}{\insertframenumber}

\newcommand{\id}{\operatorname{id}}
\newcommand{\im}{\operatorname{im}}
\newcommand{\Aut}{\operatorname{Aut}}
\newcommand{\Inn}{\operatorname{Inn}}

\newcommand{\blank}[1]{\underline{\hspace*{#1}}}

\newcommand{\abar}{\overline{a}}
\newcommand{\bbar}{\overline{b}}
\newcommand{\cbar}{\overline{c}}

\newcommand{\nml}{\unlhd}

%%%%%%%%%%
%Custom Theorem Environments
%%%%%%%%%%
\theoremstyle{definition}
\newtheorem{exercise}{Exercise}
\newtheorem{question}[exercise]{Question}
\newtheorem{warmup}{Warm-Up}
\newtheorem*{defn}{Definition}
\newtheorem*{exa}{Example}
\newtheorem*{disc}{Group Discussion}
\newtheorem*{nb}{Note}
\newtheorem*{recall}{Recall}
\renewcommand{\emph}[1]{{\color{blue}\texttt{#1}}}

\definecolor{Gold}{RGB}{237, 172, 26}
%Statement block
\newenvironment{statementblock}[1]{%
	\setbeamercolor{block body}{bg=Gold!20}
	\setbeamercolor{block title}{bg=Gold}
	\begin{block}{\textbf{#1.}}}{\end{block}}
\newenvironment{thm}[1]{%
	\setbeamercolor{block body}{bg=Gold!20}
	\setbeamercolor{block title}{bg=Gold}
	\begin{block}{\textbf{Theorem #1.}}}{\end{block}}


%%%%%%%%%%
%Custom Environment Wrappers
%%%%%%%%%%
\newcommand{\enumarabic}[1]{
	\begin{enumerate}[label=\textbf{\arabic*.}]
		#1
	\end{enumerate}
}
\newcommand{\enumalph}[1]{
	\begin{enumerate}[label=(\alph*)]
		#1
	\end{enumerate}
}
\newcommand{\bulletize}[1]{
	\begin{itemize}[label=$\bullet$]
		#1
	\end{itemize}
}
\newcommand{\circtize}[1]{
	\begin{itemize}[label=$\circ$]
		#1
	\end{itemize}
}
\newcommand{\slide}[1]{
	\begin{frame}{\fn}
		#1
	\end{frame}
}
\newcommand{\slidec}[1]{
\begin{frame}[c]{\fn}
	#1
\end{frame}
}
\newcommand{\slidet}[2]{
	\begin{frame}{\fn\ - #1}
		#2
	\end{frame}
}


\newcommand{\startdoc}{
		\title{Math 425: Abstract Algebra 1}
		\author{Mckenzie West}
		\date{Last Updated: \today}
		\begin{frame}
			\maketitle
		\end{frame}
}

\newcommand{\topics}[2]{
	\begin{frame}{\insertframenumber}
		\begin{block}{\textbf{Last Section.}}
			\begin{itemize}[label=--]
				#1
			\end{itemize}
		\end{block}
		\begin{block}{\textbf{This Section.}}
			\begin{itemize}[label=--]
				#2
			\end{itemize}
		\end{block}
	\end{frame}
}

\begin{document} 
	\startdoc
	
	\topics{
		\item Kernels
		\item The first isomorphism theorem
	}{
		\item Rings
		\item Commutative Rings
		\item Fields
		\item Subrings
		\item Ring Isomorphisms
	}

\slide{
	\begin{defn}
		Suppose $R$ is a set and it has two binary operations on it (written as $+$ and $\cdot$), then the set $R$ is a \emph{ring} if 
		\enumarabic{
			\item $(R,+)$ is an abelian group
			\item $\cdot$ is associative (i.e., $r_1(r_2r_3)=(r_1r_2)r_3$)
			\item the distributive laws hold:
				\bulletize{\item $r_1(r_2+r_3)=r_1r_2+r_1r_3$\item $(r_1+r_2)r_3=r_1r_3+r_2r_3$}
		}
	\end{defn}
}

\slide{
	\begin{exa}
		\enumarabic{\setlength{\itemsep}{2em}
			\item $(\Z,+,\cdot)$, $(\Q,+,\cdot)$, $(\R,+,\cdot)$, $(\C,+,\cdot)$
			\item  $(2\Z,+,\cdot)$
			\item $\Z[\sqrt{2}]=\{a+b\sqrt{2}\ | a,b\in\Z\}$
			\item $\Z[i]=\{a+bi\ |\ a,b\in\Z\}$\quad $\leftarrow$\quad The book calls this $\Z(i)$
			\item $(\Z_n,+,\cdot)$
			\item $M_2(\R)=\left\{\begin{pmatrix}a&b\\c&d\end{pmatrix}\ |\ a,b,c,d\in\R\right\}$
		}
	\end{exa}
}

\slide{
	\begin{exa}
		The \emph{direct product} $R_1\times R_2$ of rings $R_1$ and $R_2$ is also a ring with componentwise operations:
		\bulletize{\item $(a,b)+(c,d)=(a+c,b+d)$\item $(a,b)\cdot(c,d)=(ac,bd)$} 
	\end{exa}
}
\slide{
	\begin{defn}
		Given a ring $(R,+,\cdot)$,
		\enumarabic{
			\item If $\cdot$ is commutative, then we call $R$ a \emph{commutative ring}.
			\item The \emph{additive identity} element in $R$ is denoted $0$ or $0_R$.
			\item If there exists a \emph{multiplicative identity} element in $R$, it is denoted $1$ or $1_R$.  A ring that has a $1_R$ is called a \emph{ring with unity}.
			\item A non-zero element $a\in R$ is called a \emph{zero-divisor} if there is some non-zero $b\in R$ such that $ab=0$ or $ba=0$.
			\item An element $a\in R$ is called \emph{nilpotent} if there is some $n\in \Z^+$ such that $a^n=0$.
			\item Suppose $R$ is a rings with unity.  Then an element $a\in R$ is called a \emph{unit} if there is some $b\in R$ such that $ab=ba=1$.
		}
	\end{defn}
}
\slide{
	\begin{defn}
		\enumarabic{
			\item[\textbf{7.}] The \emph{center} $Z(R)$ of a ring $R$ is defined to be 
				\[Z(R)=\{x\in R\ |\ xr=rx\ \forall r\in R\}.\]
		}
	\end{defn}
	\begin{question}
		Why don't we care about all the $x\in R$ such that $x+r=r+x$ for all $r\in R$?
	\end{question}	
}
\slide{
	\begin{defn}
		\enumarabic{
			\item[\textbf{8.}] A ring $R\neq\{0\}$ is called a \emph{division ring} if every non-zero element in $R$ is a unit.
			\item[\textbf{9.}] A \emph{field} is a commutative division ring.
		}
	\end{defn}
}
\slide{
	\begin{exercise}
		Examine these definitions for $(\Z_6,+,\cdot)$?
		\enumarabic{
			\item commutative
			\item additive identity
			\item multiplicative identity
			\item zero-divisors
			\item nilpotent elements
			\item units
			\item trivial ring
			\item center
			\item division ring
			\item field
		}
	\end{exercise}
}

\slide{
	\begin{exa}[A non-commutative ring]
		$\quats$, the quaternions
		
		$\quats=\{a+bi+cj+dk\ |\ a,b,c,d\in\R\}$ given $i^2=j^2=k^2=-1$, $ij=-ji=k$, $jk=-kj=i$, $ki=-ik=j$
		\vskip 2.5in\mbox{}
	\end{exa}
}
\slide{
	\begin{exa}[Some popular commutative division rings.]
		$\Q$, $\R$ and $\C$
		\vskip 2em
		$\Z_p$ where $p$ is prime\vskip 2in\mbox{}
	\end{exa}
}
\slide{
	\begin{thm}{3.1.1}
		If $0$ is the zero of a ring $R$, then $0r=0=r0$ for every $r\in R$.
	\end{thm}
	\begin{proof}
		\vskip 2.5in
	\end{proof}
}
\slide{
	\begin{thm}{3.1.2}
		Let $r$ and $s$ be arbitrary elements of a ring $R$.
		\enumarabic{
			\item $(-r)s=r(-s)=-rs$
			\item $(-r)(-s)=rs$
			\item $(mr)(ns)=(mn)(rs)$ for all integers $m$ and $n$
		}
	\end{thm}
}
\slide{
	\begin{defn}
		A subset $S$ of a ring $(R,+,\cdot)$ is called a \emph{subring} if $(S,+,\cdot)$ is also a ring.
	\end{defn}
	\begin{exa}
		$\Z\subseteq\Q\subseteq\R\subseteq \C$
	\end{exa}
}

\slide{
	\begin{statementblock}{The Subring Test}
		Let $(R,+,\cdot)$ be a ring and $S$ a non-empty subset of $R$.  Then $S$ is a subring of $R$ if
		\enumarabic{
			\item $r_1-r_2\in S$ for all $r_1,r_2\in S$
			\item $r_1r_2\in S$ for all $r_1,r_2\in S$
			\item $1_R\in S$
		}
	\end{statementblock}
}
\slide{
	\begin{exa}
		Prove $\Z[i]$ is a subring of $\C$.\vskip 3in\mbox{}
	\end{exa}
}
\slide{
	\begin{exa}
		Prove $T_2(\R)=\left\{\begin{bmatrix}a&b\\0&c\end{bmatrix}\ |\ a,b,c\in\R\right\}$ is a subring of $M_2(\R)$.\vskip 3in\mbox{}
	\end{exa}
}

\slide{
	\begin{defn}
		Let $R$ and $S$ be rings. A \emph{ring isomorphism} is a bijective map $\phi:R\to S$ such that for all $r_1,r_2\in R$,
		\vskip 1em
		\enumarabic{\setlength{\itemsep}{2em}\item $\phi(r_1+r_2)=$\item $\phi(r_1r_2)=$
			\item $\phi(1_R)=1_S$}
		\vskip 1em
		In this case we say $R$ and $S$ are \emph{isomorphic} and write $R\cong S$.
	\end{defn}
}
\slide{
	\begin{statementblock}{Some Observations}
		Let $\phi: R\to S$ be a ring isomorphism.
		\enumarabic{
			\item $\phi(0_R)=0_S$
			\item $\phi(-r)=-\phi(r)$
			\item $\phi(kr)=k\phi(r)$ for all $k\in\Z$
			\item If $R$ and $S$ are rings with unity, then $\phi(1_R)=1_S$.
			\item If $\phi$ is an isomorphism, then it preserves the addition and multiplication tables of both rings.
		}
	\end{statementblock}
}
\slide{
	\begin{exa}
		Prove that $\Z_6$ and $\Z_2\times \Z_3$ are isomorphic as rings.\vskip 3in\mbox{}
	\end{exa}
}

\slide{
	\begin{defn}
		If there is some finite $n$ for which $$n(1_R)=\underbrace{1_R+1_R+\cdots+1_R}_{n\text{ times}}$$
		then we say the \emph{characteristic} of a ring $R$ is the smallest such $n$ (aka, the order of $1_R$ in the additive group $(R,+)$.)  Otherwise we say the \emph{characteristic} of $R$ is $0$.  Denote this value by $\operatorname{char} R$.
	\end{defn}
	\begin{exercise}
		\begin{multicols}{2}\enumalph{\setlength{\itemsep}{2em}\item $\operatorname{char} \Z_3=$\item $\operatorname{char}\R=$\item $\operatorname{char}\{0\}=$\item $\operatorname{char}\Z_4\times\Z_6$=}
		\end{multicols}
		\vskip 2em\mbox{}
	\end{exercise}
}

\slide{
	\begin{thm}{3.1.3}
		If $R$ is a ring and $\operatorname{char} R=n$, then
		\enumarabic{\item If $\operatorname{char} R=n>0$, then $kR=\{0\}$ if and only if $n$ divides $k$.\item If $\operatorname{char} R=0$, then $kR=0$ if and only if $k=0$.}
	\end{thm}
}
\slide{
	\begin{statementblock}{Fun Fact}
		If $r\in R$ is nilpotent, then $1-r$ is a unit.
	\end{statementblock}
	\begin{proof}
		\vskip 3in
	\end{proof}
}

\end{document}

	